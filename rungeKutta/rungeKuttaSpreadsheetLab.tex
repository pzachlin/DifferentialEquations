\documentclass{ximera}
%% You can put user macros here
%% However, you cannot make new environments

%\listfiles

% Get the 'old' hints/expandables, for use on ximera.osu.edu
%\def\xmNotHintAsExpandable{true}
%\def\xmNotExpandableAsAccordion{true}



%\graphicspath{{./}{firstExample/}{secondExample/}}
\graphicspath{{./}
{aboutDiffEq/}
{applicationsLeadingToDiffEq/}
{applicationsToCurves/}
{autonomousSecondOrder/}
{basicConcepts/}
{bernoulli/}
{constCoeffHomSysI/}
{constCoeffHomSysII/}
{constCoeffHomSysIII/}
{constantCoeffWithImpulses/}
{constantCoefficientHomogeneousEquations/}
{convolution/}
{coolingActivity/}
{directionFields/}
{drainingTank/}
{epidemicActivity/}
{eulersMethod/}
{exactEquations/}
{existUniqueNonlinear/}
{frobeniusI/}
{frobeniusII/}
{frobeniusIII/}
{global.css/}
{growthDecay/}
{heatingCoolingActivity/}
{higherOrderConstCoeff/}
{homogeneousLinearEquations/}
{homogeneousLinearSys/}
{improvedEuler/}
{integratingFactors/}
{interactExperiment/}
{introToLaplace/}
{introToSystems/}
{inverseLaplace/}
{ivpLaplace/}
{laplaceTable/}
{lawOfCooling/}
{linSysOfDiffEqs/}
{linearFirstOrderDiffEq/}
{linearHigherOrder/}
{mixingProblems/}
{motionUnderCentralForce/}
{nonHomogeneousLinear/}
{nonlinearToSeparable/}
{odesInSage/}
{piecewiseContForcingFn/}
{population/}
{reductionOfOrder/}
{regularSingularPts/}
{reviewOfPowerSeries/}
{rlcCircuit/}
{rungeKutta/}
{secondLawOfMotion/}
{separableEquations/}
{seriesSolNearOrdinaryPtI/}
{seriesSolNearOrdinaryPtII/}
{simplePendulum/}
{springActivity/}
{springProblemsI/}
{springProblemsII/}
{undCoeffHigherOrderEqs/}
{undeterminedCoeff/}
{undeterminedCoeff2/}
{unitStepFunction/}
{varParHigherOrder/}
{varParamNonHomLinSys/}
{variationOfParameters/}
}


\usepackage{tikz}
%\usepackage{tkz-euclide}
\usepackage{tikz-3dplot}
\usepackage{tikz-cd}
\usetikzlibrary{shapes.geometric}
\usetikzlibrary{arrows}
\usetikzlibrary{decorations.pathmorphing,patterns}
\usetikzlibrary{backgrounds} % added by Felipe
% \usetkzobj{all}   % NOT ALLOWED IN RECENT TeX's ...
\pgfplotsset{compat=1.13} % prevents compile error.

\pdfOnly{\renewcommand{\theHsection}{\thepart.section.\thesection}}  %% MAKES LINKS WORK should be added to CLS
\pdfOnly{\renewcommand{\part}[1]{\chapterstyle\title{#1}\begin{abstract}\end{abstract}\maketitle\def\thechaptertitle{#1}}}


\renewcommand{\vec}[1]{\mathbf{#1}}
\newcommand{\RR}{\mathbb{R}}
\providecommand{\dfn}{\textit}
\renewcommand{\dfn}{\textit}
\newcommand{\dotp}{\cdot}
\newcommand{\id}{\text{id}}
\newcommand\norm[1]{\left\lVert#1\right\rVert}
\newcommand{\dst}{\displaystyle}
 
\newtheorem{general}{Generalization}
\newtheorem{initprob}{Exploration Problem}

\tikzstyle geometryDiagrams=[ultra thick,color=blue!50!black]

\usepackage{mathtools}

\title{Spreadsheet Lab} \license{CC BY-NC-SA 4.0}

\begin{document}

\begin{abstract}
\end{abstract}
\maketitle

\begin{onlineOnly}
\section*{Spreadsheet Lab}
\end{onlineOnly}
 
Spreadsheets are a great way to analyze iterative processes such as the numerical methods presented here in Chapter 3.  In this lab you will implement the 4th order Runge Kutta method using a spreadsheet.
 
 
\begin{exploration}\label{lab3.3:exp1}
Example~\ref{example:3.1.2} asked us to use Euler's method with step sizes $h=0.1$, $h=0.05$, and $h=0.025$ to
find approximate values of the solution of the initial value problem
$$
y'+2y=x^3e^{-2x},\quad y(0)=1
$$
at $x=0, 0.1, 0.2, 0.3, \ldots, 1.0$.
 
The link below takes you to a spreadsheet which can be used to complete this exercise.  (It is the implementation of the improved Euler method we did in Exploration~\ref{lab3.2:exp1})  To use the spreadsheet, SAVE A COPY of the sheet.  The yellow boxes control the initial condition and the step size. 
     
\href{https://docs.google.com/spreadsheets/d/1UZyfR-vJ-ALm6GqvgHudH_0z-KIjuA2OZvAzZBKa4hc/edit?usp=sharing}{LINK TO SPREADSHEET}

This time we do not provide a video, as it is valuable for you to try to implement this method on your own!  You have everything you need to take on this task.  Here we remind you of the process as described before Example~\ref{exer:3.3.1}.

The Runge-Kutta method computes approximate values
$y_1, y_2, \dots, y_n$ of the solution of $\eqref{eq:3.3.1}$
at $x_0, x_0+h, \dots, x_0+nh$ as follows: Given $y_i$,
compute
\begin{eqnarray*} k_{1i}&=&f(x_i,y_i),\\
k_{2i}&=&f\left(x_i+\frac{h}{2},y_i+\frac{h}{2}k_{1i}\right),\\
k_{3i}&=&f\left(x_i+\frac{h}{2},y_i+\frac{h}{2}k_{2i}\right),\\
k_{4i}&=&f(x_i+h,y_i+hk_{3i}),
\end{eqnarray*}
and
$$
y_{i+1}=y_i+\frac{h}{6}(k_{1i}+2k_{2i}+2k_{3i}+k_{4i}).
$$

\end{exploration}



 
 




 
\end{document}