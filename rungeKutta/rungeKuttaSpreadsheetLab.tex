\documentclass{ximera}
\input{../preamble.tex}

\title{Spreadsheet Lab} \license{CC BY-NC-SA 4.0}

\begin{document}

\begin{abstract}
\end{abstract}
\maketitle

\begin{onlineOnly}
\section*{Spreadsheet Lab}
\end{onlineOnly}
 
Spreadsheets are a great way to analyze iterative processes such as the numerical methods presented here in Chapter 3.  In this lab you will implement the 4th order Runge Kutta method using a spreadsheet.
 
 
\begin{exploration}\label{lab3.3:exp1}
Example~\ref{example:3.1.2} asked us to use Euler's method with step sizes $h=0.1$, $h=0.05$, and $h=0.025$ to
find approximate values of the solution of the initial value problem
$$
y'+2y=x^3e^{-2x},\quad y(0)=1
$$
at $x=0, 0.1, 0.2, 0.3, \ldots, 1.0$.
 
The link below takes you to a spreadsheet which can be used to complete this exercise.  (It is the implementation of the improved Euler method we did in Exploration~\ref{lab3.2:exp1})  To use the spreadsheet, SAVE A COPY of the sheet.  The yellow boxes control the initial condition and the step size. 
     
\href{https://docs.google.com/spreadsheets/d/1UZyfR-vJ-ALm6GqvgHudH_0z-KIjuA2OZvAzZBKa4hc/edit?usp=sharing}{LINK TO SPREADSHEET}

This time we do not provide a video, as it is valuable for you to try to implement this method using a spreadsheet.
\end{exploration}



 
 




 
\end{document}