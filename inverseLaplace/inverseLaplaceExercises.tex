\documentclass{ximera}
%% You can put user macros here
%% However, you cannot make new environments

%\listfiles

% Get the 'old' hints/expandables, for use on ximera.osu.edu
%\def\xmNotHintAsExpandable{true}
%\def\xmNotExpandableAsAccordion{true}



%\graphicspath{{./}{firstExample/}{secondExample/}}
\graphicspath{{./}
{aboutDiffEq/}
{applicationsLeadingToDiffEq/}
{applicationsToCurves/}
{autonomousSecondOrder/}
{basicConcepts/}
{bernoulli/}
{constCoeffHomSysI/}
{constCoeffHomSysII/}
{constCoeffHomSysIII/}
{constantCoeffWithImpulses/}
{constantCoefficientHomogeneousEquations/}
{convolution/}
{coolingActivity/}
{directionFields/}
{drainingTank/}
{epidemicActivity/}
{eulersMethod/}
{exactEquations/}
{existUniqueNonlinear/}
{frobeniusI/}
{frobeniusII/}
{frobeniusIII/}
{global.css/}
{growthDecay/}
{heatingCoolingActivity/}
{higherOrderConstCoeff/}
{homogeneousLinearEquations/}
{homogeneousLinearSys/}
{improvedEuler/}
{integratingFactors/}
{interactExperiment/}
{introToLaplace/}
{introToSystems/}
{inverseLaplace/}
{ivpLaplace/}
{laplaceTable/}
{lawOfCooling/}
{linSysOfDiffEqs/}
{linearFirstOrderDiffEq/}
{linearHigherOrder/}
{mixingProblems/}
{motionUnderCentralForce/}
{nonHomogeneousLinear/}
{nonlinearToSeparable/}
{odesInSage/}
{piecewiseContForcingFn/}
{population/}
{reductionOfOrder/}
{regularSingularPts/}
{reviewOfPowerSeries/}
{rlcCircuit/}
{rungeKutta/}
{secondLawOfMotion/}
{separableEquations/}
{seriesSolNearOrdinaryPtI/}
{seriesSolNearOrdinaryPtII/}
{simplePendulum/}
{springActivity/}
{springProblemsI/}
{springProblemsII/}
{undCoeffHigherOrderEqs/}
{undeterminedCoeff/}
{undeterminedCoeff2/}
{unitStepFunction/}
{varParHigherOrder/}
{varParamNonHomLinSys/}
{variationOfParameters/}
}


\usepackage{tikz}
%\usepackage{tkz-euclide}
\usepackage{tikz-3dplot}
\usepackage{tikz-cd}
\usetikzlibrary{shapes.geometric}
\usetikzlibrary{arrows}
\usetikzlibrary{decorations.pathmorphing,patterns}
\usetikzlibrary{backgrounds} % added by Felipe
% \usetkzobj{all}   % NOT ALLOWED IN RECENT TeX's ...
\pgfplotsset{compat=1.13} % prevents compile error.

\pdfOnly{\renewcommand{\theHsection}{\thepart.section.\thesection}}  %% MAKES LINKS WORK should be added to CLS
\pdfOnly{\renewcommand{\part}[1]{\chapterstyle\title{#1}\begin{abstract}\end{abstract}\maketitle\def\thechaptertitle{#1}}}


\renewcommand{\vec}[1]{\mathbf{#1}}
\newcommand{\RR}{\mathbb{R}}
\providecommand{\dfn}{\textit}
\renewcommand{\dfn}{\textit}
\newcommand{\dotp}{\cdot}
\newcommand{\id}{\text{id}}
\newcommand\norm[1]{\left\lVert#1\right\rVert}
\newcommand{\dst}{\displaystyle}
 
\newtheorem{general}{Generalization}
\newtheorem{initprob}{Exploration Problem}

\tikzstyle geometryDiagrams=[ultra thick,color=blue!50!black]

\usepackage{mathtools}

\title{Exercises} \license{CC BY-NC-SA 4.0}

\begin{document}

\begin{abstract}
\end{abstract}
\maketitle

\begin{onlineOnly}
\section*{Exercises}
\end{onlineOnly}

\begin{problem}\label{exer:8.2.1}
 Use the table of Laplace transforms  to find the inverse Laplace
transform.

\begin{enumerate}
    \item $\frac{3}{(s-7)^4}$
    \item $\frac{2s-4}{s^2-4s+13}$
    \item $\frac{1}{s^2+4s+20}$
    \item $\frac{2}{s^2+9}$
    \item $\frac{s^2-1}{(s^2+1)^2}$
    \item $\frac{1}{(s-2)^2-4}$
    \item $\frac{12s-24}{(s^2-4s+85)^2}$
    \item $\frac{2}{(s-3)^2-9}$
    \item $\frac{s^2-4s+3}{(s^2-4s+5)^2}$
\end{enumerate}
\end{problem}

\begin{problem}\label{exer:8.2.2}
 Use Theorem~\ref{thmtype:8.2.1} and the table of Laplace transforms
  to find the inverse Laplace transform.

\begin{enumerate}
    \item $\frac{2s+3}{(s-7)^4}$

\begin{solution}
$\frac{2s+3}{(s-7)^4}=\frac{2(s-7)+17}{(s-7)^4}=
\frac{2}{(s-7)^3}+\frac{17}{(s-7)^4}
=\frac{2!}{(s-7)^3}+\frac{17}{6}\frac{3!}{(s-7)^4}
\leftrightarrow e^{7t}\left(t^2+\frac{17}{6}t^3\right)$.
\end{solution}

    \item $\frac{s^2-1}{(s-2)^6}$

\begin{solution}
$\frac{s^2-1}{(s-2)^6}=\frac{[(s-2)+2]^2-1}{(s-2)^6}
=\frac{(s-2)^2+4(s-2)+3}{(s-2)^6}
=\frac{1}{(s-2)^4}+\frac{4}{(s-2)^5}+\frac{3}{(s-2)^6}
=\frac{1}{6}\frac{3!}{(s-2)^4}+\frac{1}{6}\frac{4!}{(s-2)^5}
+\frac{1}{40} \frac{5!}{(s-2)^6}\leftrightarrow\left(\frac{1}{6}t^3+\frac{1}{6}t^4+\frac{1}{40}t^5\right)e^{2t}$.
\end{solution}
    
    \item $\frac{s+5}{s^2+6s+18}$

\begin{solution}
$\frac{s+5}{s^2+6s+18}=\frac{(s+3)}{(s+3)^2+9}
+\frac{2}{3}\frac{3}{(s+3)^2+9}
\leftrightarrow e^{-3t}\left(\cos 3t+\frac{2}{3}\sin 3t\right)$.
\end{solution}
    
    \item $\frac{2s+1}{s^2+9}$

\begin{solution}
$\frac{2s+1}{s^2+9}=2\frac{s}{s^2+9}+\frac{1}{3}\frac{3}{s^2+9}\leftrightarrow
2\cos 3t+\frac{1}{3}\sin 3t$.
\end{solution}

    \item $\frac{s}{s^2+2s+1}$
\begin{solution}
$\frac{s}{s^2+2s+1}=\frac{(s+1)-1}{(s+1)^2}=
\frac{1}{s+1}-\frac{1}{(s+1)^2}\leftrightarrow
(1-t)e{^{-t}}$. 
\end{solution}

    \item $\frac{s+1}{s^2-9}$

\begin{solution}
$\frac{s+1}{s^2-9}=\frac{s}{s^2-9}+\frac{1}{3}\frac{3}{s^2-9}\leftrightarrow {\cosh 3t+\frac{1}{3}\sinh 3t}$.   
\end{solution}


    \item $\frac{s^3+2s^2-s-3}{(s+1)^4}$

\begin{solution}
Expand the numerator in powers of $s+1$:
$s^3+2s^2-s-3=[(s+1)-1]^3+2[(s+1)-1]^2-[(s+1)-1]-3=
(s+1)^3-(s+1)^2-2(s+1)-1$; therefore
$\frac{s^3+2s^2-s-3}{(s+1)^4}=\frac{1}{s+1}-\frac{1}{(s+1)^2}
-\frac{2}{(s+1)^3}-\frac{1}{6}\frac{6}{(s+1)^4}\leftrightarrow
{\left(1-t-t^2-\frac{1}{6}t^3\right)e^{-t}}$.
\end{solution}

    \item $\frac{2s+3}{(s-1)^2+4}$

\begin{solution}
$\frac{2s+3}{(s-1)^2+4}=2\frac{(s-1)}{(s-1)^2+4}
+\frac{5}{2}\frac{2}{(s-1)^2+4}\leftrightarrow
{e^t\left(2\cos 2t+\frac{5}{2}\sin 2t\right)}$.  
\end{solution}

    \item $\frac{1}{s}-\frac{s}{s^2+1}$

\begin{solution}
$\frac{1}{s}-\frac{s}{s^2+1}\leftrightarrow 1-\cos t$. 
\end{solution}

    \item $\frac{3s+4}{s^2-1}$

\begin{solution}
$\frac{3s+4}{s^2-1}=\frac{3s}{s^2-1}+\frac{4}{s^2-1}
\leftrightarrow3\cosh t+4\sinh t$.
Alternatively,
$\frac{3s+4}{s^2-1}=\frac{3s+4}{(s-1)(s+1)}=\frac{1}{2}\left[\frac{7}{s-1}-\frac{1}{s+1}\right]
\leftrightarrow\frac{7e^t-e^{-t}}{2}$. 
\end{solution}

    \item $\frac{3}{s-1}+\frac{4s+1}{s^2+9}$

\begin{solution}
$\frac{3}{s-1}+\frac{4s+1}{s^2+9}=3\frac{1}{s-1}
+4\frac{s}{s^2+9}+\frac{1}{3}\frac{3}{s^2+9}\leftrightarrow
3e^t+4\cos 3t+\frac{1}{3}\sin 3t$. 
\end{solution}

    \item $\frac{3}{(s+2)^2}-\frac{2s+6}{s^2+4}$

\begin{solution}
$\frac{3}{(s+2)^2}-\frac{2s+6}{s^2+4}=
3\frac{1}{(s+2)^2}-2\frac{s}{s^2+4}-3\frac{2}{s^2+4}\leftrightarrow
3te^{-2t}-2\cos 2t-3\sin 2t$.
\end{solution}
\end{enumerate}
\end{problem}

\begin{problem}\label{exer:8.2.3}
 Use Heaviside's method to find the inverse Laplace transform.

\begin{enumerate}
    \item $\frac{3-(s+1)(s-2)}{(s+1)(s+2)(s-2)}$
    \item $\frac{7+(s+4)(18-3s)}{(s-3)(s-1)(s+4)}$
    \item $\frac{2+(s-2)(3-2s)}{(s-2)(s+2)(s-3)}$
    \item $\frac{3-(s-1)(s+1)}{(s+4)(s-2)(s-1)}$
    \item $\frac{3+(s-2)(10-2s-s^2)}{(s-2)(s+2)(s-1)(s+3)}$
    \item $\frac{3+(s-3)(2s^2+s-21)}{(s-3)(s-1)(s+4)(s-2)}$
\end{enumerate}
\end{problem}

\begin{problem}\label{exer:8.2.4}
 Find the inverse Laplace transform.

\begin{enumerate}
    \item $\frac{2+3s}{(s^2+1)(s+2)(s+1)}$

\begin{solution}
$$
\frac{2+3s}{(s^2+1)(s+2)(s+1)}=
\frac{A}{s+2}+\frac{B}{s+1}+\frac{Cs+D}{s^2+1},
$$
where
$$
A(s^2+1)(s+1)+
B(s^2+1)(s+2)+(Cs+D)(s+2)(s+1)=2+3s.
$$
$$
\begin{array}{rcrl}
-5A&=&-4&(\mbox{set }s=-2);\\
2B&=&-1& (\mbox{set }s=-1);\\
A+2B+2D&=&2&(\mbox{set }s=0);\\
A+B+C&=&0&(\mbox{equate coefficients of }s^3).
\end{array}
$$
Solving this system yields $A=\frac{4}{5}$, $B=-\frac{1}{2}$,
$C=-\frac{3}{10}$, $D=\frac{11}{10}$. Therefore,
\begin{eqnarray*}
\frac{2+3s}{(s^2+1)(s+2)(s+1)}&=&
\frac{4}{5}\frac{1}{s+2}-\frac{1}{2}\frac{1}{s+1}-\frac{1}{10}\frac{3s-11}{s^2+1}\\
&\leftrightarrow&
\frac{4}{5}e^{-2t}-\frac{1}{2}e^{-t}-\frac{3}{10}\cos t
+\frac{11}{10}\sin t.
\end{eqnarray*}
\end{solution}

    \item $\frac{3s^2+2s+1}{(s^2+1)(s^2+2s+2)}$
\begin{solution}
$$
\frac{3s^2+2s+1}{(s^2+1)(s^2+2s+2)}=
\frac{As+B}{s^2+1}+\frac{C(s+1)+D}{(s+1)^2+1},
$$
where
$$
(As+B)((s+1)^2+1)+(C(s+1)+D)(s^2+1)=3s^2+2s+1.
$$
$$
\begin{array}{rcrl}
2B+C+D&=&1&(\mbox{set }s=0);\\
-A+B+2D&=&2& (\mbox{set }s=-1);\\
2B+C+D&=&1&(\mbox{set }s=0);\\
A+C&=&0&(\mbox{equate coefficients of }s^3).
\end{array}
$$
Solving this system yields
$A=6/5$, $B=2/5$, $C=-6/5$, $D=7/5$.
Therefore,
\begin{eqnarray*}
\frac{3s^2+2s+1}{(s^2+1)(s^2+2s+2)}&=&
\frac{1}{5}\left[\frac{6s+2}{s^2+1}-\frac{6(s+1)-7}{(s+1)^2+1}\right]\\
&\leftrightarrow&
 \frac{6}{5}\cos t+\frac{2}{5}\sin t
-\frac{6}{5}e^{-t}\cos t+\frac{7}{5}e^{-t}\sin t.
\end{eqnarray*}
\end{solution}

    \item $\frac{3s+2}{(s-2)(s^2+2s+5)}$

\begin{solution}
$s^2+2s+5=(s+1)^2+4$;
$$
\frac{3s+2}{(s-2)((s+1)^2+4)}=\frac{A}{s-2}+
\frac{B(s+1)+C}{(s+1)^2+4},
$$
where
$$
A\left((s+1)^2)+4\right)+
\left(B(s+1)+C\right)(s-2)=3s+2.
$$
$$
\begin{array}{rcrl}
13A&=&8&(\mbox{set }s=2);\\
4A-3C&=&-1&(\mbox{set }s=-1);\\
A+B&=&0&(\mbox{equate coefficients of }s^2).
\end{array}
$$
Solving this system yields $A=\frac{8}{13}$, $B=-\frac{8}{13}$,
$C=\frac{15}{13}$. Therefore,
\begin{eqnarray*}
\frac{3s+2}{(s-2)((s+1)^2+4)}&=&\frac{1}{13}\left[\frac{8}{s-2}-
\frac{8(s-1)-15}{(s+1)^2+4}\right]\\
&\leftrightarrow&
\frac{8}{13}e^{2t}-\frac{8}{13}e^{-t}\cos 2t+\frac{15}{26}
e^{-t}\sin 2t.
\end{eqnarray*}
\end{solution}

    \item $\frac{3s^2+2s+1}{(s-1)^2(s+2)(s+3)}$

\begin{solution}
$$
\frac{3s^2+2s+1}{(s-1)^2(s+2)(s+3)}=
\frac{A}{s-1}+\frac{B}{(s-1)^2}+\frac{C}{s+2}+\frac{D}{s+3},
$$
where
$$
(A(s-1)+B)(s+2)(s+3)+(C(s+3)+D(s+2))(s-1)^2=3s^2+2s+1.
$$
$$
\begin{array}{rcrl}
12B&=&6&(\mbox{set }s=1);\\
9C&=&9& (\mbox{set }s=-2);\\
-16D&=&22&(\mbox{set }s=-3);\\
A+C+D&=&0&(\mbox{equate coefficients of }s^3).
\end{array}
$$
Solving this system yields
$A=3/8$, $B=1/2$, $C=1$, $D=-11/8$.
Therefore,
\begin{eqnarray*}
\frac{3s^2+2s+1}{(s-1)^2(s+2)(s+3)}&=&
\frac{3}{8}\frac{1}{s-1}+\frac{1}{2}\frac{1}{(s-1)^2}+\frac{1}{s+2}-\frac{11}{8}\frac{1}{s+3}\\
&\leftrightarrow&
\frac{3}{8}e^t+\frac{1}{2}te^t+e^{-2t}-\frac{11}{8}e^{-3t}.
\end{eqnarray*}
\end{solution}
    
    \item $\frac{2s^2+s+3}{(s-1)^2(s+2)^2}$

\begin{solution}
$$
\frac{2s^2+s+3}{(s-1)^2(s+2)^2}=
\frac{A}{s-1}+\frac{B}{(s-1)^2}+\frac{C}{s+2}+\frac{D}{(s+2)^2},
$$
where
$$
(A(s-1)+B)(s+2)^2+(C(s+2)+D)(s-1)^2=2s^2+s+3.
$$
$$
\begin{array}{rcrl}
9B&=&6&(\mbox{set }s=1);\\
9D&=&9& (\mbox{set }s=-2);\\
-4A+4B+2C+D&=&3&(\mbox{set }s=0);\\
A+C&=&0&(\mbox{equate coefficients of }s^3).
\end{array}
$$
Solving this system yields
$A=1/9$, $B=2/3$, $C=-1/9$, $D=1$.
Therefore,
\begin{eqnarray*}
\frac{2s^2+s+3}{(s-1)^2(s+2)^2}&=&
\frac{1}{9}\frac{1}{s-1}+\frac{2}{3}\frac{1}{(s-1)^2}-\frac{1}{9}\frac{1}{s+2}+\frac{1}{(s+2)^2}\\
&\leftrightarrow&
\frac{1}{9}e^t+\frac{2}{3}te^t-\frac{1}{9}e^{-2t}+te^{-2t}.
\end{eqnarray*}
\end{solution}
    
    \item $\frac{3s+2}{(s^2+1)(s-1)^2}$

\begin{solution}
$$
\frac{3s+2}{(s^2+1)(s-1)^2}=\frac{A}{s-1}+\frac{B}{(s-1)^2}+
\frac{Cs+D}{s^2+1},
$$
where
$$
A(s-1)(s^2+1)+B(s^2+1)+(Cs+D)(s-1)^2=3s+2.
\text{(A)}
$$
Setting $s=1$ yields $2B=5$, so $B=\frac{5}{2}$.
Substituting this into (A) shows that
\begin{eqnarray*}
A(s-1)(s^2+1)+(Cs+D)(s-1)^2&=&3s+2-\frac{5}{2}(s^2+1)\\
&=&-\frac{5s^2-6s+1}{2}=-\frac{(s-1)(5s-1)}{2}.
\end{eqnarray*}
Therefore,
$$
A(s^2+1)+(Cs+D)(s-1)=\frac{1-5s}{2}.
$$
$$
\begin{array}{rcrl}
2A&=&-2&(\mbox{set }s=1);\\
A-D&=&1/2&(\mbox{set }s=0);\\
A+C&=&0&(\mbox{equate coefficients of }s^2).
\end{array}
$$
Solving this system yields $A=-1$, $C=1$,
$D=-\frac{3}{2}$. Therefore,
\begin{eqnarray*}
\frac{3s+2}{(s^2+1)(s-1)^2}&=&-\frac{1}{s-1}+\frac{5}{2}\frac{1}{(s-1)^2}
+\frac{s-3/2}{s^2+1}\\
&\leftrightarrow&
-e^t+\frac{5}{2}te^t+\cos t-\frac{3}{2}\sin t.
\end{eqnarray*}
\end{solution}
\end{enumerate}
\end{problem}

\begin{problem}\label{exer:8.2.5}
 Use the method of Example~\ref{example:8.2.9}  to find the
inverse Laplace transform.

\begin{enumerate}
    \item $\frac{3s+2}{(s^2+4)(s^2+9)}$
    \item $\frac{-4s+1}{(s^2+1)(s^2+16)}$
    \item $\frac{5s+3}{(s^2+1)(s^2+4)}$
    \item $\frac{-s+1}{(4s^2+1)(s^2+1)}$
    \item $\frac{17s-34}{(s^2+16)(16s^2+1)}$
    \item $\frac{2s-1}{(4s^2+1)(9s^2+1)}$
\end{enumerate}
\end{problem}

\begin{problem}\label{exer:8.2.6}
 Find the inverse Laplace transform.

\begin{enumerate}
    \item $\frac{17 s-15}{(s^2-2s+5)(s^2+2s+10)}$

\begin{solution}
$$
\frac{17s-15}{(s^2-2s+5)(s^2+2s+10)}
=\frac{A(s-1)+B}{(s-1)^2+4}+\frac{C(s+1)+D}{(s+1)^2+9}
$$
where
$$
(A(s-1)+B)((s+1)^2+9)+(C(s+1)+D)((s-1)^2+4)=17 s-15.
$$
$$
\begin{array}{rcrl}
13B+8C+4D&=&2&(\mbox{set }s=1);\\
-18A+9B+8D&=&-32& (\mbox{set }s=-1);\\
-10A+10B+5C+5D&=&-15&(\mbox{set }s=0);\\
A+C&=&0&(\mbox{equate coefficients of }s^3).
\end{array}
$$
Solving this system yields $A=1$, $B=2$,
$C=-1$, $D=-4$. Therefore,
\begin{eqnarray*}
\frac{17s-15}{(s^2-2s+5)(s^2+2s+10)}
&=&
\frac{(s-1)+2}{(s-1)^2+4}-\frac{(s+1)+4}{(s+1)^2+9}
\\&\leftrightarrow&
 e^t(\cos 2t+\sin 2t)-e^{-t}\left(\cos
3t+\frac{4}{3}\sin 3t\right).
\end{eqnarray*}
\end{solution}

    \item $\frac{8s+56}{(s^2-6s+13)(s^2+2s+5)}$

\begin{solution}
$$\frac{8s+56}{(s^2-6s+13)(s^2+2s+5)}
=\frac{A(s-3)+B}{(s-3)^2+4}+\frac{C(s+1)+D}{(s+1)^2+4}
$$
where
$$
(A(s-3)+B)((s+1)^2+4)+(C(s+1)+D)((s-3)^2+4)=8s+56.
$$
$$
\begin{array}{rcrl}
20B+16C+4D&=&80&(\mbox{set }s=3);\\
-16A+4B+20D&=&48& (\mbox{set }s=-1);\\
-15A+5B+13C+13D&=&56&(\mbox{set }s=0);\\
A+C&=&0&(\mbox{equate coefficients of }s^3).
\end{array}
$$
Solving this system yields $A=-1$, $B=3$,
$C=1$, $D=1$. Therefore,
\begin{eqnarray*}
\frac{8s+56}{(s^2-6s+13)(s^2+2s+5)}
&=&\frac{-(s-3)+3}{(s-3)^2+4}+\frac{(s+1)+1}{(s+1)^2+4}
\\&\leftrightarrow&
{e^{3t}\left(-\cos 2t+\frac{3}{2}\sin
2t\right)+e^{-t}\left(\cos 2t+\frac{1}{2}\sin 2t\right)}.
\end{eqnarray*}
\end{solution}

    \item $\frac{s+9}{(s^2+4s+5)(s^2-4s+13)}$

\begin{solution}
$$\frac{s+9}{(s^2+4s+5)(s^2-4s+13)}
=\frac{A(s+2)+B}{(s+2)^2+1}+\frac{C(s-2)+D}{(s-2)^2+9}
$$
where
$$
(A(s+2)+B)((s-2)^2+9)+(C(s-2)+D)((s+2)^2+1)=s+9.
$$
$$
\begin{array}{rcrl}
25B-4C+D&=&7&(\mbox{set }s=-2);\\
36A+9B+17D&=&11& (\mbox{set }s=2);\\
26A+13B-10C+5D&=&9&(\mbox{set }s=0);\\
A+C&=&0&(\mbox{equate coefficients of }s^3).
\end{array}
$$
Solving this system yields $A=1/8$, $B=1/4$,
$C=-1/8$, $D=1/4$. Therefore,
\begin{eqnarray*}
\frac{s+9}{(s^2+4s+5)(s^2-4s+13)}
&=&
=\left[\frac{1}{8}\frac{(s+2)+2}{(s+2)^2+1}-\frac{(s-2)-2}{(s-2)^2+3}\right]
\\&\leftrightarrow&
{e^{-2t}\left(\frac{1}{8}\cos t+\frac{1}{4}\sin t\right)-e^{2t}\left(\frac{1}{8}\cos 3t-\frac{1}{12}\sin
3t\right)}.
\end{eqnarray*}
\end{solution}
    
    \item $\frac{3s-2}{(s^2-4s+5)(s^2-6s+13)}$

\begin{solution}
$$\frac{3s-2}{(s^2-4s+5)(s^2-6s+13)}
=\frac{A(s-2)+B}{(s-2)^2+1}+\frac{C(s-3)+D}{(s-3)^2+4}
$$
where
$$
(A(s-2)+B)((s-3)^2+4)+(C(s-3)+D)((s-2)^2+1)=3s-2.
$$
$$
\begin{array}{rcrl}
5B-C+D&=&4&(\mbox{set }s=2);\\
4A+4B+2D&=&7& (\mbox{set }s=3);\\
-26A+13B-15C+5D&=&-2&(\mbox{set }s=0);\\
A+C&=&0&(\mbox{equate coefficients of }s^3).
\end{array}
$$
Solving this system yields $A=1$, $B=1/2$,
$C=-1$, $D=1/2$. Therefore,
\begin{eqnarray*}
\frac{3s-2}{(s^2-4s+5)(s^2-6s+13)}
&=&
=\frac{1}{2}\left[\frac{2(s-2)+1}{(s-2)^2+1}-\frac{2(s-3)-1}{(s-3)^2+4}\right]
\\&\leftrightarrow&
e^{2t}\left(\cos t+\frac{1}{2}\sin
t\right)-e^{3t}\left(\cos 2t-\frac{1}{4}\sin 2t\right).
\end{eqnarray*}
\end{solution}
    
    \item $\frac{3s-1}{(s^2-2s+2)(s^2+2s+5)}$

\begin{solution}
$$
\frac{3s-1}{(s^2-2s+2)(s^2+2s+5)}
=\frac{A(s-1)+B}{(s-1)^2+1}+\frac{C(s+1)+D}{(s+1)^2+4}
$$
where
$$
(A(s-1)+B)((s+1)^2+4)+(C(s+1)+D)((s-1)^2+1)=3s-1.
$$
$$
\begin{array}{rcrl}
8B+2C+D&=&2&(\mbox{set }s=1);\\
-8A+4B+5D&=&-4& (\mbox{set }s=-1);\\
-5A+5B+2C+2D&=&-1&(\mbox{set }s=0);\\
A+5B+C&=&0&(\mbox{equate coefficients of }s^3).
\end{array}
$$
Solving this system yields $A=1/5$, $B=2/5$,
$C=-1/5$, $D=-4/5$. Therefore,
\begin{eqnarray*}
\frac{3s-1}{(s^2-2s+2)(s^2+2s+5)}
&=&
\frac{1}{5}\left[\frac{(s-1)+2}{(s-1)^2+1}-\frac{(s+1)+4}{(s+1)^2+4}\right].
\\&\leftrightarrow&
e^t\left(\frac{1}{5}\cos t+\frac{2}{5} \sin t\right)
-e^{-t}\left(\frac{1}{5}\cos 2t+\frac{2}{5}\sin 2t\right).
\end{eqnarray*}
\end{solution}
    
    \item $\frac{20s+40}{(4s^2-4s+5)(4s^2+4s+5)}$

\begin{solution}
$$
\frac{20s+40}{(4s^2-4s+5)(4s^2+4s+5)}
=[\frac{A(s-1/2)+B}{(s-1/2)^2+1}+
\frac{C(s+1/2)+D}{(s+1/2)^2+1}
$$
where
$$
(A(s-1/2)+B)((s+1/2)^2+1)+(C(s+1/2)+D)((s-1/2)^2+1)=\frac{5s+10}{4}.
$$
$$
\begin{array}{rcrl}
2B+C+D&=&25/8&(\mbox{set }s=1/2);\\
-A+B+2D&=&15/8& (\mbox{set }s=-1/2);\\
-5A+10B+5C+10D&=&20&(\mbox{set }s=0);\\
A+C&=&0&(\mbox{equate coefficients of }s^3).
\end{array}
$$
Solving this system yields $A=-1$, $B=9/8$,
$C=1$, $D=-1/8$. Therefore,
\begin{eqnarray*}
\frac{20s+40}{(4s^2-4s+5)(4s^2+4s+5)}
&=&
\frac{1}{8}\left[\frac{-8(s-1/2)+9}{(s-1/2)^2+1}+\frac{8(s+1/2)
}{(s+1/2)^2+1}\right] %check this term
\\&\leftrightarrow&
e^{t/2}\left(-\cos t+\frac{9}{8}\sin
t\right)+e^{-t/2}\left(\cos t-\frac{1}{8}\sin t\right).
\end{eqnarray*}
\end{solution}    
\end{enumerate}
\end{problem}

\begin{problem}\label{exer:8.2.7}
Find the inverse Laplace transform.

\begin{enumerate}
    \item $\frac{1}{s(s^2+1)}$
    \item $\frac{1}{(s-1)(s^2-2s+17)}$
    \item $\frac{3s+2}{(s-2)(s^2+2s+10)}$
    \item $\frac{34-17s}{(2s-1)(s^2-2s+5)}$
    \item $\frac{s+2}{(s-3)(s^2+2s+5)}$
    \item $\frac{2s-2}{(s-2)(s^2+2s+10)}$
\end{enumerate}
\end{problem}

\begin{problem}\label{exer:8.2.8}
 Find the inverse Laplace transform.
\begin{enumerate}
\item $\frac{2s+1}{(s^2+1)(s-1)(s-3)}$

\begin{solution}
$$
\frac{2s+1}{(s^2+1)(s-1)(s-3)}
=\frac{A}{s-1}+\frac{B}{s-3}+\frac{Cs+D}{s^2+1}
$$
where
$$
(A(s-3)+B(s-1))(s^2+1)+(Cs+D)(s-1)(s-3)=2s+1.
$$
$$
\begin{array}{rcrl}
-4A&=&3&(\mbox{set }s=1);\\
20B&=&7& (\mbox{set }s=3);\\
-3A-B+3D&=&1& (\mbox{set }s=0);\\
A+B+C&=&0&(\mbox{equate coefficients of }s^3).
\end{array}
$$
Solving this system yields $A=-3/4$, $B=7/20$,
$C=2/5$, $D=-3/10$. Therefore,
\begin{eqnarray*}
\frac{2s+1}{(s^2+1)(s-1)(s-3)}
&=&
-\frac{3}{4}\frac{1}{s-1}+\frac{7}{20}\frac{1}{s-3}+\frac{2}{5}\frac{s}{s^2+1}-\frac{3}{10}\frac{1}{s^2+1}
\\&\leftrightarrow&
-\frac{3}{4}e^t+\frac{7}{20}e^{3t}+\frac{2}{5}\cos t-\frac{3}{10}\sin t.
\end{eqnarray*}
\end{solution}

\item $\frac{s+2}{(s^2+2s+2)(s^2-1)}$



\begin{solution}
$$
\frac{s+2}{(s^2+2s+2)(s^2-1)}
=\frac{A}{s-1}+\frac{B}{s+1}+\frac{C(s+1)+D}{(s+1)^2+1}
$$
where
$$
(A(s+1)+B(s-1))((s+1)^2+1)+(C(s+1)+D)(s^2-1)=s+2.
$$
$$
\begin{array}{rcrl}
10A&=&3&(\mbox{set }s=1);\\
-2B&=&1& (\mbox{set }s=-1);\\
2A-2B-C-D&=&2& (\mbox{set }s=0);\\
A+B+C&=&0&(\mbox{equate coefficients of }s^3).
\end{array}
$$
Solving this system yields $A=3/10$, $B=-1/2$,
$C=1/5$, $D=-3/5$. Therefore,
\begin{eqnarray*}
\frac{s+2}{(s^2+2s+2)(s^2-1)}
&=&
\frac{3}{10}\frac{1}{s-1}-\frac{1}{2}\frac{1}{s+1}+\frac{1}{5}\frac{s+1}{(s+1)^2+1}
-\frac{3}{5}\frac{1}{(s+1)^2+1}
\\&\leftrightarrow&
\frac{3}{10}e^{t}-\frac{1}{2}e^{-t}
+\frac{1}{5}e^{-t}\cos t-\frac{3}{5}e^{-t}\sin t.
\end{eqnarray*}
\end{solution}

\item $\frac{2s-1}{(s^2-2s+2)(s+1)(s-2)}$

\begin{solution}
$$
\frac{2s-1}{(s^2-2s+2)(s+1)(s-2)}
=\frac{A}{s-2}+\frac{B}{s+1}+\frac{C(s-1)+D}{(s-1)^2+1}
$$
where
$$
(A(s+1)+B(s-2))((s-1)^2+1)+(C(s-1)+D)(s-2)(s+1)=2s-1.
$$
$$
\begin{array}{rcrl}
6A&=&3&(\mbox{set }s=2);\\
-15B&=&-3& (\mbox{set }s=-1);\\
2A-4B+2C-2D&=&-1& (\mbox{set }s=0);\\
A+B+C&=&0&(\mbox{equate coefficients of }s^3).
\end{array}
$$
Solving this system yields $A=1/2$, $B=1/5$,
$C=-7/10$, $D=-1/10$. Therefore,
\begin{eqnarray*}
\frac{2s-1}{(s^2-2s+2)(s+1)(s-2)}
&=&
=\frac{1}{2}\frac{1}{s-2}+\frac{1}{5}\frac{1}{s+1}-\frac{7}{10}\frac{s-1}{(s-1)^2+1}
-\frac{1}{10}\frac{1}{(s-1)^2+1}
\\&\leftrightarrow&
\frac{1}{2}e^{2t}+\frac{1}{5} e^{-t}
-\frac{7}{10}e^t\cos t-\frac{1}{10}e^t\sin t.
\end{eqnarray*}
\end{solution}

\item $\frac{s-6}{(s^2-1)(s^2+4)}$

\begin{solution}
$$
\frac{s-6}{(s^2-1)(s^2+4)}
=\frac{A}{s-1}+\frac{B}{s+1}+\frac{Cs+D}{s^2+4}
$$
where
$$
(A(s+1)+B(s-1))(s^2+4)+(Cs+D)(s^2-1)=s-6.
$$
$$
\begin{array}{rcrl}
10A&=&-5&(\mbox{set }s=1);\\
-10B&=&-7& (\mbox{set }s=-1);\\
4A-4B-D&=&-6& (\mbox{set }s=0);\\
A+B+C&=&0&(\mbox{equate coefficients of }s^3).
\end{array}
$$
Solving this system yields $A=-1/2$, $B=7/10$,
$C=-1/5$, $D=6/5$. Therefore,
\begin{eqnarray*}
\frac{s-6}{(s^2-1)(s^2+4)}
&=&
=-\frac{1}{2}\frac{1}{s-1}+\frac{7}{10}\frac{1}{s+1}-\frac{1}{5}\frac{s}{s^2+4}
+\frac{3}{5}+\frac{1}{s^2+4}
\\&\leftrightarrow&
-\frac{1}{2}e^t+\frac{7}{10}e^{-t}-\frac{1}{5}\cos 2t+\frac{3}{5}\sin 2t.
\end{eqnarray*}
\end{solution}

\item $\frac{2s-3}{s(s-2)(s^2-2s+5)}$

\begin{solution}
$$
\frac{2s-3}{s(s-2)(s^2-2s+5)}
=\frac{A}{s}+\frac{B}{s-2}+\frac{C(s-1)+D}{(s-1)^2+4}
$$
where
$$
(A(s-2)+Bs)((s-1)^2+4)+(C(s-1)+D)s(s-2)=2s-3.
$$
$$
\begin{array}{rcrl}
-10A&=&-3&(\mbox{set }s=0);\\
10B&=&1& (\mbox{set }s=2);\\
-4A+4B-D&=&-1& (\mbox{set }s=1);\\
A+B+C&=&0&(\mbox{equate coefficients of }s^3).
\end{array}
$$
Solving this system yields $A=3/10$, $B=1/10$,
$C=-2/5$, $D=1/5$. Therefore,
\begin{eqnarray*}
\frac{2s-3}{s(s-2)(s^2-2s+5)}
&=&
=\frac{3}{10s}+\frac{1}{10}\frac{1}{s-2}-\frac{2}{5}\frac{s-1}{(s-1)^2+4}
+\frac{1}{5}\frac{1}{(s-1)^2+4}
\\&\leftrightarrow&
\frac{3}{10}+\frac{1}{10}e^{2t}-\frac{2}{5}e^t\cos 2t
+\frac{1}{10}e^t\sin2t.
\end{eqnarray*}
\end{solution}

\item $\frac{5s-15}{(s^2-4s+13)(s-2)(s-1)}$

\begin{solution}
$$
\frac{5s-15}{(s^2-4s+13)(s-2)(s-1)}
=\frac{A}{s-1}+\frac{B}{s-2}+\frac{C(s-2)+D}{(s-2)^2+9}
$$
where
$$
(A(s-2)+B(s-1))((s-2)^2+9)+(C(s-2)+D)(s-1)(s-2)=5s-15.
$$
$$
\begin{array}{rcrl}
-10A&=&-10&(\mbox{set }s=1);\\
9B&=&-5& (\mbox{set }s=2);\\
-26A-13B-4C+2D&=&-15& (\mbox{set }s=0);\\
A+B+C&=&0&(\mbox{equate coefficients of }s^3).
\end{array}
$$
Solving this system yields $A=1$, $B=-5/9$,
$C=-4/9$, $D=1$. Therefore,
\begin{eqnarray*}
\frac{5s-15}{(s^2-4s+13)(s-2)(s-1)}
&=&
=\frac{1}{s-1}-\frac{5}{9}\frac{1}{s-2}-\frac{4}{9}\frac{s-2}{(s-2)^2+9}
+\frac{1}{(s-2)^2+9}
\\&\leftrightarrow&
e^t-\frac{5}{9}e^{2t}-\frac{4}{9}e^{2t}\cos 3t+\frac{1}{3}e^{2t}\sin 3t.
\end{eqnarray*}
\end{solution}

\end{enumerate}
\end{problem}

\begin{problem}\label{exer:8.2.9}
 Given that $f(t)\leftrightarrow F(s)$, find the inverse
Laplace transform of $F(as-b)$, where $a>0$.
\end{problem}

\begin{problem}\label{exer:8.2.10}
\begin{enumerate}
\item  % (a)
If $s_1$, $s_2$, \dots, $s_n$ are distinct and $P$ is a polynomial of
degree less than $n$, then
$$
\frac{P(s)}{(s-s_1)(s-s_2)\cdots(s-s_n)}=
\frac{A_1}{s-s_1}+\frac{A_2}{s-s_2}+\cdots+\frac{A_n}{s-s_n}.
$$
Multiply through by $s-s_i$ to show that
 $A_i$ can be obtained by ignoring the factor $s-s_i$ on the
left and setting $s=s_i$ elsewhere.

\begin{solution}
Let $i=1$. (The proof for $i=2,\dots,n$) is similar.
Multiplying the given equation through by $s-s_1$ yields
$$
\frac{P(s)}{(s-s_2)\cdots(s-s_n)}=
A_1+(s-s_1)\left[\frac{A_2}{s-s_2}+\cdots+\frac{A_n}{s-s_n}\right],
$$
and setting $s=s_1$ yields
$A_1=\frac{P(s_1)}{(s_1-s_2)\cdots(s_2-s_n)}$.
\end{solution}

\item % (b)
Suppose $P$ and $Q_1$ are polynomials such that
$\text{degree}(P)\le\text{degree}(Q_1)$ and $Q_1(s_1)\ne0$.
Show that the coefficient of $1/(s-s_1)$ in the partial fraction expansion of
$$
F(s)=\frac{P(s)}{(s-s_1)Q_1(s)}
$$
is $P(s_1)/Q_1(s_1)$.

\begin{solution}
From calculus we know that $F$  has a partial
fraction expansion of the form
$\frac{P(s)}{(s-s_1)Q_1(s)}=\frac{A}{s-s_1}+G(s)$ where $G$ is
continuous at $s_1$. Multiplying through by $s-s_1$ shows that
$\frac{P(s)}{Q_1(s)}=A+(s-s_1)G(s)$. Now set $s=s_1$ to obtain
$A=\frac{P(s_1)}{Q(s_1)}$.
\end{solution}

\item % (c)
Explain how the two results above are related.

\begin{solution}
The second result is generalization of the first, since it shows that if $s_1$ is a simple zero
of the denominator of the rational function, then Heaviside's
method can be used to determine the coefficient of $1/(s-s_1)$
in the partial fraction expansion even if some of the other zeros
of the denominator are repeated or complex.
\end{solution}
\end{enumerate}
\end{problem}
\end{document}