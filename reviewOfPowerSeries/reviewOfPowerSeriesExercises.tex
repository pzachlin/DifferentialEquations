\documentclass{ximera}
%% You can put user macros here
%% However, you cannot make new environments

%\listfiles

% Get the 'old' hints/expandables, for use on ximera.osu.edu
%\def\xmNotHintAsExpandable{true}
%\def\xmNotExpandableAsAccordion{true}



%\graphicspath{{./}{firstExample/}{secondExample/}}
\graphicspath{{./}
{aboutDiffEq/}
{applicationsLeadingToDiffEq/}
{applicationsToCurves/}
{autonomousSecondOrder/}
{basicConcepts/}
{bernoulli/}
{constCoeffHomSysI/}
{constCoeffHomSysII/}
{constCoeffHomSysIII/}
{constantCoeffWithImpulses/}
{constantCoefficientHomogeneousEquations/}
{convolution/}
{coolingActivity/}
{directionFields/}
{drainingTank/}
{epidemicActivity/}
{eulersMethod/}
{exactEquations/}
{existUniqueNonlinear/}
{frobeniusI/}
{frobeniusII/}
{frobeniusIII/}
{global.css/}
{growthDecay/}
{heatingCoolingActivity/}
{higherOrderConstCoeff/}
{homogeneousLinearEquations/}
{homogeneousLinearSys/}
{improvedEuler/}
{integratingFactors/}
{interactExperiment/}
{introToLaplace/}
{introToSystems/}
{inverseLaplace/}
{ivpLaplace/}
{laplaceTable/}
{lawOfCooling/}
{linSysOfDiffEqs/}
{linearFirstOrderDiffEq/}
{linearHigherOrder/}
{mixingProblems/}
{motionUnderCentralForce/}
{nonHomogeneousLinear/}
{nonlinearToSeparable/}
{odesInSage/}
{piecewiseContForcingFn/}
{population/}
{reductionOfOrder/}
{regularSingularPts/}
{reviewOfPowerSeries/}
{rlcCircuit/}
{rungeKutta/}
{secondLawOfMotion/}
{separableEquations/}
{seriesSolNearOrdinaryPtI/}
{seriesSolNearOrdinaryPtII/}
{simplePendulum/}
{springActivity/}
{springProblemsI/}
{springProblemsII/}
{undCoeffHigherOrderEqs/}
{undeterminedCoeff/}
{undeterminedCoeff2/}
{unitStepFunction/}
{varParHigherOrder/}
{varParamNonHomLinSys/}
{variationOfParameters/}
}


\usepackage{tikz}
%\usepackage{tkz-euclide}
\usepackage{tikz-3dplot}
\usepackage{tikz-cd}
\usetikzlibrary{shapes.geometric}
\usetikzlibrary{arrows}
\usetikzlibrary{decorations.pathmorphing,patterns}
\usetikzlibrary{backgrounds} % added by Felipe
% \usetkzobj{all}   % NOT ALLOWED IN RECENT TeX's ...
\pgfplotsset{compat=1.13} % prevents compile error.

\pdfOnly{\renewcommand{\theHsection}{\thepart.section.\thesection}}  %% MAKES LINKS WORK should be added to CLS
\pdfOnly{\renewcommand{\part}[1]{\chapterstyle\title{#1}\begin{abstract}\end{abstract}\maketitle\def\thechaptertitle{#1}}}


\renewcommand{\vec}[1]{\mathbf{#1}}
\newcommand{\RR}{\mathbb{R}}
\providecommand{\dfn}{\textit}
\renewcommand{\dfn}{\textit}
\newcommand{\dotp}{\cdot}
\newcommand{\id}{\text{id}}
\newcommand\norm[1]{\left\lVert#1\right\rVert}
\newcommand{\dst}{\displaystyle}
 
\newtheorem{general}{Generalization}
\newtheorem{initprob}{Exploration Problem}

\tikzstyle geometryDiagrams=[ultra thick,color=blue!50!black]

\usepackage{mathtools}

\title{Exercises} \license{CC BY-NC-SA 4.0}

\begin{document}

\begin{abstract}
\end{abstract}
\maketitle

\begin{onlineOnly}
\section*{Exercises}
\end{onlineOnly}


\begin{problem}\label{exer:7.1.1}
\begin{enumerate}

\item Use Theorem~\ref{thmtype:7.1.3} to find the radius of convergence $R$. If $R>0$, find the open interval of convergence.

$\sum_{n=0}^\infty {(-1)^n\over2^nn}(x-1)^n$

\item Use Theorem~\ref{thmtype:7.1.3} to find the radius of convergence $R$. If $R>0$, find the open interval of convergence.

$\sum_{n=0}^\infty 2^nn(x-2)^n$

\item Use Theorem~\ref{thmtype:7.1.3} to find the radius of convergence $R$. If $R>0$, find the open interval of convergence.
$\sum_{n=0}^\infty \frac{n!}{9^n}x^n$

\item Use Theorem~\ref{thmtype:7.1.3} to find the radius of convergence $R$. If $R>0$, find the open interval of convergence.

$\sum_{n=0}^\infty\frac{n(n+1)}{16^n}(x-2)^n$

\item Use Theorem~\ref{thmtype:7.1.3} to find the radius of convergence $R$. If $R>0$, find the open interval of convergence.

$\sum_{n=0}^\infty (-1)^n\frac{7^n}{n!}x^n$

\item Use Theorem~\ref{thmtype:7.1.3} to find the radius of convergence $R$. If $R>0$, find the open interval of convergence.

$\sum_{n=0}^\infty \frac{3^n}{4^{n+1}(n+1)^2}(x+7)^n$
\end{enumerate}
\end{problem}

\begin{problem}\label{exer:7.1.2}
Suppose there is an integer $M$ such that $b_m\ne0$ for $m\ge M$,  and
$$
\lim_{m\to\infty}\left|\frac{b_{m+1}}{b_m}\right|=L,
$$
where $0\le L\le\infty$. Show that the radius of convergence of
$$
\sum_{m=0}^\infty b_m(x-x_0)^{2m}
$$
is $R=1/\sqrt L$,
which is interpreted to mean that $R=0$ if $L=\infty$ or
$R=\infty$ if $L=0$.
\begin{hint}
 Apply Theorem~$\ref{thmtype:7.1.3}$ to the series $\sum_{m=0}^\infty
b_mz^m$ and then let $z=(x-x_0)^2$.   
\end{hint}
\end{problem}

\begin{problem}\label{exer:7.1.3}
\begin{enumerate}

\item Use the result of Exercise~\ref{exer:7.1.2} to find the radius of convergence $R$. If $R>0$, find the open interval of convergence.

$\sum_{m=0}^\infty (-1)^m(3m+1)(x-1)^{2m+1}$

\item Use the result of Exercise~\ref{exer:7.1.2} to find the radius of convergence $R$. If $R>0$, find the open interval of convergence.
$\sum_{m=0}^\infty (-1)^m\frac{m(2m+1)}{2^m}(x+2)^{2m}$ 

\item Use the result of Exercise~\ref{exer:7.1.2} to find the radius of convergence $R$. If $R>0$, find the open interval of convergence.
$\sum_{m=0}^\infty \frac{m!}{(2m)!}(x-1)^{2m}$

\item Use the result of Exercise~\ref{exer:7.1.2} to find the radius of convergence $R$. If $R>0$, find the open interval of convergence.
$\sum_{m=0}^\infty (-1)^m\frac{m!}{9^m}(x+8)^{2m}$

\item Use the result of Exercise~\ref{exer:7.1.2} to find the radius of convergence $R$. If $R>0$, find the open interval of convergence.
$\sum_{m=0}^\infty(-1)^m\frac{(2m-1)}{3^m}x^{2m+1}$

\item Use the result of Exercise~\ref{exer:7.1.2} to find the radius of convergence $R$. If $R>0$, find the open interval of convergence.
$\sum_{m=0}^\infty(x-1)^{2m}$

\end{enumerate}
\end{problem}

\begin{problem}\label{exer:7.1.4}
Suppose there is an integer $M$ such that $b_m\ne0$ for $m\ge M$,
and
$$
\lim_{m\to\infty}\left|\frac{b_{m+1}}{b_m}\right|=L,
$$
where $0\le L\le\infty$. Let $k$ be a positive integer. Show that the
radius of convergence of
$$
\sum_{m=0}^\infty b_m(x-x_0)^{km}
$$
is $R=1/\sqrt[k]L$, which is interpreted to mean that $R=0$ if
$L=\infty$ or $R=\infty$ if $L=0$.

\begin{hint}
Apply Theorem~$\ref{thmtype:7.1.3}$
to the series $\sum_{m=0}^\infty b_mz^m$ and then let $z=(x-x_0)^k$.    
\end{hint}
\end{problem}

\begin{problem}\label{exer:7.1.5}

\begin{enumerate}
\item Use the result of Exercise~\ref{exer:7.1.4} to find the radius of convergence $R$. If $R>0$, find the open interval of convergence.
    
 $\sum_{m=0}^\infty\frac{(-1)^m}{(27)^m}(x-3)^{3m+2}$
 
\item Use the result of Exercise~\ref{exer:7.1.4} to find the radius of convergence $R$. If $R>0$, find the open interval of convergence.
 
 $\sum_{m=0}^\infty\frac{x^{7m+6}}{m}$ 
 
\item Use the result of Exercise~\ref{exer:7.1.4} to find the radius of convergence $R$. If $R>0$, find the open interval of convergence.
 
 $\sum_{m=0}^\infty\frac{9^m(m+1)}{(m+2)}(x-3)^{4m+2}$
 
\item Use the result of Exercise~\ref{exer:7.1.4} to find the radius of convergence $R$. If $R>0$, find the open interval of convergence.
 
 $\sum_{m=0}^\infty(-1)^m\frac{2^m}{m!}x^{4m+3}$
 
\item Use the result of Exercise~\ref{exer:7.1.4} to find the radius of convergence $R$. If $R>0$, find the open interval of convergence.
 
 $\sum_{m=0}^\infty\frac{m!}{(26)^m}(x+1)^{4m+3}$
 
\item Use the result of Exercise~\ref{exer:7.1.4} to find the radius of convergence $R$. If $R>0$, find the open interval of convergence.
 
 $\sum_{m=0}^\infty\frac{(-1)^m}{8^mm(m+1)}(x-1)^{3m+1}$
\end{enumerate}
\end{problem}

\begin{problem}\label{exer:7.1.6}  
Graph $y=\sin x$ and the Taylor polynomial
$$
T_{2M+1}(x)=\sum_{n=0}^M{(-1)^nx^{2n+1}\over(2n+1)!}
$$
on the interval $(-2\pi,2\pi)$ for $M=1$, $2$, $3$, \dots, until you find
a value of $M$ for which there is no perceptible difference between the
two graphs.
\end{problem}


\begin{problem}\label{exer:7.1.7}  
Graph $y=\cos x$ and the Taylor polynomial
$$
T_{2M}(x)=\sum_{n=0}^M{(-1)^nx^{2n}\over(2n)!}
$$
on the interval $(-2\pi,2\pi)$ for $M=1$, $2$, $3$, \dots, until you find a
value of $M$ for which there is no perceptible difference between the
two graphs.
\end{problem}

\begin{problem}\label{exer:7.1.8}  
Graph $y=1/(1-x)$ and the Taylor polynomial
$$
T_N(x)=\sum_{n=0}^Nx^n
$$
on the interval $[0,.95]$ for $N=1$, $2$, $3$, \dots, until you find a
value of $N$ for which there is no perceptible difference between the
two graphs. Choose the scale on the $y$-axis so that $0\le y\le20$.
\end{problem}

\begin{problem}\label{exer:7.1.9}  
Graph $y=\cosh x$ and the Taylor polynomial
$$
T_{2M}(x)=\sum_{n=0}^M{x^{2n}\over(2n)!}
$$
on the interval $(-5,5)$ for $M=1$, $2$, $3$, \dots, until you find a
value of $M$ for which there is no perceptible difference between the
two graphs.
 Choose the scale on the $y$-axis so that $0\le y\le75$.
\end{problem}

\begin{problem}\label{exer:7.1.10}  
Graph $y=\sinh x$ and the Taylor polynomial
$$
T_{2M+1}(x)=\sum_{n=0}^M{x^{2n+1}\over(2n+1)!}
$$
on the interval $(-5,5)$ for $M=0$, $1$, $2$, \dots, until you find a
value of $M$ for which there is no perceptible difference between the
two graphs.  Choose the scale on the $y$-axis so that~$-75~\le~y\le~75$.
 \end{problem}

\begin{problem}\label{exer:7.1.11}
Find a power series solution $y(x)=\sum_{n=0}^\infty a_nx^n$.

$(2+x)y''+xy'+3y$
\end{problem}

\begin{problem}\label{exer:7.1.12}
Find a power series solution $y(x)=\sum_{n=0}^\infty a_nx^n$.

$(1+3x^2)y''+3x^2y'-2y$
\end{problem}

\begin{problem}\label{exer:7.1.13} 
Find a power series solution $y(x)=\sum_{n=0}^\infty a_nx^n$.

$(1+2x^2)y''+(2-3x)y'+4y$
\end{problem}

\begin{problem}\label{exer:7.1.14}
Find a power series solution $y(x)=\sum_{n=0}^\infty a_nx^n$.

$(1+x^2)y''+(2-x)y'+3y$
\end{problem}

\begin{problem}\label{exer:7.1.15}
Find a power series solution $y(x)=\sum_{n=0}^\infty a_nx^n$.

$(1+3x^2)y''-2xy'+4y$
\end{problem}

\begin{problem}\label{exer:7.1.16} Suppose $y(x)=\sum_{n=0}^\infty a_n(x+1)^n$
on an open interval that contains $x_0~=~-1$. Find a power series in
$x+1$ for
$$
xy''+(4+2x)y'+(2+x)y.
$$
\end{problem}

\begin{problem}\label{exer:7.1.17}  Suppose $y(x)=\sum_{n=0}^\infty
a_n(x-2)^n$ on an open interval  that contains $x_0~=~2$.
Find a power series in $x-2$ for
$$
x^2y''+2xy'-3xy.
$$
\end{problem}

\begin{problem}\label{exer:7.1.18}   
Do the following experiment for various choices of
real numbers  $a_0$ and $a_1$.
\begin{enumerate}
\item % (a)
Use software  to solve the initial value problem
$$
(2-x)y''+2y=0,\quad y(0)=a_0,\quad y'(0)=a_1,
$$
numerically on $(-1.95,1.95)$. Choose the most accurate method
your software provides.
(See Section~10.1 for a brief discussion of one such
method.)
\item % (b)
For $N=2$, $3$, $4$, \dots, compute $a_2$, \dots, $a_N$
from Eqn.\eqref{eq:7.1.18} and graph
$$
T_N(x)=\sum_{n=0}^N a_nx^n
$$
and the solution obtained above on the same axes.
Continue increasing $N$ until it is obvious that there is
no point in continuing.
(This sounds vague, but you will know when to stop.)
\end{enumerate}
\end{problem}

\begin{problem}\label{exer:7.1.19}  
Do the following experiment for various choices of
real numbers  $a_0$ and $a_1$.
\begin{enumerate}
\item % (a)
Use software  to solve the initial value problem
$$
(1+x)y''+2(x-1)^2y'+3y=0,\quad y(1)=a_0,\quad y'(1)=a_1,
$$
on the interval $(0,2)$. Choose the most accurate method
your software provides.
(See Section~10.1 for a brief discussion of one such
method.)
\item % (b)
For $N=2$, $3$, $4$, \dots, compute $a_2$, \dots, $a_N$
from Eqn.\eqref{eq:7.1.18} and graph
$$
T_N(x)=\sum_{n=0}^N a_nx^n
$$
and the solution obtained above on the same axes.
Continue increasing $N$ until it is obvious that there is
no point in continuing.
(This sounds vague, but you will know when to stop.)
\end{enumerate}
\item Use Eqns.~\eqref{eq:7.1.24} and \eqref{eq:7.1.25}
to compute $\{a_n\}$.
\end{problem}

\begin{problem}\label{exer:7.1.20}
Suppose the series $\sum_{n=0}^\infty a_nx^n$ converges on an
open interval $(-R,R)$, let $r$ be an arbitrary real number, and
define
$$
y(x)=x^r\sum_{n=0}^\infty a_nx^n=\sum_{n=0}^\infty a_nx^{n+r}
$$
on $(0,R)$. Use Theorem~\ref{thmtype:7.1.4} and the rule for
differentiating the product of two functions to show that
\begin{eqnarray*}
y'(x)&=&\sum_{n=0}^\infty  (n+r)a_nx^{n+r-1},\\
y''(x)&=&\sum_{n=0}^\infty(n+r)(n+r-1)a_nx^{n+r-2},\\
&\vdots&\\
y^{(k)}(x)&=&\sum_{n=0}^\infty(n+r)(n+r-1)\cdots(n+r-k)a_nx^{n+r-k}
\end{eqnarray*}
on $(0,R)$
\end{problem}

\begin{problem}\label{exer:7.1.21}
Let $y$ be as defined in Exercise~\ref{exer:7.1.20}, and write the given
expression in the form $x^r\sum_{n=0}^\infty b_nx^n$.

$x^2(1-x)y''+x(4+x)y'+(2-x)y$
\end{problem}

\begin{problem}\label{exer:7.1.22}
Let $y$ be as defined in Exercise~\ref{exer:7.1.20}, and write the given
expression in the form $x^r\sum_{n=0}^\infty b_nx^n$.

$x^2(1+x)y''+x(1+2x)y'-(4+6x)y$
\end{problem}

\begin{problem}\label{exer:7.1.23}
Let $y$ be as defined in Exercise~\ref{exer:7.1.20}, and write the given
expression in the form $x^r\sum_{n=0}^\infty b_nx^n$.

$x^2(1+x)y''-x(1-6x-x^2)y'+(1+6x+x^2)y$
\end{problem}

\begin{problem}\label{exer:7.1.24}
Let $y$ be as defined in Exercise~\ref{exer:7.1.20}, and write the given
expression in the form $x^r\sum_{n=0}^\infty b_nx^n$.

$x^2(1+3x)y''+x(2+12x+x^2)y'+2x(3+x)y$
\end{problem}

\begin{problem}\label{exer:7.1.25}
Let $y$ be as defined in Exercise~\ref{exer:7.1.20}, and write the given
expression in the form $x^r\sum_{n=0}^\infty b_nx^n$.

$x^2(1+2x^2)y''+x(4+2x^2)y'+2(1-x^2)y$
\end{problem}

\begin{problem}\label{exer:7.1.26}
Let $y$ be as defined in Exercise~\ref{exer:7.1.20}, and write the given expression in the form $x^r\sum_{n=0}^\infty b_nx^n$.

$x^2(2+x^2)y''+2x(5+x^2)y'+2(3-x^2)y$
\end{problem}

\end{document}