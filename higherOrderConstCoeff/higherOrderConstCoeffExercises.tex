\documentclass{ximera}
%% You can put user macros here
%% However, you cannot make new environments

%\listfiles

% Get the 'old' hints/expandables, for use on ximera.osu.edu
%\def\xmNotHintAsExpandable{true}
%\def\xmNotExpandableAsAccordion{true}



%\graphicspath{{./}{firstExample/}{secondExample/}}
\graphicspath{{./}
{aboutDiffEq/}
{applicationsLeadingToDiffEq/}
{applicationsToCurves/}
{autonomousSecondOrder/}
{basicConcepts/}
{bernoulli/}
{constCoeffHomSysI/}
{constCoeffHomSysII/}
{constCoeffHomSysIII/}
{constantCoeffWithImpulses/}
{constantCoefficientHomogeneousEquations/}
{convolution/}
{coolingActivity/}
{directionFields/}
{drainingTank/}
{epidemicActivity/}
{eulersMethod/}
{exactEquations/}
{existUniqueNonlinear/}
{frobeniusI/}
{frobeniusII/}
{frobeniusIII/}
{global.css/}
{growthDecay/}
{heatingCoolingActivity/}
{higherOrderConstCoeff/}
{homogeneousLinearEquations/}
{homogeneousLinearSys/}
{improvedEuler/}
{integratingFactors/}
{interactExperiment/}
{introToLaplace/}
{introToSystems/}
{inverseLaplace/}
{ivpLaplace/}
{laplaceTable/}
{lawOfCooling/}
{linSysOfDiffEqs/}
{linearFirstOrderDiffEq/}
{linearHigherOrder/}
{mixingProblems/}
{motionUnderCentralForce/}
{nonHomogeneousLinear/}
{nonlinearToSeparable/}
{odesInSage/}
{piecewiseContForcingFn/}
{population/}
{reductionOfOrder/}
{regularSingularPts/}
{reviewOfPowerSeries/}
{rlcCircuit/}
{rungeKutta/}
{secondLawOfMotion/}
{separableEquations/}
{seriesSolNearOrdinaryPtI/}
{seriesSolNearOrdinaryPtII/}
{simplePendulum/}
{springActivity/}
{springProblemsI/}
{springProblemsII/}
{undCoeffHigherOrderEqs/}
{undeterminedCoeff/}
{undeterminedCoeff2/}
{unitStepFunction/}
{varParHigherOrder/}
{varParamNonHomLinSys/}
{variationOfParameters/}
}


\usepackage{tikz}
%\usepackage{tkz-euclide}
\usepackage{tikz-3dplot}
\usepackage{tikz-cd}
\usetikzlibrary{shapes.geometric}
\usetikzlibrary{arrows}
\usetikzlibrary{decorations.pathmorphing,patterns}
\usetikzlibrary{backgrounds} % added by Felipe
% \usetkzobj{all}   % NOT ALLOWED IN RECENT TeX's ...
\pgfplotsset{compat=1.13} % prevents compile error.

\pdfOnly{\renewcommand{\theHsection}{\thepart.section.\thesection}}  %% MAKES LINKS WORK should be added to CLS
\pdfOnly{\renewcommand{\part}[1]{\chapterstyle\title{#1}\begin{abstract}\end{abstract}\maketitle\def\thechaptertitle{#1}}}


\renewcommand{\vec}[1]{\mathbf{#1}}
\newcommand{\RR}{\mathbb{R}}
\providecommand{\dfn}{\textit}
\renewcommand{\dfn}{\textit}
\newcommand{\dotp}{\cdot}
\newcommand{\id}{\text{id}}
\newcommand\norm[1]{\left\lVert#1\right\rVert}
\newcommand{\dst}{\displaystyle}
 
\newtheorem{general}{Generalization}
\newtheorem{initprob}{Exploration Problem}

\tikzstyle geometryDiagrams=[ultra thick,color=blue!50!black]

\usepackage{mathtools}

\title{Exercises} \license{CC BY-NC-SA 4.0}

\begin{document}

\begin{abstract}
\end{abstract}
\maketitle

\begin{onlineOnly}
\section*{Exercises}
\end{onlineOnly}


\begin{problem}\label{exer:9.2.1}  Find the general solution.

$$y'''-3y''+3y'-y=0$$
\end{problem}

\begin{problem}\label{exer:9.2.2} Find the general solution.

$$y^{(4)}+8y''-9y=0$$

\end{problem}

\begin{problem}\label{exer:9.2.3} Find the general solution.

$$y'''-y''+16y'-16y=0$$

\end{problem}


\begin{problem}\label{exer:9.2.4} Find the general solution.

$$2y'''+3y''-2y'-3y=0$$
\end{problem}

\begin{problem}\label{exer:9.2.5} Find the general solution.

$$y'''+5y''+9y'+5y=0$$

\end{problem}


\begin{problem}\label{exer:9.2.6}  Find the general solution.

$$4y'''-8y''+5y'-y=0$$

\end{problem}

\begin{problem}\label{exer:9.2.7} Find the general solution.

$$27y'''+27y''+9y'+y=0$$

\end{problem}


\begin{problem}\label{exer:9.2.8} Find the general solution.

$$y^{(4)}+y''=0$$

\end{problem}

 \begin{problem}\label{exer:9.2.9}  Find the general solution.

$$y^{(4)}-16y=0$$

\end{problem}


\begin{problem}\label{exer:9.2.10} Find the general solution.

$$y^{(4)}+12y''+36y=0$$
\end{problem}



\begin{problem}\label{exer:9.2.11} Find the general solution.

$$16y^{(4)}-72y''+81y=0$$

\end{problem}

\begin{problem}\label{exer:9.2.12} Find the general solution.

$$6y^{(4)}+5y'''+7y''+5y'+y=0$$

\end{problem}


\begin{problem}\label{exer:9.2.13}  Find the general solution.

$$4y^{(4)}+12y'''+3y''-13y'-6y=0$$
\end{problem}


\begin{problem}\label{exer:9.2.14}  Find the general solution.

$$y^{(4)}-4y'''+7y''-6y'+2y=0$$

\end{problem}


\begin{problem}\label{exer:9.2.15}
Solve the
initial value problem.

$y'''-2y''+4y'-8y=0, \quad  y(0)=2,\quad y'(0)=-2,\;  y''(0)=0$
\end{problem}

\begin{problem}\label{exer:9.2.16}  Solve the
initial value problem.

$y'''+3y''-y'-3y=0, \quad  y(0)=0,\quad y'(0)=14,\quad y''(0)=-40$
\end{problem}

\begin{problem}\label{exer:9.2.17} Solve the
initial value problem, and graph the solution.
$y'''-y''-y'+y=0, \quad  y(0)=-2,\quad y'(0)=9,\quad y''(0)=4$
\end{problem}

\begin{problem}\label{exer:9.2.18} Solve the
initial value problem, and graph the solution.

$y'''-2y'-4y=0, \quad  y(0)=6,\quad y'(0)=3,\quad y''(0)=22$
\end{problem}

\begin{problem}\label{exer:9.2.19} Solve the
initial value problem, and graph the solution.

$3y'''-y''-7y'+5y=0, \quad  y(0)=\frac{14}{5},\quad y'(0)=0,\quad y''(0)=10$
\end{problem}

\begin{problem}\label{exer:9.2.20}  Solve the
initial value problem.

$y'''-6y''+12y'-8y=0, \quad  y(0)=1,\quad y'(0)=-1,\quad y''(0)=-4$
\end{problem}

\begin{problem}\label{exer:9.2.21}  Solve the
initial value problem.

$2y'''-11y''+12y'+9y=0, \quad  y(0)=6,\quad y'(0)=3,\quad y''(0)=13$
\end{problem}

\begin{problem}\label{exer:9.2.22}  Solve the
initial value problem.

$8y'''-4y''-2y'+y=0, \quad  y(0)=4,\quad y'(0)=-3,\quad y''(0)=-1$
\end{problem}

\begin{problem}\label{exer:9.2.23}  Solve the
initial value problem.

$y^{(4)}-16y=0, \quad   y(0)=2,\;  y'(0)=2,\;  y''(0)=-2,\;  y'''(0)=0$
\end{problem}

\begin{problem}\label{exer:9.2.24}  Solve the
initial value problem.

$y^{(4)}-6y'''+7y''+6y'-8y=0, \quad  y(0)=-2,\quad y'(0)=-8,\quad y''(0)=-14$,

 $y'''(0)=-62$
\end{problem}

\begin{problem}\label{exer:9.2.25}  Solve the
initial value problem.

$4y^{(4)}-13y''+9y=0, \quad  y(0)=1,\quad y'(0)=3,\quad y''(0)=1,\quad y'''(0)=3$
\end{problem}

\begin{problem}\label{exer:9.2.26}  Solve the
initial value problem.

$y^{(4)}+2y'''-2y''-8y'-8y=0, \quad  y(0)=5,\quad y'(0)=-2,\quad y''(0)=6,\quad y'''(0)=8$
\end{problem}

\begin{problem}\label{exer:9.2.27}  Solve the
initial value problem, and graph the solution.
$4y^{(4)}+8y'''+19y''+32y'+12y=0, \quad  y(0)=3,\quad y'(0)=-3,\quad y''(0)=
-\frac{7}{2}$,
 $y'''(0)=\frac{31}{4}$
\end{problem}

\begin{problem}\label{exer:9.2.28}
Find a fundamental set of solutions of the given equation, and verify that it is a fundamental set by evaluating its Wronskian at $x=0$.

\begin{enumerate}
    \item $(D-1)^2(D-2)y=0$
    \item $(D^2+4)(D-3)y=0$
    \item $(D^2+2D+2)(D-1)y=0$
    \item $D^3(D-1)y=0$
    \item $(D^2-1)(D^2+1)y=0$
    \item $(D^2-2D+2)(D^2+1)y=0$
\end{enumerate}
\end{problem}

\begin{problem}\label{exer:9.2.29} Find a
fundamental set of solutions.

$(D^2+6D+13)(D-2)^2D^3y=0$
\end{problem}

\begin{problem}\label{exer:9.2.30} Find a
fundamental set of solutions.

$(D-1)^2(2D-1)^3(D^2+1)y=0$
\end{problem}

\begin{problem}\label{exer:9.2.31} Find a
fundamental set of solutions.

$(D^2+9)^3D^2y=0$
\end{problem}

\begin{problem}\label{exer:9.2.32}  Find a
fundamental set of solutions.

$(D-2)^3(D+1)^2Dy=0$
\end{problem}


\begin{problem}\label{exer:9.2.33}  Find a
fundamental set of solutions.

$(D^2+1)(D^2+9)^2(D-2)y=0$
\end{problem}

\begin{problem}\label{exer:9.2.34} Find a
fundamental set of solutions.

$(D^4-16)^2y=0$
\end{problem}

\begin{problem}\label{exer:9.2.35}  Find a
fundamental set of solutions.

$(4D^2+4D+9)^3y=0$
\end{problem}

\begin{problem}\label{exer:9.2.36} Find a
fundamental set of solutions.

$D^3(D-2)^2(D^2+4)^2y=0$
\end{problem}

\begin{problem}\label{exer:9.2.37}  Find a
fundamental set of solutions.

$(4D^2+1)^2(9D^2+4)^3y=0$
\end{problem}

\begin{problem}\label{exer:9.2.38} Find a
fundamental set of solutions.

$\left[(D-1)^4-16\right]y=0$
\end{problem}


\begin{problem}\label{exer:9.2.39}
It can be shown that
$$
\left|\begin{array}{cccc}
1&1&\cdots&1\\
a_1&a_2&\cdots&a_n\\
a^2_1&a^2_2&\cdots&a^2_n\\
\vdots&\vdots&\ddots&\vdots\\
a^{n-1}_1&a^{n-1}_2&\cdots&a^{n-1}_n\end{array}\right|=
\prod_{1\le i<j\le n}(a_j-a_i),
\text{(A)}
$$
where the left side is  the
\href{http://www-history.mcs.st-and.ac.uk/Mathematicians/Vandermonde.html}{Vandermonde
  determinant} and
the right side is the product of all factors of the form $(a_j-a_i)$
with $i$ and $j$ between $1$ and $n$ and $i<j$.

\begin{enumerate}
\item %(a)
Verify  (A) for $n=2$ and $n=3$.

\item %(b)
Find the Wronskian of $\{e^{{a_1}x}, \quad  e^{{a_2}x},\dots, e^{{a_n}x}\}$.
\end{enumerate}
\end{problem}

\begin{problem}\label{exer:9.2.40}
A theorem from algebra says that if $P_1$ and $P_2$ are polynomials
with no common factors then there are polynomials $Q_1$ and $Q_2$
such that
$$
Q_1P_1+Q_2P_2=1.
$$
This implies that
$$
Q_1(D)P_1(D)y+Q_2(D)P_2(D)y=y
$$
for every function $y$ with enough derivatives for the left side to be
defined.

\begin{enumerate}
\item %(a)
Use this to show that if $P_1$ and $P_2$ have no common factors and
$$
P_1(D)y=P_2(D)y=0
$$
then $y=0$.

\item %(b)
Suppose $P_1$ and $P_2$ are polynomials with no common factors.
Let $u_1$, \dots, $u_r$ be linearly independent solutions of $P_1(D)y=0$
and let $v_1$, \dots, $v_s$ be linearly independent solutions of
$P_2(D)y=0$. Use the previous part to show that $\{u_1,\dots,u_r,\,
v_1,\dots,v_s\}$ is a linearly independent set.

\item %(c)
Suppose the characteristic polynomial of the constant coefficient
equation
$$
a_0y^{(n)}+a_1y^{(n-1)}+\cdots+a_ny=0
\text{(A)}
$$
has the factorization
$$
p(r)=a_0p_1(r)p_2(r)\cdots p_k(r),
$$
where each $p_j$ is of the form
$$
p_j(r)=(r-r_j)^{n_j} \mbox{ or }
p_j(r)=[(r-\lambda_j)^2+w^2_j]^{m_j}\quad  (\omega_j>0)
$$
and no two of the polynomials $p_1$, $p_2$, \dots, $p_k$ have a common
factor. Show that we can find a fundamental set of solutions
$\{y_1,y_2,\dots,y_n\}$ of {\rm(A)} by finding a
fundamental set of solutions of each of the equations
$$
p_j(D)y=0,\quad 1\le j\le k,
$$
 and taking $\{y_1,y_2,\dots,y_n\}$ to be the set of all
functions in these separate fundamental sets.
\end{enumerate}
\end{problem}

\begin{problem}\label{exer:9.2.41}
\begin{enumerate}
\item % (a)
Show that if
$$
z=p(x)\cos\omega x+q(x)\sin\omega x,
\text{(A)}
$$
where $p$ and $q$ are polynomials of degree $\le k$, then
$$
(D^2+\omega^2)z=p_1(x)\cos\omega x+q_1(x)\sin\omega x,
$$
where $p_1$ and $q_1$ are polynomials of degree $\le k-1$.

\item % (b)
Apply the previous part $m$ times to show that if $z$ is of
the form (A) where $p$ and $q$ are polynomial of
degree $\le m-1$, then
$$
(D^2+\omega^2)^mz=0.
\text{(B)}
$$

\item %(c)
Use Eqn.~\eqref{eq:9.2.17} to show that if $y=e^{\lambda x}z$ then
$$
[(D-\lambda)^2+\omega^2]^my=e^{\lambda
x}(D^2+\omega^2)^mz.
$$

\item % (d)
Conclude that if $p$ and $q$ are arbitrary polynomials of degree $\le m-1$ then
$$
y=e^{\lambda x}(p(x)\cos\omega x+q(x)\sin\omega x)
$$
is a solution of
$$
[(D-\lambda)^2+\omega^2]^my=0.
\text{(C)}
$$

\item % (e)
Use this conclusion to show that the functions
$$
\begin{array}{rl}
e^{\lambda x}\cos\omega x, xe^{\lambda x}\cos\omega x,
&\dots, x^{m-1}e^{\lambda x}\cos\omega x,\\
e^{\lambda x}\sin\omega x, xe^{\lambda x}\sin\omega x,&
\dots, x^{m-1}e^{\lambda x}\sin\omega x
\end{array}
\text{(D)}
$$
are all solutions of (C).

\item % (f)
Complete the proof of Theorem~\ref{thmtype:9.2.2} by showing that the
functions in (D) are linearly independent.
\end{enumerate}
\end{problem}

\begin{problem}\label{exer:9.2.42}
\begin{enumerate}
\item % (a)
Use the trigonometric identities
\begin{eqnarray*}
\cos(A+B)&=&\cos A\cos B-\sin A\sin B\\
\sin(A+B)&=&\cos A\sin B+\sin A\cos B
\end{eqnarray*}
to show that
$$
(\cos A+i\sin A)(\cos B+i\sin B)=\cos(A+B)+i\sin(A+B).
$$
\item % (b)
Apply the previous part repeatedly to show that if $n$ is a positive
integer then
$$
\prod_{k=1}^n(\cos A_k+i\sin A_k)=\cos(A_1+A_2+\cdots+A_n)
+i\sin(A_1+A_2+\cdots+A_n).
$$
\item % (c)
Infer that if $n$ is a positive integer then
$$
(\cos\theta+i\sin\theta)^n=\cos n\theta+i\sin n\theta.
\text{(A)}
$$
\item % (d)
Show that (A) also holds if $n=0$ or a negative integer.
\begin{hint}
Verify by direct calculation that
$$
(\cos\theta+i\sin\theta)^{-1}=(\cos\theta-i\sin\theta).
$$
Then replace $\theta$ by $-\theta$  in (A).
\end{hint}
\item\label{exer:9.2.42e} % (e)
Now suppose $n$ is a positive integer. Infer from (A)
that if
$$
z_k=\cos\left(2k\pi\over n\right)+i\sin\left(2k\pi\over n\right)
,\quad k=0,1,\dots,n-1,
$$
and
$$
\zeta_k=\cos\left((2k+1)\pi\over
n\right)+i\sin\left((2k+1)\pi\over n\right)
,\quad k=0,1,\dots,n-1,
$$
then
$$
z_k^n=1\quad\mbox{ and }\quad\zeta_k^n=-1,\quad k=0,1,\dots,n-1.
$$
(Why don't we also consider other integer values for $k$?)

\item % (f)
Let $\rho$ be a positive number. Use the previous part to show that
$$
z^n-\rho=(z-\rho^{1/n} z_0)(z-\rho^{1/n}z_1)\cdots(z-\rho^{1/n} z_{n-1})
$$
and
$$
z^n+\rho=(z-\rho^{1/n} \zeta_0)(z-\rho^{1/n} \zeta_1)\cdots(z-\rho^{1/n}
\zeta_{n-1}).
$$
\end{enumerate}
\end{problem}

\begin{problem}\label{exer:9.2.43}
Use \ref{exer:9.2.42e} to find a fundamental set of
solutions of the given equation.

\begin{enumerate}
\item $y'''-y=0$
\item $y'''+y=0$
\item $y^{(4)}+64y=0$
\item $y^{(6)}-y=0$
\item $y^{(6)}+64y=0$
\item $\left[(D-1)^6-1\right]y=0$
\item $y^{(5)}+y^{(4)}+y'''+y''+y'+y=0$
\end{enumerate}
\end{problem}

\begin{problem}\label{exer:9.2.44}
An equation of the form
$$
 a_0x^ny^{(n)}+a_1x^{n-1}y^{(n-1)}+\cdots
+a_{n-1}xy'+a_ny=0,\quad x>0,
\text{(A)}
$$
where $a_0$, $a_1$, \dots, $a_n$ are constants, is an \emph{Euler} or
\emph{equidimensional} equation.

Show that if
$$
 x=e^t \quad \mbox{ and } \quad Y(t)=y(x(t)),
\text{(B)}
$$
then
\begin{eqnarray*}
x \frac{dy}{dx}&=&\frac{dY}{dt}\\
x^2\frac{d^2y}{dx^2}&=&\frac{d^2Y}{dt^2}-\frac{dY}{dt}\\
x^3\frac{d^3y}{dx^3}&=&\frac{d^3Y}{dt^3}-3\frac{d^2Y}{dt^2}+2\frac{dY}{dt}.
\end{eqnarray*}
In general, it can be shown that if $r$ is any integer $\ge2$ then
$$
x^r \frac{d^ry}{dx^r}=\frac{d^rY}{dt^r}+
 A_{1r}\frac{d^{r-1}Y}{dt^{r-1}}+\cdots+A_{r-1,r}
 \frac{dY}{dt}
$$
where $A_{1r}$, \dots, $A_{r-1,r}$ are integers.  Use these results to
show that the substitution (B) transforms (A) into a
constant coefficient equation for $Y$ as a function of $t$.
\end{problem}

\begin{problem}\label{exer:9.2.45}
Use Exercise~\ref{exer:9.2.44}  to show that a function $y=y(x)$ satisfies
the equation
$$
 a_0x^3y'''+a_1x^2y''+a_2xy'+a_3y=0,
\text{(A)}
$$
on $(0,\infty)$
if and only if the function $Y(t)=y(e^t)$ satisfies
$$
a_0\frac{d^3Y}{dt^3}+(a_1-3a_0) \frac{d^2Y}{dt^2}+(a_2-a_1+2a_0)
\frac{dY}{dt}+a_3Y=0.
$$
Assuming that $a_0$, $a_1$, $a_2$, $a_3$ are real and $a_0 \ne0$, find
the possible forms for the general solution of (A).
\end{problem}

\end{document}