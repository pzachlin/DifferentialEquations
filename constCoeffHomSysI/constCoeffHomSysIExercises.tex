\documentclass{ximera}
%% You can put user macros here
%% However, you cannot make new environments

%\listfiles

% Get the 'old' hints/expandables, for use on ximera.osu.edu
%\def\xmNotHintAsExpandable{true}
%\def\xmNotExpandableAsAccordion{true}



%\graphicspath{{./}{firstExample/}{secondExample/}}
\graphicspath{{./}
{aboutDiffEq/}
{applicationsLeadingToDiffEq/}
{applicationsToCurves/}
{autonomousSecondOrder/}
{basicConcepts/}
{bernoulli/}
{constCoeffHomSysI/}
{constCoeffHomSysII/}
{constCoeffHomSysIII/}
{constantCoeffWithImpulses/}
{constantCoefficientHomogeneousEquations/}
{convolution/}
{coolingActivity/}
{directionFields/}
{drainingTank/}
{epidemicActivity/}
{eulersMethod/}
{exactEquations/}
{existUniqueNonlinear/}
{frobeniusI/}
{frobeniusII/}
{frobeniusIII/}
{global.css/}
{growthDecay/}
{heatingCoolingActivity/}
{higherOrderConstCoeff/}
{homogeneousLinearEquations/}
{homogeneousLinearSys/}
{improvedEuler/}
{integratingFactors/}
{interactExperiment/}
{introToLaplace/}
{introToSystems/}
{inverseLaplace/}
{ivpLaplace/}
{laplaceTable/}
{lawOfCooling/}
{linSysOfDiffEqs/}
{linearFirstOrderDiffEq/}
{linearHigherOrder/}
{mixingProblems/}
{motionUnderCentralForce/}
{nonHomogeneousLinear/}
{nonlinearToSeparable/}
{odesInSage/}
{piecewiseContForcingFn/}
{population/}
{reductionOfOrder/}
{regularSingularPts/}
{reviewOfPowerSeries/}
{rlcCircuit/}
{rungeKutta/}
{secondLawOfMotion/}
{separableEquations/}
{seriesSolNearOrdinaryPtI/}
{seriesSolNearOrdinaryPtII/}
{simplePendulum/}
{springActivity/}
{springProblemsI/}
{springProblemsII/}
{undCoeffHigherOrderEqs/}
{undeterminedCoeff/}
{undeterminedCoeff2/}
{unitStepFunction/}
{varParHigherOrder/}
{varParamNonHomLinSys/}
{variationOfParameters/}
}


\usepackage{tikz}
%\usepackage{tkz-euclide}
\usepackage{tikz-3dplot}
\usepackage{tikz-cd}
\usetikzlibrary{shapes.geometric}
\usetikzlibrary{arrows}
\usetikzlibrary{decorations.pathmorphing,patterns}
\usetikzlibrary{backgrounds} % added by Felipe
% \usetkzobj{all}   % NOT ALLOWED IN RECENT TeX's ...
\pgfplotsset{compat=1.13} % prevents compile error.

\pdfOnly{\renewcommand{\theHsection}{\thepart.section.\thesection}}  %% MAKES LINKS WORK should be added to CLS
\pdfOnly{\renewcommand{\part}[1]{\chapterstyle\title{#1}\begin{abstract}\end{abstract}\maketitle\def\thechaptertitle{#1}}}


\renewcommand{\vec}[1]{\mathbf{#1}}
\newcommand{\RR}{\mathbb{R}}
\providecommand{\dfn}{\textit}
\renewcommand{\dfn}{\textit}
\newcommand{\dotp}{\cdot}
\newcommand{\id}{\text{id}}
\newcommand\norm[1]{\left\lVert#1\right\rVert}
\newcommand{\dst}{\displaystyle}
 
\newtheorem{general}{Generalization}
\newtheorem{initprob}{Exploration Problem}

\tikzstyle geometryDiagrams=[ultra thick,color=blue!50!black]

\usepackage{mathtools}

\title{Exercises} \license{CC BY-NC-SA 4.0}

\begin{document}

\begin{abstract}
\end{abstract}
\maketitle

\begin{onlineOnly}
\section*{Exercises}
\end{onlineOnly}




\begin{problem}\label{exer:10.4.1} 
Find the general solution.

$ {\bf y}'=
\begin{bmatrix} 1&2\\2&1\end{bmatrix}{\bf y}$
\end{problem}

\begin{problem}\label{exer:10.4.2} 
Find the general solution.

$ {\bf y}'=
\frac{1}{4}\begin{bmatrix}-5&3 \\3&-5\end{bmatrix}{\bf
y}$

\begin{solution}
    $\frac{1}{4}\begin{vmatrix}-5-4\lambda&3\\
3&-5-4\lambda\end{vmatrix}
=(\lambda+1/2)(\lambda+2)$.
Eigenvectors  associated with $\lambda_1=-1/2$  satisfy
$\begin{bmatrix}-3&3\\-4&4
\end{bmatrix}\begin{bmatrix}
x_1\\x_2\end{bmatrix}=\begin{bmatrix}0\\0\end{bmatrix}$,
so $x_1=x_2$.  Taking $x_2=1$ yields
${\bf y}_1=\begin{bmatrix}1 \\
1\end{bmatrix}e^{-t/2}$.
Eigenvectors  associated with $\lambda_2=-2$
satisfy
$\begin{bmatrix}3/4&3/4\\1&1
\end{bmatrix}\begin{bmatrix}
x_1\\x_2\end{bmatrix}=\begin{bmatrix} 0\\0\end{bmatrix}$,
so $x_1=-x_2$.  Taking $x_2=1$ yields
${\bf y}_2=\begin{bmatrix} -1\\
1\end{bmatrix}e^{-2t}$. Hence
 ${\bf y}= c_1\begin{bmatrix} 1\\1\end{bmatrix}e^{-t/2}+c_2\begin{bmatrix}-1\\1\end{bmatrix}e^{-2t}$.
\end{solution}
\end{problem}

\begin{problem}\label{exer:10.4.3}
Find the general solution.

$ {\bf y}'=
\frac{1}{5}\begin{bmatrix}-4&3\\
-2&-11\end{bmatrix}{\bf y}$
\end{problem}

\begin{problem}\label{exer:10.4.4}
Find the general solution.

$ {\bf y}'=
\begin{bmatrix}-1&-4\\-1&-1\end{bmatrix}{\bf y}$

\begin{solution}
    $\begin{vmatrix}-1-\lambda&-4\\-1&-1-\lambda\end{vmatrix}
=(\lambda-1)(\lambda+3)$.
Eigenvectors  associated with $\lambda_1=-3$ satisfy
$\begin{bmatrix}2&-4\\1&-2
\end{bmatrix}\begin{bmatrix}
x_1\\x_2\end{bmatrix}=\begin{bmatrix}0\\0\end{bmatrix}$,
so $x_1=2x_2$.  Taking $x_2=1$ yields
${\bf y}_1=\begin{bmatrix}
2\\1\end{bmatrix}e^{-3t}$.
Eigenvectors  associated with $\lambda_2=1$  satisfy
$\begin{bmatrix}-2&-4\\-1&-2
\end{bmatrix}\begin{bmatrix}
x_1\\x_2\end{bmatrix}=\begin{bmatrix}0\\0\end{bmatrix}$,
so $x_1=-2x_2$.  Taking $x_2=1$ yields
${\bf y}_2=\begin{bmatrix} -2\\1\end{bmatrix}e^t$.
Hence  ${\bf y}= c_1\begin{bmatrix} 2\\1\end{bmatrix}e^{-3t}+c_2\begin{bmatrix}{r}-2\\1\end{bmatrix}e^t$.
\end{solution}
\end{problem}


\begin{problem}\label{exer:10.4.5} 
Find the general solution.

$ {\bf y}'=
\begin{bmatrix} 2&-4\\-1&-1\end{bmatrix}{\bf y}$
\end{problem}

\begin{problem}\label{exer:10.4.6} 
Find the general solution.

$ {\bf y}'=
\begin{bmatrix} 4&-3\\2&-1\end{bmatrix}{\bf y}$

\begin{solution}
    $ \begin{vmatrix}4-\lambda&-3\\2&-1-\lambda\end{vmatrix}
=(\lambda-2)(\lambda-1)$.
Eigenvectors  associated with $\lambda_1=2$  satisfy
$  \begin{bmatrix}2&-3\\2&-3
 \end{bmatrix} \begin{bmatrix}
x_1\\x_2 \end{bmatrix}= \begin{bmatrix} 0\\0 \end{bmatrix}$,
so $x_1=\frac{3 }{2}x_2$.  Taking $x_2=2$ yields
${\bf y}_1=  \begin{bmatrix}3\\2 \end{bmatrix}e^{2t}$.
Eigenvectors  associated with $\lambda_2=1$ satisfy
$  \begin{bmatrix}3&-3\\2&-2
 \end{bmatrix} \begin{bmatrix}
x_1\\x_2 \end{bmatrix}= \begin{bmatrix} 0\\0 \end{bmatrix}$,
so $x_1=x_2$.  Taking $x_2=1$ yields
${\bf y}_2=  \begin{bmatrix}1\\1 \end{bmatrix}e^t$.
Hence
 ${\bf y}= c_1 \begin{bmatrix}3\\2 \end{bmatrix}e^{2t}+c_2 \begin{bmatrix}1\\1 \end{bmatrix}e^t$.
\end{solution}

\end{problem}


\begin{problem}\label{exer:10.4.7}
Find the general solution.

$ {\bf y}'=
\begin{bmatrix}-6&-3\\1&-2\end{bmatrix}{\bf y}$
\end{problem}

\begin{problem}\label{exer:10.4.8}
Find the general solution.

$ {\bf y}'=
\begin{bmatrix}
1&-1&-2\\1&-2&-3\\-4&1&-1\end{bmatrix}
{\bf y}$

\begin{solution}
    $  \begin{vmatrix}1-\lambda&-1&-2\\1&-2-\lambda&-3 \\
-4&1&-1-\lambda\end{vmatrix}=-(\lambda+3)(\lambda+1)(\lambda-2)$.
The eigenvectors associated with
 with $\lambda_1=-3$ satisfy the system with  augmented matrix
$  \begin{bmatrix}4&-1&-2&\vdots&0\\1&1
&-3&\vdots&0\\-4&1&2&\vdots&0
 \end{bmatrix}$
which is row equivalent to
$  \begin{bmatrix}1&0&-1&\vdots&0\\0&1&-2&
\vdots&0\\0&0&0&\vdots&0 \end{bmatrix}$.
Hence $x_1=x_3$ and $x_2=2x_3$.  Taking $x_3=1$ yields
${\bf y}_1=  \begin{bmatrix} 1\\2\\1
 \end{bmatrix}e^{-3t}$.
The eigenvectors associated with
 with $\lambda_2=-1$ satisfy the system with  augmented matrix
$  \begin{bmatrix}2&-1&-2&\vdots&0\\1&-1
&-3&\vdots&0\\-4&1&0&\vdots&0
 \end{bmatrix}$,
which is row equivalent to
$  \begin{bmatrix}1&0&1&\vdots&0\\0&1&4&
\vdots&0\\0&0&0&\vdots&0 \end{bmatrix}$.
Hence $x_1=-x_3$ and $x_2=-4x_3$.  Taking $x_3=1$ yields
${\bf y}_2=  \begin{bmatrix}-1\\-4\\ 1
 \end{bmatrix}e^{-t}$.
The eigenvectors associated with
 with $\lambda_3=2$ satisfy the system with  augmented matrix
$  \begin{bmatrix}-1&-1&-2&\vdots&0\\1&
-4&-3&\vdots&0\\-4&1&-3&\vdots&0
 \end{bmatrix}$,
which is row equivalent to
$  \begin{bmatrix}1&0&1&\vdots&0\\0&1&1&
\vdots&0\\0&0&0&\vdots&0 \end{bmatrix}$.
Hence $x_1=-x_3$ and $x_2=-x_3$.  Taking $x_3=1$ yields
${\bf y}_3=  \begin{bmatrix}-1\\-1\\1
 \end{bmatrix}e^{2t}$. Hence
 ${\bf y}= c_1 \begin{bmatrix} 1\\2\\1
 \end{bmatrix}e^{-3t}+c_2 \begin{bmatrix}{r}-1\\-4\\1 \end{bmatrix}e^{-t}+c_3 \begin{bmatrix}-1\\-1\\1 \end{bmatrix}e^{2t}$.
\end{solution}
\end{problem}


\begin{problem}\label{exer:10.4.9}
Find the general solution.

$ {\bf y}'=
\begin{bmatrix}
-6&-4&-8\\-4&0&-4\\-8&-4&-6\end{bmatrix}{\bf y}$
\end{problem}

\begin{problem}\label{exer:10.4.10}
Find the general solution.

$ {\bf y}'=
\begin{bmatrix}3&5&8\\1&-1&
-2\\-1&-1&-1\end{bmatrix}{\bf y}$

\begin{solution}
    $ \begin{vmatrix}3-\lambda&5&8\\1&-1-\lambda&-2\\
-1&-1&-1-\lambda\end{vmatrix}=-(\lambda-1)(\lambda+2)(\lambda-2)$.
The eigenvectors associated with
 with $\lambda_1=1$ satisfy the system with  augmented matrix
$ \begin{bmatrix}2&5&8&\vdots&0\\1&-2
&-2&\vdots&0\\-1&-1&-2&\vdots&0
\end{bmatrix}$,
which is row equivalent to
$ \begin{bmatrix}1&0&2/3&\vdots&0\\0&1&4/3&
\vdots&0\\0&0&0&\vdots&0\end{bmatrix}$.
Hence $x_1=-\frac{2 }{3}x_3$ and $x_2=-\frac{4 }{3}x_3$.  Taking $x_3=3$
yields
${\bf y}_1= \begin{bmatrix}-2\\-4\\3
\end{bmatrix}e^t$.
The eigenvectors associated with
 with $\lambda_2=-2$ satisfy the system with  augmented matrix
$ \begin{bmatrix}5&5&8&\vdots&0\\1&1
&-2&\vdots&0\\-1&-1&1&\vdots&0
\end{bmatrix}$,
which is row equivalent to
$ \begin{bmatrix}1&1&0&\vdots&0\\0&0&1&
\vdots&0\\0&0&0&\vdots&0\end{bmatrix}$.
Hence $x_1=-x_2$ and $x_3=0$.  Taking $x_2=1$ yields
${\bf y}_2= \begin{bmatrix} -1\\1\\0
\end{bmatrix}e^{-2t}$.
The eigenvectors associated with
 with $\lambda_3=2$ satisfy the system with  augmented matrix
$ \begin{bmatrix}1&5&8&\vdots&0\\1&-3
&-2&\vdots&0\\-1&-1&-3&\vdots&0
\end{bmatrix}$,
which is row equivalent to
$ \begin{bmatrix}1&0&7/4&\vdots&0\\0&1&5/4&
\vdots&0\\0&0&0&\vdots&0\end{bmatrix}$.
Hence $x_1=-\frac{7 }{4}x_3$ and $x_2=-\frac{5 }{4}x_3$.  Taking $x_3=4$
yields
${\bf y}_3= \begin{bmatrix}-7\\-5\\4
\end{bmatrix}e^{2t}$. Hence
${\bf y}=  c_1\begin{bmatrix}-2\\-4\\
3\end{bmatrix}e^t+c_2\begin{bmatrix}-1\\1\\0
\end{bmatrix}e^{-2t}+c_3\begin{bmatrix}-7\\-5\\4
\end{bmatrix}e^{2t}$.
\end{solution}
\end{problem}


\begin{problem}\label{exer:10.4.11}
Find the general solution.

$ {\bf y}'=
\begin{bmatrix}
1&-1&2\\12&-4
& 10\\-6&1&-7
\end{bmatrix}{\bf y}$
\end{problem}

\begin{problem}\label{exer:10.4.12}
Find the general solution.

$ {\bf y}'=
\begin{bmatrix}
4&-1&-4\\4&-3&-2\\1&-1&-1\end{bmatrix}{\bf y}$

\begin{solution}
    $ \begin{vmatrix}4-\lambda&-1&-4\\4&-3-\lambda&-2\\
1&-1&-1-\lambda\end{vmatrix}=-(\lambda-3)(\lambda+2)(\lambda+1)$.
The eigenvectors associated
 with $\lambda_1=3$ satisfy the system with  augmented matrix
$  \begin{bmatrix}1&-1&-4&\vdots&0\\4&-6
&-2&\vdots&0\\1&-1&-4&\vdots&0
 \end{bmatrix}$,
which is row equivalent to
$  \begin{bmatrix}1&0&-11&\vdots&0\\0&1&-7&
\vdots&0\\0&0&0&\vdots&0 \end{bmatrix}$.
Hence  $x_1=11x_3$ and $x_2=7x_3$.  Taking $x_3=1$ yields
${\bf y}_1=  \begin{bmatrix}11\\7\\1
 \end{bmatrix}e^{3t}$.
The eigenvectors associated
 with $\lambda_2=-2$ satisfy the system with  augmented matrix
$  \begin{bmatrix}6&-1&-4&\vdots&0\\4&-1
&-2&\vdots&0\\1&-1&1&\vdots&0
 \end{bmatrix}$,
which is row equivalent to
$  \begin{bmatrix}1&0&-1&\vdots&0\\0&1&-2&
\vdots&0\\0&0&0&\vdots&0 \end{bmatrix}$.
Hence $x_1=x_3$ and $x_2=2x_3$.  Taking $x_3=1$ yields
${\bf y}_2=  \begin{bmatrix}1\\2\\1
 \end{bmatrix}e^{-2t}$.
The eigenvectors associated
 with $\lambda_3=-1$ satisfy the system with  augmented matrix
$  \begin{bmatrix}5&-1&-4&\vdots&0\\4&-2
&-2&\vdots&0\\1&-1&0&\vdots&0
 \end{bmatrix}$,
which is row equivalent to
$  \begin{bmatrix}1&0&-1&\vdots&0\\0&1&-1&
\vdots&0\\0&0&0&\vdots&0 \end{bmatrix}$.
Hence  $x_1=x_3$ and $x_2=x_3$.  Taking $x_3=1$ yields
${\bf y}_3=  \begin{bmatrix}1\\1\\1
 \end{bmatrix}e^{-t}$. Hence
 ${\bf y}= c_1 \begin{bmatrix}11\\\phantom{1}7
\\\phantom{1}1 \end{bmatrix}e^{3t}+c_2 \begin{bmatrix}1\\2
\\1 \end{bmatrix}e^{-2t}+c_3 \begin{bmatrix}1\\1\\1
 \end{bmatrix}e^{-t}$.
\end{solution}
\end{problem}


\begin{problem}\label{exer:10.4.13} 
Find the general solution.

$ {\bf y}'=
\begin{bmatrix}-2&2&-6\\2&6&2\\-2&-2&
2\end{bmatrix}{\bf y}$
\end{problem}

\begin{problem}\label{exer:10.4.14} 
Find the general solution.

$ {\bf y}'=
\begin{bmatrix}3&2&-2\\-2&7&-2\\
-10&10&-5\end{bmatrix}{\bf y}$ 

\begin{solution}
    $ \begin{vmatrix}3-\lambda&2&-2\\-2&7-\lambda&-2\\-10
&10&-5-\lambda\end{vmatrix}=-(\lambda+5)(\lambda-5)^2$.
The eigenvectors associated
 with $\lambda_1=-5$ satisfy the system with  augmented matrix
$  \begin{bmatrix}8&2&-2&\vdots&0\\-2&12
&-2&\vdots&0\\-10&10&0&\vdots&0
 \end{bmatrix}$,
which is row equivalent to
$  \begin{bmatrix}1&0&-1/5&\vdots&0\\0&1&-1/5&
\vdots&0\\0&0&0&\vdots&0 \end{bmatrix}$.
Hence  $x_1=\frac{1 }{5}x_3$ and $x_2=\frac{1 }{5}x_3$.  Taking
$x_3=5$ yields
${\bf y}_1=  \begin{bmatrix}1\\1\\5
 \end{bmatrix}e^{-5t}$.
The eigenvectors associated
 with $\lambda_2=5$ satisfy the system with  augmented matrix
$  \begin{bmatrix}-2&2&-2&\vdots&0\\-2&2
&-2&\vdots&0\\-10&10&-10&\vdots&0
 \end{bmatrix}$,
which is row equivalent to
$  \begin{bmatrix}1&-1&1&\vdots&0\\0&0&0&
\vdots&0\\0&0&0&\vdots&0 \end{bmatrix}$.
Hence $x_1=x_2-x_3$.  Taking $x_2=0$ and $x_3=1$ yields
${\bf y}_2=  \begin{bmatrix}-1\\0\\1
 \end{bmatrix}e^{5t}$.
 Taking $x_2=1$ and $x_3=0$ yields
${\bf y}_3=  \begin{bmatrix}1\\1\\0
 \end{bmatrix}e^{5t}$. Hence
 ${\bf y}= c_1 \begin{bmatrix}1\\1\\5
 \end{bmatrix}e^{-5t}+c_2 \begin{bmatrix}-1\\0\\
 1 \end{bmatrix}e^{5t}+c_3 \begin{bmatrix}1\\1\\0 \end{bmatrix}e^{5t}$.
\end{solution}
\end{problem}


\begin{problem}\label{exer:10.4.15} 
Find the general solution.

$ {\bf y}'=
\begin{bmatrix}3&1&-1\\3&5&1\\-6&2&4
\end{bmatrix}{\bf y}$
\end{problem}


\begin{problem}\label{exer:10.4.16}
Solve the initial value problem.

$ {\bf y}'=\begin{bmatrix}-7&4\\-6&7\end{bmatrix}{\bf y},\quad{\bf
y}(0)=\begin{bmatrix}2\\-4\end{bmatrix}$

\begin{solution}
    $ \left|\begin{array}{cc}-7-\lambda&4\\-6&7-\lambda\end{array}\right|
=(\lambda-5)(\lambda+5)$.
Eigenvectors  associated with $\lambda_1=5$  satisfy
$ \begin{bmatrix}-12&4\\-6&2
\end{bmatrix}\begin{bmatrix}
x_1\\x_2\end{bmatrix}=\begin{bmatrix} 0\\0\end{bmatrix}$,
so $x_1=\frac{x_2 }{3}$.  Taking $x_2=3$ yields
${\bf y}_1= \begin{bmatrix} 1\\3\end{bmatrix}e^{5t}$.
Eigenvectors  associated with $\lambda_2=5$
satisfy
$ \begin{bmatrix}-2&4\\-6&12
\end{bmatrix}\begin{bmatrix}
x_1\\x_2\end{bmatrix}=\begin{bmatrix}{c} 0\\0\end{bmatrix}$,
so $x_1=2x_2$.  Taking $x_2=1$ yields
${\bf y}_2= \begin{bmatrix}2\\1\end{bmatrix}e^{-5t}$.
The general solution is
${\bf y}=c_1 \begin{bmatrix}
1\\3\end{bmatrix}e^{5t}+
c_2 \begin{bmatrix}2\\1\end{bmatrix}e^{-5t}$.
Now ${\bf y}(0)= \begin{bmatrix}2\\-4\end{bmatrix}\Rightarrow
c_1 \begin{bmatrix}
1\\3\end{bmatrix}+
c_2 \begin{bmatrix}2\\1\end{bmatrix}= \begin{bmatrix}2\\-4\end{bmatrix}$,
so $c_1=-2$ and $c_2=2$. Therefore,
${\bf y}= -\begin{bmatrix}2\\6\end{bmatrix}e^{5t}+\begin{bmatrix}4\\2\end{bmatrix}e^{-5t}$.
\end{solution}
\end{problem}

\begin{problem}\label{exer:10.4.17}
Solve the initial value problem.

$ {\bf y}'=\frac{1}{6}\begin{bmatrix}7 &2\\-2& 2\end{bmatrix}{\bf y},\quad{\bf
y}(0)= \begin{bmatrix}0\\-3\end{bmatrix}$
\end{problem}

\begin{problem}\label{exer:10.4.18}
Solve the initial value problem.

$ {\bf y}'=\begin{bmatrix}21&-12\\24&-15\end{bmatrix}{\bf y},\quad{\bf
y}(0)= \begin{bmatrix}5\\3\end{bmatrix}$

\begin{solution}
    $ \begin{vmatrix}21-\lambda&-12\\24&-15-\lambda\end{vmatrix}
=(\lambda-9)(\lambda+3)$.
Eigenvectors  associated with $\lambda_1=9$  satisfy
$ \begin{bmatrix}12&-12\\24&-24
\end{bmatrix}\begin{bmatrix}
x_1\\x_2\end{bmatrix}=\begin{bmatrix} 0\\0\end{bmatrix}$,
so $x_1=x_2$.  Taking $x_2=1$ yields
${\bf y}_1= \begin{bmatrix}1\\1\end{bmatrix}e^{9t}$.
Eigenvectors  associated with $\lambda_2=-3$
$ \begin{bmatrix}24&-12\\24&-12
\end{bmatrix}\begin{bmatrix}
x_1\\x_2\end{bmatrix}=\begin{bmatrix} 0\\0\end{bmatrix}$,
so $x_1=\frac{1 }{2}x_2$.  Taking $x_2=2$ yields
${\bf y}_2= \begin{bmatrix}1\\2\end{bmatrix}e^{-3t}$.
The general solution is
${\bf y}=c_1 \begin{bmatrix}1\\1\end{bmatrix}e^{9t}+c_2 \begin{bmatrix}1\\2\end{bmatrix}e^{-3t}$.
Now  ${\bf y}(0)= \begin{bmatrix}5\\3\end{bmatrix}\Rightarrow
c_1 \begin{bmatrix}1\\1\end{bmatrix}+c_2 \begin{bmatrix}1\\2\end{bmatrix}= \begin{bmatrix}5\\3\end{bmatrix}$, so $c_1=7$
and $c_2=-2$. Therefore,
${\bf y}= \begin{bmatrix}7\\7\end{bmatrix}e^{9t}-\begin{bmatrix}2\\4\end{bmatrix}e^{-3t}$.
\end{solution}
\end{problem}

\begin{problem}\label{exer:10.4.19}
Solve the initial value problem.

$ {\bf y}'=\begin{bmatrix}-7&4\\-6&7\end{bmatrix}{\bf y},\quad{\bf
y}(0)= \begin{bmatrix}-1\\7\end{bmatrix}$
\end{problem}

\begin{problem}\label{exer:10.4.20}
Solve the initial value problem.

$ {\bf y}'=\frac{1}{6} \begin{bmatrix}1&2&0\\4&-1&0\\0&0&3\end{bmatrix}{\bf y},\quad{\bf
y}(0)= \begin{bmatrix}4\\7\\1\end{bmatrix}$

\begin{solution}
    $ \left|\begin{array}{ccc}1/6-\lambda&1/3&0\\
2/3&-1/6-\lambda&0\\
0&0&1/2-\lambda\end{array}\right|=-(\lambda+1/2)(\lambda-1/2)^2$.
The eigenvectors associated with
 with $\lambda_1=-1/2$ satisfy the system with  augmented matrix
$  \begin{bmatrix}2/3&1/3&0&\vdots&0
\\2/3&1/3&0&\vdots&0\\0&0&1&\vdots&0
 \end{bmatrix}$,
which is row equivalent to
$  \begin{bmatrix}1&1/2&0&\vdots&0\\0&0&1&
\vdots&0\\0&0&0&\vdots&0 \end{bmatrix}$.
Hence $x_1=-\frac{x_2 }{2}$ and $x_3=0$.  Taking $x_2=2$ yields
${\bf y}_1=  \begin{bmatrix}-1\\2\\0
 \end{bmatrix}e^{-t/2}$.
The eigenvectors associated with
 with $\lambda_2=\lambda_3=1/2$ satisfy the system with  augmented
matrix
$  \begin{bmatrix}-1/3&1/3&0&\vdots&0\\
2/3&-2/3&0&\vdots&0\\0&0&0&\vdots&0
 \end{bmatrix}$,
which is row equivalent to
$  \begin{bmatrix}1&-1&0&\vdots&0\\0&0&0&
\vdots&0\\0&0&0&\vdots&0 \end{bmatrix}$.
Hence $x_1=x_2$ and $x_3$ is arbitrary.  Taking $x_2=1$ and $x_3=0$
yields
${\bf y}_2=  \begin{bmatrix}1\\1\\0
 \end{bmatrix}e^{t/2}$.
Taking $x_2=0$ and $x_3=1$ yields
${\bf y}_3=  \begin{bmatrix}0\\0\\1
 \end{bmatrix}e^{t/2}$.
 The general solution is
${\bf y}= c_1 \begin{bmatrix}-1\\2\\0\end{bmatrix}e^{-t/2}+
c_2 \begin{bmatrix}1\\1\\0\end{bmatrix}e^{t/2}+c_3 \begin{bmatrix}0\\0\\1\end{bmatrix}e^{t/2}$.
Now ${\bf y}(0)=  \begin{bmatrix}4\\7\\1\end{bmatrix}\Rightarrow
 c_1 \begin{bmatrix}-1\\2\\0\end{bmatrix}+
c_2 \begin{bmatrix}1\\1\\0\end{bmatrix}+c_3 \begin{bmatrix}0\\0\\1\end{bmatrix}e^{t/2}=  \begin{bmatrix}4\\7\\1\end{bmatrix}$,
so $c_1=1$, $c_2=5$, and $c_3=1$. Hence
${\bf y}=  \begin{bmatrix}-1\\2\\0\end{bmatrix}e^{-t/2}+
 \begin{bmatrix}5\\5\\0\end{bmatrix}e^{t/2}+ \begin{bmatrix}0\\0\\1\end{bmatrix}e^{t/2}$.
\end{solution}
\end{problem}

\begin{problem}\label{exer:10.4.21}
Solve the initial value problem.

$ {\bf y}'=\frac{1}{3} \begin{bmatrix}2&-2&3\\-4&4&3\\2&1&0\end{bmatrix}{\bf
y},\quad{\bf y}(0)= \begin{bmatrix}1\\1\\5\end{bmatrix}$
\end{problem}

\begin{problem}\label{exer:10.4.22}
Solve the initial value problem.

$ {\bf y}'= \begin{bmatrix}6&-3&-8\\2&1&-2\\3&-3&-5\end{bmatrix}{\bf
y},\quad{\bf y}(0)= \begin{bmatrix}0\\-1\\-1\end{bmatrix}$

\begin{solution}
    $ \begin{vmatrix}6-\lambda&-3&-8\\2&1-\lambda&-2\\
3&-3&-5-\lambda\end{vmatrix}=-(\lambda-1)(\lambda+2)(\lambda-3)$.
The eigenvectors associated
 with $\lambda_1=1$ satisfy the system with  augmented matrix
$  \begin{bmatrix}5&-3&-8&\vdots&0\\2&0
&-2&\vdots&0\\3&-3&-6&\vdots&0
 \end{bmatrix}$,
which is row equivalent to
$  \begin{bmatrix}1&0&-1&\vdots&0\\0&1&1&
\vdots&0\\0&0&0&\vdots&0 \end{bmatrix}$.
Hence  $x_1=x_3$ and $x_2=-x_3$.  Taking $x_3=$ yields
${\bf y}_1=  \begin{bmatrix}1\\-1\\1
 \end{bmatrix}e^t$.
The eigenvectors associated
 with $\lambda_2=-2$ satisfy the system with  augmented matrix
$  \begin{bmatrix}8&-3&-8&\vdots&0\\2&3
&-2&\vdots&0\\3&-3&-3&\vdots&0
 \end{bmatrix}$,
which is row equivalent to
$  \begin{bmatrix}1&0&-1&\vdots&0\\0&1&0&
\vdots&0\\0&0&0&\vdots&0 \end{bmatrix}$.
Hence $x_1=x_3$ and $x_2=0$.  Taking $x_3=1$ yields
${\bf y}_2=  \begin{bmatrix}1\\0\\1
 \end{bmatrix}e^{-2t}$.
The eigenvectors associated
 with $\lambda_3=3$ satisfy the system with  augmented matrix
$  \begin{bmatrix}3&-3&-8&\vdots&0\\2&-2
&-2&\vdots&0\\3&-3&-8&\vdots&0
 \end{bmatrix}$,
which is row equivalent to
$  \begin{bmatrix}1&-1&0&\vdots&0\\0&0&1&
\vdots&0\\0&0&0&\vdots&0 \end{bmatrix}$.
Hence  $x_1=x_2$ and $x_3=0$.  Taking $x_2=1$ yields
${\bf y}_3=  \begin{bmatrix}1\\1\\0
 \end{bmatrix}e^{3t}$.
The general solution is
${\bf y}=c_1  \begin{bmatrix}1\\-1\\1
 \end{bmatrix}e^t+
c_2  \begin{bmatrix}1\\0\\1
 \end{bmatrix}e^{-2t}
+c_3  \begin{bmatrix}1\\1\\0
 \end{bmatrix}e^{3t}$.
Now ${\bf y}(0)=  \begin{bmatrix}0\\-1\\-1\end{bmatrix}\Rightarrow
c_1  \begin{bmatrix}1\\-1\\1
 \end{bmatrix}+
c_2  \begin{bmatrix}1\\0\\1
 \end{bmatrix}
+c_3  \begin{bmatrix}1\\1\\0
 \end{bmatrix}=  \begin{bmatrix}0\\-1\\-1\end{bmatrix}$, so
$c_1=2$, $c_2=-3$, and $c_3=1$. Therefore,
${\bf
y}=  \begin{bmatrix}2\\-2\\2\end{bmatrix}e^t- \begin{bmatrix}3\\0\\3\end{bmatrix}e^{-2t}+ \begin{bmatrix}1\\1\\0\end{bmatrix}e^{3t}$.
\end{solution}
\end{problem}

\begin{problem}\label{exer:10.4.23}
Solve the initial value problem.

$ {\bf y}'=\frac{1}{3} \begin{bmatrix}2&4&-7\\1&5&-5\\-4&4&-1\end{bmatrix}{\bf
y},\quad{\bf y}(0)= \begin{bmatrix}4\\1\\3\end{bmatrix}$
\end{problem}

\begin{problem}\label{exer:10.4.24} 
Solve the initial value problem.

$ {\bf y}'= \begin{bmatrix}3&0&1\\11&-2&7\\1&0&3\end{bmatrix}{\bf y},\quad
{\bf y}(0)= \begin{bmatrix}2\\7\\6\end{bmatrix}$

\begin{solution}
    $ \begin{vmatrix}3-\lambda&0&1\\11&-2-\lambda&7\\
1&0&3-\lambda\end{vmatrix}=-(\lambda-2)(\lambda+2)(\lambda-4)$.
The eigenvectors associated with
 with $\lambda_1=2$ satisfy the system with  augmented matrix
$  \begin{bmatrix}1&0&1&\vdots&0\\11&-4
&7&\vdots&0\\1&0&1&\vdots&0
 \end{bmatrix}$,
which is row equivalent to
$  \begin{bmatrix}1&0&1&\vdots&0\\0&1&1&
\vdots&0\\0&0&0&\vdots&0 \end{bmatrix}$.
Hence $x_1=-x_3$ and $x_2=-x_3$.  Taking $x_3=1$ yields
${\bf y}_1=  \begin{bmatrix}-1\\-1\\1
 \end{bmatrix}e^{2t}$.
The eigenvectors associated with
 with $\lambda_2=-2$ satisfy the system with  augmented matrix
$  \begin{bmatrix}5&0&1&\vdots&0\\11&0
&7&\vdots&0\\1&0&5&\vdots&0
 \end{bmatrix}$,
which is row equivalent to
$  \begin{bmatrix}1&0&0&\vdots&0\\0&0&1&
\vdots&0\\0&0&0&\vdots&0 \end{bmatrix}$.
Hence $x_1=x_3=0$ and $x_2$ is arbitrary.  Taking $x_3=1$ yields
${\bf y}_2=  \begin{bmatrix}0\\1\\0
 \end{bmatrix}e^{-2t}$.
The eigenvectors associated with
 with $\lambda_3=4$ satisfy the system with  augmented matrix
$  \begin{bmatrix}-1&0&1&\vdots&0\\11&-6
&7&\vdots&0\\1&0&-1&\vdots&0
 \end{bmatrix}$,
which is row equivalent to
$  \begin{bmatrix}1&0&-1&\vdots&0\\0&1&-3&
\vdots&0\\0&0&0&\vdots&0 \end{bmatrix}$.
Hence $x_1=x_3$ and $x_2=3x_3$.  Taking $x_3=1$ yields
${\bf y}_3=  \begin{bmatrix}1\\3\\1
 \end{bmatrix}e^{4t}$.
 The general solution is
${\bf y}=c_1  \begin{bmatrix}-1\\-1\\1
 \end{bmatrix}e^{2t}
+c_2  \begin{bmatrix}0\\1\\0
 \end{bmatrix}e^{-2t}
+c_3  \begin{bmatrix}1\\3\\1
 \end{bmatrix}e^{4t}$.
Now ${\bf y}(0)=  \begin{bmatrix}2\\7\\6\end{bmatrix}\Rightarrow
c_1  \begin{bmatrix}-1\\-1\\1
 \end{bmatrix}
+c_2  \begin{bmatrix}0\\1\\0
 \end{bmatrix}
+c_3  \begin{bmatrix}1\\3\\1
 \end{bmatrix}=  \begin{bmatrix}2\\7\\6\end{bmatrix}$, so $c_1=2$, $c_2=-3$, and
$c_3=4$. Hence
${\bf y}=  \begin{bmatrix}-2\\-2\\2\end{bmatrix}e^{2t}- \begin{bmatrix}0\\3\\0\end{bmatrix}e^{-2t}+
 \begin{bmatrix}4\\12\\4\end{bmatrix}e^{4t}$

\end{solution}
\end{problem}

\begin{problem}\label{exer:10.4.25} 
Solve the initial value problem.

$ {\bf y}'= \begin{bmatrix}-2&-5&-1\\-4&-1&1\\4&5&3\end{bmatrix}{\bf y},\quad
{\bf y}(0)= \begin{bmatrix}8\\-10\\-4\end{bmatrix}$
\end{problem}

\begin{problem}\label{exer:10.4.26}  
Solve the initial value problem.

$ {\bf y}'= \begin{bmatrix}3&-1&0\\4&-2&0\\4&-4&2\end{bmatrix}{\bf y},\quad
{\bf y}(0)= \begin{bmatrix}7\\10\\2\end{bmatrix}$

\begin{solution}
    $\left|\begin{array}{cccc}3-\lambda&-1&0\\4&-2-\lambda&0
\\4&-4&2-\lambda\end{array}\right|
=-(\lambda+1)(\lambda-2)^2$.
The eigenvectors associated
 with $\lambda_1=-1$ satisfy the system with  augmented matrix
$ \begin{bmatrix}4&-1&0&\vdots&0\\4&-1&0&\vdots&0\\4&-4&3&\vdots&0 \end{bmatrix}$,
which is row equivalent to
$ \begin{bmatrix}1&0&-1/4&\vdots&0\\0&1&-1&\vdots&0\\0&0&0&\vdots&0 \end{bmatrix}$
Hence $x_1=x_2/4$ and $x_2=x_3$.  Taking $x_3=4$ yields
${\bf y}_1=  \begin{bmatrix}1\\4\\4
 \end{bmatrix}e^{-t}$.
The eigenvectors associated with
 with $\lambda_2=\lambda_3=2$ satisfy the system with  augmented
matrix
$ \begin{bmatrix}1&-1&0&\vdots&0\\4&-4&0&\vdots&0\\4&-4&0&\vdots&0 \end{bmatrix}$,
which is row equivalent to
$ \begin{bmatrix}1&-1&0&\vdots&0\\0&0&0&\vdots&0\\0&0&0&\vdots&0 \end{bmatrix}$.
Hence $x_1=x_2$ and $x_3$ is arbitrary.  Taking $x_2=1$ and $x_3=0$
yields
${\bf y}_2=  \begin{bmatrix}1\\1\\0
 \end{bmatrix}e^{2t}$.
Taking $x_2=0$ and $x_3=1$ yields
${\bf y}_3=  \begin{bmatrix}0\\0\\1
 \end{bmatrix}e^{2t}$.
 The general solution is
${\bf y}=c_1  \begin{bmatrix}1\\4\\4
 \end{bmatrix}e^{-t}+
c_2  \begin{bmatrix}1\\1\\0
 \end{bmatrix}e^{2t}+
c_3  \begin{bmatrix}0\\0\\1
 \end{bmatrix}e^{2t}$.
Now ${\bf y}(0)=  \begin{bmatrix}7\\10\\2\end{bmatrix}\Rightarrow
c_1  \begin{bmatrix}1\\4\\4
 \end{bmatrix}+
c_2  \begin{bmatrix}1\\1\\0
 \end{bmatrix}+
c_3  \begin{bmatrix}0\\0\\1
 \end{bmatrix}=  \begin{bmatrix}7\\10\\2\end{bmatrix}$,
 so $c_1=1$, $c_2=6$, and
$c_3=-2$. Hence
 ${\bf y}=
 \begin{bmatrix}1\\4\\4\end{bmatrix}e^{-t}+ \begin{bmatrix}6\\6\\-2\end{bmatrix}e^{2t}$.

\end{solution}
\end{problem}

\begin{problem}\label{exer:10.4.27}
Solve the initial value problem.

$ {\bf y}'=
\begin{bmatrix}-2&2&6\\2&6&2\\-2&-2&
2\end{bmatrix}{\bf y},\quad{\bf y}(0)= \begin{bmatrix}6\\-10\\7\end{bmatrix}$
\end{problem}

\begin{problem}\label{exer:10.4.28}
Let $A$ be an $n\times n$ constant matrix. Then
Theorem~\ref{thmtype:10.2.1} implies that the solutions
of
$$
{\bf y}'=A{\bf y}\quad\quad
\text{(A)}
$$
are all defined on $(-\infty,\infty)$.
\begin{enumerate}
\item % (a)
Use Theorem~\ref{thmtype:10.2.1} to show that the only
solution
of (A) that can ever equal the zero vector is ${\bf y}\equiv{\bf0}$.

\begin{solution}
    If ${\bf y}(t_0)=0$, then ${\bf y}$ is the solution
of the initial value problem ${\bf y}'=A{\bf y},\ {\bf y}(t_0)={\bf
0}$. Since ${\bf y}\equiv0$ is a solution of this problem,
Theorem~\ref{thmtype:10.2.1} implies the conclusion.
\end{solution}

\item % (b)
Suppose ${\bf y}_1$ is a solution of (A) and ${\bf y}_2$ is
defined by ${\bf y}_2(t)={\bf y}_1(t-\tau)$, where $\tau$ is an
arbitrary real number. Show that ${\bf y}_2$ is also a solution of
(A).

\begin{solution}
    It is given that ${\bf y}_1'(t)=A{\bf y}_1(t)$ for all $t$.
Replacing $t$ by $t-\tau$ shows that ${\bf y}_1'(t-\tau)=A{\bf
y}_1(t-\tau)=A{\bf y}_2(t)$ for all $t$. Since ${\bf y}_2'(t)={\bf
y}_1'(t-\tau)$ by the chain rule, this implies that ${\bf
y}_2'(t)=A{\bf y}_2(t)$ for all $t$.
\end{solution}

\item % (c)
Suppose ${\bf y}_1$ and ${\bf y}_2$ are solutions of (A) and
there are real numbers $t_1$ and $t_2$ such that ${\bf y}_1(t_1)={\bf
y}_2(t_2)$. Show that ${\bf y}_2(t)={\bf y}_1(t-\tau)$ for all $t$,
where $\tau=t_2-t_1$. 
\begin{hint}Show that ${\bf y}_1(t-\tau)$ and ${\bf
y}_2(t)$ are solutions of the same initial value problem for (A),
and apply the uniqueness assertion of
Theorem~\ref{thmtype:10.2.1}.
\end{hint}

\begin{solution}
    If ${\bf z}(t)={\bf y}_1(t-\tau)$, then ${\bf z}(t_2)={\bf
y}_1(t_1)={\bf y}_2(t_2)$; therefore ${\bf z}$ and ${\bf y}_2$ are
both solutions of the initial value problem ${\bf y}'=A{\bf y},\ {\bf
y}(t_2)={\bf k}$, where ${\bf k}={\bf y}_2(t_2)$.
\end{solution}
\end{enumerate}
\end{problem}


\begin{problem}\label{exer:10.4.29}  
Describe and graph trajectories of the given system.

${\bf y}'=\begin{bmatrix}1&1\\1&-1\end{bmatrix}{\bf y}$
\end{problem}

\begin{problem}\label{exer:10.4.30}
Describe and graph trajectories of the given system.

${\bf y}'=\begin{bmatrix}-4&3\\-2&-11\end{bmatrix}{\bf
y}$
\end{problem}


\begin{problem}\label{exer:10.4.31} 
Describe and graph trajectories of the given system.

${\bf y}'=\begin{bmatrix}9&-3\\-1&11\end{bmatrix}{\bf
y}$
\end{problem}

\begin{problem}\label{exer:10.4.32} 
Describe and graph trajectories of the given system.

${\bf
y}'=\begin{bmatrix}-1&-10\\-5&4\end{bmatrix}{\bf y}$
\end{problem}


\begin{problem}\label{exer:10.4.33}
Describe and graph trajectories of the given system.

${\bf y}'= \begin{bmatrix}5&-4\\1&10\end{bmatrix}{\bf
y}$
\end{problem}

\begin{problem}\label{exer:10.4.34} 
Describe and graph trajectories of the given system.

${\bf
y}'= \begin{bmatrix}-7&1\\3 &-5\end{bmatrix}{\bf y}$
\end{problem}

\begin{problem}\label{exer:10.4.35}
Suppose the eigenvalues of the $2\times 2$
matrix $A$ are $\lambda=0$ and $\mu\ne0$, with
corresponding  eigenvectors ${\bf x}_1$ and ${\bf x}_2$.
Let $L_1$ be the line through the origin parallel to ${\bf x}_1$.
\begin{enumerate}
\item % (a)
Show that every point on $L_1$ is the trajectory of a constant
solution of ${\bf y}'=A{\bf y}$.
\item % (b)
Show that the trajectories of nonconstant solutions of ${\bf y}'=A{\bf
y}$ are half-lines parallel to ${\bf x}_2$ and on either side of
$L_1$, and that the direction of motion along these trajectories is
away from $L_1$ if $\mu>0$, or toward $L_1$ if $\mu<0$.
\end{enumerate}
\end{problem}


\begin{problem}\label{exer:10.4.36}
The matrix of the systems in
in this exercise is singular. Describe
and graph the trajectories of nonconstant solutions  of the given
systems.

${\bf y}'= \begin{bmatrix}-1&1\\1&-1\end{bmatrix}{\bf y}$
\end{problem}

\begin{problem}\label{exer:10.4.37}
The matrix of the systems in
in this exercise is singular. Describe
and graph the trajectories of nonconstant solutions  of the given
systems.

${\bf y}'= \begin{bmatrix}-1 &-3\\2 &6\end{bmatrix}{\bf
y}$
\end{problem}


\begin{problem}\label{exer:10.4.38}
The matrix of the systems in
in this exercise is singular. Describe
and graph the trajectories of nonconstant solutions  of the given
systems.

${\bf y}'= \begin{bmatrix}1 &-3\\-1 &3\end{bmatrix}{\bf
y}$
\end{problem}

\begin{problem}\label{exer:10.4.39}
The matrix of the systems in
in this exercise is singular. Describe
and graph the trajectories of nonconstant solutions  of the given
systems.

${\bf y}'= \begin{bmatrix}1 &-2\\-1 &2\end{bmatrix}{\bf y}$
\end{problem}


\begin{problem}\label{exer:10.4.40} 
The matrix of the systems in
in this exercise is singular. Describe
and graph the trajectories of nonconstant solutions  of the given
systems.

${\bf y}'= \begin{bmatrix}-4 &-4\\1&1\end{bmatrix}{\bf
y}$
\end{problem}

\begin{problem}\label{exer:10.4.41}
The matrix of the systems in
in this exercise is singular. Describe
and graph the trajectories of nonconstant solutions  of the given
systems.

${\bf
y}'= \begin{bmatrix}3& -1\\-3 &1\end{bmatrix}{\bf y}$
\end{problem}


\begin{problem}\label{exer:10.4.42} 
Let $P=P(t)$ and $Q=Q(t)$ be the populations of two species at time
$t$, and assume that each population would grow exponentially if the
other didn't exist; that is, in the absence of competition,
$$
P'=aP \mbox{\quad and \quad}Q'=bQ
\quad\quad \text{(A)}
$$
where $a$ and $b$ are positive constants. One way to model the effect
of competition is to assume that the growth rate per individual of
each population is reduced by an amount proportional to the other
population, so (A) is replaced by
\begin{eqnarray*}
P'&=&\phantom{-}aP-\alpha Q\\
Q'&=&-\beta P+bQ,
\end{eqnarray*}
where $\alpha$ and $\beta$ are positive constants. (Since negative
population doesn't make sense, this system holds only while $P$ and
$Q$ are both positive.) Now suppose $P(0)=P_0>0$ and
$Q(0)=Q_0>0$.

\begin{enumerate}
\item % (a)
For several choices of $a$, $b$, $\alpha$, and $\beta$, verify
experimentally
(by graphing trajectories of (A) in the $P$-$Q$ plane) that there's a
constant $\rho>0$ (depending upon $a$, $b$, $\alpha$, and $\beta$) with the
following properties:
\begin{enumerate}
\item % (i)
If $Q_0>\rho P_0$, then $P$ decreases monotonically to zero
in finite time, during which $Q$ remains positive.
\item % (ii)
If $Q_0<\rho P_0$, then $Q$ decreases monotonically to zero in
finite time, during which $P$ remains positive.
\end{enumerate}
\item % (b)
Conclude from  part (a) that exactly one of the species
becomes extinct in finite time if $Q_0\ne\rho P_0$. Determine
experimentally what happens if $Q_0=\rho P_0$.
\item % (c)
Confirm your experimental results and determine $\gamma$ by expressing
the eigenvalues and associated eigenvectors of
$$
A=\begin{bmatrix} a &-\alpha\\-\beta&b\end{bmatrix}
$$
in terms of $a$, $b$, $\alpha$, and $\beta$, and applying the geometric
arguments developed at the end of this section.
\end{enumerate}

\begin{solution}
    The characteristic polynomial of $A$ is
$p(\lambda)=\lambda^2-(a+b)+ab-\alpha\beta$, so the eigenvalues of $A$
are $\lambda_1=\frac{a+b-\gamma}{2}$ and
$\lambda_1=\frac{a+b+\gamma}{2}$, where
$\gamma=\sqrt{(a-b)^2+4\alpha\beta}$; ${\bf
x}_1=\begin{bmatrix}b-a+\gamma\\2\beta\end{bmatrix}$ and ${\bf
x}_2=\begin{bmatrix}b-a-\gamma\\2\beta\end{bmatrix}$ are associated eigenvectors. Since
$\gamma>|b-a|$, if $L_1$ and $L_2$ are lines through the origin
parallel to ${\bf x}_1$ and ${\bf x}_2$, then $L_1$ is in the first and
third quadrants and $L_2$ is in the second and fourth quadrants. The
slope of $L_1$ is $\rho=\frac{2\beta}{ b-a+\gamma}>0$. If $Q_0=\rho
P_0$ there are three possibilities: (i) if $\alpha\beta=ab$, then
$\lambda_1=0$ and $P(t)=P_0$, $Q(t)=Q_0$ for all $t>0$; (ii) if
$\alpha\beta<ab$, then $\lambda_1>0$ and
$\lim_{t\to\infty}P(t)=\lim_{t\to\infty}Q(t)=\infty$ (monotonically);
(iii) if $\alpha\beta>ab$, then $\lambda_1<0$ and
$\lim_{t\to\infty}P(t)=\lim_{t\to\infty}Q(t)=0$ (monotonically). Now
suppose $Q_0\ne\rho P_0$, so that the trajectory cannot intersect
$L_1$, and assume for the moment that (A) makes sense for all $t>0$;
that is, even if one or the other of $P$ and $Q$ is negative. Since
$\lambda_2>0$ it follows that either $\lim_{t\to\infty}P(t)=\infty$ or
$\lim_{t\to\infty}Q(t)=\infty$ (or both), and the trajectory is
asymptotically parallel to $L_2$. Therefore,the trajectory must cross
into the third quadrant (so $P(T)=0$ and $Q(T)>0$ for some finite $T$)
if $Q_0>\rho P_0$, or into the fourth quadrant (so $Q(T)=0$ and
$P(T)>0$ for some finite $T$) if $Q_0<\rho P_0$.

\end{solution}
\end{problem}

\end{document}