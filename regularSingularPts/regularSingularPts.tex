\documentclass{ximera}

%% You can put user macros here
%% However, you cannot make new environments

%\listfiles

% Get the 'old' hints/expandables, for use on ximera.osu.edu
%\def\xmNotHintAsExpandable{true}
%\def\xmNotExpandableAsAccordion{true}



%\graphicspath{{./}{firstExample/}{secondExample/}}
\graphicspath{{./}
{aboutDiffEq/}
{applicationsLeadingToDiffEq/}
{applicationsToCurves/}
{autonomousSecondOrder/}
{basicConcepts/}
{bernoulli/}
{constCoeffHomSysI/}
{constCoeffHomSysII/}
{constCoeffHomSysIII/}
{constantCoeffWithImpulses/}
{constantCoefficientHomogeneousEquations/}
{convolution/}
{coolingActivity/}
{directionFields/}
{drainingTank/}
{epidemicActivity/}
{eulersMethod/}
{exactEquations/}
{existUniqueNonlinear/}
{frobeniusI/}
{frobeniusII/}
{frobeniusIII/}
{global.css/}
{growthDecay/}
{heatingCoolingActivity/}
{higherOrderConstCoeff/}
{homogeneousLinearEquations/}
{homogeneousLinearSys/}
{improvedEuler/}
{integratingFactors/}
{interactExperiment/}
{introToLaplace/}
{introToSystems/}
{inverseLaplace/}
{ivpLaplace/}
{laplaceTable/}
{lawOfCooling/}
{linSysOfDiffEqs/}
{linearFirstOrderDiffEq/}
{linearHigherOrder/}
{mixingProblems/}
{motionUnderCentralForce/}
{nonHomogeneousLinear/}
{nonlinearToSeparable/}
{odesInSage/}
{piecewiseContForcingFn/}
{population/}
{reductionOfOrder/}
{regularSingularPts/}
{reviewOfPowerSeries/}
{rlcCircuit/}
{rungeKutta/}
{secondLawOfMotion/}
{separableEquations/}
{seriesSolNearOrdinaryPtI/}
{seriesSolNearOrdinaryPtII/}
{simplePendulum/}
{springActivity/}
{springProblemsI/}
{springProblemsII/}
{undCoeffHigherOrderEqs/}
{undeterminedCoeff/}
{undeterminedCoeff2/}
{unitStepFunction/}
{varParHigherOrder/}
{varParamNonHomLinSys/}
{variationOfParameters/}
}


\usepackage{tikz}
%\usepackage{tkz-euclide}
\usepackage{tikz-3dplot}
\usepackage{tikz-cd}
\usetikzlibrary{shapes.geometric}
\usetikzlibrary{arrows}
\usetikzlibrary{decorations.pathmorphing,patterns}
\usetikzlibrary{backgrounds} % added by Felipe
% \usetkzobj{all}   % NOT ALLOWED IN RECENT TeX's ...
\pgfplotsset{compat=1.13} % prevents compile error.

\pdfOnly{\renewcommand{\theHsection}{\thepart.section.\thesection}}  %% MAKES LINKS WORK should be added to CLS
\pdfOnly{\renewcommand{\part}[1]{\chapterstyle\title{#1}\begin{abstract}\end{abstract}\maketitle\def\thechaptertitle{#1}}}


\renewcommand{\vec}[1]{\mathbf{#1}}
\newcommand{\RR}{\mathbb{R}}
\providecommand{\dfn}{\textit}
\renewcommand{\dfn}{\textit}
\newcommand{\dotp}{\cdot}
\newcommand{\id}{\text{id}}
\newcommand\norm[1]{\left\lVert#1\right\rVert}
\newcommand{\dst}{\displaystyle}
 
\newtheorem{general}{Generalization}
\newtheorem{initprob}{Exploration Problem}

\tikzstyle geometryDiagrams=[ultra thick,color=blue!50!black]

\usepackage{mathtools}

\title{Regular Singular Points:   Euler Equations}%\label{Module 7-ADEF}


\begin{document}

\begin{abstract}
We consider the utilization of power series to determine solutions to differential equations near a singular point.  We also study Euler equations.
\end{abstract}

\maketitle

\section*{Regular Singular Points:   Euler Equations}

\begin{remark}This section sets the stage for Sections  \href{https://xerxes.ximera.org/differentialequations/main/frobeniusI/frobeniusI}{7.5},
\href{https://xerxes.ximera.org/differentialequations/main/frobeniusII/frobeniusII}{7.6},
and \href{https://xerxes.ximera.org/differentialequations/main/frobeniusIII/frobeniusIII}{7.7}. If you are not interested in those sections, but wish
to learn about Euler equations, omit the introductory paragraphs
and start reading at Definition~\ref{thmtype:7.4.2}.
\end{remark}

In the next three sections we will continue to study  equations of the form
\begin{equation}\label{eq:7.4.1}
P_0(x)y''+P_1(x)y'+P_2(x)y=0
\end{equation}
where $P_0$, $P_1$, and $P_2$ are polynomials, but the emphasis will
be different from that of \href{https://ximera.osu.edu/ode/main/seriesSolNearOrdinaryPtI/seriesSolNearOrdinaryPtI}{Section 7.2} and \href{https://ximera.osu.edu/ode/main/seriesSolNearOrdinaryPtII/seriesSolNearOrdinaryPtII}{7.3},
where we obtained solutions of $\eqref{eq:7.4.1}$ near an ordinary point $x_0$ in the
form of power series in $x-x_0$. If $x_0$ is a singular point of
\eqref{eq:7.4.1} (that is, if $P(x_0)=0$),  the solutions cannot in general be represented by power series in $x-x_0$. Nevertheless, it is
often necessary in physical applications to study the behavior of
solutions of $\eqref{eq:7.4.1}$ near a singular point. Although this can be difficult in the absence of some sort of assumption on the nature of
the singular point, equations that satisfy the requirements of the
next definition can be solved by series methods discussed in the
next three sections. Fortunately, many equations arising in applications satisfy these requirements.

\begin{definition}\label{thmtype:7.4.1}
Let $P_0$, $P_1$, and $P_2$ be polynomials with no common factor and
suppose  $P_0(x_0)=0$. Then $x_0$ is a \textit{regular singular
point} of the equation
\begin{equation} \label{eq:7.4.2}
P_0(x)y''+P_1(x)y'+P_2(x)y=0
\end{equation}
if $\eqref{eq:7.4.2}$ can be written as
\begin{equation} \label{eq:7.4.3}
(x-x_0)^2A(x)y''+(x-x_0)B(x)y'+C(x)y=0
\end{equation}
where $A$, $B$, and $C$ are polynomials and $A(x_0)\neq0$;   otherwise,
$x_0$ is an \textit{irregular} singular point of $\eqref{eq:7.4.2}$.
\end{definition}

\begin{example}\label{example:7.4.1}
Bessel's equation,
\begin{equation} \label{eq:7.4.4}
x^2y''+xy'+(x^2-\nu^2)y=0,
\end{equation}
has the singular point $x_0=0$. Since this equation is of the form
$\eqref{eq:7.4.3}$ with $x_0=0$, $A(x)=1$, $B(x)=1$, and $C(x)=x^2-\nu^2$,
it follows that $x_0=0$ is a regular singular point of
$\eqref{eq:7.4.4}$.
 \end{example}

\begin{example}\label{example:7.4.2}
Legendre's equation,
\begin{equation} \label{eq:7.4.5}
(1-x^2)y''-2xy'+\alpha(\alpha+1)y=0,
\end{equation}
has the singular points $x_0=\pm1$. Mutiplying through by $1-x$
yields
$$
(x-1)^2(x+1)y''+2x(x-1)y'-\alpha(\alpha+1)(x-1)y=0,
$$
which is of the form $\eqref{eq:7.4.3}$ with $x_0=1$, $A(x)=x+1$,
$B(x)=2x$, and $C(x)=-\alpha(\alpha+1)(x-1)$. Therefore $x_0=1$
is a regular singular point of $\eqref{eq:7.4.5}$. We leave it to you to show that $x_0=-1$ is also a regular singular point of
$\eqref{eq:7.4.5}$.
\end{example}

\begin{example}\label{example:7.4.3}
The equation
$$
x^3y''+xy'+y=0
$$
has an irregular  singular point at $x_0=0$. (Verify.)
\end{example}

For convenience we restrict our attention to the case where $x_0=0$ is a regular singular point of $\eqref{eq:7.4.2}$. This is not really a restriction, since if $x_0\neq0$ is a regular singular point of
$\eqref{eq:7.4.2}$, then  introducing the new independent variable $t=x-x_0$
and the new unknown $Y(t)=y(t+x_0)$ leads to a differential equation
with polynomial coefficients that has a regular singular point at $t_0=0$. 
This is illustrated in Exercise~\ref{exer:7.4.22} for Legendre's equation, and in Exercise~\ref{exer:7.4.23} for the general case.

\subsection*{Euler Equations}

The simplest kind of equation with a regular singular point at $x_0=0$
is the Euler equation, defined as follows.

\begin{definition}\label{thmtype:7.4.2}

\href{https://en.wikipedia.org/wiki/Cauchy%E2%80%93Euler_equation}{Euler's equidimensional equation} is an equation that can be written in the form
\begin{equation} \label{eq:7.4.6}
ax^2y''+bxy'+cy=0,
\end{equation}
where $a,b$, and $c$ are real constants and $a\neq0$.
\end{definition}

Theorem~\ref{thmtype:5.1.1} implies that $\eqref{eq:7.4.6}$ has solutions defined on $(0,\infty)$ and $(-\infty,0)$,
since $\eqref{eq:7.4.6}$ can be rewritten as
$$
ay''+\frac{b}{x}y'+\frac{c}{x^2}y=0.
$$
For convenience we will restrict our attention to the interval
$(0,\infty)$. 
(Exercise~\ref{exer:7.4.19} deals with solutions of $\eqref{eq:7.4.6}$ on $(-\infty,0)$.) 
The key to finding solutions on
$(0,\infty)$ is  that if $x>0$ then $x^r$ is defined as a
real-valued function on $(0,\infty)$ for all values of $r$, and
substituting $y=x^r$ into $\eqref{eq:7.4.6}$ produces
\begin{equation} \label{eq:7.4.7}
\begin{array}{lcl}
ax^2(x^r)''+bx(x^r)'+cx^r&=&ax^2r(r-1)x^{r-2}+bxrx^{r-1}+cx^r\\
&=&[ar(r-1)+br+c]x^r.
\end{array}
\end{equation}
The polynomial
$$
p(r)=ar(r-1)+br+c
$$
is called the \textit{indicial polynomial} of $\eqref{eq:7.4.6}$, and
$p(r)=0$
is its \textit{indicial equation}. From $\eqref{eq:7.4.7}$ we can
see that
$y=x^r$ is a solution of $\eqref{eq:7.4.6}$ on $(0,\infty)$ if and only if
$p(r)=0$. Therefore, if the indicial equation has distinct real roots
$r_1$ and $r_2$ then $y_1=x^{r_1}$ and $y_2=x^{r_2}$ form a
fundamental set of solutions of $\eqref{eq:7.4.6}$ on
$(0,\infty)$, since $y_2/y_1=x^{r_2-r_1}$ is nonconstant.
In this case
$$
y=c_1x^{r_1}+c_2x^{r_2}
$$
is the general solution of $\eqref{eq:7.4.6}$ on $(0,\infty)$.

\begin{example}\label{example:7.4.4}
 Find the general solution of
\begin{equation} \label{eq:7.4.8}
x^2y''-xy'-8y=0
\end{equation}
on  $(0,\infty)$.
\begin{explanation}
The indicial polynomial of $\eqref{eq:7.4.8}$ is
$$
p(r)=r(r-1)-r-8=(r-4)(r+2).
$$
Therefore $y_1=x^4$ and $y_2=x^{-2}$ are solutions of $\eqref{eq:7.4.8}$ on
$(0,\infty)$, and its general solution on $(0,\infty)$ is
$$
y=c_1x^4+\frac{c_2}{x^2}.
$$
\end{explanation}
\end{example}

\begin{example}\label{example:7.4.5}
  Find the general solution of
\begin{equation} \label{eq:7.4.9}
6x^2y''+5xy'-y=0
\end{equation}
on  $(0,\infty)$.
\begin{explanation}
The indicial polynomial of $\eqref{eq:7.4.9}$ is
$$
p(r)=6r(r-1)+5r-1=(2r-1)(3r+1).
$$
Therefore the general solution of $\eqref{eq:7.4.9}$ on
$(0,\infty)$ is
$$
y=c_1x^{1/2}+c_2x^{-1/3}.
$$
\end{explanation}
\end{example}



If the indicial equation  has a repeated root $r_1$, then
$y_1=x^{r_1}$ is a solution of
\begin{equation} \label{eq:7.4.10}
ax^2y''+bxy'+cy=0,
\end{equation}
on $(0,\infty)$, but $\eqref{eq:7.4.10}$ has no other solution of the form
$y=x^r$. If the indicial equation has complex conjugate zeros then
$\eqref{eq:7.4.10}$ has no real--valued solutions of the form $y=x^r$.
Fortunately, we can use the results of 
\href{https://xerxes.ximera.org/differentialequations/main/constantCoefficientHomogeneousEquations/constantCoefficientHomogeneousEquations}{Section 5.2} 
for constant coefficient equations to solve $\eqref{eq:7.4.10}$ in any case.

\begin{theorem}\label{thmtype:7.4.3}
Suppose the roots of the indicial equation
\begin{equation}
\label{eq:7.4.11} ar(r-1)+br+c=0
\end{equation}
are $r_1$ and $r_2$. Then the general solution of the Euler equation
\begin{equation} \label{eq:7.4.12}
ax^2y''+bxy'+cy=0
\end{equation}
 on  $(0,\infty)$ is
\begin{eqnarray*}
y&=&c_1x^{r_1}+c_2x^{r_2}\mbox{ if $r_1$ and $r_2$ are distinct
real numbers  };
\\ y&=&x^{r_1}(c_1+c_2\ln x)\mbox{ if
$r_1=r_2$  };
\\ y&=&x^{\lambda}\left[c_1\cos\left(\omega\ln x\right)+
c_2\sin\left(\omega\ln x \right)\right]\mbox{ if
$r_1,r_2=\lambda\pm i\omega$ with $\omega>0$}.
\end{eqnarray*}
\end{theorem}

\begin{proof}
We first show that $y=y(x)$ satisfies $\eqref{eq:7.4.12}$ on
$(0,\infty)$ if and only if $Y(t)=y(e^t)$ satisfies the constant
coefficient equation
\begin{equation} \label{eq:7.4.13}
a\frac{d^2Y}{dt^2}+(b-a)\frac{dY}{dt}+cY=0
\end{equation} on
$(-\infty,\infty)$. To do this, it's convenient to write $x=e^t$, or,
equivalently, $t=\ln x$;   thus, $Y(t)=y(x)$, where $x=e^t$. From the
chain rule,
$$
\frac{dY}{dt}=\frac{dy}{dx}\frac{dx}{dt}
$$
and, since
$$
\frac{dx}{dt}=e^t=x,
$$
it follows that
\begin{equation} \label{eq:7.4.14}
\frac{dY}{dt}=x\frac{dy}{dx}.
\end{equation}
Differentiating this with
respect to $t$ and using the chain rule again yields
\begin{eqnarray*}
\frac{d^2Y}{dt^2}&=&\frac{d}{dt}\left(\frac{dY}{dt}\right)=\frac{d}{dt}\left(x\frac{dy}{dx}\right)\\
&=&\frac{dx}{dt}\frac{dy}{dx}+x\frac{d^2y}{dx^2}\frac{dx}{dt}\\
&=&x\frac{dy}{dx}+x^2\frac{d^2y}{dx^2}\quad\left(\mbox{ since } \frac{dx}{dt}=x\right).
\end{eqnarray*}
From this and $\eqref{eq:7.4.14}$,
$$
x^2\frac{d^2y}{dx^2}=\frac{d^2Y}{dt^2}-\frac{dY}{dt}.
$$
Substituting this and $\eqref{eq:7.4.14}$ into $\eqref{eq:7.4.12}$ yields
$\eqref{eq:7.4.13}$. Since $\eqref{eq:7.4.11}$ is the characteristic equation of
$\eqref{eq:7.4.13}$, Theorem~\ref{thmtype:5.2.1} implies that the general solution
of $\eqref{eq:7.4.13}$ on $(-\infty,\infty)$ is
\begin{eqnarray*}
Y(t)&=&c_1e^{r_1t}+c_2e^{r_2t}\mbox{ if $r_1$ and
$r_2$ are distinct real numbers;  }\\
 Y(t)&=&e^{r_1t}(c_1+c_2t)\mbox{ if
$r_1=r_2$;  }\\
Y(t)&=&e^{\lambda t }\left(c_1\cos\omega
t+c_2\sin\omega t \right)\mbox{ if $r_1,r_2=\lambda\pm i\omega$ with
$\omega\neq0$}.
\end{eqnarray*}
Since $Y(t)=y(e^t)$, substituting $t=\ln x$ in the last three equations shows that the general solution of $\eqref{eq:7.4.12}$ on
$(0,\infty)$ has the form stated in the theorem.
\end{proof}

\begin{example}\label{example:7.4.6}
 Find the general solution of
\begin{equation} \label{eq:7.4.15}
x^2y''-5xy'+9y=0
\end{equation}
on $(0,\infty)$.
\begin{explanation}
The indicial polynomial of $\eqref{eq:7.4.15}$ is
$$
p(r)=r(r-1)-5r+9=(r-3)^2.
$$
Therefore the general solution of $\eqref{eq:7.4.15}$ on $(0,\infty)$ is
$$
y=x^3(c_1+c_2 \ln x).
$$
\end{explanation}
\end{example}

\begin{example}\label{example:7.4.7}
  Find the general solution of
\begin{equation} \label{eq:7.4.16}
x^2y''+3xy'+2y=0
\end{equation}
on  $(0,\infty)$.
\begin{explanation}
The indicial polynomial of $\eqref{eq:7.4.16}$ is
$$
p(r)=r(r-1)+3r+2=(r+1)^2+1.
$$
The roots of the indicial equation are $r=-1 \pm i$ and the general
solution of $\eqref{eq:7.4.16}$ on $(0,\infty)$ is
$$
y=\frac{1}{x}\left[c_1\cos(\ln x)+c_2\sin(\ln x)\right].
$$
\end{explanation}
\end{example}

\section*{Text Source}
Trench, William F., "Elementary Differential Equations" (2013). Faculty Authored and Edited Books \& CDs. 8. (CC-BY-NC-SA)

\href{https://digitalcommons.trinity.edu/mono/8/}{https://digitalcommons.trinity.edu/mono/8/}


\end{document}