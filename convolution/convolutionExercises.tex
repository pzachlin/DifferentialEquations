\documentclass{ximera}
%% You can put user macros here
%% However, you cannot make new environments

%\listfiles

% Get the 'old' hints/expandables, for use on ximera.osu.edu
%\def\xmNotHintAsExpandable{true}
%\def\xmNotExpandableAsAccordion{true}



%\graphicspath{{./}{firstExample/}{secondExample/}}
\graphicspath{{./}
{aboutDiffEq/}
{applicationsLeadingToDiffEq/}
{applicationsToCurves/}
{autonomousSecondOrder/}
{basicConcepts/}
{bernoulli/}
{constCoeffHomSysI/}
{constCoeffHomSysII/}
{constCoeffHomSysIII/}
{constantCoeffWithImpulses/}
{constantCoefficientHomogeneousEquations/}
{convolution/}
{coolingActivity/}
{directionFields/}
{drainingTank/}
{epidemicActivity/}
{eulersMethod/}
{exactEquations/}
{existUniqueNonlinear/}
{frobeniusI/}
{frobeniusII/}
{frobeniusIII/}
{global.css/}
{growthDecay/}
{heatingCoolingActivity/}
{higherOrderConstCoeff/}
{homogeneousLinearEquations/}
{homogeneousLinearSys/}
{improvedEuler/}
{integratingFactors/}
{interactExperiment/}
{introToLaplace/}
{introToSystems/}
{inverseLaplace/}
{ivpLaplace/}
{laplaceTable/}
{lawOfCooling/}
{linSysOfDiffEqs/}
{linearFirstOrderDiffEq/}
{linearHigherOrder/}
{mixingProblems/}
{motionUnderCentralForce/}
{nonHomogeneousLinear/}
{nonlinearToSeparable/}
{odesInSage/}
{piecewiseContForcingFn/}
{population/}
{reductionOfOrder/}
{regularSingularPts/}
{reviewOfPowerSeries/}
{rlcCircuit/}
{rungeKutta/}
{secondLawOfMotion/}
{separableEquations/}
{seriesSolNearOrdinaryPtI/}
{seriesSolNearOrdinaryPtII/}
{simplePendulum/}
{springActivity/}
{springProblemsI/}
{springProblemsII/}
{undCoeffHigherOrderEqs/}
{undeterminedCoeff/}
{undeterminedCoeff2/}
{unitStepFunction/}
{varParHigherOrder/}
{varParamNonHomLinSys/}
{variationOfParameters/}
}


\usepackage{tikz}
%\usepackage{tkz-euclide}
\usepackage{tikz-3dplot}
\usepackage{tikz-cd}
\usetikzlibrary{shapes.geometric}
\usetikzlibrary{arrows}
\usetikzlibrary{decorations.pathmorphing,patterns}
\usetikzlibrary{backgrounds} % added by Felipe
% \usetkzobj{all}   % NOT ALLOWED IN RECENT TeX's ...
\pgfplotsset{compat=1.13} % prevents compile error.

\pdfOnly{\renewcommand{\theHsection}{\thepart.section.\thesection}}  %% MAKES LINKS WORK should be added to CLS
\pdfOnly{\renewcommand{\part}[1]{\chapterstyle\title{#1}\begin{abstract}\end{abstract}\maketitle\def\thechaptertitle{#1}}}


\renewcommand{\vec}[1]{\mathbf{#1}}
\newcommand{\RR}{\mathbb{R}}
\providecommand{\dfn}{\textit}
\renewcommand{\dfn}{\textit}
\newcommand{\dotp}{\cdot}
\newcommand{\id}{\text{id}}
\newcommand\norm[1]{\left\lVert#1\right\rVert}
\newcommand{\dst}{\displaystyle}
 
\newtheorem{general}{Generalization}
\newtheorem{initprob}{Exploration Problem}

\tikzstyle geometryDiagrams=[ultra thick,color=blue!50!black]

\usepackage{mathtools}

\title{Exercises} \license{CC BY-NC-SA 4.0}

\begin{document}

\begin{abstract}
\end{abstract}
\maketitle

\begin{onlineOnly}
\section*{Exercises}
\end{onlineOnly}

\begin{problem}\label{exer:8.6.1}

\begin{enumerate}

\item Express the inverse transform as an integral.  

$\frac{1}{s^2(s^2+4)}$

\item Express the inverse transform as an integral.

$\frac{s}{(s+2)(s^2+9)}$

\item Express the inverse transform as an integral.

$\frac{s}{(s^2+4)(s^2+9)}$

\item Express the inverse transform as an integral.

$\frac{s}{(s^2+1)^2}$

\item Express the inverse transform as an integral.

$\frac{1}{s(s-a)}$

\item Express the inverse transform as an integral.

$\frac{1}{(s+1)(s^2+2s+2)}$

\item Express the inverse transform as an integral.

$\frac{1}{(s+1)^2(s^2+4s+5)}$

\item Express the inverse transform as an integral.

$\frac{1}{(s-1)^3(s+2)^2}$

\item Express the inverse transform as an integral.

$\frac{s-1}{s^2(s^2-2s+2)}$

\item Express the inverse transform as an integral.

$\frac{s(s+3)}{(s^2+4)(s^2+6s+10)}$

\item Express the inverse transform as an integral.

$\frac{1}{(s-3)^5s^6}$

\item Express the inverse transform as an integral.

$\frac{1}{(s-1)^3(s^2+4)}$

\item Express the inverse transform as an integral.

$\frac{1}{s^2(s-2)^3}$

\item Express the inverse transform as an integral.

$\frac{1}{s^7(s-2)^6}$
\end{enumerate}
\end{problem}

\begin{problem}\label{exer:8.6.2}

\begin{enumerate}

\item Find the Laplace transform.

$\int_0^t\sin a\tau\cos b(t-\tau)\, d\tau$

\begin{solution}
$\sin at\leftrightarrow\frac{a}{s^2+a^2}$ and
$\cos bt\leftrightarrow\frac{s}{s^2+b^2}$, so
   $H(s)=\frac{a s}{(s^2+a^2)(s^2+b^2)}$.
\end{solution}

\item Find the Laplace transform.

$\int_0^t e^\tau\sin a(t-\tau)\,d\tau$ 

\begin{solution}
$e^t\leftrightarrow\frac{1}{s-1}$ and
$\sin at\leftrightarrow\frac{a}{s^2+a^2}$, so
$H(s)=\frac{a}{(s-1)(s^2+a^2)}$.
\end{solution}

\item Find the Laplace transform. 

$\int_0^t\sinh a\tau\cosh a(t-\tau)\,d\tau$

\begin{solution}
$\sinh at\leftrightarrow\frac{a}{s^2-a^2}$ and
$\cosh at\leftrightarrow\frac{1}{s^2-a^2}$, so
$H(s)=\frac{as}{(s^2-a^2)^2}$.
\end{solution}

\item Find the Laplace transform. 

$\int_0^t\tau(t-\tau)\sin \omega\tau\cos\omega
(t-\tau)\,d\tau$

\begin{solution}
$t\sin\omega t\leftrightarrow\frac{2\omega
s}{(s^2+\omega^2)^2}$ and $t\cos\omega
t\leftrightarrow\frac{s^2-\omega^2}{(s^2+\omega^2)^2}$, so
$H(s)=\frac{2\omega s (s^2-\omega^2)}{(s^2+\omega^2)^4}$.
\end{solution}

\item Find the Laplace transform. 

$e^t\int_0^t\sin\omega\tau
\cos\omega (t-\tau)\,d\tau$

\begin{solution}
$e^t\int_0^t\sin\omega\tau\cos\omega
(t-\tau)\,d\tau=\int_0^t
\left(e^\tau\sin\omega\tau\right)\left(e^{(t-\tau)}\cos\omega
(t-\tau)\right)\,d\tau$; $e^t\sin\omega
t\leftrightarrow\frac{\omega}{(s-1)^2+\omega^2}$ and $e^t\cos\omega
t\leftrightarrow\frac{s-1}{(s-1)^2+\omega^2}$, so
$H(s)=\frac{(s-1)\omega}{\left((s-1)^2+\omega^2\right)^2}$.
\end{solution}

\item Find the Laplace transform. 

$e^t\int_0^t\tau^2 (t-\tau)e^\tau\,d\tau$

\begin{solution}
$e^t\int_0^t\tau^2
(t-\tau)e^\tau\,d\tau = \int_0^t\tau^2e^{2\tau}(t-\tau)e^{(t-\tau)}\,d\tau$;
$t^2e^{2t}\leftrightarrow\frac{2}{(s-2)^3}$ and
$te^t\leftrightarrow\frac{1}{(s-1)^2}$, so
 $H(s)=\frac{2}{(s-2)^3 (s-1)^2}$.
\end{solution}

\item Find the Laplace transform. 

$e^{-t}\int_0^t e^{-\tau}\tau\cos\omega (t-\tau)\,d\tau$

\begin{solution}
$e^{-t}\int_0^t e^{-\tau}\tau\cos\omega
(t-\tau)\,d\tau=
\int_0^t\tau e^{-2\tau}e^{-(t-\tau)}\cos\omega(t-\tau)\,d\tau$;
$te^{-2t}\leftrightarrow\frac{1}{(s+2)^2}$ and
$e^{-t}\cos\omega t\leftrightarrow\frac{s+1}{(s+1)^2+\omega^2}$, so
$H(s)=\frac{s+1}{(s+2)^2\left[(s+1)^2+
\omega^2\right]}$.
\end{solution}

\item Find the Laplace transform. 

$e^t\int_0^t e^{2\tau}\sinh
(t-\tau)\,d\tau$

\begin{solution}
$e^t\int_0^t e^{2\tau}\sinh (t-\tau)\,d\tau=
\int_0^t e^{3\tau}\left(e^{(t-\tau)}\sinh
(t-\tau)\right)\,d\tau$; $e^{3t}\leftrightarrow\frac{1}{s-3}$ and
$e^t\sinh t\leftrightarrow\frac{1}{(s-1)^2-1}$, so
$H(s)=\frac{1}{(s-3)\left((s-1)^2-1\right)}$.
\end{solution}

\item Find the Laplace transform. 

$\int_0^t\tau e^{2\tau}\sin
2(t-\tau)\,d\tau$

\begin{solution}
$te^{2t}\leftrightarrow\frac{1}{(s-2)^2}$ and
$\sin2t\leftrightarrow\frac{2}{s^2+4}$, so
 $H(s)=\frac{2}{(s-2)^2(s^2+4)}$.
\end{solution}

\item Find the Laplace transform. 

$\int_0^t (t-\tau)^3 e^\tau\, d\tau$

\begin{solution}
$t^3\leftrightarrow\frac{6}{s^4}$  and
$e^t\leftrightarrow\frac{1}{s-1}$, so
$H(s)=\frac{6}{s^4(s-1)}$.
\end{solution}

\item Find the Laplace transform. 

$\int_0^t\tau^6
e^{-(t-\tau)}\sin 3(t-\tau)\,d\tau$

\begin{solution}
 $t^6\leftrightarrow \frac{6!}{s^7}$ and
$e^{-t}\sin3t \leftrightarrow\frac{3}{(s+1)^2+9}$, so
$H(s)=\frac{3\cdot 6!}{s^7\left[(s+1)^2+
9\right]}$.
\end{solution}

\item Find the Laplace transform. 

$\int_0^t\tau^2 (t-\tau)^3\,
d\tau$

\begin{solution}
$t^2\leftrightarrow\frac{2}{s^3}$ and
$t^3\leftrightarrow\frac{6}{s^4}$, so
 $H(s)=\frac{12}{s^7}$.
\end{solution}

\item Find the Laplace transform. 

$\int_0^t (t-\tau)^7 e^{-\tau}
\sin 2\tau\,d\tau$

\begin{solution}
$t^7\leftrightarrow\frac{7!}{s^8}$ and
$e^{-t}\sin2t \leftrightarrow\frac{2}{(s+1)^2+4}$, so
  $H(s)=\frac{2\cdot 7!}{s^8\left[(s+1)^2+
4\right]}$.
\end{solution}

\item Find the Laplace transform. 

$\int_0^t (t-\tau)^4\sin
2\tau\,d\tau$

\begin{solution}
$t^4\leftrightarrow\frac{24}{s^5}$ and
$\sin2t\leftrightarrow\frac{2}{s^2+4}$, so
 $H(s)=\frac{48}{s^5(s^2+4)}$.
\end{solution}
\end{enumerate}
\end{problem}

\begin{problem}\label{exer:8.6.3}

\begin{enumerate}
\item Find a formula for the solution of the initial value problem.

$y''+3y'+y=f(t),\quad y(0)=0,\quad y'(0)=0$

\item Find a formula for the solution of the initial value problem.

$y''+4y=f(t),\quad y(0)=0,\quad y'(0)=0$

\item Find a formula for the solution of the initial value problem.

$y''+2y'+y=f(t),\quad y(0)=0,\quad y'(0)=0$

\item Find a formula for the solution of the initial value problem.

$y''+k^2y=f(t),\quad y(0)=1,\quad y'(0)=-1$

\item Find a formula for the solution of the initial value problem.

$y''+6y'+9y=f(t),\quad y(0)=0,\quad y'(0)=-2$

\item Find a formula for the solution of the initial value problem.

$y''-4y=f(t),\quad y(0)=0,\quad y'(0)=3$

\item Find a formula for the solution of the initial value problem.

$y''-5y'+6y=f(t),\quad y(0)=1,\quad y'(0)=3$

\item Find a formula for the solution of the initial value problem.

$y''+\omega^2y=f(t),\quad y(0)=k_0,\quad y'(0)=k_1$
\end{enumerate}
\end{problem}

\begin{problem}\label{exer:8.6.4}

\begin{enumerate}
\item Solve the integral equation.

$y(t)=t-\int_0^t (t-\tau)
y(\tau)\,d\tau$

\begin{solution}
$Y(s)=\frac{1}{s^2}-\frac{Y(s)}{s^2}$;
$Y(s)\left(1+\frac{1}{s^2}\right)=\frac{1}{s^2}$; $Y(s)\frac{s^2+1}{s^2}=\frac{1}{s^2}$; $Y(s)=\frac{1}{s^2+1}$, so $y=\sin t$.
\end{solution}

\item Solve the integral equation.

$y(t)=\sin t-2
\int_0^t\cos (t-\tau) y (\tau)\,d\tau$

\begin{solution}
$Y(s)=\frac{1}{s^2+1}-\frac{2sY(s)}{s^2+1}$;
$Y(s)\left(1+\frac{2s}{s^2+1}\right)=\frac{1}{s^2+1}$; $Y(s)\frac{(s+1)^2}{s^2+1}=\frac{1}{s^2+1}$; $Y(s)=\frac{1}{(s+1)^2}$, so
$y=te^{-t}$.
\end{solution}

\item Solve the integral equation.

$y(t)=1+2
\int_0^ty(\tau)\cos(t-\tau)\,d\tau$

\begin{solution}
$Y(s)=\frac{1}{s}+\frac{2sY(s)}{s^2+1}$;
$Y(s)\left(1-\frac{2s}{s^2+1}\right)=\frac{1}{s}$; $Y(s)\frac{(s-1)^2}{s^2+1}=\frac{1}{s}$;
$Y(s)=\frac{(s^2+1)}{s(s-1)^2}=\frac{A}{s}+\frac{B}{s^2}
+\frac{C}{(s-1)^2}$, where $A(s-1)^2+Bs(s-1)+Cs=s^2+1$.
Setting $s=0$ and $s=1$ shows that $A=1$ and $C=2$; equating
coefficients of $s^2$ yields $A+B=1$, so $B=0$.  Therefore,
$Y(s)=\frac{1}{s}
+\frac{1}{(s-1)^2}$, so $y=1+2te^t$.
\end{solution}

\item Solve the integral equation.

$y(t)=t+\int_0^t
y(\tau)e^{-(t-\tau)}\,d\tau$

\begin{solution}
$Y(s)=\frac{1}{s^2}+\frac{Y(s)}{s+1}$;
$Y(s)\left(1-\frac{1}{s+1}\right)=\frac{1}{s^2}$;
$Y(s)\left(\frac{s}{s+1}\right)=\frac{1}{s^2}$;
$Y(s)=\frac{s+1}{s^3}
=\frac{1}{s^2}+\frac{1}{s^3}$, so
$y=t+\frac{t^2}{2}$.
\end{solution}

\item Solve the integral equation. 

$y'(t)=t+\int_0^t y(\tau)\cos
(t-\tau)\,d\tau,\, y(0)=4$

\begin{solution}
$sY(s)-4=\frac{1}{s^2}+\frac{sY(s)}{s^2+1}$;
$Y(s)\left(s-\frac{s}{s^2+1}\right)=4+\frac{1}{s^2}$; $Y(s)\frac{s^3}{s^2+1}=\frac{4s^2+1}{s^2}$;
$Y(s)=\frac{(4s^2+1)(s^2+1)}{s^5}=\frac{4s^4+5s^2+1}{s^5}
=\frac{4}{s}+\frac{5}{s^3}+\frac{1}{s^5}$, so
$y=4+\frac{5}{2}t^2+\frac{1}{24}t^4$.
\end{solution}

\item Solve the integral equation.

$y(t)=\cos t-\sin t+
\int_0^t y(\tau)\sin (t-\tau)\,d\tau$

\begin{solution}
$Y(s)=\frac{s-1}{s^2+1}+\frac{Y(s)}{s^2+1}$;
$Y(s)\left(1-\frac{1}{s^2+1}\right)=\frac{s-1}{s^2+1}$; $Y(s)\frac{s^2}{s^2+1}=\frac{s-1}{s^2+1}$; $Y(s)=\frac{s-1}{s^2}
=\frac{1}{s}-\frac{1}{s^2}$, so
 $y=1-t$.
\end{solution}

\end{enumerate}
\end{problem}

\begin{problem}\label{exer:8.6.5}

\begin{enumerate}

\item Use the convolution theorem to evaluate the integral.

$\int_0^t (t-\tau)^7\tau^8\,
d\tau$

\item Use the convolution theorem to evaluate the integral.

$\int_0^t(t-\tau)^{13}\tau^7\,d\tau$

\item Use the convolution theorem to evaluate the integral.

$\int_0^t(t-\tau)^6\tau^7\,
d\tau$

\item Use the convolution theorem to evaluate the integral.

$\int_0^te^{-\tau}\sin(t-\tau)\,d\tau$

\item Use the convolution theorem to evaluate the integral.

 $\int_0^t\sin\tau\cos2(t-\tau)\,d\tau$
\end{enumerate}
\end{problem}

\begin{problem}\label{exer:8.6.6}
 Show that
$$
\int_0^tf(t-\tau)g(\tau)\,d\tau=\int_0^tf(\tau)g(t-\tau)\,d\tau
$$
by introducing the new variable of integration $x=t-\tau$ in the first
integral.

\begin{solution}
Substituting $x=t-\tau$ yields
$\int_0^tf(t-\tau)g(\tau)\,d\tau=-\int_t^0f(x)g(t-x)(-\,dx)=
\int_0^tf(x)g(t-x)\,dx=\int_0^tf(\tau)g(t-\tau)\,d\tau$.
\end{solution}
\end{problem}

\begin{problem}\label{exer:8.6.7} Use the convolution theorem to show that if
$f(t)\leftrightarrow F(s)$ then
$$
\int_0^tf(\tau)\,d\tau\leftrightarrow \frac{F(s)}{s}.
$$
\end{problem}

\begin{problem}\label{exer:8.6.8}
 Show that if $p(s)=as^2+bs+c$ has
distinct real zeros $r_1$ and $r_2$ then the solution of
$$
ay''+by'+cy=f(t),\quad y(0)=k_0,\quad y'(0)=k_1
$$
is
\begin{eqnarray*}
y(t)&=&\; k_0\frac{r_2e^{r_1t}-r_1e^{r_2t}}{r_2-r_1}+k_1\frac{e^{r_2t}-e^{r_1t}
}{r_2-r_1}
\\
&&+\frac{1}{a(r_2-r_1)}\int_0^t(e^{r_2\tau}-e^{r_1\tau})f(t-\tau)\,d\tau.
\end{eqnarray*}

\begin{solution}
$p(s)Y(s)=F(s)+a(k_1+k_0s)+bk_0$, so
(A) $Y(s)=\frac{F(s)}{p(s)}+\frac{k_0(as+b)+k_1a}{p(s)}$.
Since $p(s)=a(s-r_1)(s-r_2)$ and therefore $b=-a(r_1+r_2)$,
(A) can be rewritten as
$$
Y(s)=\frac{F(s)}{a(s-r_1)(s-r_2)}+\frac{k_0(s-r_1-r_2)}{(s-r_1)(s-r_2)}
+\frac{k_1}{(s-r_1)(s-r_2)}.
$$
$$
\frac{1}{(s-r_1)(s-r_2)}=\frac{1}{r_2-r_1}\left(\frac{1}{s-r_2}
-\frac{1}{s-r_1}\right)\leftrightarrow\frac{e^{r_2t}-e^{r_1t}}{r_2-r_1},
$$
so the convolution theorem implies that
$$
\frac{F(s)}{a(s-r_1)(s-r_2)}\leftrightarrow
\frac{1}{a}\int_0^t\frac{e^{r_2\tau}-e^{r_1\tau}}{r_2-r_1}f(t-\tau)\,d\tau.
$$
$$
\frac{s-r_1-r_2}{(s-r_1)(s-r_2)}=\frac{r_2}{r_2-r_1}\frac{1}{s-r_1}
-\frac{r_1}{r_2-r_1}\frac{1}{s-r_2}\leftrightarrow
\frac{r_2e^{r_1t}-r_1e^{r_2}t}{r_2-r_1}.
$$
Therefore,
$$
y(t)=\ k_0\frac{r_2e^{r_1t}-r_1e^{r_2t}}{r_2-r_1}+k_1\frac{e^{r_2t}-e^{r_1t}}{r_2-r_1}
+\frac{1}{a}
\int_0^t\frac{e^{r_2\tau}-e^{r_1\tau}}{r_2-r_1}f(t-\tau)\,d\tau.
$$
\end{solution}
\end{problem}

\begin{problem}\label{exer:8.6.9}
Show that if $p(s)=as^2+bs+c$ has
a repeated  real zero $r_1$ then
 the solution of
$$
ay''+by'+cy=f(t),\quad y(0)=k_0,\quad y'(0)=k_1
$$
is
$$
y(t)=\; k_0(1-r_1t)e^{r_1t}+k_1te^{r_1t}
+\frac{1}{a}\int_0^t\tau
e^{r_1\tau}f(t-\tau)\,d\tau.
$$
\end{problem}

\begin{problem}\label{exer:8.6.10}
 Show that if $p(s)=as^2+bs+c$ has
 complex conjugate zeros $\lambda\pm i\omega$
then the solution of
$$
ay''+by'+cy=f(t),\quad y(0)=k_0,\quad y'(0)=k_1
$$
is
\begin{eqnarray*}
y(t)&=&\;  e^{\lambda t}\left[k_0(\cos\omega t-\frac{\lambda}{\omega}\sin\omega
t)+\frac{k_1}{\omega}\sin\omega t\right]
\\
&&+\frac{1}{a\omega}\int_0^te^{\lambda t}f(t-\tau)\sin\omega\tau\,
d\tau.
\end{eqnarray*}

\begin{solution}
$p(s)Y(s)=F(s)+a(k_1+k_0s)+bk_0$, so
(A) $Y(s)=\frac{F(s)}{p(s)}+\frac{k_0(as+b)+k_1a}{p(s)}$.
Since $p(s)=a(s-\lambda)^2+\omega^2$ and therefore $b=-2a\lambda$,
(A) can be rewritten as
$$
Y(s)=\frac{F(s)}{a[(s-\lambda)^2+\omega^2}]+\frac{k_0(s-2\lambda)}{(s-\lambda)^2+\omega^2}
+\frac{k_1}{(s-\lambda)^2+\omega^2}.
$$
$\frac{1}{(s-\lambda)^2+\omega^2}\leftrightarrow
\frac{1}{\omega}e^{\lambda t}\sin\omega t$, so the convolution theorem
implies that
$$
\frac{F(s)}{a[(s-\lambda)^2+\omega^2]}\leftrightarrow
\frac{1}{a\omega}\int_0^te^{\lambda t}f(t-\tau)\sin\omega\tau\,
d\tau.
$$
$$
\frac{s-2\lambda}{(s-\lambda)^2+\omega^2}=
\frac{(s-\lambda)-\lambda}{(s-\lambda)^2+\omega^2}\leftrightarrow
e^{\lambda t}\left(\cos\omega
t-\frac{\lambda}{\omega}\sin\omega t\right).
$$
Therefore,
$$
y(t)=e^{\lambda t}\left[k_0\left(\cos\omega
t-\frac{\lambda}{\omega}\sin\omega
t\right)+\frac{k_1}{\omega}\sin\omega t\right]
+\frac{1}{a\omega}\int_0^te^{\lambda t}f(t-\tau)\sin\omega\tau\,
d\tau.
$$
\end{solution}
\end{problem}

\begin{problem}\label{exer:8.6.11}
Let
$$
w={\cal L}^{-1}\left(\frac{1}{as^2+bs+c}\right),
$$
where $a,b$, and $c$ are constants and $a\ne0$.
\begin{enumerate}
\item % (a)
Show that $w$ is the solution of
$$
aw''+bw'+cw=0,\quad w(0)=0,\quad w'(0)=\frac{1}{a}.
$$
\item\label{exer:8.6.11b} % (b)
Let $f$ be continuous on $[0,\infty)$  and define
$$
h(t)=\int_0^t w(t-\tau)f(\tau)\,d\tau.
$$
Use
\href{http://www-history.mcs.st-and.ac.uk/Mathematicians/Leibniz.html}{Leibniz's rule} for differentiating an integral with
respect to a parameter to show  that $h$ is the solution of
$$
ah''+bh'+ch=f,\quad h(0)=0,\quad h'(0)=0.
$$
\item % (c)
 Show that the function $y$ in  Eqn.~\eqref{eq:8.6.14} is the solution of
Eqn.~\eqref{eq:8.6.13} provided that $f$ is continuous on $[0,\infty)$;
thus, it's not necessary to assume that $f$ has a Laplace transform.
\end{enumerate}
\end{problem}

\begin{problem}\label{exer:8.6.12}
Consider the initial value problem
$$
ay''+by'+cy=f(t),\quad y(0)=0,\quad y'(0)=0,
\text{ (A)}
$$
where $a,b$, and $c$ are constants, $a\ne0$, and
$$
f(t)=\left\{\begin{array}{cc}f_0(t),&0\le t<t_1,\\
f_1(t),&t\ge t_1.\end{array}\right.
$$
Assume that $f_0$ is continuous and of exponential order on $[0,\infty)$
and $f_1$ is continuous and of exponential order on $[t_1,\infty)$.
Let
$$
p(s)=as^2+bs+c.
$$
\begin{enumerate}
\item % (a)
Show that the Laplace transform of the solution of  (A)
is
$$
Y(s)=\frac{F_0(s)+e^{-st_1}G(s)}{p(s)}
$$
where  $g(t)=f_1(t+t_1)-f_0(t+t_1)$.

\begin{solution}
$$
ay''+by'+cy=f_0(t)+u(t-t_1)(f_1(t)-f_0(t)),\,y(0)=0,\,y'(0)=0;
$$
$$
p(s)Y(s)=F_0(s)+{\cal L}(u(t-t_1)(f_1(t)-f_0(t)))
=F_0(s)+e^{-st_1}{\cal L}(g);
$$
$$
Y(s)=\frac{F_0(s)+e^{-st_1}G(s)}{p(s)}.
\text{ (B)}
$$
\end{solution}

\item % (b)
Let $w$ be as in Exercise~\ref{exer:8.6.11}.
Use Theorem~\ref{thmtype:8.4.2} and the convolution theorem
to show that the solution of (A) is
$$
y(t)=\int_0^t
w(t-\tau)f_0(\tau)\,d\tau+u(t-t_1)\int_0^{t-t_1}
w(t-t_1-\tau)g(\tau)\,d\tau
$$
for $t>0$.

\begin{solution}
Since $F_0(s)\leftrightarrow f_0(t)$, $G(s)\leftrightarrow
g(t)$, and $\frac{1}{p(s)}\leftrightarrow w(t)$, the convolution
theorem implies that
$$
\frac{F_0(s)}{p(s)}\leftrightarrow
\int_0^tw(t-\tau)f_0(\tau)\,d\tau\mbox{\quad and \quad}
 \frac{G(s)}{p(s)}\leftrightarrow
\int_0^tw(t-\tau)g(\tau)\,d\tau.
$$
 Now Theorem~\ref{thmtype:8.4.2} implies that
 $\frac{e^{-st_1}G(s)}{p(s)}\leftrightarrow
u(t-t_1)\int_0^tw(t-t_1-\tau)g(\tau)\,d\tau$, and (B) implies
that
$$
y(t)=\int_0^t
w(t-\tau)f_0(\tau)\,d\tau+u(t-t_1)\int_0^{t-t_1}
w(t-t_1-\tau)g(\tau)\,d\tau.
$$
\end{solution}

\item % (c)
Henceforth, assume only that $f_0$ is continuous on $[0,\infty)$ and $f_1$
 is continuous on $[t_1,\infty)$.
 Use  Exercise~\ref{exer:8.6.11} and what you did above to show that
$$
y'(t)=\int_0^t
w'(t-\tau)f_0(\tau)\,d\tau+u(t-t_1)\int_0^{t-t_1}
w'(t-t_1-\tau)g(\tau)\,d\tau
$$
for $t>0$, and
$$
y''(t)=\frac{f(t)}{a}+\int_0^t
w''(t-\tau)f_0(\tau)\,d\tau+u(t-t_1)\int_0^{t-t_1}
w''(t-t_1-\tau)g(\tau)\,d\tau
$$
for $0<t<t_1$ and $t>t_1$. Also, show
 $y$ satisfies the differential equation in
(A) on$(0,t_1)$ and $(t_1,\infty)$.

\begin{solution}
Let $z_0(t)=\int_0^tw(t-\tau)f_0(\tau)\,d\tau$
and $z_1(t)=\int_0^tw(t-\tau)g(\tau)\,d\tau$.
Then $y(t)=z_0(t)+u(t-t_1)z_1(t-t_1)$. Using Leibniz's rule as in the
solution of Exercise~\ref{exer:8.6.11b} shows that
$$
z_0'(t)=\int_0^tw'(t-\tau)f_0(\tau)\,d\tau,\quad
z_1'(t)=\dst\int_0^tw'(t-\tau)g(\tau)\,d\tau,\quad t>0,
$$
$$
z_0''(t)=\frac{f_0(t)}{a}+\int_0^tw''(t-\tau)f_0(\tau)\,d\tau,
z_1''(t)=\frac{g(t)}{a}+\int_0^tw''(t-\tau)g(\tau)\,d\tau,\quad
t>0,
$$
 if $t>0$, and that
$$
az_0''+bz_0'+cz_0=f_0(t)\quad \mbox{and} \quad
az_1''+bz_1'+cz_1=f_1(t+t_1)-f_0(t+t_1),\quad  t>0.
$$
This implies the stated conclusion for $y'$ and $y''$  on $(0,t)$ and
$(t,\infty)$, and that $ay''+by'+cy=f(t)$ on these intervals.
\end{solution}

\item % (d)
Show that $y$ and $y'$ are continuous on $[0,\infty)$.

\begin{solution}
Since  the functions $z_0(t)$ and  $h(t)=u(t-t_1)z_1(t-t_1)$
are both continuous on $[0,\infty)$ and $h(t)=0$ if $0\le t\le t_1$,
$y$ is continuous on $[0,\infty)$. From the previous part, $y'$ is continuous
on $[0,t_1)$ and $(t_1,\infty)$, so we need only show that $y'$
is continuous at $t_1$.
For this it suffices to show  that $h'(t_1)=0$.
 Since $h(t_1)=0$ if $t\le t_1$,
(B) $\lim_{t\to t_1-}\frac{h(t)-h(t_1)}{t-t_1}=0$.
If $t>t_1$, then $h(t)=\int_0^{t-t_1}w(t-t_1\tau)g(\tau)\,d\tau$.
Since $h(t_1)=0$,
$$
\left|\frac{h(t)-h(t_1)}{t-t_1}\right|\le\int_0^{t-t_1}
|w(t-t_1-\tau)g(\tau)|\,d\tau.
\text{ (B)}
$$
 Since $g$ is continuous from the right at
$0$, we can choose constants $T>0$  and $M>0$ so that
$|g(\tau)|<M$ if $0\le \tau\le T$. Then (B) implies that
$$
\left|{h(t)-h(t_1)\over t-t_1}\right|\le M\int_0^{t-t_1}
|w(t-t_1-\tau)|\,d\tau,\quad t_1<t<t_1+T.
\text{ (C)}
$$
Now suppose $\epsilon>0$. Since $w(0)=0$, we can choose $T_1$
such that $0<T_1<T$ and $|w(x)|<\epsilon/M$ if $0\le x<T_1$.
If $t_1<t<t_1+T_1$ and $0\le\tau\le t-t_1$, then $0\le t-t_1-\tau<T_1$,
so (C) implies that
$$
\left|\frac{h(t)-h(t_1)}{t-t_1}\right|< \epsilon,\quad
t_1<t<t_1+T.
$$
Therefore,
 $\lim_{t\to t_1+}\frac{h(t)-h(t_1)}{t-t_1}=0$. This and
(B) imply that $h'(t_1)=0$.

\end{solution}
\end{enumerate}
\end{problem}

\begin{problem}\label{exer:8.6.13}
Suppose
$$
f(t)=\left\{\begin{array}{cl}
 f_0(t),&0\le t < t_1,\\[5pt]
 f_1(t),&t_1\le t < t_2,\\
&\vdots\\
f_{k-1}(t),&t_{k-1}\le t < t_k,\\
 f_k(t),&t\ge t_k,
\end{array}\right.
$$
where  $f_m$ is continuous on $[t_m,\infty)$ for $m=0,\dots,k$
(let $t_0=0$),  and define
$$
g_m(t)=f_m(t+t_m)-f_{m-1}(t+t_m) ,\, m=1,\dots,k.
$$
Extend the results of Exercise~\ref{exer:8.6.12} to show that
the solution of
$$
ay''+by'+cy=f(t),\quad y(0)=0,\quad y'(0)=0
$$
is
$$
y(t)=\int_0^t w(t-\tau)f_0(\tau)\,d\tau+\sum_{m=1}^ku(t-t_m)
\int_0^{t-t_m}w(t-t_m-\tau)g_m(\tau)\,d\tau.
$$
\end{problem}

\begin{problem}\label{exer:8.6.14}
Let $\{t_m\}_{m=0}^\infty$  be a sequence of points such that $t_0=0$,
$t_{m+1}>t_m$,
and $\lim_{m\to\infty}t_m=\infty$. For each nonegative integer $m$
let $f_m$ be continuous on $[t_m,\infty)$, and let
 $f$ be defined on $[0,\infty)$ by
$$
f(t)=f_m(t),\quad t_m\le t<t_{m+1}\quad  m=0,1,2\dots.
$$
Let
$$
g_m(t)=f_m(t+t_m)-f_{m-1}(t+t_m),\quad  m=1,\dots,k.
$$
Extend the results of Exercise~\ref{exer:8.6.13} to show that the solution
of
$$
ay''+by'+cy=f(t),\quad y(0)=0,\quad y'(0)=0
$$
is
$$
y(t)=\int_0^t w(t-\tau)f_0(\tau)\,d\tau+\sum_{m=1}^\infty u(t-t_m)
\int_0^{t-t_m}w(t-t_m-\tau)g_m(\tau)
\,d\tau.
$$
\begin{hint}
  See Exercise~\ref{exer:8.4.30}. 
\end{hint}

\end{problem}


\end{document}