\documentclass{ximera}
%% You can put user macros here
%% However, you cannot make new environments

%\listfiles

% Get the 'old' hints/expandables, for use on ximera.osu.edu
%\def\xmNotHintAsExpandable{true}
%\def\xmNotExpandableAsAccordion{true}



%\graphicspath{{./}{firstExample/}{secondExample/}}
\graphicspath{{./}
{aboutDiffEq/}
{applicationsLeadingToDiffEq/}
{applicationsToCurves/}
{autonomousSecondOrder/}
{basicConcepts/}
{bernoulli/}
{constCoeffHomSysI/}
{constCoeffHomSysII/}
{constCoeffHomSysIII/}
{constantCoeffWithImpulses/}
{constantCoefficientHomogeneousEquations/}
{convolution/}
{coolingActivity/}
{directionFields/}
{drainingTank/}
{epidemicActivity/}
{eulersMethod/}
{exactEquations/}
{existUniqueNonlinear/}
{frobeniusI/}
{frobeniusII/}
{frobeniusIII/}
{global.css/}
{growthDecay/}
{heatingCoolingActivity/}
{higherOrderConstCoeff/}
{homogeneousLinearEquations/}
{homogeneousLinearSys/}
{improvedEuler/}
{integratingFactors/}
{interactExperiment/}
{introToLaplace/}
{introToSystems/}
{inverseLaplace/}
{ivpLaplace/}
{laplaceTable/}
{lawOfCooling/}
{linSysOfDiffEqs/}
{linearFirstOrderDiffEq/}
{linearHigherOrder/}
{mixingProblems/}
{motionUnderCentralForce/}
{nonHomogeneousLinear/}
{nonlinearToSeparable/}
{odesInSage/}
{piecewiseContForcingFn/}
{population/}
{reductionOfOrder/}
{regularSingularPts/}
{reviewOfPowerSeries/}
{rlcCircuit/}
{rungeKutta/}
{secondLawOfMotion/}
{separableEquations/}
{seriesSolNearOrdinaryPtI/}
{seriesSolNearOrdinaryPtII/}
{simplePendulum/}
{springActivity/}
{springProblemsI/}
{springProblemsII/}
{undCoeffHigherOrderEqs/}
{undeterminedCoeff/}
{undeterminedCoeff2/}
{unitStepFunction/}
{varParHigherOrder/}
{varParamNonHomLinSys/}
{variationOfParameters/}
}


\usepackage{tikz}
%\usepackage{tkz-euclide}
\usepackage{tikz-3dplot}
\usepackage{tikz-cd}
\usetikzlibrary{shapes.geometric}
\usetikzlibrary{arrows}
\usetikzlibrary{decorations.pathmorphing,patterns}
\usetikzlibrary{backgrounds} % added by Felipe
% \usetkzobj{all}   % NOT ALLOWED IN RECENT TeX's ...
\pgfplotsset{compat=1.13} % prevents compile error.

\pdfOnly{\renewcommand{\theHsection}{\thepart.section.\thesection}}  %% MAKES LINKS WORK should be added to CLS
\pdfOnly{\renewcommand{\part}[1]{\chapterstyle\title{#1}\begin{abstract}\end{abstract}\maketitle\def\thechaptertitle{#1}}}


\renewcommand{\vec}[1]{\mathbf{#1}}
\newcommand{\RR}{\mathbb{R}}
\providecommand{\dfn}{\textit}
\renewcommand{\dfn}{\textit}
\newcommand{\dotp}{\cdot}
\newcommand{\id}{\text{id}}
\newcommand\norm[1]{\left\lVert#1\right\rVert}
\newcommand{\dst}{\displaystyle}
 
\newtheorem{general}{Generalization}
\newtheorem{initprob}{Exploration Problem}

\tikzstyle geometryDiagrams=[ultra thick,color=blue!50!black]

\usepackage{mathtools}

\title{Exercises} \license{CC BY-NC-SA 4.0}

\begin{document}

\begin{abstract}
\end{abstract}
\maketitle

\begin{onlineOnly}
\section*{Exercises}
\end{onlineOnly}

\begin{problem}\label{exer:1.2.1}
Find the order of the   equation.
\begin{enumerate}
\item  $\frac{d^2}{dx^2}+2\frac{dy}{dx}\frac{d^3y}{dx^3}+x=0$
\item    $y''-3y'+2y=x^7$
\item   $y'-y^7=0$
\item   $y''y-(y')^2=2$
\end{enumerate}
\end{problem}

\begin{problem}\label{exer:1.2.2}
Verify that the  function is a solution of the
differential equation on some interval, for any choice of
the arbitrary constants appearing in the function.

\begin{enumerate}
\item % (a)
 $y=ce^{2x};     \quad y'=2y$

Click below to see the answer.

\begin{expandable}
    If $y=ce^{2x}$, then $y'=2ce^{2x}=2y$.
\end{expandable}
 
\item % (b)
 $y=\frac{x^2}{3}
+\frac{c}{x};   \quad xy'+y=x^2$

Click below to see the answer.

\begin{expandable}
    $y=\frac{x^2}{3}+\frac{c}{x}$, then
$y'=\frac{2x}{3}-\frac{c}{x^2}$, so $xy'+y=\frac{2x^2}{3}-\frac{c}{x
}+\frac{x^2}{3}+\frac{c}{x}=x^2$.
\end{expandable}

\item %(c)
$y=\frac{1}{2}+ce^{-x^2};    \quad y'+2xy=x$

Click below to see the answer.

\begin{expandable}
    If
$$
y=\frac{1}{2}+ce^{-x^2}, \quad \text{then} \quad
y'=-2xce^{-x^2}
$$
and
$$
y'+2xy=-2xce^{-x^2}+2x\left(\frac{1}{2}+ce^{-x^2}\right)
=-2xce^{-x^2}+x+2cxe^{-x^2}=x.
$$
\end{expandable}

\item %(d)
$y=(1+ce^{-x^2/2})(1-ce^{-x^2/2})^{-1};   \quad
2y'+x(y^2-1)=0$

Click below to see the answer.

\begin{expandable}
    If
$$
y=\frac{1+ce^{-x^2/2}}{1-ce^{-x^2/2}}
$$
then
\begin{eqnarray*}
y'&=&\frac{(1-ce^{-x^2/2})(-cxe^{-x^2/2})-(1+ce^{-x^2/2})cxe^{-x^2/2}}{(1-cxe^{-x^2/2})^2}\\
&=&\frac{-2cxe^{-x^2/2}}{(1-ce^{-x^2/2})^2}
\end{eqnarray*}
and
\begin{eqnarray*}
y^2-1&=&\left(\frac{1+ce^{-x^2/2}}{1-ce^{-x^2/2}}\right)^2-1\\
&=&\frac{(1+ce^{-x^2/2})^2-(1-ce^{-x^2/2})^2}{(1-ce^{-x^2/2})^2}\\
&=&\frac{4ce^{-x^2/2}}{(1-ce^{-x^2/2})^2}
\end{eqnarray*}

so
$$
2y'+x(y^2-1)=\frac{-4cx+4cx}{(1-ce^{-x^2/2})^2}=0.
$$
\end{expandable}

\item %(e)
$y=\tan\left(\frac{x^3}{3}+c\right);     \quad
y'=x^2(1+y^2)$

Click below to see the answer.

\begin{expandable}
    If $y=\tan\left( \frac{x^3}{3}+c\right)$, then
$y'=x^2\sec^2\left(\frac{x^3}{3}+c\right)=
x^2\left(1+\tan^2\left(\frac{x^3}{3+c}\right)\right)=
x^2(1+y^2)$.
\end{expandable}

\item %(f)
$y=(c_1+c_2x)e^x+\sin x+x^2;    \quad
y''-2y'+y=-2 \cos x+x^2-4x+2$

Click below to see the answer.

\begin{expandable}
    If
\begin{eqnarray*}
y&=&(c_1+c_2x)e^x+\sin x+x^2,\mbox{\quad then}\\
y'&=&(c_1+2c_2x)e^x+\cos x+2x,\\
y'&=&(c_1+3c_2x)e^x-\sin x+2
\end{eqnarray*}
and
\begin{eqnarray*}
y''-2y'+y&=&c_1e^x(1-2+1)+c_2xe^x(3-4+1)\\
&&-\sin x-2\cos x+\sin x+2-4x+x^2\\
&=&-2 \cos x+x^2-4x+2.
\end{eqnarray*}
\end{expandable}

\item %(g)
$y=c_1e^x+c_2x+\frac{2}{ x};     \quad
(1-x)y''+xy'- y=4(1-x-x^2)x^{-3}$

Click below to see the answer.

\begin{expandable}
    If $y=c_1e^x+c_2x+\frac{2}{x}$, then
$y'=c_1e^x+c_2-\frac{2}{x^2}$ and $y''=c_1e^x+\frac{4}{x^3}$, so
$(1-x)y''+xy'- y=c_1(1-x+x-1)+c_2(x-x)+\frac{4(1-x)}{x^3}-\frac{2}{
x}-\frac{2}{x}= \frac{4(1-x-x^2)}{x^3}$
\end{expandable}

\item %(h)
$y=x^{-1/2}(c_1\sin x+c_2 \cos x)+4x+8$;  
$x^2y''+xy'+\frac{\left(x^2-{1}{4}\right)}y=4x^3+8x^2+3x-2$

Click below to see the answer.

\begin{expandable}
    If $y=\frac{c_1\sin x+c_2 \cos x}{x^{1/2}}+4x+8$
then
 $y'=\frac{c_1\cos x-c_2 \sin x}{x^{1/2}}
-\frac{c_1\sin x+c_2 \cos x}{2x^{3/2}}+4$ and
$y''=-\frac{c_1\sin x+c_2 \cos x}{x^{1/2}}
-\frac{c_1\sin x-c_2 \cos x}{x^{3/2}}+\frac{3}{4}
\frac{c_1\sin x+c_2 \cos x}{x^{5/2}}$, so
$ x^2y''+xy'+\left(x^2-\frac{1}{4}\right)y=
c_1\left(-x^{-3/2}\sin x-x^{1/2}\cos x+
\frac{3}{4}x^{-1/2}\sin x +x^{1/2}\cos x-
\frac{1}{2}x^{-1/2}\sin x+x^{3/2}\sin
x-\frac{1}{4}x^{-1/2}
\sin x\right)+
 c_2\left(-x^{-3/2}\cos x+x^{1/2}\sin x
+\frac{3}{4}x^{-1/2}\cos x-x^{1/2}\sin x-\frac{1}{2}x^{-1/2}\cos x+
 x^{3/2}\cos x-\frac{1}{4}x^{-1/2}\cos x\right)+4x+\left(x^2-\frac{1}{4}\right)
(4x+8)=4x^3+8x^2+3x-2$.
\end{expandable}
\end{enumerate}
\end{problem}

\begin{problem}\label{exer:1.2.3}
Find all solutions of the  equation.

\begin{enumerate}
\item $y'=-x$
\item $y'=-x \sin x$
\item $y''=x \cos x$
\item $y''=2xe^x$
\item $y''=2x+\sin x+e^x$
\item $y'''=-\cos x$
\item $y'''=-x^2+e^x$
\item $y'''=7e^{4x}$
\end{enumerate}
\end{problem}

\begin{problem}\label{exer:1.2.4}
Solve the  initial value problem.

\begin{enumerate}
\item % (a)
 $y'=-xe^x, \quad y(0)=1$

Click below to see the answer.

\begin{expandable}
    If $y'=-xe^x$, then
$y=-xe^x+\int e^x\,dx+c=(1-x)e^x+c$,
and $y(0)=1\Rightarrow 1=1+c$, so $c=0$ and $y=(1-x)e^x$.
\end{expandable}
 
\item % (b)
 $y'=x \sin x^2, \quad y\left(\sqrt{\frac{\pi}{2}}\right)=1$

 Click below to see the answer.

 \begin{expandable}
     If
$y'=x\sin x^2$, then
$y=-\frac{1}{2}\cos x^2+c$;
$y\left({\sqrt\frac{\pi}{2}}\right)=1 \Rightarrow 1=0+c$,
so $c=1$ and $y=1-\frac{1}{2}\cos x^2$.
 \end{expandable}

\item %(c)
$y'=\tan x, \quad y(\pi/4)=3$

Click below to see the answer.

\begin{expandable}
    Write $y'=\tan x=\frac{\sin x}{\cos x}=-\frac{1}{\cos
x}\frac{d}{dx}(\cos x)$. Integrating this yields $y=-\ln|\cos x|+c$;
 $y(\pi/4)=3\Rightarrow 3=-\ln\left(\cos(\pi/4)\right)+c$, or
$3=\ln\sqrt2+c$, so $c=3-\ln\sqrt2$, so
$y=-\ln(|\cos x|)+3-\ln\sqrt2=3-\ln(\sqrt2|\cos x|)$.
\end{expandable}

\item %(d)
$y''=x^4, \quad y(2)=-1, \quad y'(2)=-1$

Click below to see the answer.

\begin{expandable}
    If $y''=x^4$, then $y'=\frac{x^5}{5}+c_1$; $y'(2)=-1
\Rightarrow\frac{32}{5}+c_1=-1\Rightarrow c_1=-\frac{37}{15}$,
so $y'=\frac{x^5}{5}-\frac{37}{15}$. Therefore, $y=\frac{x^6}{30}
-\frac{37}{15}(x-2)+c_2$; $y(2)=-1\Rightarrow\frac{64}{30}+c_2=-1
\Rightarrow c_2=-\frac{47}{15}$, so
$y=-\frac{47}{15}-\frac{37}{5}(x-2)+\frac{x^6}{30}$.
\end{expandable}

\item %(e)
$y''=xe^{2x}, \quad y(0)=7, \quad y'(0)=1$

Click below to see the answer.

\begin{expandable}
    $\int xe^{2x}\,dx=\frac{xe^{2x}}{2}-\frac{1}{2}
\int e^{2x}\,dx=\frac{xe^{2x}}{2}-\frac{e^{2x}}{4}$. Therefore,
$y'=\frac{xe^{2x}}{2}-\frac{e^{2x}}{4}+c_1$; $y'(0)=1\Rightarrow
-\frac{1}{4}+c_1=\frac{5}{4}\Rightarrow c_1=\frac{5}{4}$,
so $y'=\frac{xe^{2x}}{2}-\frac{e^{2x}}{4}+\frac{5}{4}$;

$y=\frac{xe^{2x}}{4}-\frac{e^{2x}}{8}-\frac{e^{2x}}{8}+\frac{5}{4}x+c_2
=\frac{xe^{2x}}{4}-\frac{e^{2x}}{4}+\frac{5}{4}x+c_2$; $y(0)=7
\Rightarrow-\frac{1}{4}+c_2=7\Rightarrow c_2=\frac{29}{4}$, so
$y=\frac{xe^{2x}}{4}-\frac{e^{2x}}{4}+\frac{5}{4}x+\frac{29}{4}$.
\end{expandable}

\item %(f)
$y''=- x \sin x, \quad y(0)=1, \quad y'(0)=-3$

Click below to see the answer.

\begin{expandable}
   (A) $\int x\sin x\,dx=-x\cos x+\int \cos x\,dx=-x\cos x+\sin
x$ and (B) $\int x\cos x\,dx=x\sin x-\int\sin x\,dx=x\sin x+\cos x$.
If $y''=-x\sin x$, then (A) implies that $y'=x\cos x-\sin x+c_1$;
$y'(0)=-3\Rightarrow c=-3$, so $y'=x\cos x-\sin x-3$. Now (B) implies
that $y=x\sin x+\cos x+\cos x-3x+c_2=x\sin x+2\cos x-3x+c_2$;
$y(0)=1\Rightarrow 2+c_2=1\Rightarrow c_2=-1$, so $y=x \sin x+2 \cos x-3x-1$.
\end{expandable}

\item %(g)
$y'''=x^2e^x, \quad y(0)=1, \quad y'(0)=-2, \quad
y''(0)=3$

Click below to see the answer.

\begin{expandable}
    If $y'''=x^2e^x$, then $y''=\int x^2e^x\,dx=x^2e^x-2\int
xe^x\,dx=x^2e^x-2xe^x+2e^x+c_1$; $y''(0)=3\Rightarrow
2+c_1=3\Rightarrow c_1=1$, so (A) $y''=(x^2-2x+2)e^x+1$. Since $\int
(x^2-2x+2)e^x\,dx=(x^2-2x+2)e^x-\int (2x-2)e^x\,dx
=(x^2-2x+2)e^x-(2x-2)e^x+2e^x=(x^2-4x+6)e^x$, (A) implies that
$y'=(x^2-4x+6)e^x+x+c_2$; $y'(0)=-2\Rightarrow 6+c_2=-2\Rightarrow
c_2=-8$, so (B) $y'=(x^2-4x+6)e^x+x-8$; Since $\int
(x^2-4x+6)e^x\,dx=(x^2-4x+6)e^x-\int (2x-4)e^x\,dx
=(x^2-4x+6)e^x-(2x-4)e^x+2e^x=(x^2-6x+12)e^x$, (B) implies that
$y=(x^2-6x+12)e^x+\frac{x^2}{2}-8x+c_3$; $y(0)=1\Rightarrow
12+c_3=1\Rightarrow c_3=-11$, so
$y=(x^2-6x+12)e^x+\frac{x^2}{2}-8x-11$.
\end{expandable}

\item %(h)
$y'''=2+\sin 2x, \quad y(0)=1, \quad y'(0)=-6, \quad
y''(0)=3$

Click below to see the answer.

\begin{expandable}
    If $y'''=2+\sin2x$, then $y''=2x-\frac{\cos 2x}{2}+c_1$;
$y''(0)=3\Rightarrow -\frac{1}{2}+c_1=3\Rightarrow c_1=\frac{7}{2}$, so $y''=2x-\frac{\cos 2x}{2}+\frac{7}{2}$. Then
$y'=x^2-\frac{\sin 2x}{4}+\frac{7}{2}x+c_2$; $y'(0)=-6\Rightarrow
c_2=-6$, so $y'=x^2-\frac{\sin 2x}{4}+\frac{7}{2}x-6$. Then
$y=\frac{x^3}{3}+\frac{\cos 2x}{8}+\frac{7}{4}x^2-6x+c_3$;
$y(0)=1\Rightarrow \frac{1}{8}+c_3=1\Rightarrow c_3=\frac{7}{8}$, so $y=\frac{x^3}{3}+\frac{\cos 2x}{8}+
\frac{7}{4}x^2-6x+\frac{7}{8}$.
\end{expandable}

\item %(i)
$y'''=2x+1, \quad y(2)=1, \quad
y'(2)=-4, \quad y''(2)=7$

Click below to see the answer.

\begin{expandable}
    If $y'''=2x+1$, then $y''=x^2+x+c_1$; $y''(2)=7\Rightarrow
6+c_1=7\Rightarrow c_1=1$; so $y''=x^2+x+1$. Then $y'=\frac{x^3}{3}+\frac{x^2}{2}+(x-2)+c_2$; $y'(2)=-4\Rightarrow \frac{14}{3}+c_2=-4\Rightarrow c_2=-\frac{26}{3}$, so $y'=\frac{x^3}{3}+\frac{x^2}{2}+(x-2)-\frac{26}{3}$. Then $y=\frac{x^4}{12}+\frac{x^3}{6}+\frac{1}{2}(x-2)^2-\frac{26}{3}(x-2)+c_3$;
$y(2)=1\Rightarrow \frac{8}{3}+c_3=1\Rightarrow c_3=-\frac{5}{3}$, so $y=\frac{x^4}{12}+\frac{x^3}{6}+\frac{1}{2} (x-2)^2-\frac{26}{3}
(x-2)-\frac{5}{3}$.
\end{expandable}
\end{enumerate}
\end{problem}

\begin{problem}\label{exer:1.2.5}
Verify that the  function is a solution of
the initial value problem.

\begin{enumerate}
\item %(a)
$y=x\cos x;     \quad y'=\cos x-y\tan x, \quad
y(\pi/4)=\frac{\pi}{4\sqrt{2}}$

\item %(b)
$y=\frac{1+2\ln x}{ x^2}+\frac{1}{2};     \quad
y'=\frac{x^2-2x^2y+2}{ x^3},   \quad y(1)=\frac{3}{2}$

\item %(c)
$y=\tan\left(\frac{x^2}{2}\right);
\quad y'=x(1+y^2),   \quad y(0)=0$

\item %(d)
$y=\frac{2}{ x-2};     \quad y'=\frac{-y(y+1)}{ x}, \quad
y(1)=-2$
\end{enumerate}
\end{problem}

\begin{problem}\label{exer:1.2.6}
Verify that the  function is a solution of
the initial value problem.

\begin{enumerate}
\item %(a)
$y=x^2(1+\ln x);     \quad y''=\frac{3xy'-4y}{ x^2}, \quad
y(e)=2e^2, \quad y'(e)=5e$

Click below to see the answer.

\begin{expandable}
    If $y=x^2(1+\ln x)$, then $y(e)=e^2(1+\ln e)=2e^2$;
$y'=2x(1+\ln x)+x=3x+2x\ln x$, so $y'(e)=3e+2e\ln e=5e$; (A)
$y''=3+2+2\ln x=5+2\ln x$. Now, $3xy'-4y=3x(3x+2x\ln x)-4x^2(1+\ln
x)=5x^2+2x^2\ln x=x^2y''$, from (A).
\end{expandable}

\item %(b)
$y=\frac{x^2}{3}+x-1;     \quad y''=\frac{x^2-xy'+y+1}{
x^2}, \quad y(1)=\frac{1}{3}, \quad y'(1)=\frac{5}{3}$

Click below to see the answer.

\begin{expandable}
    If $y=\frac{x^2}{3}+x-1$, then $y(1)=\frac{1}{3}+1-1=\frac{1}{3}$; $y'=\frac{2}{3}x+1$, so $y'(1)=\frac{2}{3}+1=\frac{5}{3}$; (A) $y''=\frac{2}{3}$. Now
$x^2-xy'+y+1=x^2- x\left(\frac{2}{3}x+1\right)+\frac{x^2}{3}+x-1+1=\frac{2}{3}x^2=x^2y''$, from (A).
\end{expandable}

\item %(c)
$y=(1+x^2)^{-1/2};     \quad
y''=\frac{(x^2-1)y-x(x^2+1)y'}{ (x^2+1)^2}, \quad y(0)=1,$
$y'(0)=0$

Click below to see the answer.

\begin{expandable}
    If $y=(1+x^2)^{-1/2}$, then $y(0)=(1+0^2)^{-1/2}=1$;
$y'=-x(1+x^2)^{-3/2}$, so $y'(0)=0$; (A) $y''=(2x^2-1)(1+x^2)^{-5/2}$.
Now,
$(x^2-1)y-x(x^2+1)y'=(x^2-1)(1+x^2)^{-1/2}-x(x^2+1)(-x)(1+x^2)^{-3/2}
=(2x^2-1)(1+x^2)^{-1/2}=y''(1+x^2)^2$ from (A), so
$y''=\frac{(x^2-1)y-x(x^2+1)y'}{(x^2+1)^2}$.
\end{expandable}

\item %(d)
$y=\frac{x^2}{ 1-x};    \quad y''=\frac{2(x+y)(xy'-y)}{ x^3},
\quad y(1/2)=1/2, \quad y'(1/2)=3$

Click below to see the answer.

\begin{expandable}
    If $y=\frac{x^2}{1-x}$, then $y(1/2)=\frac{1/4}{1-1/2}=\frac{1}{2}$; $y'=-\frac{x(x-2)}{(1-x)^2}$, so
$y'(1/2)=\frac{(-1/2)(-3/2)}{(1-1/2)^2}=3$; (A) $y''=\frac{2}{(1-x)^3}$. Now, (B) $x+y=x+\frac{x^2}{1-x}=\frac{x}{1-x}$ and (C)
$xy'-y=-\frac{x^2(x-2)}{(1-x)^2}-\frac{x^2}{1-x}=\frac{x^2}{(1-x)^2}$.
From (B) and (C), $(x+y)(xy'-y)=\frac{x^3}{(1-x)^3}=\frac{x^3}{2}y''$,
so $y''=\frac{2(x+y)(xy'-y)}{x^3}$.
\end{expandable}

\end{enumerate}

\end{problem}

\begin{problem}\label{exer:1.2.7}
Suppose an object is launched from a point 320 feet
above the earth with an initial velocity of 128 ft/sec
upward, and  the only force acting on it thereafter is
gravity. Take $g=32$ ft/sec$^2$.

\begin{enumerate}
\item %(a)
Find the highest altitude attained by the object.

\item %(b)
Determine how long it takes for the object to fall to the
ground.
\end{enumerate}
\end{problem}

\begin{problem}\label{exer:1.2.8}

Let $a$ be a nonzero real number.
\begin{enumerate}
\item % (a)
Verify that if $c$ is an arbitrary constant then
\begin{equation}\label{eq:eqA1.2.8}
y=(x-c)^a 
\end{equation}
is a solution of
\begin{equation}\label{eq:eqB1.2.8}
y'=ay^{(a-1)/a}
\end{equation}
on $(c,\infty)$.

Click below to see the answer.

\begin{expandable}
    $y=(x-c)^a$ is defined and  $x-c=y^{1/a}$ on $(c,\infty)$;
moreover, $y'=a(x-c)^{a-1}=a\left(y^{1/a}\right)^{a-1}=ay^{(a-1)/a}$.
\end{expandable}

\item % (b)
Suppose $a<0$ or $a>1$. Can you think of a solution of
(\ref{eq:eqB1.2.8}) that isn't of the form (\ref{eq:eqA1.2.8})?

Click below to see the answer.

\begin{expandable}
    If $a>1$ or $a<0$, then $y\equiv0$  is a solution of
(B) on $(-\infty,\infty)$.
\end{expandable}
\end{enumerate}
\end{problem}

\begin{problem}\label{exer:1.2.9}
Verify that
$$
y=
\left\{ \begin{array}{cl}
e^x-1,& x \ge 0, \\
1-e^{-x},& x < 0, \end{array}\right.
$$
is a solution of
$$
y'=|y|+1
$$
on $(-\infty,\infty)$. 
\begin{hint}Use the definition of derivative
at $x=0$.
\end{hint}
\end{problem}

\begin{problem}\label{exer:1.2.10}
\begin{enumerate}
\item %(a)
Verify that if $c$ is any real number then
\begin{equation}\label{eq:eqA1.2.10}
y=c^2+cx+2c+1
\end{equation}
satisfies
\begin{equation}\label{eq:eqB1.2.10}
 y'=\frac{-(x+2)+\sqrt{x^2+4x+4y}}{2}
\end{equation}
on some open interval. Identify the open interval.

Click below to see the answer.

\begin{expandable}
    Since $y'=c$ we must show that the right side of
(B)  reduces to $c$ for all values of $x$
in some interval. If $y=c^2+cx+2c+1$,
\begin{eqnarray*}
x^2+4x+4y&=&x^2+4x+4c^2+4cx+8c+4\\
&=&x^2+4(1+c)x+4(c^2+2c+1)\\
&=&x^2+4(1+c)+2(c+1)^2=(x+2c+2)^2.
\end{eqnarray*}

Therefore, $\sqrt{x^2+4x+4y}=x+2c+2$ and the right side of
(B) reduces to $c$ if
$x>-2c-2$.
\end{expandable}

\item %(b)
Verify that
$$
y_1=\frac{-x(x+4)}{4}
$$
also satisfies  (\ref{eq:eqB1.2.10}) on some open interval, and
identify the open interval. (Note that $y_1$ can't be obtained
by selecting a value of $c$ in (\ref{eq:eqA1.2.10}).)

Click below to see the answer.

\begin{expandable}
    If $y_1=-\frac{x(x+4)}{4}$, then $y_1'=-\frac{x+2}{2}$
and $x^2+4x+4y=0$ for all $x$. Therefore, $y_1$ satisfies
(A) on $(-\infty,\infty)$.
\end{expandable}
\end{enumerate}
\end{problem}

\end{document}