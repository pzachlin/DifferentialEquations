\documentclass{ximera}
%% You can put user macros here
%% However, you cannot make new environments

%\listfiles

% Get the 'old' hints/expandables, for use on ximera.osu.edu
%\def\xmNotHintAsExpandable{true}
%\def\xmNotExpandableAsAccordion{true}



%\graphicspath{{./}{firstExample/}{secondExample/}}
\graphicspath{{./}
{aboutDiffEq/}
{applicationsLeadingToDiffEq/}
{applicationsToCurves/}
{autonomousSecondOrder/}
{basicConcepts/}
{bernoulli/}
{constCoeffHomSysI/}
{constCoeffHomSysII/}
{constCoeffHomSysIII/}
{constantCoeffWithImpulses/}
{constantCoefficientHomogeneousEquations/}
{convolution/}
{coolingActivity/}
{directionFields/}
{drainingTank/}
{epidemicActivity/}
{eulersMethod/}
{exactEquations/}
{existUniqueNonlinear/}
{frobeniusI/}
{frobeniusII/}
{frobeniusIII/}
{global.css/}
{growthDecay/}
{heatingCoolingActivity/}
{higherOrderConstCoeff/}
{homogeneousLinearEquations/}
{homogeneousLinearSys/}
{improvedEuler/}
{integratingFactors/}
{interactExperiment/}
{introToLaplace/}
{introToSystems/}
{inverseLaplace/}
{ivpLaplace/}
{laplaceTable/}
{lawOfCooling/}
{linSysOfDiffEqs/}
{linearFirstOrderDiffEq/}
{linearHigherOrder/}
{mixingProblems/}
{motionUnderCentralForce/}
{nonHomogeneousLinear/}
{nonlinearToSeparable/}
{odesInSage/}
{piecewiseContForcingFn/}
{population/}
{reductionOfOrder/}
{regularSingularPts/}
{reviewOfPowerSeries/}
{rlcCircuit/}
{rungeKutta/}
{secondLawOfMotion/}
{separableEquations/}
{seriesSolNearOrdinaryPtI/}
{seriesSolNearOrdinaryPtII/}
{simplePendulum/}
{springActivity/}
{springProblemsI/}
{springProblemsII/}
{undCoeffHigherOrderEqs/}
{undeterminedCoeff/}
{undeterminedCoeff2/}
{unitStepFunction/}
{varParHigherOrder/}
{varParamNonHomLinSys/}
{variationOfParameters/}
}


\usepackage{tikz}
%\usepackage{tkz-euclide}
\usepackage{tikz-3dplot}
\usepackage{tikz-cd}
\usetikzlibrary{shapes.geometric}
\usetikzlibrary{arrows}
\usetikzlibrary{decorations.pathmorphing,patterns}
\usetikzlibrary{backgrounds} % added by Felipe
% \usetkzobj{all}   % NOT ALLOWED IN RECENT TeX's ...
\pgfplotsset{compat=1.13} % prevents compile error.

\pdfOnly{\renewcommand{\theHsection}{\thepart.section.\thesection}}  %% MAKES LINKS WORK should be added to CLS
\pdfOnly{\renewcommand{\part}[1]{\chapterstyle\title{#1}\begin{abstract}\end{abstract}\maketitle\def\thechaptertitle{#1}}}


\renewcommand{\vec}[1]{\mathbf{#1}}
\newcommand{\RR}{\mathbb{R}}
\providecommand{\dfn}{\textit}
\renewcommand{\dfn}{\textit}
\newcommand{\dotp}{\cdot}
\newcommand{\id}{\text{id}}
\newcommand\norm[1]{\left\lVert#1\right\rVert}
\newcommand{\dst}{\displaystyle}
 
\newtheorem{general}{Generalization}
\newtheorem{initprob}{Exploration Problem}

\tikzstyle geometryDiagrams=[ultra thick,color=blue!50!black]

\usepackage{mathtools}

\title{Exercises} \license{CC BY-NC-SA 4.0}

\begin{document}

\begin{abstract}
\end{abstract}
\maketitle

\begin{onlineOnly}
\section*{Exercises}
\end{onlineOnly}


In  Exercises \ref{exer:4.5.1}--\ref{exer:4.5.8} find a first order
differential equation for the given family of curves.

\begin{problem}\label{exer:4.5.1} 
Find a first order
differential equation for the family of curves $y(x^2+y^2)=c$
\end{problem}

\begin{problem}\label{exer:4.5.2} Find a first order
differential equation for the family of curves  $e^{xy}=cy$

\begin{solution}
Differentiating (A) $e^{xy}=cy$ yields (B) $(xy'+y)e^{xy}=cy'$. From
(A), $c=\frac{e^{xy}}{y}$. Substituting this into (B) and
cancelling $e^{xy}$ yields $xy'+y=\frac{y'}{y}$, so
$y'=-\frac{y^2}{(xy-1)}$.
\end{solution}
\end{problem}

 \begin{problem}\label{exer:4.5.3} Find a first order
differential equation for the family of curves  $\ln |xy|=c(x^2+y^2)$
\end{problem}

 \begin{problem}\label{exer:4.5.4} Find a first order
differential equation for the family of curves 
 $y=x^{1/2}+cx$

\begin{solution}
Solving $y=x^{1/2}+cx$ for $c$ yields $c=\frac{y}{x}-x^{-1/2}$, and
differentiating yields $0=\frac{y'}{x}-\frac{y}{x^2}+\frac{x^{-3/2}}{2}$, or $xy'-y=-\frac{x^{1/2}}{2}$.
\end{solution}
\end{problem}

\begin{problem}\label{exer:4.5.5} Find a first order
differential equation for the family of curves 
 $y=e^{x^2}+ce^{-x^2}$
 \end{problem}
 
 \begin{problem}\label{exer:4.5.6} Find a first order
differential equation for the family of curves 
 $y=x^3+\frac{c}{x}$

\begin{solution}
Rewriting $y=x^3+\frac{c}{x}$ as $xy=x^4+c$ and differentiating
yields $xy'+y=4x^3$.
\end{solution}
\end{problem}

\begin{problem}\label{exer:4.5.7} Find a first order
differential equation for the family of curves 
 $y=\sin x+ce^x$
\end{problem}

\begin{problem}\label{exer:4.5.8} Find a first order
differential equation for the family of curves 
 $y=e^x+c(1+x^2)$

\begin{solution}
Rewriting $y=e^x+c(1+x^2)$ as
$\frac{y}{1+x^2}=\frac{e^x}{1+x^2}+c$ and differentiating yields
$\frac{y'}{1+x^2}-\frac{2xy}{(1+x^2)^2}=
\frac{e^x}{1+x^2}-\frac{2xe^x}{(1+x^2)^2}$, so
$(1+x^2)y'-2xy=(1-x)^2e^x$.
\end{solution}
\end{problem}

\begin{problem}\label{exer:4.5.9}
Show that the family of circles
$$
(x-x_0)^2+y^2=1,\;-\infty<x_0<\infty,
$$
can be obtained by joining integral curves of two first order
differential equations. More specifically, find differential equations
for the families of semicircles
$$
(x-x_0)^2+y^2=1,\; x_0<x<x_0+1,\;-\infty<x_0<\infty,
$$
$$
(x-x_0)^2+y^2=1,\; x_0-1<x<x_0,\;-\infty<x_0<\infty.
$$
\end{problem}

\begin{problem}\label{exer:4.5.10}
Suppose $f$ and $g$ are differentiable for all $x$. Find a
differential equation for the family of functions $y=f+cg$
($c$=constant).

\begin{solution}
If (A) $y=f+cg$, then (B) $y'=f'+cg'$. Mutiplying (A) by $g'$
and (B) by $g$ yields
(C) $yg'=fg'+cgg'$ and  (D) $y'g=f'g+cg'g$, and subtracting (C)
from (D) yields
$y'g-yg'=f'g-fg'$.
\end{solution}
\end{problem}

\begin{problem}\label{exer:4.5.11}
Find a
first order differential equation for the family of lines through a given point $(x_0,y_0)$.
\end{problem}

\begin{problem}\label{exer:4.5.12}
Find a
first order differential equation for the family of circles through $(-1,0)$ and $(1,0)$.

\begin{solution}
Let $(x_0,y_0)$ be the center
and $r$ be the radius of a circle in the family. Since $(-1,0)$ and
$(1,0)$ are on the circle, $(x_0+1)^2+y_0^2=(x_0-1)^2+y_0^2$, which
implies that $x_0=0$. Therefore,the equation of the circle is (A)
$x^2+(y-y_0)^2=r^2$. Since $(1,0)$ is on the circle, $r^2=1+y_0^2$.
Substituting this into (A) shows that the equation of the circle is
$x^2+y^2-2yy_0=1$, so $2y_0=\frac{x^2+y^2-1}{y}$. Differentiating
$y(2x+2yy')-y'(x^2+y^2-1)=0$, so $y'(y^2-x^2+1)+2xy=0$.
\end{solution}
\end{problem}

\begin{problem}\label{exer:4.5.13}
Find a
first order differential equation for the family of circles through $(0,0)$ and $(0,2)$.
\end{problem}

\begin{problem}\label{exer:4.5.14}
Use the method Example~\ref{example:4.5.6}{\bf (a)} to find the equations
of lines through the given points tangent to the parabola $y=x^2$.
Also, find the points of tangency.
\begin{enumerate}
\item $(5,9)$

\begin{solution}
From Example~4.5.6 the equation of the line tangent to the
parabola at $(x_0,x_0^2)$ is (A) $y=-x_0^2+2x_0x$.  From (A), $(x,y)=(5,9)$ is on the tangent line through
$(x_0,x_0^2)$ if and only if $9=-x_0^2+10x_0$, or
$x_0^2-10x_0+9=(x_0-1)(x_0-9)=0$. Letting $x_0=1$ in (A) yields the
line $y=-1+2x$, tangent to the parabola at $(x_0,x_0^2)=(1,1)$.
Letting $x_0=9$ in (A) yields the line $y=-81+18x$, tangent to the
parabola at $(x_0,x_0^2)=(9,81)$.
\end{solution}
    
\item $(6,11)$

\begin{solution}
From Example~4.5.6 the equation of the line tangent to the
parabola at $(x_0,x_0^2)$ is (A) $y=-x_0^2+2x_0x$.  From (A), $(x,y)=(6,11)$ is on the tangent line through
$(x_0,x_0^2)$ if and only if $11=-x_0^2+12x_0$, or
$x_0^2-12x_0+11=(x_0-1)(x_0-11)=0$. Letting $x_0=1$ in (A) yields the
line $y=-1+2x$, tangent to the parabola at $(x_0,x_0^2)=(1,1)$.
Letting $x_0=11$ in (A) yields the line $y=-121+22x$, tangent to the
parabola at $(x_0,x_0^2)=(11,21)$.
\end{solution}

\item $(-6,20)$

\begin{solution}
From Example~4.5.6 the equation of the line tangent to the
parabola at $(x_0,x_0^2)$ is (A) $y=-x_0^2+2x_0x$.  From (A), $(x,y)=(-6,20)$ is on the tangent line through
$(x_0,x_0^2)$ if and only if $20=-x_0^2-12x_0$, or
$x_0^2+12x_0+20=(x_0+2)(x_0+10)=0$. Letting $x_0=-2$ in (A) yields the
line $y=-4-4x$, tangent to the parabola at $(x_0,x_0^2)=(-2,4)$.
Letting $x_0=-10$ in (A) yields the line $y=-100-20x$, tangent to the
parabola at $(x_0,x_0^2)=(-10,100)$.
\end{solution}
    
\item $(-3,5)$

\begin{solution}
From Example~4.5.6 the equation of the line tangent to the
parabola at $(x_0,x_0^2)$ is (A) $y=-x_0^2+2x_0x$.  From (A), $(x,y)=(-3,5)$ is on the tangent line through
$(x_0,x_0^2)$ if and only if $5=-x_0^2-6x_0$, or
$x_0^2+6x_0+5=(x_0+1)(x_0+5)=0$. Letting $x_0=-1$ in (A) yields the
line $y=-1-2x$, tangent to the parabola at $(x_0,x_0^2)=(-1,1)$.
Letting $x_0=-5$ in (A) yields the line $y=-25-10x$, tangent to the
parabola at $(x_0,x_0^2)=(-5,25)$.
\end{solution}
\end{enumerate}
\end{problem}

\begin{problem}\label{exer:4.5.15}
\begin{enumerate}
\item % (a)
Show that the equation of the line tangent to the circle
\begin{equation}\label{eq:eqA4.5.15}
x^2+y^2=1
\end{equation}
at a point $(x_0,y_0)$  on the circle is
\begin{equation}\label{eq:eqB4.5.15}
y=\frac{1-x_0x}{ y_0}
\end{equation}
if 
\begin{equation}
  x_0\ne\pm1.  
\end{equation}

\begin{solution}
If $(x_0,y_0)$ is any point on the circle such that
$x_0\ne\pm1$ (and therefore $y_0\ne0$), then differentiating \ref{eq:eqA4.5.15}
yields $2x_0+2y_0y_0'=0$, so $y_0'=-\frac{x_0}{y_0}$. Therefore,the
equation of the tangent line is $y=y_0-\frac{x_0}{y_0}(x-x_0)$.
Since $x_0^2+y_0^2=1$, this is equivalent to \ref{eq:eqB4.5.15}.
\end{solution}
 
\item % (b)
Show that if $y'$ is the slope of a nonvertical tangent line to the
circle \ref{eq:eqA4.5.15} and $(x,y)$ is a point on the tangent line, then
\begin{equation}\label{eq:eqC4.5.15}
(y')^2(x^2-1)-2xyy'+y^2-1=0.
\end{equation}

\begin{solution}
Since $y'=-\frac{x_0}{y_0}$ on the tangent line, we can rewrite \ref{eq:eqB4.5.15}
as $y-xy'=\frac{1}{y_0}$. Hence (F) $\frac{1}{(y-xy')^2}=y_0^2$
and (G) $x_0^2=1-y_0^2=\frac{(y-xy')^2-1}{(y-xy')^2}$. Since
$(y')^2=\frac{x_0^2}{y_0^2}$, (F) and (G) imply that
$(y')^2=(y-xy')^2-1$, which implies \ref{eq:eqC4.5.15}.
\end{solution}

\item % (c)
Show that the segment of the tangent line \ref{eq:eqB4.5.15} on which $(x-x_0)/y_0>0$
is an integral curve of the differential equation
\begin{equation}\label{eq:eqD4.5.15}
y'=\frac{xy-\sqrt{x^2+y^2-1}}{x^2-1},
\end{equation}
while the segment on which $(x-x_0)/y_0<0$ is an integral curve of the
differential equation
\begin{equation}\label{eq:eqE4.5.15}
y'=\frac{xy+\sqrt{x^2+y^2-1}}{x^2-1}.
\end{equation}
\begin{hint}
Use the quadratic formula to solve \ref{eq:eqC4.5.15} for
$y'$. Then substitute \ref{eq:eqB4.5.15} for $y$ and choose the $\pm$ sign in the quadratic formula so that the
resulting expression for $y'$ reduces to the known slope
$y'=-x_0/y_0$.
\end{hint}

\begin{solution}
Using the quadratic formula to solve \ref{eq:eqC4.5.15} for $y'$
yields
\begin{equation}\label{eq:eqH4.5.16}
y'=\frac{xy\pm\sqrt{x^2+y^2-1}}{x^2-1}
\end{equation}
if $(x,y)$ is on a tangent  line with slope $y'$.
If $y=\frac{1-x_0x}{y_0}$, then
$x^2+y^2-1=x^2+\left(\frac{1-x_0x}{y_0}\right)^2-1=\left(\frac{x-x_0}{y_0}\right)^2$
(since $x_0^2+y_0^2=1$). Since $y'=-\frac{x_0}{y_0}$, this implies
that \ref{eq:eqH4.5.15} is equivalent to
$-\frac{x_0}{y_0}=\frac{1}{x^2-1}\left[\frac{x(1-x_0x)}{y_0}
\pm\left|\frac{x-x_0}{y_0}\right|\right]$,
which holds if and only if we choose the ``$\pm$" so that
 $\pm\left|\frac{x-x_0}{y_0}\right|=-\left(\frac{x-x_0}{y_0}\right)$.  Therefore,we must choose $\pm=-$ if $\frac{x-x_0}{y_0}>0$, so \ref{eq:eqH4.5.15} reduces to \ref{eq:eqD4.5.15},
or  $\pm=+$ if $\frac{x-x_0}{y_0}<0$, so \ref{eq:eqH4.5.15} reduces to \ref{eq:eqE4.5.15}.
\end{solution}

\item % (d)
Show that the upper and lower semicircles of \ref{eq:eqA4.5.15} are also integral
curves of \ref{eq:eqD4.5.15} and \ref{eq:eqE4.5.15}.

\begin{solution}
Differentiating \ref{eq:eqA4.5.15}
yields $2x+2yy'=0$, so $y'=-\frac{x}{y}$ on either semicircle.
Since \ref{eq:eqD4.5.15} and \ref{eq:eqE4.5.15} both reduce to $y'=\frac{xy}{1-x^2}=-\frac{x}{y
}$ (since $x^2+y^2=1$) on both semicircles, the conclusion follows.
\end{solution}

\item % (e)
Find the equations of two lines through (5,5) tangent to the circle
\ref{eq:eqA4.5.15}, and find the points of tangency.

\begin{solution}
From \ref{eq:eqD4.5.15} and \ref{eq:eqE4.5.15} the slopes of tangent lines from (5,5)
tangent to the circle are
$y'=\frac{25\pm\sqrt49}{24}=\frac{3}{4},\frac{4}{3}$. Therefore,
tangent lines are $y=5+\frac{3}{4}(x-5)=\frac{1+3x/5}{4/5}$ and
$y=5+\frac{4}{3}(x-5)=\frac{1-4x/5}{-3/5}$, which intersect the
circle at $(-3/5,4/5)$ $(4/5,-3/5)$, respectively. (See \ref{eq:eqB4.5.15} ).
\end{solution}
\end{enumerate}
\end{problem}

\begin{problem}\label{exer:4.5.16}
\begin{enumerate}
\item % (a)
Show that the equation of the line tangent to  \begin{equation}\label{eq:eqA4.5.16}
x=y^2
\end{equation}
at a point $(x_0,y_0)\ne(0,0)$  on the parabola is
\begin{equation}\label{eq:eqB4.5.16}
y=\frac{y_0}{2}+\frac{x}{2y_0}.
\end{equation}

\begin{solution}
If $(x_0,y_0)$ is any point on the parabola such that
$x_0>0$ (and therefore $y_0\ne0$), then differentiating \ref{eq:eqA4.5.16} yields
$1=2y_0y_0'$, so $y_0'=\frac{1}{2y_0}$. Therefore,the equation of
the tangent line is $y=y_0+\frac{1}{2y_0}(x-x_0)$. Since
$x_0=y_0^2$, this is equivalent to \ref{eq:eqB4.5.16}.
\end{solution}


\item % (b)
Show that if $y'$ is the slope of a nonvertical tangent line
to the parabola \ref{eq:eqA4.5.16} and $(x,y)$ is a point on the tangent
line then
\begin{equation}\label{eq:eqC4.5.16}
4x^2(y')^2-4xyy'+x=0.
\end{equation}

\begin{solution}
Since $y'=\frac{1}{2y_0}$ on the tangent line, we can rewrite \ref{eq:eqB4.5.16} as
$\frac{y_0}{2}=y-xy'$. Substituting this into \ref{eq:eqB4.5.16} yields
$y=(y-xy')+\frac{x}{4(y-xy')}$, which implies~\ref{eq:eqC4.5.16}.
\end{solution}


\item % (c)
Show that the segment of the tangent line defined in the first part of this problem on which
$x>x_0$ is an integral curve of the differential equation
\begin{equation}\label{eq:eqD4.5.16}
y'={y+\sqrt{y^2-x}}{2x}.
\end{equation}
while the segment on which $x<x_0$ is an integral curve of the
differential equation
\begin{equation}\label{eq:eqE4.5.16}
y'={y-\sqrt{y^2-x}}{2x},
\end{equation}
\begin{hint}
    Use the quadratic formula to solve \ref{eq:eqC4.5.16} for $y'$. Then
substitute \ref{eq:eqB4.5.16} for $y$ and choose the $\pm$ sign in the
quadratic formula so that the resulting expression for $y'$ reduces to
the known slope $y'=\frac{1}{2y_0}$.
\end{hint}

\begin{solution}
Using the quadratic formula to solve \ref{eq:eqC4.5.16} for $y'$ yields
\begin{equation}\label{eq:eqF4.5.16}
y'=\frac{y\pm\sqrt{y^2-x}}{2x}
\end{equation}
if $(x,y)$ is on a tangent line with slope $y'$. If
$y=\frac{y_0}{2}+\frac{x}{2y_0}$, then
$y^2-x=\frac{1}{4}\left(y_0-\frac{x}{y_0}\right)^2$ so \ref{eq:eqF4.5.16} is
equivalent to $\frac{1}{2y_0}= \frac{y_0+\frac{x}{y_0}\pm\left|y_0-\frac{x}{y_0}\right|}{4x}$ which holds if and
only if we choose the ``$\pm$" so that $\pm\left|y_0-\frac{x}{y_0}
\right|=-\left(y_0-\frac{x}{y_0}\right)$. Therefore,we must
choose $\pm=+$ if $x>y_0^2=x_0$, so \ref{eq:eqF4.5.16} reduces to \ref{eq:eqD4.5.16}, or $\pm=-$ if
$x<y_0^2=x_0$, so \ref{eq:eqF4.5.16} reduces to \ref{eq:eqE4.5.16}.

\end{solution}

\item % (d)
Show that the upper and lower halves of the parabola \ref{eq:eqA4.5.16}, given by
$y=\sqrt x$ and $y=-\sqrt x$ for $x>0$, are also integral curves of
\ref{eq:eqD4.5.16} and \ref{eq:eqE4.5.16}.
\end{enumerate}

\begin{solution}
Differentiating \ref{eq:eqA4.5.16} yields $1=2yy'$, so $y'=\frac{1}{2y}$
on either half of the parabola. Since \ref{eq:eqD4.5.16} and \ref{eq:eqE4.5.16} both reduce to this
if $x=y^2$, the conclusion follows
\end{solution}
\end{problem}

\begin{problem}\label{exer:4.5.17}
Use the results of Exercise \ref{exer:4.5.16} to find the equations of two
lines tangent to the parabola $x=y^2$ and passing through the given
point. Also find the points of tangency.

\begin{enumerate}
    \item $(-5,2)$
    \item $(-4,0)$
    \item $(7,4)$
    \item $(5,-3)$
\end{enumerate}
\end{problem}

\begin{problem}\label{exer:4.5.18}
Find a curve $y=y(x)$ through (1,2) such that the tangent to the curve
at any point $(x_0,y(x_0))$ intersects the $x$ axis at
$x_I=\frac{x_0}{2}$.

\begin{solution}
The equation of the line tangent to the curve at $(x_0,y(x_0))$
is $y=y(x_0)+y'(x_0)(x-x_0)$. Since $y(x_0/2)=0$,
$y(x_0)-\frac{y'(x_0)x_0}{2}=0$. Since $x_0$ is arbitrary,
it follows that $y'=\frac{2y}{x}$, so
$\frac{y'}{y}=\frac{2}{x}$, $\ln|y|=2\ln|x|+k$, and
$y=cx^2$. Since $(1,2)$ is on the curve, $c=2$. Therefore,$y=2x^2$.
\end{solution}
\end{problem}

\begin{problem}\label{exer:4.5.19}
Find all curves $y=y(x)$ such that the tangent to the curve at any
point $(x_0,y(x_0))$ intersects the $x$ axis at $x_I=x^3_0$.
\end{problem}

\begin{problem}\label{exer:4.5.20}
Find all curves $y=y(x)$ such that the tangent to the curve at any
point passes through a given point $(x_1,y_1)$.

\begin{solution}
The equation of the line tangent to the curve at $(x_0,y(x_0))$ is
$y=y(x_0)+y'(x_0)(x-x_0)$. Since $(x_1,y_1)$ is on the line,
$y(x_0)+y'(x_0)(x_1-x_0)=y_1$. Since $x_0$ is arbitrary, it follows
that $y+y'(x_1-x)=y_1$, so $\frac{y'}{y-y_1}=\frac{1}{x-x_1}$,
$\ln|y-y_1|=\ln|x-x_1|+k$, and $y-y_1=c(x-x_1)$.
\end{solution}
\end{problem}

\begin{problem}\label{exer:4.5.21}
Find a curve $y=y(x)$ through $(1,-1)$ such that the tangent to the
curve at any point $(x_0,y(x_0))$ intersects the $y$ axis at
$y_I=x^3_0$.
\end{problem}

\begin{problem}\label{exer:4.5.22}
Find all curves $y=y(x)$ such that the tangent to the curve at any
point $(x_0,y(x_0))$ intersects the $y$ axis at $y_I=x_0$.

\begin{solution}
The equation of the line tangent to the curve at $(x_0,y(x_0))$ is
$y=y(x_0)+y'(x_0)(x-x_0)$. Since $y(0)=x_0$, $x_0=y(x_0)-y'(x_0)x_0$.
Since $x_0$ is arbitrary, it follows that $x=y-xy'$, so
$y'-\frac{y}{x}=-1$. The solutions of this equation are of the form $y=ux$,
where $u'x=-1$, so $u'=-\frac{1}{x}$. Therefore,$u=-\ln|x|+c$ and
$y=-x\ln|x|+cx$.
\end{solution}
\end{problem}

\begin{problem}\label{exer:4.5.23}
Find a curve $y=y(x)$ through $(0,2)$ such that the normal to the
curve at any point $(x_0,y(x_0))$ intersects the $x$ axis at
$x_I=x_0+1$.
\end{problem}

\begin{problem}\label{exer:4.5.24}
Find a curve $y=y(x)$ through $(2,1)$ such that the normal to the
curve at any point $(x_0,y(x_0))$ intersects the $y$ axis at
$y_I=2y(x_0)$.

\begin{solution}
The equation of the line normal to the curve at $(x_0,y_0)$ is
$y=y(x_0)-\frac{x-x_0}{y'(x_0)}$. Since $y(0)=2y(x_0)$,
$y(x_0)+\frac{x_0}{y'(x_0)}=2y(x_0)$. Since $x_0$ is arbitrary, it
follows that $y'y=x$, so
$\frac{y^2}{2}=\frac{x^2}{2}+\frac{c}{2}$ and $y^2=x^2+c$. Now
$y(2)=1\leftrightarrow c=-3$. Therefore,$y=\sqrt{x^2-3}$.
\end{solution}
\end{problem}

\begin{problem}\label{exer:4.5.25} Find the
orthogonal trajectories of the family of curves given by $x^2+2y^2=c^2$.
\end{problem}

\begin{problem}\label{exer:4.5.26} Find the
orthogonal trajectories of the family of curves given by $x^2+4xy+y^2=c$.

\begin{solution}
Differentiating the given equation yields $2x+4y+4xy'+2yy'=0$, so
$y'=-\frac{x+2y}{2x+y}$ is a differential equation for the given
family, and $y'=\frac{2x+y}{x+2y}$ is a differential equation
for the orthogonal trajectories. Substituting $y=ux$ yields
$u'x+u=\frac{2+u}{1+2u}$, so $u'x=-\frac{2(u^2-1)}{1+2u}$ and
$\frac{1+2u}{(u-1)(u+1)}u'=-\frac{2}{x}$, or $\left[\frac{3}{u-1}+\frac{1}{u+1}\right]u'=-\frac{4}{x}$. Therefore,
$3\ln|u-1|+\ln|u+1|=-4\ln|x|+K$, so $(u-1)^3(u+1)=\frac{k}{x^4}$.
Substituting $u=\frac{y}{x}$ yields the orthogonal trajectories
$(y-x)^3(y+x)=k$.
\end{solution}
\end{problem}

\begin{problem}\label{exer:4.5.27} Find the
orthogonal trajectories of the family of curves given by $y=ce^{2x}$.
\end{problem}

\begin{problem}\label{exer:4.5.28} Find the
orthogonal trajectories of the family of curves given by $xye^{x^2}=c$.

\begin{solution}
Differentiating yields $ye^{x^2}(1+2x^2)+xe^{x^2}y'=0$, so
$y'=\frac{y(1+2x^2)}{x}$ is a differential equation for the given
family. Therefore,$y'=-\frac{x}{y(1+2x^2)}$ is a differential
equation for the orthogonal trajectories. This yields
$yy'=-\frac{x}{1+2x^2}$, so
$\frac{y^2}{2}=-\frac{1}{4}\ln(1+2x^2)+\frac{k}{2}$, and the
orthogonal trajectories are given by
$y^2=-\frac{1}{2}\ln(1+2x^2)+k$.
\end{solution}
\end{problem}

\begin{problem}\label{exer:4.5.29}
Find the
orthogonal trajectories of the family of curves given by $y=\frac{ce^x}{x}$.
\end{problem}

\begin{problem}\label{exer:4.5.30}
Find a curve through $(-1,3)$ orthogonal to every parabola of the form
$$
y=1+cx^2
$$
that it intersects. Which of these parabolas does the desired curve
intersect?

\begin{solution}
Differentiating (A) $y=1+cx^2$ yields (B) $y'=2cx$. From (A),
$c=\frac{y-1}{x^2}$. Substituting this into (B) yields the
differential equation $y'=\frac{2(y-1)}{x}$ for the given family of
parabolas. Therefore,$y'=-\frac{x}{2(y-1)}$ is a differential
equation for the orthogonal trajectories. Separating variables yields
$2(y-1)y'=-x$, so $(y-1)^2=-\frac{x^2}{2}+k$. Now
$y(-1)=3\leftrightarrow k=\frac{9}{2}$, so
$(y-1)^2=-\frac{x^2}{2}+\frac{9}{2}$. Therefore,(D)
$y=1+\sqrt{\frac{9-x^2}{2}}$. This curve intersects the parabola
(A) if and only if the equation (C) $cx^2=\sqrt{\frac{9-x^2}{2}}$ has
a solution $x^2$ in $(0,9)$. Therefore,$c>0$ is a necessary condition for intersection. We will show that it is also sufficient. Squaring both sides of (C) and simplifying yields $2c^2x^4+x^2-9=0$. Using the quadratic formula to solve this for $x^2$ yields
$x^2=-1+\sqrt{\frac{1+72c^2}{4c^2}}$. The condition $x^2<9$ holds if
and only if $-1+\sqrt{1+72c^2}<36c^4$, which is equivalent to
$1+72c^2<(1+36c^2)^2=1+72c^2+1296c^4$, which holds for all $c>0$.
\end{solution}
\end{problem}

\begin{problem}\label{exer:4.5.31}
Show that the orthogonal trajectories of
$$
x^2+2axy+y^2=c
$$
satisfy
$$
|y-x|^{a+1}|y+x|^{a-1}=k.
$$
\end{problem}

\begin{problem}\label{exer:4.5.32}
If lines $L$ and $L_1$ intersect at $(x_0,y_0)$ and $\alpha$ is the
smallest angle through which $L$ must be rotated counterclockwise
about $(x_0,y_0)$ to bring it into coincidence with $L_1$, we say that
$\alpha$ is the \emph{angle from $L$ to $L_1$};   thus,
$0\le\alpha<\pi$. If $L$ and $L_1$ are tangents to curves $C$ and
$C_1$, respectively, that intersect at $(x_0,y_0)$, we say that $C_1$
intersects $C$ at the angle $\alpha$. Use the identity
$$
\tan(A+B)=\frac{\tan A+\tan B}{1-\tan A\tan B}
$$
to show that if $C$ and $C_1$ are intersecting integral curves of
$$
y'=f(x,y)$$
and
$$y'={f(x,y)+\tan\alpha}{
1-f(x,y)\tan\alpha} \left( \alpha \ne \frac{\pi}{2}\right),
$$
respectively, then $C_1$ intersects $C$ at the angle $\alpha$.

\begin{solution}
The angles $\theta$ and $\theta_1$  from the $x$-axis to the tangents
to $C$ and $C_1$ satisfy $\tan\theta=f(x_0,y_0)$ and
$\tan\theta_1=\frac{f(x_0,y_0)+\tan\alpha}{1-f(x_0,y_0)\tan\alpha}=
\frac{\tan\theta+\tan\alpha}{1-\tan\theta\tan\alpha}=\tan(\theta+\alpha)$.
Therefore, assuming $\theta$ and $\theta_1$ are both in $[0,2\pi)$,
$\theta_1=\theta+\alpha$.
\end{solution}
\end{problem}

\begin{problem}\label{exer:4.5.33}
Use the result of Exercise~\ref{exer:4.5.32} to find a family of curves
that intersect every nonvertical line through the origin at the angle
$\alpha=\pi/4$.
\end{problem}

\begin{problem}\label{exer:4.5.34}
Use the result of Exercise~\ref{exer:4.5.32} to find a family of curves
that intersect every circle centered at the origin at a given angle
$\alpha \ne \pi/2$.

\begin{solution}
Circles centered at the origin are given by $x^2+y^2=r^2$.
Differentiating yields $2x+2yy'=0$, so $y'=-\frac{x}{y}$ is a
differential equation for the given family, and
$y'=\frac{-(x/y)+\tan\alpha}{1+(x/y)\tan\alpha}$ is a differential
equation for the desired family. Substituting $y=ux$ yields
$u'x+u=\frac{-1/u+\tan\alpha}{1+(1/u)\tan\alpha}=\frac{-1+u\tan\alpha}{u+\tan\alpha}$. Therefore,$u'x=-\frac{1+u^2}{u+\tan\alpha}$,
$\frac{u+\tan\alpha}{1+u^2}u'=-\frac{1}{x}$ and
$\frac{1}{2}\ln(1+u^2)+\tan\alpha\tan^{-1}u=-\ln|x|+k$. Substituting
$u=\frac{y}{x}$ yields $\frac{1}{2}\ln (x^2+y^2)+(\tan\alpha)
\tan^{-1}\frac{y}{x}=k$.
\end{solution}
\end{problem}

\end{document}