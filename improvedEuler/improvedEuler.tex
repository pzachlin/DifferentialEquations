\documentclass{ximera}

%% You can put user macros here
%% However, you cannot make new environments

%\listfiles

% Get the 'old' hints/expandables, for use on ximera.osu.edu
%\def\xmNotHintAsExpandable{true}
%\def\xmNotExpandableAsAccordion{true}



%\graphicspath{{./}{firstExample/}{secondExample/}}
\graphicspath{{./}
{aboutDiffEq/}
{applicationsLeadingToDiffEq/}
{applicationsToCurves/}
{autonomousSecondOrder/}
{basicConcepts/}
{bernoulli/}
{constCoeffHomSysI/}
{constCoeffHomSysII/}
{constCoeffHomSysIII/}
{constantCoeffWithImpulses/}
{constantCoefficientHomogeneousEquations/}
{convolution/}
{coolingActivity/}
{directionFields/}
{drainingTank/}
{epidemicActivity/}
{eulersMethod/}
{exactEquations/}
{existUniqueNonlinear/}
{frobeniusI/}
{frobeniusII/}
{frobeniusIII/}
{global.css/}
{growthDecay/}
{heatingCoolingActivity/}
{higherOrderConstCoeff/}
{homogeneousLinearEquations/}
{homogeneousLinearSys/}
{improvedEuler/}
{integratingFactors/}
{interactExperiment/}
{introToLaplace/}
{introToSystems/}
{inverseLaplace/}
{ivpLaplace/}
{laplaceTable/}
{lawOfCooling/}
{linSysOfDiffEqs/}
{linearFirstOrderDiffEq/}
{linearHigherOrder/}
{mixingProblems/}
{motionUnderCentralForce/}
{nonHomogeneousLinear/}
{nonlinearToSeparable/}
{odesInSage/}
{piecewiseContForcingFn/}
{population/}
{reductionOfOrder/}
{regularSingularPts/}
{reviewOfPowerSeries/}
{rlcCircuit/}
{rungeKutta/}
{secondLawOfMotion/}
{separableEquations/}
{seriesSolNearOrdinaryPtI/}
{seriesSolNearOrdinaryPtII/}
{simplePendulum/}
{springActivity/}
{springProblemsI/}
{springProblemsII/}
{undCoeffHigherOrderEqs/}
{undeterminedCoeff/}
{undeterminedCoeff2/}
{unitStepFunction/}
{varParHigherOrder/}
{varParamNonHomLinSys/}
{variationOfParameters/}
}


\usepackage{tikz}
%\usepackage{tkz-euclide}
\usepackage{tikz-3dplot}
\usepackage{tikz-cd}
\usetikzlibrary{shapes.geometric}
\usetikzlibrary{arrows}
\usetikzlibrary{decorations.pathmorphing,patterns}
\usetikzlibrary{backgrounds} % added by Felipe
% \usetkzobj{all}   % NOT ALLOWED IN RECENT TeX's ...
\pgfplotsset{compat=1.13} % prevents compile error.

\pdfOnly{\renewcommand{\theHsection}{\thepart.section.\thesection}}  %% MAKES LINKS WORK should be added to CLS
\pdfOnly{\renewcommand{\part}[1]{\chapterstyle\title{#1}\begin{abstract}\end{abstract}\maketitle\def\thechaptertitle{#1}}}


\renewcommand{\vec}[1]{\mathbf{#1}}
\newcommand{\RR}{\mathbb{R}}
\providecommand{\dfn}{\textit}
\renewcommand{\dfn}{\textit}
\newcommand{\dotp}{\cdot}
\newcommand{\id}{\text{id}}
\newcommand\norm[1]{\left\lVert#1\right\rVert}
\newcommand{\dst}{\displaystyle}
 
\newtheorem{general}{Generalization}
\newtheorem{initprob}{Exploration Problem}

\tikzstyle geometryDiagrams=[ultra thick,color=blue!50!black]

\usepackage{mathtools}

\title{The Improved Euler Method and Related Methods}%\label{Module 7-ADEF}


\begin{document}

\begin{abstract}
We explore some ways to improve upon Euler's method for approximating the solution of a differential equation.
\end{abstract}

\maketitle

\section*{The Improved Euler Method and Related Methods}

In \href{https://ximera.osu.edu/ode/main/eulersMethod/eulersMethod}{Trench 3.1} we saw that the global truncation error of
Euler's method is $O(h)$, which would seem to imply that we can
achieve arbitrarily accurate results with Euler's method by simply
choosing the step size sufficiently small. However, this
isn't  a good idea, for two reasons. First,
after a certain point decreasing the step size will
increase roundoff errors to the point where the accuracy
will deteriorate rather than improve.
 The second
and more important reason is that in most applications of numerical
methods to an initial value problem
\begin{equation} \label{eq:3.2.1}
y'=f(x,y),\quad y(x_0)=y_0,
\end{equation}
the expensive part of the computation is the evaluation of $f$.
Therefore we want methods that give good results for a given number
of such evaluations. This is what motivates us to look for numerical
methods better than Euler's.

To clarify this point, suppose  we want to approximate the value of $e$
by applying Euler's method to the initial value problem
$$
y'=y,\quad y(0)=1,\quad\mbox{(with solution $y=e^x$)}
$$
on $[0,1]$, with $h=1/12$, $1/24$, and $1/48$, respectively. Since
each step in Euler's method requires one evaluation of $f$, the number
of evaluations of $f$ in each of these attempts is $n=12$, $24$, and
$48$, respectively. In each case we accept $y_n$ as an approximation
to $e$.  The table below shows the results. 


$$
\begin{array}{|c|c|c|}
\hline
n&
\text{Euler}&
\text{Exact}\\ \hline
12 & 2.613035290  & 2.718281828 \\
24 & 2.663731258   & 2.718281828 \\
48 & 2.690496599  & 2.718281828
\\
\hline
\end{array}
$$



The first column of the table
indicates
the number of evaluations of $f$ required to obtain the approximation, and
the last column contains the value of $e$ rounded to ten significant
figures.

In this section we study the
\dfn{improved Euler method}, which
requires two evaluations of $f$ at each step. We have used this method
with $h=1/6$, $1/12$, and $1/24$. The required number of evaluations
of $f$ were 12, 24, and $48$, as in the three applications of Euler's
method;     however, you can see from the third column of
the table below that the approximation to $e$ obtained by the
improved Euler method with only 12 evaluations of $f$ is better than
the approximation obtained by Euler's method with 48 evaluations.

\begin{center}
$$
\begin{array}{|c|c|c|c|}
\hline
n&
\text{Euler}&
\text{Improved Euler}&
\text{Exact}\\ \hline
12 & 2.613035290  &2.707188994  & 2.718281828 \\
24 & 2.663731258   &2.715327371   & 2.718281828 \\
48 & 2.690496599  &2.717519565  & 2.718281828
\\
\hline
\end{array}
$$
\end{center}

In \href{https://ximera.osu.edu/ode/main/rungeKutta/rungeKutta}{Trench 3.} we will study the
\href{https://en.wikipedia.org/wiki/Runge%E2%80%93Kutta_methods}{Runge-Kutta method},
which requires four evaluations of $f$ at each step. We have used this
method with $h=1/3$, $1/6$, and $1/12$. The required number of
evaluations of $f$ were again $12$, $24$, and $48$, as in the three
applications of Euler's method and the improved Euler method;     however,
you can see from the fourth column of the table below that the
approximation to $e$ obtained by the Runge-Kutta method with only 12
evaluations of $f$ is better than the approximation obtained by the
improved Euler method with 48 evaluations.

\begin{center}
$$
\begin{array}{|c|c|c|c|c|}
\hline
n&
\text{Euler}&
\text{Improved Euler}&
\text{Runge-Kutta}&
\text{Exact}\\ \hline
12 & 2.613035290  &2.707188994  &2.718069764 & 2.718281828 \\
24 & 2.663731258   &2.715327371   &2.718266612 & 2.718281828 \\
48 & 2.690496599  &2.717519565  &2.718280809 & 2.718281828
\\
\hline
\end{array}
$$
\end{center}

\subsection*{The Improved Euler Method}

The
\dfn{improved Euler method} for solving the initial value
problem $\eqref{eq:3.2.1}$ is based on approximating the integral curve of
$\eqref{eq:3.2.1}$ at $(x_i,y(x_i))$ by the line through $(x_i,y(x_i))$ with
slope
$$
m_i=\frac{f(x_i,y(x_i))+f(x_{i+1},y(x_{i+1}))}{2};
$$
that is, $m_i$ is the average of the slopes of the tangents to the
integral curve at the endpoints of $[x_i,x_{i+1}]$. The equation of
the approximating line is therefore
\begin{equation} \label{eq:3.2.2}
y=y(x_i)+\frac{f(x_i,y(x_i))+f(x_{i+1},y(x_{i+1}))}{2}(x-x_i).
\end{equation}
Setting $x=x_{i+1}=x_i+h$ in $\eqref{eq:3.2.2}$ yields
\begin{equation} \label{eq:3.2.3}
y_{i+1}=y(x_i)+\frac{h}{2}\left(f(x_i,y(x_i))+f(x_{i+1},y(x_{i+1}))\right)
\end{equation}
as an approximation to $y(x_{i+1})$. As in our derivation of Euler's
method, we replace $y(x_i)$ (unknown if $i>0$) by its
approximate value $y_i$; then $\eqref{eq:3.2.3}$ becomes
$$
y_{i+1}=y_i+\frac{h}{2}\left(f(x_i,y_i)+f(x_{i+1},y(x_{i+1})\right).
$$
However, this still won't work, because we don't know $y(x_{i+1})$,
which appears on the right. We overcome this by replacing $y(x_{i+1})$
by $y_i+hf(x_i,y_i)$, the value that the  Euler method would
assign to $y_{i+1}$. Thus,  the improved Euler method
starts with the known value $y(x_0)=y_0$ and computes
$y_1, y_2, \dots, y_n$ successively with the formula
\begin{equation} \label{eq:3.2.4}
y_{i+1}=y_i+\frac{h}{2}\left(f(x_i,y_i)+f(x_{i+1},y_i+hf(x_i,y_i))\right).
\end{equation}
The computation indicated here can be conveniently organized as
follows: given $y_i$, compute
\begin{eqnarray*}
k_{1i}&=&f(x_i,y_i),\\
k_{2i}&=&f\left(x_i+h,y_i+hk_{1i}\right),\\
y_{i+1}&=&y_i+\frac{h}{2}(k_{1i}+k_{2i}).
\end{eqnarray*}

The improved Euler method requires two evaluations of $f(x,y)$ per
step, while Euler's method requires only one. However, we'll see at
the end of this section that if $f$ satisfies appropriate assumptions,
 the local truncation error with the improved Euler method is
$O(h^3)$, rather than $O(h^2)$ as with Euler's method. Therefore
the global truncation error with the improved Euler method is
$O(h^2)$;     however, we won't prove this.

We note that the magnitude of the local truncation error in the
improved Euler method and other methods discussed in this section is
determined by the third derivative $y'''$ of the solution of the
initial value problem. Therefore the local truncation error will be
larger where $|y'''|$ is large, or smaller where $|y'''|$ is small.

The next example, which deals with the initial value problem
considered in Example~\ref{example:3.1.1}, illustrates
the computational procedure indicated in the improved Euler method.

\begin{example}\label{example:3.2.1}
Use the improved Euler method with $h=0.1$ to find approximate values
of the solution of the initial value problem
\begin{equation}\label{eq:3.2.5}
y'+2y=x^3e^{-2x},\quad y(0)=1
\end{equation}
at $x=0.1,0.2,0.3$.

\begin{explanation}
As in Example~\ref{example:3.1.1}, we rewrite $\eqref{eq:3.2.5}$  as
$$
y'=-2y+x^3e^{-2x},\quad y(0)=1,
$$
which is of the form $\eqref{eq:3.2.1}$, with
$$
f(x,y)=-2y+x^3e^{-2x},\quad x_0=0,\quad\mbox{and}\quad y_0=1.
$$
The improved Euler method yields

\begin{eqnarray*}
k_{10} & = & f(x_0,y_0)
  = f(0,1)=-2,\\
k_{20} & = & f(x_1,y_0+hk_{10})=f(.1,1+(.1)(-2))\\
 &=& f(.1,.8)=-2(.8)+(.1)^3e^{-.2}=-1.599181269,\\
y_1&=&y_0+\frac{h}{2}(k_{10}+k_{20}),\\
&=&1+(.05)(-2-1.599181269)=.820040937,\\
k_{11} & = & f(x_1,y_1)
  = f(.1,.820040937)= -2(.820040937)+(.1)^3e^{-.2}=-1.639263142,\\
k_{21} & = & f(x_2,y_1+hk_{11})=f(.2,.820040937+.1(-1.639263142)),\\
 &=&
f(.2,.656114622)=-2(.656114622)+(.2)^3e^{-.4}=-1.306866684,\\
y_2&=&y_1+\frac{h}{2}(k_{11}+k_{21}),\\
&=&.820040937+(.05)(-1.639263142-1.306866684)=.672734445,\\
k_{12} & = & f(x_2,y_2)
  = f(.2,.672734445)= -2(.672734445)+(.2)^3e^{-.4}=-1.340106330,\\
k_{22} & = & f(x_3,y_2+hk_{12})=f(.3,.672734445+.1(-1.340106330)),\\
 &=&
f(.3,.538723812)=-2(.538723812)+(.3)^3e^{-.6}=-1.062629710,\\
y_3&=&y_2+\frac{h}{2}(k_{12}+k_{22})\\
&=&.672734445+(.05)(-1.340106330-1.062629710)=.552597643.
\end{eqnarray*}
\end{explanation}
\end{example}

\begin{example}\label{example:3.2.2}
The table below shows results of using the improved Euler method
with step sizes $h=0.1$ and $h=0.05$ to find approximate values of the
solution of the initial value problem
$$
y'+2y=x^3e^{-2x},\quad y(0)=1
$$
at $x=0, 0.1, 0.2, 0.3, \dots, 1.0$. For comparison, it also shows
the
corresponding approximate values obtained with Euler's method in
\ref{example:3.1.2}, and the values of the exact solution
$$
y=\frac{e^{-2x}}{4}(x^4+4).
$$
The results
obtained by the improved Euler method with $h=0.1$ are better than
those obtained by Euler's method with $h=0.05$.

$$
\begin{array}{|c|c|c|c|c|c|}
\hline
 &\text{Euler} &\text{Euler} &\text{Improved} &\text{Improved}& \text{Exact}\\
 & & &\text{Euler}&\text{Euler}& \\
x&h=0.1&h=0.05&h=0.1&h=0.05&
\\ \hline
0.0 & 1.000000000& 1.000000000& 1.000000000   &1.000000000 & 1.000000000 \\
0.1 & 0.800000000& 0.810005655& 0.820040937   &0.819050572 & 0.818751221 \\
0.2 & 0.640081873& 0.656266437& 0.672734445   &0.671086455 & 0.670588174 \\
0.3 & 0.512601754& 0.532290981& 0.552597643   &0.550543878 & 0.549922980 \\
0.4 & 0.411563195& 0.432887056& 0.455160637   &0.452890616 & 0.452204669 \\
0.5 & 0.332126261& 0.353785015& 0.376681251   &0.374335747 & 0.373627557 \\
0.6 & 0.270299502& 0.291404256& 0.313970920   &0.311652239 & 0.310952904 \\
0.7 & 0.222745397& 0.242707257& 0.264287611   &0.262067624 & 0.261398947 \\
0.8 & 0.186654593& 0.205105754& 0.225267702   &0.223194281 & 0.222570721 \\
0.9 & 0.159660776& 0.176396883& 0.194879501   &0.192981757 & 0.192412038 \\
1.0 & 0.139778910& 0.154715925& 0.171388070   &0.169680673 & 0.169169104\\
\hline
\end{array}
$$
The following interactive illustrates the difference graphically.

\begin{center}
    \geogebra{mjqkq58n}{800}{600}
\end{center}
\end{example}

\begin{example}\label{example:3.2.3}
The table below shows analogous results for the nonlinear
initial value problem
$$
y'=-2y^2+xy+x^2,\ y(0)=1.
$$
We applied Euler's  method to this problem in
Example~\ref{example:3.1.3}.


\[
\begin{array}{|c|c|c|c|c|c|}
\hline
 &\text{Euler} &\text{Euler} &\text{Improved} &\text{Improved}& \text{ ``Exact'' }\\
 & & &\text{Euler}&\text{Euler}& \\
x&h=0.1&h=0.05&h=0.1&h=0.05&
\\ \hline
0.0 & 1.000000000 & 1.000000000 &1.000000000 & 1.000000000 & 1.000000000 \\
0.1 & 0.800000000 & 0.821375000 &0.840500000 & 0.838288371 & 0.837584494 \\
0.2 & 0.681000000 & 0.707795377 &0.733430846 & 0.730556677 & 0.729641890 \\
0.3 & 0.605867800 & 0.633776590 &0.661600806 & 0.658552190 & 0.657580377 \\
0.4 & 0.559628676 & 0.587454526 &0.615961841 & 0.612884493 & 0.611901791 \\
0.5 & 0.535376972 & 0.562906169 &0.591634742 & 0.588558952 & 0.587575491 \\
0.6 & 0.529820120 & 0.557143535 &0.586006935 & 0.582927224 & 0.581942225 \\
0.7 & 0.541467455 & 0.568716935 &0.597712120 & 0.594618012 & 0.593629526 \\
0.8 & 0.569732776 & 0.596951988 &0.626008824 & 0.622898279 & 0.621907458 \\
0.9 & 0.614392311 & 0.641457729 &0.670351225 & 0.667237617 & 0.666250842 \\
1.0 & 0.675192037 & 0.701764495 &0.730069610 & 0.726985837 & 0.726015790\\
\hline
\end{array}
\]
\end{example}

\begin{example}\label{example:3.2.4}
Use step sizes $h=0.2$, $h=0.1$, and $h=0.05$ to find approximate
values of the solution of
\begin{equation} \label{eq:3.2.6}
y'-2xy=1,\quad y(0)=3
\end{equation}
at $x=0, 0.2, 0.4, 0.6, \dots, 2.0$ by
\begin{enumerate}
\item\label{item:3.2.4a}
the improved Euler method;
\item\label{item:3.2.4b}
the improved Euler
semilinear method. 
\end{enumerate}
(We used Euler's method and the Euler semilinear
method on this problem in~\ref{example:3.1.4}.)

\begin{explanation}
\ref{item:3.2.4a}
Rewriting $\eqref{eq:3.2.6}$ as
$$
y'=1+2xy,\quad y(0)=3
$$
and applying the improved Euler method with $f(x,y)=1+2xy$ yields
the results shown in the table.

$$
\begin{array}{|c|c|c|c|c|}
\hline
x&
h=0.2&
h=0.1&
h=0.05&
\text{``Exact''}\\ \hline
0.0 &   3.000000000 &   3.000000000 &   3.000000000 &   3.000000000 \\
0.2 &   3.328000000 &   3.328182400 &   3.327973600 &   3.327851973 \\
0.4 &   3.964659200 &   3.966340117 &   3.966216690 &   3.966059348 \\
0.6 &   5.057712497 &   5.065700515 &   5.066848381 &   5.067039535 \\
0.8 &   6.900088156 &   6.928648973 &   6.934862367 &   6.936700945 \\
1.0 &  10.065725534 &  10.154872547 &  10.177430736 &  10.184923955 \\
1.2 &  15.708954420 &  15.970033261 &  16.041904862 &  16.067111677 \\
1.4 &  26.244894192 &  26.991620960 &  27.210001715 &  27.289392347 \\
1.6 &  46.958915746 &  49.096125524 &  49.754131060 &  50.000377775 \\
1.8 &  89.982312641 &  96.200506218 &  98.210577385 &  98.982969504 \\
2.0 & 184.563776288 & 203.151922739 & 209.464744495 & 211.954462214 \\
\hline
\end{array}
$$

\ref{item:3.2.4b}
Since $y_1=e^{x^2}$ is a solution of the complementary equation
$y'-2xy=0$, we can  apply the improved Euler semilinear method to
$\eqref{eq:3.2.6}$, with
$$
y=ue^{x^2}\quad\mbox{and}\quad
u'=e^{-x^2},\quad u(0)=3.
$$
The results listed in the table below are clearly better than
those obtained by the improved Euler method.


$$
\begin{array}{|c|c|c|c|c|}
\hline
x&
h=0.2&
h=0.1&
h=0.05&
\text{``Exact''}\\ \hline
0.0 &   3.000000000 &  3.000000000  &   3.000000000 &   3.000000000 \\
0.2 &   3.326513400 &  3.327518315  &   3.327768620 &   3.327851973 \\
0.4 &   3.963383070 &  3.965392084  &   3.965892644 &   3.966059348 \\
0.6 &   5.063027290 &  5.066038774  &   5.066789487 &   5.067039535 \\
0.8 &   6.931355329 &  6.935366847  &   6.936367564 &   6.936700945 \\
1.0 &  10.178248417 & 10.183256733  &  10.184507253 &  10.184923955 \\
1.2 &  16.059110511 & 16.065111599  &  16.066611672 &  16.067111677 \\
1.4 &  27.280070674 & 27.287059732  &  27.288809058 &  27.289392347 \\
1.6 &  49.989741531 & 49.997712997  &  49.999711226 &  50.000377775 \\
1.8 &  98.971025420 & 98.979972988  &  98.982219722 &  98.982969504 \\
2.0 & 211.941217796 &211.951134436  & 211.953629228 & 211.954462214 \\
\hline
\end{array}
$$
\end{explanation}
\end{example}


\subsection*{A Family of Methods with $O(h^3)$ Local Truncation Error}

We'll now derive a class of methods with $O(h^3)$ local truncation
error for solving $\eqref{eq:3.2.1}$. For simplicity, we
assume that $f$, $f_x$, $f_y$, $f_{xx}$, $f_{yy}$, and $f_{xy}$ are
continuous and bounded for all $(x,y)$. This implies that if $y$ is
the solution of $\eqref{eq:3.2.1}$ then $y''$ and $y'''$ are bounded.
%(Exercise~\ref{exer:3.2.31}).

We begin by approximating the integral curve of $\eqref{eq:3.2.1}$ at
$(x_i,y(x_i))$ by the line through $(x_i,y(x_i))$ with slope
$$
m_i=\sigma y'(x_i)+\rho y'(x_i+\theta h),
$$
where $\sigma$, $\rho$, and $\theta$ are constants that we'll soon
specify;     however, we insist at the outset that $0<\theta\leq 1$, so
that
$$
x_i<x_i+\theta h\leq x_{i+1}.
$$
The equation of the approximating line is
\begin{equation} \label{eq:3.2.7}
\begin{array}{rcl}
y&=&y(x_i)+m_i(x-x_i)\\
&=&y(x_i)+\left[\sigma y'(x_i)+\rho y'(x_i+\theta h)\right](x-x_i).
\end{array}
\end{equation}
Setting $x=x_{i+1}=x_i+h$ in $\eqref{eq:3.2.7}$ yields
$$
\hat y_{i+1}=y(x_i)+h\left[\sigma y'(x_i)+\rho y'(x_i+\theta h)\right]
$$
as an approximation to $y(x_{i+1})$.

To determine $\sigma$, $\rho$, and $\theta$ so that the error
\begin{equation} \label{eq:3.2.8}
\begin{array}{rcl}
E_i&=&y(x_{i+1})-\hat y_{i+1}\\
&=&y(x_{i+1})-y(x_i)-h\left[\sigma y'(x_i)+\rho y'(x_i+\theta
h)\right]
\end{array}
\end{equation}
in this approximation is $O(h^3)$, we begin by recalling from Taylor's
theorem that
$$
y(x_{i+1})=y(x_i)+hy'(x_i)+\frac{h^2}{2}y''(x_i)+\frac{h^3}{6}y'''(\hat
x_i),
$$
where $\hat x_i$ is in $(x_i,x_{i+1})$. Since $y'''$ is bounded this
implies that
$$
 y(x_{i+1})-y(x_i)-hy'(x_i)-\frac{h^2}{2}y''(x_i)=O(h^3).
$$
Comparing this with $\eqref{eq:3.2.8}$ shows that $E_i=O(h^3)$ if
\begin{equation} \label{eq:3.2.9}
\sigma y'(x_i)+\rho y'(x_i+\theta h)=y'(x_i)+\frac{h}{2}y''(x_i)
+O(h^2).
\end{equation}
However, applying Taylor's theorem to $y'$ shows that
$$
y'(x_i+\theta h)=y'(x_i)+\theta h y''(x_i)+\frac{(\theta
h)^2}{2}y'''(\overline x_i),
$$
where $\overline x_i$ is in $(x_i,x_i+\theta h)$. Since $y'''$ is
bounded, this implies that
$$
y'(x_i+\theta h)=y'(x_i)+\theta h y''(x_i)+O(h^2).
$$
Substituting this into $\eqref{eq:3.2.9}$ and noting that the sum of
two $O(h^2)$ terms is again $O(h^2)$ shows that
$E_i=O(h^3)$ if
$$
(\sigma+\rho)y'(x_i)+\rho\theta h y''(x_i)=
y'(x_i)+\frac{h}{2}y''(x_i),
$$
which is true if
\begin{equation} \label{eq:3.2.10}
\sigma+\rho=1 \quad\mbox{and}\quad \rho\theta=\frac{1}{2}.
\end{equation}

Since $y'=f(x,y)$, we can now conclude from $\eqref{eq:3.2.8}$ that
\begin{equation} \label{eq:3.2.11}
y(x_{i+1})=y(x_i)+h\left[\sigma f(x_i,y_i)+\rho f(x_i+\theta
h,y(x_i+\theta h))\right]+O(h^3)
\end{equation}
if $\sigma$, $\rho$, and $\theta$ satisfy $\eqref{eq:3.2.10}$. However,
this formula would not be useful even if we knew $y(x_i)$ exactly (as
we would for $i=0$), since we still wouldn't know $y(x_i+\theta h)$
exactly. To overcome this difficulty, we again use Taylor's theorem to
write
$$
y(x_i+\theta h)=y(x_i)+\theta h y'(x_i)+\frac{h^2}{2}y''(\tilde x_i),
$$
where $\tilde x_i$ is in $(x_i,x_i+\theta h)$. Since
$y'(x_i)=f(x_i,y(x_i))$ and $y''$ is bounded, this implies that
\begin{equation} \label{eq:3.2.12}
|y(x_i+\theta h)-y(x_i)-\theta h f(x_i,y(x_i))|\leq Kh^2
\end{equation}
for some constant $K$. Since $f_y$ is bounded, the mean value theorem
implies that
$$
|f(x_i+\theta h,u)-f(x_i+\theta h,v)|\leq M|u-v|
$$
for some constant $M$. Letting
$$
u=y(x_i+\theta h)\quad\mbox{and}\quad
v=y(x_i)+\theta h f(x_i,y(x_i))
$$
and recalling $\eqref{eq:3.2.12}$ shows that
$$
f(x_i+\theta h,y(x_i+\theta h))=f(x_i+\theta h,y(x_i)+\theta h
f(x_i,y(x_i)))+O(h^2).
$$
Substituting this into $\eqref{eq:3.2.11}$   yields
\begin{eqnarray*}
y(x_{i+1})&=&y(x_i)+h\left[\sigma f(x_i,y(x_i))+\right.\\&&\left.\rho
f(x_i+\theta h,y(x_i)+\theta hf(x_i,y(x_i)))\right]+O(h^3).
\end{eqnarray*}
This implies that the formula
$$
y_{i+1}=y_i+h\left[\sigma f(x_i,y_i)+\rho f(x_i+\theta
h,y_i+\theta hf(x_i,y_i))\right]
$$
has $O(h^3)$ local truncation error if $\sigma$, $\rho$, and $\theta$
satisfy $\eqref{eq:3.2.10}$. Substituting $\sigma=1-\rho$ and
$\theta=1/2\rho$ here yields
\begin{equation} \label{eq:3.2.13}
y_{i+1}=y_i+h\left[(1-\rho)f(x_i,y_i)+\rho
f\left(x_i+\frac{h}{2\rho},
y_i+\frac{h}{2\rho}f(x_i,y_i)\right)\right].
\end{equation}

The computation indicated here can be conveniently organized as
follows: given $y_i$, compute
\begin{eqnarray*}
k_{1i}&=&f(x_i,y_i),\\
k_{2i}&=&f\left(x_i+\frac{h}{2\rho},
y_i+\frac{h}{2\rho}k_{1i}\right),\\
y_{i+1}&=&y_i+h[(1-\rho)k_{1i}+\rho k_{2i}].
\end{eqnarray*}

Consistent with our requirement that $0<\theta<1$, we require that
$\rho\geq 1/2$. Letting $\rho=1/2$ in $\eqref{eq:3.2.13}$ yields the improved
Euler method $\eqref{eq:3.2.4}$. Letting $\rho=3/4$ yields
\href{https://en.wikipedia.org/wiki/Heun%27s_method}{Heun's method},
$$
y_{i+1}=y_i+h\left[\frac{1}{4}f(x_i,y_i)+\frac{3}{4}f\left(
x_i+\frac{2}{3}h,y_i+\frac{2}{3}hf(x_i,y_i)\right)\right],
$$
which can be organized as
\begin{eqnarray*}
k_{1i}&=&f(x_i,y_i),\\
k_{2i}&=&f\left(x_i+\frac{2h}{3},
y_i+\frac{2h}{3}k_{1i}\right),\\
y_{i+1}&=&y_i+\frac{h}{4}(k_{1i}+3k_{2i}).
\end{eqnarray*}
Letting $\rho=1$ yields the \dfn{midpoint
method},
$$
y_{i+1}=y_i+hf\left(x_i+\frac{h}{2},y_i+\frac{h}{2}f(x_i,y_i)\right),
$$
which can be organized as
\begin{eqnarray*}
k_{1i}&=&f(x_i,y_i),\\
k_{2i}&=&f\left(x_i+\frac{h}{2},
y_i+\frac{h}{2}k_{1i}\right),\\
y_{i+1}&=&y_i+hk_{2i}.
\end{eqnarray*}

%Examples involving the midpoint method and Heun's method are given in
%Exercises~\ref{exer:3.2.23}-\ref{exer:3.2.30}.



\section*{Text Source}
Trench, William F., "Elementary Differential Equations" (2013). Faculty Authored and Edited Books \& CDs. 8. (CC-BY-NC-SA)

\href{https://digitalcommons.trinity.edu/mono/8/}{https://digitalcommons.trinity.edu/mono/8/}


\end{document}