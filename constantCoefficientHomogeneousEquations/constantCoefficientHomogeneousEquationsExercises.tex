\documentclass{ximera}
%% You can put user macros here
%% However, you cannot make new environments

%\listfiles

% Get the 'old' hints/expandables, for use on ximera.osu.edu
%\def\xmNotHintAsExpandable{true}
%\def\xmNotExpandableAsAccordion{true}



%\graphicspath{{./}{firstExample/}{secondExample/}}
\graphicspath{{./}
{aboutDiffEq/}
{applicationsLeadingToDiffEq/}
{applicationsToCurves/}
{autonomousSecondOrder/}
{basicConcepts/}
{bernoulli/}
{constCoeffHomSysI/}
{constCoeffHomSysII/}
{constCoeffHomSysIII/}
{constantCoeffWithImpulses/}
{constantCoefficientHomogeneousEquations/}
{convolution/}
{coolingActivity/}
{directionFields/}
{drainingTank/}
{epidemicActivity/}
{eulersMethod/}
{exactEquations/}
{existUniqueNonlinear/}
{frobeniusI/}
{frobeniusII/}
{frobeniusIII/}
{global.css/}
{growthDecay/}
{heatingCoolingActivity/}
{higherOrderConstCoeff/}
{homogeneousLinearEquations/}
{homogeneousLinearSys/}
{improvedEuler/}
{integratingFactors/}
{interactExperiment/}
{introToLaplace/}
{introToSystems/}
{inverseLaplace/}
{ivpLaplace/}
{laplaceTable/}
{lawOfCooling/}
{linSysOfDiffEqs/}
{linearFirstOrderDiffEq/}
{linearHigherOrder/}
{mixingProblems/}
{motionUnderCentralForce/}
{nonHomogeneousLinear/}
{nonlinearToSeparable/}
{odesInSage/}
{piecewiseContForcingFn/}
{population/}
{reductionOfOrder/}
{regularSingularPts/}
{reviewOfPowerSeries/}
{rlcCircuit/}
{rungeKutta/}
{secondLawOfMotion/}
{separableEquations/}
{seriesSolNearOrdinaryPtI/}
{seriesSolNearOrdinaryPtII/}
{simplePendulum/}
{springActivity/}
{springProblemsI/}
{springProblemsII/}
{undCoeffHigherOrderEqs/}
{undeterminedCoeff/}
{undeterminedCoeff2/}
{unitStepFunction/}
{varParHigherOrder/}
{varParamNonHomLinSys/}
{variationOfParameters/}
}


\usepackage{tikz}
%\usepackage{tkz-euclide}
\usepackage{tikz-3dplot}
\usepackage{tikz-cd}
\usetikzlibrary{shapes.geometric}
\usetikzlibrary{arrows}
\usetikzlibrary{decorations.pathmorphing,patterns}
\usetikzlibrary{backgrounds} % added by Felipe
% \usetkzobj{all}   % NOT ALLOWED IN RECENT TeX's ...
\pgfplotsset{compat=1.13} % prevents compile error.

\pdfOnly{\renewcommand{\theHsection}{\thepart.section.\thesection}}  %% MAKES LINKS WORK should be added to CLS
\pdfOnly{\renewcommand{\part}[1]{\chapterstyle\title{#1}\begin{abstract}\end{abstract}\maketitle\def\thechaptertitle{#1}}}


\renewcommand{\vec}[1]{\mathbf{#1}}
\newcommand{\RR}{\mathbb{R}}
\providecommand{\dfn}{\textit}
\renewcommand{\dfn}{\textit}
\newcommand{\dotp}{\cdot}
\newcommand{\id}{\text{id}}
\newcommand\norm[1]{\left\lVert#1\right\rVert}
\newcommand{\dst}{\displaystyle}
 
\newtheorem{general}{Generalization}
\newtheorem{initprob}{Exploration Problem}

\tikzstyle geometryDiagrams=[ultra thick,color=blue!50!black]

\usepackage{mathtools}

\title{Exercises} \license{CC BY-NC-SA 4.0}

\begin{document}

\begin{abstract}
\end{abstract}
\maketitle

\begin{onlineOnly}
\section*{Exercises}
\end{onlineOnly}


\begin{problem}\label{exer:5.2.1} Find the general solution of $y''+5y'-6y=0$.
\end{problem}

\begin{problem}\label{exer:5.2.2} Find the general solution of $y''-4y'+5y=0$.
\begin{solution}
$p(r)=r^2-4r+5=(r-2)^2+1$;\;
$y=e^{2x}(c_1 \cos x+c_2 \sin x)$.
\end{solution}
\end{problem}
 \begin{problem}\label{exer:5.2.3} Find the general solution of $y''+8y'+7y=0$.
\end{problem} 

\begin{problem}\label{exer:5.2.4} Find the general solution of $y''-4y'+4y=0$.
\begin{solution}
$p(r)=r^2-4r+4=(r-2)^2$;\;
 $y=e^{2x}(c_1+c_2x)$.
\end{solution}
\end{problem} 
 \begin{problem}\label{exer:5.2.5}  Find the general solution of $y''+2y'+10y=0$.
\end{problem} 

\begin{problem}\label{exer:5.2.6}  Find the general solution of $y''+6y'+10y=0$.
\begin{solution}
$p(r)=r^2+6r+10=(r+3)^2+1$;\;
$y=e^{-3x}(c_1 \cos x+c_2 \sin x)$.
\end{solution}
\end{problem} 
 \begin{problem}\label{exer:5.2.7} Find the general solution of $y''-8y'+16y=0$.
\end{problem} 

\begin{problem}\label{exer:5.2.8}  Find the general solution of $y''+y'=0$.
\begin{solution}
$p(r)=r^2+r=r(r+1)$;\;
 $y=c_1+c_2e^{-x}$.
\end{solution}
\end{problem} 

\begin{problem}\label{exer:5.2.9}  Find the general solution of $y''-2y'+3y=0$.
\end{problem}  

\begin{problem}\label{exer:5.2.10}  Find the general solution of $y''+6y'+13y=0$.
\begin{solution}
$p(r)=r^2+6r+13y=(r+3)^2+4$;\;
 $y=e^{-3x}(c_1\cos 2x+c_2\sin 2x)$.
\end{solution}
\end{problem} 

\begin{problem}\label{exer:5.2.11}  Find the general solution of $4y''+4y'+10y=0$.
\end{problem}

\begin{problem}\label{exer:5.2.12}  Find the general solution of $10y''-3y'-y=0$.
\begin{solution}
$p(r)=10r^2-3r-1=(2r-1)(5r+1)=10(r-1/2)
(r+1/5)$;\;
$y=c_1e^{-x/5}+c_2e^{x/2}$.
\end{solution}
\end{problem} 

\begin{problem}\label{exer:5.2.13}  Solve
the initial value problem.
$y''+14y'+50y=0, \quad  y(0)=2,\quad y'(0)=-17$
\end{problem} 

\begin{problem}\label{exer:5.2.14}  Solve
the initial value problem.
$6y''-y'-y=0, \quad  y(0)=10,\quad y'(0)=0$
\begin{solution}
$p(r)=6r^2-r-1=(2r-1)(3r+1)=6(r-1/2) (r+1/3)$;\;
$y=c_1e^{-x/3}+c_2e^{x/2}$;\;
$y'=-\frac{c_1}{3}e^{-x/3}+\frac{c_2}{2}e^{x/2}$;\;
 $y(0)=10\Rightarrow c_1+c_2=10,\ y'(0)=0\Rightarrow -\frac{c_1}{3}
+\frac{c_2}{2}=0$;\ $c_1=6, c_2=4$;\;
$y=4e^{x/2}+6e^{-x/3}$.
\end{solution}
\end{problem} 

\begin{problem}\label{exer:5.2.15}  Solve
the initial value problem.
$6y''+y'-y=0, \quad  y(0)=-1,\quad y'(0)=3$
\end{problem} 

\begin{problem}\label{exer:5.2.16}  Solve
the initial value problem.
$4y''-4y'-3y=0, \quad  y(0)=\frac{13}{12},\quad y'(0)=\frac{23}{24}$
\begin{solution}
$p(r)=4r^2-4r-3=(2r-3)(2r+1)= 4(r-3/2)(r+1/2)$;
$y=c_1e^{-x/2} +c_2e^{3x/2}$;\;
$y'=-\frac{c_1}{2}e^{-x/2} +\frac{3c_2}{2}e^{3x/2}$;\;
 $y(0)=\frac{13}{12}\Rightarrow c_1+c_2=\frac{13}{12},\
y'(0)=\frac{23}{24}\Rightarrow
-\frac{c_1}{2}+\frac{3c_2}{2}=\frac{23}{24}$;\;
$c_1=\frac{1}{3}, c_2=\frac{3}{4}$;\;
$y=\frac{e^{-x/2}}{3}+\frac{3e^{3x/2}}{4}$.
\end{solution}
\end{problem} 

\begin{problem}\label{exer:5.2.17}  Solve
the initial value problem.
$4y''-12y'+9y=0, \quad  y(0)=3,\quad y'(0)=\frac{5}{2}$
\end{problem} 

\begin{problem}\label{exer:5.2.18} Solve
the initial value problem and graph the solution.
$y''+7y'+12y=0, \quad  y(0)=-1,\quad y'(0)=0$
\begin{solution}
$p(r)=r^2+7r+12=(r+3)(r+4)$;\;
$y=c_1e^{-4x}+c_2e^{-3x}$;\;
$y'=-4c_1e^{-4x}-3c_2e^{-3x}$;\;
 $y(0)=-1\Rightarrow c_1+c_2=-1,\ y'(0)=0\Rightarrow -4c_1-3c_2=0$;\;
$c_1=3$, $c_2=-4$;\;
$y=3e^{-4x}-4e^{-3x}$.
\end{solution}
\end{problem} 

\begin{problem}\label{exer:5.2.19} Solve
the initial value problem and graph the solution.
$y''-6y'+9y=0, \quad  y(0)=0,\quad y'(0)=2$
\end{problem} 

\begin{problem}\label{exer:5.2.20} Solve
the initial value problem and graph the solution.
$36y''-12y'+y=0, \quad  y(0)=3,\quad y'(0)=\frac{5}{2}$
\begin{solution}
$p(r)=36r^2-12r+1=(6r-1)^2=36(r-1/6)^2$;\;
$y=e^{x/6}(c_1+c_2x)$;\;
$y'=\frac{e^{x/6}}{6}(c_1+c_2x)+c_2e^{x/6}$;\;
$y(0)=3\Rightarrow c_1=3,\ y'(0)=\frac{5}{2}\Rightarrow
\frac{c_1}{6}+c_2=\frac{5}{2}\Rightarrow c_2=2$;\;
$y=e^{x/6}(3+2x)$.
\end{solution}
\end{problem} 

\begin{problem}\label{exer:5.2.21} Solve
the initial value problem and graph the solution.
$y''+4y'+10y=0, \quad  y(0)=3,\quad y'(0)=-2$
\end{problem} 

\begin{problem}\label{exer:5.2.22}
\begin{enumerate}
\item % (a)
Suppose $y$  is a solution of the constant coefficient homogeneous
equation \begin{equation}\label{eq:eqA5.2.22}
ay''+by'+cy=0.
\end{equation}
Let $z(x)=y(x-x_0)$, where $x_0$ is an arbitrary real number.
Show that
$$
az''+bz'+cz=0.
$$
\begin{solution}
From (A), $ay''(x)+by'(x)+cy(x)=0$ for all $x$. Replacing $x$ by
$x-x_0$ yields (C) $ay''(x-x_0)+by'(x-x_0)+cy(x-x_0)=0$. If
$z(x)=y(x-x_0)$, then the chain rule implies that $z'(x)=y'(x-x_0)$
and $z''(x)=y''(x-x_0)$, so (C) is equivalent to $az''+bz'+cz=0$.
\end{solution}

\item % (b)
Let $z_1(x)=y_1(x-x_0)$ and $z_2(x)=y_2(x-x_0)$, where $\{y_1,y_2\}$
is a fundamental set of solutions of \ref{eq:eqA5.2.22}. Show that
$\{z_1,z_2\}$ is also a fundamental set of solutions of
\ref{eq:eqA5.2.22}.
\begin{solution}
 If  $\{y_1,y_2\}$  is a fundamental set of solutions of
(A) then Theorem~5.1.6  implies that
$y_2/y_1$ is nonconstant. Therefore,
$\frac{z_2(x)}{z_1(x)}=\frac{y_2(x-x_0)}{y_1(x-x_0)}$ is also
nonconstant, so Theorem~5.1.6 implies that  $\{z_1,z_2\}$ is a
fundamental set of solutions of (A).
\end{solution}

\item % (c)
The statement of
 Theorem~\ref{thmtype:5.2.1}  is convenient for solving an
initial value problem
$$
ay''+by'+cy=0, \quad  y(0)=k_0,\quad y'(0)=k_1,
$$
where the initial conditions are imposed at $x_0=0$.
However, if the  initial value problem is \begin{equation}\label{eq:eqB5.2.22}
ay''+by'+cy=0, \quad  y(x_0)=k_0,\quad y'(x_0)=k_1,
\end{equation}
where $x_0\ne0$, then  determining  the constants in
$$
y=c_1e^{r_1x}+c_2e^{r_2x}, \quad  y=e^{r_1x}(c_1+c_2x),$$
or
$$y=e^{\lambda x}(c_1\cos\omega x+c_2\sin\omega x)
$$
(whichever is applicable) is more complicated. Use your work above to restate
Theorem~\ref{thmtype:5.2.1} in a form more convenient for solving
\ref{eq:eqB5.2.22}.
\end{enumerate}
\begin{solution}
Let $p(r)=ar^2+br+c$ be the characteristic polynomial of {\rm(A)}.
Then:
\begin{itemize}
\item
If $p(r)=0$  has distinct real roots $r_1$ and $r_2$, then the general
solution of {\rm(A)} is
$$
y=c_1e^{r_1(x-x_0)}+c_2e^{r_2(x-x_0)}.
$$
\item
If $p(r)=0$  has a repeated root  $r_1$, then
the general solution of {\rm(A)} is
$$
y=e^{r_1(x-x_0)}(c_1+c_2(x-x_0)).
$$
\item % (c)
If $p(r)=0$  has complex conjugate roots $r_1=\lambda+i\omega$
and
$r_2=\lambda-i\omega$ $($where $\omega>0)$, then the general solution
of {\rm(A)} is
$$
y=e^{\lambda (x-x_0)}(c_1\cos\omega(x-x_0)+c_2\sin\omega(x-x_0)).
$$
\end{itemize}
\end{solution}
\end{problem}

\begin{problem}\label{exer:5.2.23}  Use a method
suggested by Exercise~\ref{exer:5.2.22} to solve the initial value
problem.
$y''+3y'+2y=0, \quad  y(1)=-1,\quad y'(1)=4$
\end{problem}

\begin{problem}\label{exer:5.2.24}  Use a method
suggested by Exercise~\ref{exer:5.2.22} to solve the initial value
problem.
$y''-6y'-7y=0, \quad  y(2)=-\frac{1}{3},\quad y'(2)=-5$
\begin{solution}
$p(r)=r^2-6r-7=(r-7)(r+1)$;
\begin{eqnarray*}
y&=&c_1e^{-(x-2)}+c_2e^{7(x-2)};\\
y'&=&-c_1e^{-(x-2)}+7c_2e^{7(x-2)};
\end{eqnarray*}
$y(2)=-\frac{1}{3}\Rightarrow c_1+c_2=-\frac{1}{3},\
y'(2)=-5\Rightarrow -c_1+7c_2=-5$;\;
$c_1=\frac{1}{3}, c_2=-\frac{2}{3}$;\;
$y=\frac{1}{3}e^{-(x-2)}-\frac{2}{3}e^{7(x-2)}$.
\end{solution}
\end{problem}

\begin{problem}\label{exer:5.2.25}  Use a method
suggested by Exercise~\ref{exer:5.2.22} to solve the initial value
problem.
$y''-14y'+49y=0, \quad  y(1)=2,\quad y'(1)=11$
\end{problem}

\begin{problem}\label{exer:5.2.26}  Use a method
suggested by Exercise~\ref{exer:5.2.22} to solve the initial value
problem.
$9y''+6y'+y=0, \quad  y(2)=2,\quad y'(2)=-\frac{14}{3}$
\begin{solution}
$p(r)=9r^2+6r+1=(3r+1)^2=9(r+1/3)^2$;
\begin{eqnarray*}
y&=&e^{-(x-2)/3}\left(c_1+c_2(x-2)\right);\\
y'&=&-\frac{1}{3}e^{-(x-2)/3}\left(c_1+c_2(x-2)\right)
+c_2e^{-(x-2)/3};
\end{eqnarray*}
$y(2)=2\Rightarrow c_1=2,\
y'(2)=-\frac{14}{3}\Rightarrow -\frac{c_1}{3}+c_2=-\frac{14}{3}
\Rightarrow c_2=-4$;\;
$y=e^{-(x-2)/3}\left(2-4(x-2)\right)$.
\end{solution}
\end{problem}

\begin{problem}\label{exer:5.2.27}  Use a method
suggested by Exercise~\ref{exer:5.2.22} to solve the initial value
problem.
$9y''+4y=0, \quad  y(\pi/4)=2,\quad y'(\pi/4)=-2$
\end{problem}

\begin{problem}\label{exer:5.2.28}  Use a method
suggested by Exercise~\ref{exer:5.2.22} to solve the initial value
problem.
$y''+3y=0, \quad  y(\pi/3)=2,\quad y'(\pi/3)=-1$
\begin{solution}
$p(r)=r^2+3$;
\begin{eqnarray*}
y&=&c_1\cos \sqrt3\left(x-\frac{\pi}{3}\right)+c_2\sin
\sqrt3\left(x-\frac{\pi}{3}\right);\\
y'&=&-\sqrt3c_1\sin
\sqrt3\left(x-\frac{\pi}{3}\right)+\sqrt3c_2\cos
\sqrt3\left(x-\frac{\pi}{3}\right);
\end{eqnarray*}
$y(\pi/3)=2\Rightarrow c_1=2,\ y'(\pi/3)=-1\Rightarrow
c_2=-\frac{1}{\sqrt3}$;\;
$$
y=2\cos \sqrt3\left(x-\frac{\pi}{3}\right)-\frac{1}{\sqrt3}\sin
\sqrt3\left(x-\frac{\pi}{3}\right).
$$
\end{solution}
\end{problem}

\begin{problem}\label{exer:5.2.29}
Prove: If the characteristic equation of \begin{equation}\label{eq:eqA5.2.29}
ay''+by'+cy=0
\end{equation}
has a repeated negative root or two roots with negative real parts, then
every solution of \ref{eq:eqA5.2.29} approaches zero as $x\to\infty$.
\end{problem}

\begin{problem}\label{exer:5.2.30}
Suppose the characteristic polynomial of $ay''+by'+cy=0$
 has distinct real roots $r_1$ and $r_2$. Use a method suggested
by Exercise~\ref{exer:5.2.22}  to find a formula for the solution
of
$$
ay''+by'+cy=0, \quad  y(x_0)=k_0,\quad y'(x_0)=k_1.
$$
\begin{solution}
$y$ is a solution of $ay''+by'+cy=0$  if and only if
\begin{eqnarray*}
y&=&r_1 c_1e^{r_1(x-x_0)}+r_2 e^{r_2(x-x_0)}\\
y'&=&r_1c_1e^{r_1(x-x_0)}+r_2e^{r_2}(x-x_0).
\end{eqnarray*}
Now $y_1(x_0)=k_0$ and $y_1'(x_0)=k_1\Rightarrow c_1+c_2=k_0,
r_1c_1+r_2c_2=k_1$. Therefore,$c_1=\frac{r_2k_0-k_1}{r_2-r_1}$
and $c_2=\frac{k_1-r_1k_0}{r_2-r_1}$. Substituting $c_1$ and $c_2$
into  the above equations for $y$ and y' yields
\begin{eqnarray*}
y&=&\frac{r_2k_0-k_1}{r_2-r_1}e^{r_1(x-x_0)}+
\frac{k_1-r_1k_0}{r_2-r_1}e^{r_2(x-x_0)}\\[2\jot]
&=&\frac{k_0}{r_2-r_1}\left(r_2e^{r_1(x-x_0)}-r_1e^{r_2(x-x_0)}\right)+\frac{k_1}{r_2-r_1} \left(e^{r_2(x-x_0)}-e^{r_1(x-x_0)}\right).
\end{eqnarray*}
\end{solution}
\end{problem}

\begin{problem}\label{exer:5.2.31}
Suppose the characteristic polynomial of $ay''+by'+cy=0$ has a
repeated real root $r_1$. Use a method suggested by
Exercise~\ref{exer:5.2.22} to find a formula for the solution of
$$
ay''+by'+cy=0, \quad  y(x_0)=k_0,\quad y'(x_0)=k_1.
$$
\end{problem}

\begin{problem}\label{exer:5.2.32}
Suppose the characteristic polynomial of $ay''+by'+cy=0$ has
complex conjugate roots $\lambda\pm i\omega$. Use a method suggested
by Exercise~\ref{exer:5.2.22} to find a formula for the solution of
$$
ay''+by'+cy=0, \quad  y(x_0)=k_0,\quad y'(x_0)=k_1.
$$
\begin{solution}
$y$ is a solution of $ay''+by'+cy=0$  if and only if
\begin{equation}
y=\lambda e^{\lambda(x-x_0)}\left(c_1\cos\omega(x-x_0)+c_2\sin\omega(x-x_0)\right)
\end{equation}
and
\begin{eqnarray*}
y'&=&\lambda
e^{\lambda(x-x_0)}\left(c_1\cos\omega(x-x_0)+c_2\sin\omega(x-x_0)\right)\\
&&
+\omega
e^{\lambda(x-x_0)}\left(-c_1\sin\omega(x-x_0)+c_2\cos\omega(x-x_0)\right).
\end{eqnarray*}
Now $y_1(x_0)=k_0\Rightarrow c_1=k_0$ and $y_1'(x_0)=k_1\Rightarrow
\lambda c_1+\omega c_2=k_1$,  so $c_2=\frac{k_1-\lambda
k_0}{\omega}$.
 Substituting $c_1$ and $c_2$
into
(A) yields
$$
y=e^{\lambda(x-x_0)}\left[k_0\cos\omega(x-x_0)+\left(\frac{k_1-\lambda k_0}{\omega}\right)\sin\omega(x-x_0)\right].
$$
\end{solution}
\end{problem}

\begin{problem}\label{exer:5.2.33}
Suppose the characteristic equation of \begin{equation}\label{eq:eqA5.2.33}
ay''+by'+cy=0
\end{equation}
has a repeated real root $r_1$. Temporarily, think of $e^{rx}$
as a function of two real variables $x$ and $r$.
\begin{enumerate}
\item % (a)
Show that \begin{equation}\label{eq:eqB5.2.33}
a\frac{\partial^2}{\partial^2 x}(e^{rx})+b\frac{\partial}{\partial
x}(e^{rx}) +ce^{rx}=a(r-r_1)^2e^{rx}.
\end{equation}
\item % (b)
Differentiate \ref{eq:eqB5.2.33} with respect to $r$  to obtain \begin{equation}\label{eq:eqC5.2.33}
a\frac{\partial}{\partial r}\left(\frac{\partial^2}{\partial^2
x}(e^{rx})\right)+b\frac{\partial}{\partial r}\left(\frac{\partial
}{\partial x}(e^{rx})\right)
+c(xe^{rx})=[2+(r-r_1)x]a(r-r_1)e^{rx}.
\end{equation}
\item % (c)
Reverse the orders of the partial differentiations in the first two
terms on the left side of
\ref{eq:eqC5.2.33} to obtain \begin{equation}\label{eq:eqD5.2.33}
a\frac{\partial^2}{\partial x^2}(xe^{rx})+b\frac{\partial}{\partial
x}(xe^{rx})+c(xe^{rx})=[2+(r-r_1)x]a(r-r_1)e^{rx}.
\end{equation}
\item % (d)
Set $r=r_1$ in \ref{eq:eqC5.2.33} and \ref{eq:eqD5.2.33}
to see that $y_1=e^{r_1x}$ and $y_2=xe^{r_1x}$ are solutions
of \ref{eq:eqA5.2.33}.
\end{enumerate}
\end{problem}

\begin{problem}\label{exer:5.2.34}
In calculus you learned that $e^u$, $\cos u$, and $\sin u$
can be represented by the infinite series \begin{equation}\label{eq:eqA5.2.34}
e^u=\sum_{n=0}^\infty \frac{u^n}{n!}
=1+\frac{u}{1!}+\frac{u^2}{2!}+\frac{u^3}{3!}+\cdots+\frac{u^n}{n!}+\cdots,
\end{equation}
\begin{equation}\label{eq:eqB5.2.34}
\cos u=\sum_{n=0}^\infty (-1)^n\frac{u^{2n}}{(2n)!}
=1-\frac{u^2}{2!}+\frac{u^4}{4!}+\cdots+(-1)^n\frac{u^{2n}}{(2n)!}
+\cdots,
\end{equation}
and \begin{equation}\label{eq:eqC5.2.34}
\sin u=\sum_{n=0}^\infty (-1)^n\frac{u^{2n+1}}{(2n+1)!}
=u-\frac{u^3}{3!}+\frac{u^5}{5!}+\cdots+(-1)^n
\frac{u^{2n+1}}{(2n+1)!}
+\cdots
\end{equation}
for all real values of $u$.
Even though you have previously considered \ref{eq:eqA5.2.34}
only for real values of $u$, we can set $u=i\theta$, where $\theta$ is
real, to obtain \begin{equation}\label{eq:eqD5.2.34}
e^{i\theta}=\sum_{n=0}^\infty \frac{(i\theta)^n}{n!}.
\end{equation}
Given the proper background in the theory of infinite series with
complex terms, it can be shown that the series in
\ref{eq:eqD5.2.34} converges for all real $\theta$.
\begin{enumerate}
\item % (a)
Recalling that $i^2=-1,$
write enough  terms of the sequence $\{i^n\}$ to convince yourself
that the sequence is repetitive:
$$
1,i,-1,-i,1,i,-1,-i,1,i,-1,-i,1,i,-1,-i,\cdots.
$$
Use this to group the terms in \ref{eq:eqD5.2.34} as
\begin{eqnarray*}
e^{i\theta}&=&\left(1-\frac{\theta^2}{2}+\frac{\theta^4}{4}+\cdots\right)
+i\left(\theta-\frac{\theta^3}{3!}+\frac{\theta^5}{5!}+\cdots\right) \\
&=&\sum_{n=0}^\infty (-1)^n\frac{\theta^{2n}}{(2n)!}
+i\sum_{n=0}^\infty (-1)^n\frac{\theta^{2n+1}}{(2n+1)!}.
\end{eqnarray*}
By comparing this result with \ref{eq:eqB5.2.34} and
\ref{eq:eqC5.2.34}, conclude that \begin{equation}\label{eq:eqE5.2.34}
e^{i\theta}=\cos\theta+i\sin\theta.
\end{equation}
This is
\href{http://www-history.mcs.st-and.ac.uk/Mathematicians/Euler.html}
{Euler's identity}.
\item % (b)
Starting from
$$
e^{i\theta_1}e^{i\theta_2}=(\cos\theta_1+i\sin\theta_1)
(\cos\theta_2+i\sin\theta_2),
$$
collect the real part (the terms not multiplied by $i$)
and the imaginary part (the terms  multiplied by $i$) on the
right, and use the trigonometric identities
\begin{eqnarray*}
\cos(\theta_1+\theta_2)&=&\cos\theta_1\cos\theta_2-\sin\theta_1\sin\theta_2\\
\sin(\theta_1+\theta_2)&=&\sin\theta_1\cos\theta_2+\cos\theta_1\sin\theta_2
\end{eqnarray*}
to verify that
$$
e^{i(\theta_1+\theta_2)}=e^{i\theta_1}e^{i\theta_2},
$$
as you would expect from the use of the exponential notation
$e^{i\theta}$.
\begin{solution}
\begin{eqnarray*}
e^{i\theta_1}e^{i\theta_2}&=&(\cos\theta_1+i\sin\theta_1)
(\cos\theta_2+i\sin\theta_2)\\
&=&(\cos\theta_1\cos\theta_2-\sin\theta_1\sin\theta_2)
+i(\sin\theta_1\cos\theta_2+\cos\theta_1\sin\theta_2)\\
&=&\cos(\theta_1+\theta_2)+i\sin(\theta_1+\theta_2)=e^{i(\theta_1+\theta_2)}.
\end{eqnarray*}
\end{solution}

\item % (c)
If  $\alpha$ and $\beta$ are real numbers,
define \begin{equation}\label{eq:eqF5.2.34}
e^{\alpha+i\beta}=e^\alpha e^{i\beta}=e^\alpha(\cos\beta+i\sin\beta).
\end{equation}
Show that if $z_1=\alpha_1+i\beta_1$ and $z_2=\alpha_2+i\beta_2$ then
$$
e^{z_1+z_2}=e^{z_1}e^{z_2}.
$$
%\begin{solution}
%\begin{eqnarray*}
%e^{z_1+z_2}&=&e^{(\alpha_1+i\beta_1)+(\alpha_2+i\beta_2)}
%=e^{(\alpha_1+\alpha_2)+i(\beta_1+\beta_2)}\\
%&=&e^{(\alpha_1+\alpha_2)}e^{i(\beta_1+\beta_2)} \text{ (from (F) with } \alpha=\alpha_1+\alpha_2 \text{ and } \beta=\beta_1+\beta_2)\\
%&=&e^{\alpha_1}e^{\alpha_2}e^{i(\beta_1+\beta_2)}
%\text{ (property of the real--valued exponential function)}\\
%&=&e^{\alpha_1}e^{\alpha_2}e^{i\beta_1}e^{i\beta_2}\text{ (from \part{b})}\\
%&=&e^{\alpha_1}e^{i\beta_1}e^{\alpha_2}e^{i\beta_2}
%=e^{\alpha_1+i\beta_1}e^{\alpha_2+i\beta_2}=e^{z_1}e^{z_2}.
%\end{eqnarray*}
%\end{solution}


\item % (d)
Let $a$, $b$, and $c$ be real numbers, with $a\ne0$. Let $z=u+iv$
where $u$ and $v$ are real-valued functions of $x$. Then we say that
$z$ is a solution of \begin{equation}\label{eq:eqG5.2.34}
ay''+by'+cy=0
\end{equation}
if $u$ and $v$ are both solutions of \ref{eq:eqG5.2.34}.
Use Part C of  Theorem~\ref{thmtype:5.2.1} to verify that if the
characteristic equation of \ref{eq:eqG5.2.34} has complex conjugate roots
$\lambda\pm i\omega$ then
$z_1=e^{(\lambda+i\omega)x}$ and $z_2=e^{(\lambda-i\omega)x}$
are both solutions of \ref{eq:eqG5.2.34}.
\begin{solution}
The real and imaginary parts of $z_1=e^{(\lambda+i\omega)x}$ are
$u_1=e^{\lambda x}\cos\omega x$ and $v_1=e^{\lambda x}\sin\omega x$,
which are both solutions of $ay''+by'+cy=0$, by
Theorem~5.2.1(c). Similarly, the real and imaginary parts of
$z_2=e^{(\lambda-i\omega)x}$ are $u_2=e^{\lambda x}\cos(-\omega
x)=e^{\lambda x}\cos\omega x$ and $v_1=e^{\lambda x}\sin(-\omega
x)=-e^{\lambda x}\sin\omega x$, which are both solutions of
$ay''+by'+cy=0$, by Theorem~5.2.1,(c).
\end{solution}
\end{enumerate}


\end{problem}
\end{document}