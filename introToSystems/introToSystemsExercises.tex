\documentclass{ximera}
%% You can put user macros here
%% However, you cannot make new environments

%\listfiles

% Get the 'old' hints/expandables, for use on ximera.osu.edu
%\def\xmNotHintAsExpandable{true}
%\def\xmNotExpandableAsAccordion{true}



%\graphicspath{{./}{firstExample/}{secondExample/}}
\graphicspath{{./}
{aboutDiffEq/}
{applicationsLeadingToDiffEq/}
{applicationsToCurves/}
{autonomousSecondOrder/}
{basicConcepts/}
{bernoulli/}
{constCoeffHomSysI/}
{constCoeffHomSysII/}
{constCoeffHomSysIII/}
{constantCoeffWithImpulses/}
{constantCoefficientHomogeneousEquations/}
{convolution/}
{coolingActivity/}
{directionFields/}
{drainingTank/}
{epidemicActivity/}
{eulersMethod/}
{exactEquations/}
{existUniqueNonlinear/}
{frobeniusI/}
{frobeniusII/}
{frobeniusIII/}
{global.css/}
{growthDecay/}
{heatingCoolingActivity/}
{higherOrderConstCoeff/}
{homogeneousLinearEquations/}
{homogeneousLinearSys/}
{improvedEuler/}
{integratingFactors/}
{interactExperiment/}
{introToLaplace/}
{introToSystems/}
{inverseLaplace/}
{ivpLaplace/}
{laplaceTable/}
{lawOfCooling/}
{linSysOfDiffEqs/}
{linearFirstOrderDiffEq/}
{linearHigherOrder/}
{mixingProblems/}
{motionUnderCentralForce/}
{nonHomogeneousLinear/}
{nonlinearToSeparable/}
{odesInSage/}
{piecewiseContForcingFn/}
{population/}
{reductionOfOrder/}
{regularSingularPts/}
{reviewOfPowerSeries/}
{rlcCircuit/}
{rungeKutta/}
{secondLawOfMotion/}
{separableEquations/}
{seriesSolNearOrdinaryPtI/}
{seriesSolNearOrdinaryPtII/}
{simplePendulum/}
{springActivity/}
{springProblemsI/}
{springProblemsII/}
{undCoeffHigherOrderEqs/}
{undeterminedCoeff/}
{undeterminedCoeff2/}
{unitStepFunction/}
{varParHigherOrder/}
{varParamNonHomLinSys/}
{variationOfParameters/}
}


\usepackage{tikz}
%\usepackage{tkz-euclide}
\usepackage{tikz-3dplot}
\usepackage{tikz-cd}
\usetikzlibrary{shapes.geometric}
\usetikzlibrary{arrows}
\usetikzlibrary{decorations.pathmorphing,patterns}
\usetikzlibrary{backgrounds} % added by Felipe
% \usetkzobj{all}   % NOT ALLOWED IN RECENT TeX's ...
\pgfplotsset{compat=1.13} % prevents compile error.

\pdfOnly{\renewcommand{\theHsection}{\thepart.section.\thesection}}  %% MAKES LINKS WORK should be added to CLS
\pdfOnly{\renewcommand{\part}[1]{\chapterstyle\title{#1}\begin{abstract}\end{abstract}\maketitle\def\thechaptertitle{#1}}}


\renewcommand{\vec}[1]{\mathbf{#1}}
\newcommand{\RR}{\mathbb{R}}
\providecommand{\dfn}{\textit}
\renewcommand{\dfn}{\textit}
\newcommand{\dotp}{\cdot}
\newcommand{\id}{\text{id}}
\newcommand\norm[1]{\left\lVert#1\right\rVert}
\newcommand{\dst}{\displaystyle}
 
\newtheorem{general}{Generalization}
\newtheorem{initprob}{Exploration Problem}

\tikzstyle geometryDiagrams=[ultra thick,color=blue!50!black]

\usepackage{mathtools}

\title{Exercises} \license{CC BY-NC-SA 4.0}

\begin{document}

\begin{abstract}
\end{abstract}
\maketitle

\begin{onlineOnly}
\section*{Exercises}
\end{onlineOnly}

\begin{problem}\label{exer:10.1.1}
Tanks $T_1$ and $T_2$ contain 50 gallons and 100 gallons of salt
solutions, respectively. A solution with 2 pounds of salt per gallon
is pumped into $T_1$ from an external source at $1$~gal/min, and a
solution with $3$ pounds of salt per gallon is pumped into $T_2$ from an
external source at $2$~gal/min. The solution from $T_1$ is pumped into
$T_2$ at $3$ gal/min, and the solution from $T_2$ is pumped into $T_1$
at $4$ gal/min. $T_1$ is drained at $2$~gal/min and $T_2$ is drained at
$1$~gal/min. Let $Q_1(t)$ and $Q_2(t)$ be the number of pounds of salt
in $T_1$ and $T_2$, respectively, at time $t>0$. Derive a system of
differential equations for $Q_1$ and $Q_2$. Assume that both mixtures
are well stirred.
\end{problem}


\begin{problem}\label{exer:10.1.2}
Two 500 gallon tanks $T_1$ and $T_2$ initially contain 100
gallons each of salt solution. A solution with $2$ pounds of salt per
gallon is pumped into $T_1$ from an external source at $6$~gal/min, and
a solution with $1$ pound of salt per gallon is pumped into $T_2$ from
an external source at $5$~gal/min. The solution from $T_1$ is pumped
into $T_2$ at $2$ gal/min, and the solution from $T_2$ is pumped into
$T_1$ at $1$ gal/min. Both tanks are drained at $3$~gal/min. Let $Q_1(t)$
and $Q_2(t)$ be the number of pounds of salt in $T_1$ and $T_2$,
respectively, at time $t>0$. Derive a system of differential equations
for $Q_1$ and $Q_2$ that's valid until a tank is about to overflow.
Assume that both mixtures are well stirred.

\begin{solution}
    $Q_1'=(\mbox{rate in})_1-(\mbox{rate out})_1$ and $Q_2'=(\mbox{rate
in})_2-(\mbox{rate out})_2$.

The volumes of the solutions in $T_1$ and $T_2$ are $V_1(t)=100+2t$
and $V_2(t)=100+3t$, respectively. $T_1$ receives salt from the
external source at the rate of $\mbox{(2 lb/gal) }\times\mbox{
(6~gal/min)}=\mbox{12 lb/min}$, and from $T_2$ at the rate of
$\mbox{(lb/gal in }T_2)\times\mbox{ (1~gal/min)
}=\frac{1}{100+3t}Q_2 \mbox{ lb/min}$. Therefore, (A) $\mbox{(rate
in)}_1= 12+\frac{1}{100+3t}Q_2$. Solution leaves $T_1$ at 5~gal/min,
since 3~gal/min are drained and 2~gal/min are pumped to $T_2$; hence
(B) $(\mbox{rate out})_1=(\mbox{ lb/gal in T}_1)\times
\mbox{(5~gal/min) }
=\frac{1}{100+2t}Q_1\times5=\frac{5}{100+2t}Q_1$. Now (A) and (B)
imply that (C) $Q_1'=12-\frac{5}{100+2t}Q_1+\frac{1}{100+3t}Q_2$.

$T_2$ receives salt from the external source at the rate of $\mbox{(1
lb/gal) }\times\mbox{ (5~gal/min)}=\mbox{ 5 lb/min}$, and from $T_1$
at the rate of $\mbox{(lb/gal in }T_1)\times\mbox{ (2~gal/min)
}=\frac{1}{100+2t}Q_1\times2=\frac{1}{50+t}Q_1 \mbox{ lb/min}$.
Therefore, (D) $\mbox{(rate in)}_2= 5+\frac{1}{50+t}Q_1$. Solution
leaves $T_2$ at 4~gal/min, since 3~gal/min are drained and 1~gal/min
is pumped to $T_1$; hence (E) $(\mbox{rate out})_2=(\mbox{ lb/gal in
T}_2)\times \mbox{(4~gal/min) }
=\frac{1}{100+3t}Q_2\times4=\frac{4}{100+3t}Q_2$. Now (D) and (E)
imply that (F) $Q_2'=5+\frac{1}{50+t}Q_1-\frac{4}{100+3t}Q_2$. Now
(C) and (F) form the desired system.
\end{solution}
\end{problem}

\begin{problem}\label{exer:10.1.3}
A mass $m_1$ is suspended from a rigid support on a spring $S_1$ with
spring constant $k_1$ and damping constant $c_1$. A second mass $m_2$
is suspended from the first on a spring $S_2$ with spring constant
$k_2$ and damping constant $c_2$, and a third mass $m_3$ is suspended
from the second on a spring $S_3$ with spring constant $k_3$ and
damping
constant $c_3$. Let $y_1=y_1(t)$, $y_2=y_2(t)$, and $y_3=y_3(t)$ be the
displacements of the three masses from their equilibrium positions at
time $t$, measured positive upward. Derive a system of differential
equations for $y_1$, $y_2$ and $y_3$, assuming that the masses of the
springs are negligible and that vertical external forces $F_1$, $F_2$,
and
$F_3$ also act on the masses.
\end{problem}

\begin{problem}\label{exer:10.1.4}
Let ${\bf X}=x\,{\bf i}+y\,{\bf j}+z\,{\bf k}$ be the position vector
of an object with mass $m$, expressed in terms of a rectangular
coordinate system with origin at Earth's center
(Figure~\ref{figure:10.1.3}). Derive a system of differential equations for
$x$, $y$, and $z$, assuming that the object moves under Earth's
gravitational force (given by Newton's law of gravitation, as in
Example~\ref{example:10.1.3} ) and a resistive force proportional to the
speed of the object. Let $\alpha$ be the constant of proportionality.

\begin{solution}
    $m{\bf X}''=-\alpha{\bf X'}-mgR^2\frac{{\bf X}}{||{\bf X}||^3}$;
see Example~~\ref{example:10.1.3}.
\end{solution}
\end{problem}



\begin{problem}\label{exer:10.1.5}
 Rewrite the given system as a first order system.

\begin{enumerate}
    \item $\begin{array}{rcl} x''' &=& f(t,x,y,y')\\ 
    y'' &=&g(t,y,y') \end{array}$
    \item $\begin{array}{rcl} u' &=&f(t,u,v,v',w')\\ v''&=&g(t,u,v,v',w)
\\w''&=&h(t,u,v,v',w,w')\end{array}$
\item $y''' =f(t,y,y',y'')$
\item $y^{(4)} = f(t,y)$
\item $\begin{array}{rcl} x'' &=& f(t,x,y)\\ y'' &=&
g(t,x,y) \end{array}$
\end{enumerate}


\end{problem}

\begin{problem}\label{exer:10.1.6}
Rewrite the system (\ref{eq:10.1.14}) of differential equations derived in
Example~\ref{example:10.1.3}   as a first order system.
\end{problem}

\begin{problem}\label{exer:10.1.7}
Formulate a version of Euler's method (Section~3.1)
for the numerical solution of the initial value problem
$$
\begin{array}{rcl}
y_1'&=&g_1(t,y_1,y_2),\quad y_1(t_0)=y_{10},\\
y_2'&=&g_2(t,y_1,y_2),\quad y_2(t_0)=y_{20},
\end{array}
$$
on an interval $[t_0,b]$.
\end{problem}


\begin{problem}\label{exer:10.1.8}
Formulate a version of the improved Euler method
(Section~3.2)
for the numerical solution of the initial value problem
$$
\begin{array}{rcl}
y_1'&=&g_1(t,y_1,y_2),\quad y_1(t_0)=y_{10},\\
y_2'&=&g_2(t,y_1,y_2),\quad y_2(t_0)=y_{20},
\end{array}
$$
on an interval $[t_0,b]$.

\begin{solution}

    $\begin{array}{rcl}
I_{1i}&=&g_1(t_i,y_{1i},y_{2i}),\\
J_{1i}&=&g_2(t_i,y_{1i},y_{2i}),\\
I_{2i}&=&g_1\left(t_i+h,y_{1i}+hI_{1i},y_{2i}+hJ_{1i}\right),\\
J_{2i}&=&g_2\left(t_i+h,y_{1i}+hI_{1i},y_{2i}+hJ_{1i}\right),\\
y_{1,i+1}&=&y_{1i}+\frac{h}{2}(I_{1i}+I_{2i}),\\
y_{2,i+1}&=&y_{2i}+\frac{h}{2}(J_{1i}+J_{2i}).
\end{array}$
\end{solution}
\end{problem}

\end{document}