\documentclass{ximera}
%% You can put user macros here
%% However, you cannot make new environments

%\listfiles

% Get the 'old' hints/expandables, for use on ximera.osu.edu
%\def\xmNotHintAsExpandable{true}
%\def\xmNotExpandableAsAccordion{true}



%\graphicspath{{./}{firstExample/}{secondExample/}}
\graphicspath{{./}
{aboutDiffEq/}
{applicationsLeadingToDiffEq/}
{applicationsToCurves/}
{autonomousSecondOrder/}
{basicConcepts/}
{bernoulli/}
{constCoeffHomSysI/}
{constCoeffHomSysII/}
{constCoeffHomSysIII/}
{constantCoeffWithImpulses/}
{constantCoefficientHomogeneousEquations/}
{convolution/}
{coolingActivity/}
{directionFields/}
{drainingTank/}
{epidemicActivity/}
{eulersMethod/}
{exactEquations/}
{existUniqueNonlinear/}
{frobeniusI/}
{frobeniusII/}
{frobeniusIII/}
{global.css/}
{growthDecay/}
{heatingCoolingActivity/}
{higherOrderConstCoeff/}
{homogeneousLinearEquations/}
{homogeneousLinearSys/}
{improvedEuler/}
{integratingFactors/}
{interactExperiment/}
{introToLaplace/}
{introToSystems/}
{inverseLaplace/}
{ivpLaplace/}
{laplaceTable/}
{lawOfCooling/}
{linSysOfDiffEqs/}
{linearFirstOrderDiffEq/}
{linearHigherOrder/}
{mixingProblems/}
{motionUnderCentralForce/}
{nonHomogeneousLinear/}
{nonlinearToSeparable/}
{odesInSage/}
{piecewiseContForcingFn/}
{population/}
{reductionOfOrder/}
{regularSingularPts/}
{reviewOfPowerSeries/}
{rlcCircuit/}
{rungeKutta/}
{secondLawOfMotion/}
{separableEquations/}
{seriesSolNearOrdinaryPtI/}
{seriesSolNearOrdinaryPtII/}
{simplePendulum/}
{springActivity/}
{springProblemsI/}
{springProblemsII/}
{undCoeffHigherOrderEqs/}
{undeterminedCoeff/}
{undeterminedCoeff2/}
{unitStepFunction/}
{varParHigherOrder/}
{varParamNonHomLinSys/}
{variationOfParameters/}
}


\usepackage{tikz}
%\usepackage{tkz-euclide}
\usepackage{tikz-3dplot}
\usepackage{tikz-cd}
\usetikzlibrary{shapes.geometric}
\usetikzlibrary{arrows}
\usetikzlibrary{decorations.pathmorphing,patterns}
\usetikzlibrary{backgrounds} % added by Felipe
% \usetkzobj{all}   % NOT ALLOWED IN RECENT TeX's ...
\pgfplotsset{compat=1.13} % prevents compile error.

\pdfOnly{\renewcommand{\theHsection}{\thepart.section.\thesection}}  %% MAKES LINKS WORK should be added to CLS
\pdfOnly{\renewcommand{\part}[1]{\chapterstyle\title{#1}\begin{abstract}\end{abstract}\maketitle\def\thechaptertitle{#1}}}


\renewcommand{\vec}[1]{\mathbf{#1}}
\newcommand{\RR}{\mathbb{R}}
\providecommand{\dfn}{\textit}
\renewcommand{\dfn}{\textit}
\newcommand{\dotp}{\cdot}
\newcommand{\id}{\text{id}}
\newcommand\norm[1]{\left\lVert#1\right\rVert}
\newcommand{\dst}{\displaystyle}
 
\newtheorem{general}{Generalization}
\newtheorem{initprob}{Exploration Problem}

\tikzstyle geometryDiagrams=[ultra thick,color=blue!50!black]

\usepackage{mathtools}

\title{Exercises} \license{CC BY-NC-SA 4.0}

\begin{document}

\begin{abstract}
\end{abstract}
\maketitle

\begin{onlineOnly}
\section*{Exercises}
\end{onlineOnly}

\begin{problem}\label{exer:10.3.1}
Prove: If ${\bf y}_1$, ${\bf y}_2$, \dots, ${\bf y}_n$ are solutions of
${\bf y}'=A(t){\bf y}$ on  $(a,b)$, then any linear
combination of ${\bf y}_1$, ${\bf y}_2$, \dots, ${\bf y}_n$ is also a
solution of ${\bf y}'=A(t){\bf y}$ on $(a,b)$.
\end{problem}

\begin{problem}\label{exer:10.3.2}
In Section~5.1 the Wronskian of two
solutions $y_1$ and $y_2$ of the scalar second order equation
$$
P_0(x)y''+P_1(x)y'+P_2(x)y=0
\quad\quad\text{(A)}
$$
was defined to be
$$
W=\begin{vmatrix} y_1&y_2 \\ y'_1&y'_2\end{vmatrix}.
$$
\begin{enumerate}
\item % (a)
Rewrite  (A)   as a system of first order equations and show
that $W$ is
the Wronskian (as defined in this section) of two solutions of this system.

\begin{solution}
    The system equivalent of (A) is $$\text{(B)}\quad\quad {\bf
y}'=-\frac{1}{P_0(x)}\begin{bmatrix}0&1\\P_2(x)&P_1(x)\end{bmatrix}{\bf y}$$, where
${\bf y}=\begin{bmatrix}y\\y'\end{bmatrix}$. Let ${\bf y}_1=\begin{bmatrix}y_1\\y_1'\end{bmatrix}$ and
 ${\bf y}_1=\begin{bmatrix}y_2\\y_2'\end{bmatrix}$. Then the Wronskian of $\{{\bf
y}_1,{\bf y}_2\}$ as defined in this section is
$\begin{bmatrix}y_1&y_2\\y_1'&y_2'\end{bmatrix}=W$.
\end{solution}

\item % (b)
Apply Eqn.~\ref{eq:10.3.6} to the system  derived in part (a), and
show that
$$
W(x)=W(x_0)\exp\left\{-\int^x_{x_0}\frac{P_1(s)}{ P_0(s)}\,
ds\right\},
$$
which is the form of Abel's formula given in
Theorem~\ref{thmtype:9.1.3}.

\begin{solution}
    The trace of the matrix in (B) is $-P_1(x)/P_0(x)$, so Eqn.~\ref{eq:10.3.6} implies
that
$W(x)=W(x_0)\exp\left\{-\int^x_{x_0}\frac{P_1(s)}{P_0(s)}\,
ds\right\}$.
\end{solution}

\end{enumerate}
\end{problem}

\begin{problem}\label{exer:10.3.3}
In Section~9.1 the Wronskian of $n$
solutions $y_1$, $y_2$, \dots, $y_n$  of the $n-$th order
equation
$$
P_0(x)y^{(n)}+P_1(x)y^{(n-1)}+\cdots+P_n(x)y=0
\text{(A)}
$$
was defined to be
$$W=\begin{vmatrix}
           y_1&y_2&\cdots&y_n \\
y'_1&y'_2&\cdots&y_n'\\
\vdots&\vdots&\ddots&\vdots\\
y_1^{(n-1)}&y_2^{(n-1)}&\cdots&y_n^{(n-1)}
         \end{vmatrix}$$


\begin{enumerate}
\item % (a)
Rewrite (A) as a system of first order equations and
show that $W$ is the Wronskian (as defined in this section) of $n$
solutions of this system.
\item % (b)
Apply Eqn.~\ref{eq:10.3.6} to the system  derived in part (a), and
show that
$$
W(x)=W(x_0)\exp\left\{-\int^x_{x_0}\frac{P_1(s)}{ P_0(s)}\,
ds\right\},
$$
which is the form of Abel's formula given in
Theorem~\ref{thmtype:9.1.3}.
\end{enumerate}

\end{problem}


\begin{problem}\label{exer:10.3.4}
 Suppose
$$
{\bf y}_1=\begin{bmatrix}y_{11}\\y_{21}\end{bmatrix}\mbox{\quad and \quad}
{\bf y}_2=\begin{bmatrix}y_{12}\\y_{22}\end{bmatrix}
$$
are solutions of the $2\times 2$ system ${\bf y}'=A{\bf y}$ on
$(a,b)$, and let
$$
 Y=\begin{bmatrix} y_{11} & y_{12}\\ y_{21} & y_{22}\end{bmatrix}\mbox{\quad and \quad}
W=\begin{vmatrix} y_{11}&y_{12}\\y_{21}&y_{22}\end{vmatrix};
$$ thus, $W$ is the Wronskian of $\{{\bf y}_1,{\bf y}_2\}$.
\begin{enumerate}
\item % (a)
 Deduce from the definition of  determinant that
$$
W'=\left|\begin{array}{cc} {y'_{11}}&{y'_{12}}\\ {y_{21}}&
{y_{22}}\end{array}\right|
+\left|\begin{array}{cc} {y_{11}}&{y_{12}}\\
 {y'_{21}}&{y'_{22}}\end{array}\right|.
$$
\begin{solution}
    See the solution of Exercise~{exer:9.1.18}.
\end{solution}
\item % (b)
 Use the equation $Y'=A(t)Y$ and
the definition of matrix multiplication to show that
$$
[y'_{11}\quad y'_{12}]=a_{11} [y_{11}\quad y_{12}]+a_{12} [y_{21}
\quad y_{22}]
$$
and
$$
[y'_{21}\quad  y'_{22}]=a_{21} [y_{11}\quad y_{12}]+a_{22}
[y_{21}\quad y_{22}].
$$
\item % (c)
 Use  properties of determinants to deduce from part (a) and part (b)
 that
$$
\left|\begin{array}{cc} {y'_{11}}&{y'_{12}}\\ {y_{21}}&
{y_{22}}\end{array}\right|=a_{11}W\mbox{\quad and \quad}
\left|\begin{array}{cc} {y_{11}}&{y_{12}}\\
 {y'_{21}}&{y'_{22}}\end{array}\right|=a_{22}W.
$$

\begin{solution}
    $
\begin{vmatrix}
y_{11}'&y_{12}'\\y_{21}&y_{22}
\end{vmatrix}=
\begin{vmatrix}
a_{11}y_{11}+a_{12}y_{21}&a_{11}y_{12}+a_{12}y_{22}\\
y_{21}&y_{22}
\end{vmatrix}=
a_{11}\begin{vmatrix}
y_{11}&y_{12}\\y_{21}&y_{22}
\end{vmatrix}
+a_{12}\begin{vmatrix}
y_{21}&y_{22}\\y_{21}&y_{22}
\end{vmatrix}=a_{11}W+a_{12}0=a_{11}W$.
Similarly,
$
\begin{vmatrix}
y_{11}&y_{12}\\y_{21}'&y_{22}'
\end{vmatrix}=a_{22}W$.
\end{solution}

\item % (d)
 Conclude from part (c) that
$$
W'=(a_{11}+a_{22})W,
$$
and use this to show that if $a<t_0<b$  then
$$
W(t)=W(t_0)\exp\left(\int^t_{t_0} \left[a_{11}(s)+a_{22} (s)
\right]\, ds\right)\quad a<t<b.
$$
\end{enumerate}

\end{problem}

\begin{problem}\label{exer:10.3.5}
Suppose  the $n\times n$ matrix $A=A(t)$ is continuous on $(a,b)$. Let
$$
 Y=
\begin{bmatrix} y_{11}&y_{12}&\cdots&y_{1n} \\
y_{21}&y_{22}&\cdots&y_{2n} \\
\vdots&\vdots&\ddots&\vdots \\
y_{n1}&y_{n2}&\cdots&y_{nn}
\end{bmatrix},
$$
where the columns of $Y$ are solutions of ${\bf y}'=A(t){\bf y}$. Let
$$
r_i=[y_{i1}\, y_{i2}\, \dots\, y_{in}]
$$
be the $i$th row of $Y$, and let $W$ be the determinant of $Y$.

\begin{enumerate}
\item % (a)
 Deduce from the definition of  determinant that
$$
W'=W_1+W_2+\cdots+W_n,
$$
where, for $1 \le m \le n$, the $i$th row of $W_m$ is $r_i$ if $i \ne m$,
and $r'_m$ if $i=m$.

\item % (b)
 Use the equation $Y'=A Y$
and the definition of matrix multiplication to show that
$$
r'_m=a_{m1}r_1+a_{m2} r_2+\cdots+a_{mn}r_n.
$$

\item % (c)
 Use  properties of determinants to deduce
from part (b) that
$$
\det (W_m)=a_{mm}W.
$$

\item % (d)
 Conclude from part (a) and part (c) that
$$
W'=(a_{11}+a_{22}+\cdots+a_{nn})W,
$$
and use this to  show that  if $a<t_0<b$ then
$$
W(t)=W(t_0)\exp\left(
\int^t_{t_0}\big[a_{11}(s)+a_{22}(s)+\cdots+a_{nn}(s)]\,
ds\right), \quad a < t < b.
$$
\end{enumerate}
\end{problem}

\begin{problem}\label{exer:10.3.6}
Suppose the $n\times n$ matrix $A$ is continuous on $(a,b)$
and $t_0$ is a point in  $(a,b)$. Let $Y$  be a fundamental matrix for
${\bf y}'=A(t){\bf y}$ on  $(a,b)$.
\begin{enumerate}
\item % (a)
Show that $Y(t_0)$  is invertible.

\begin{solution}
    From  the equivalence of Theorem~\ref{thmtype:10.3.3} parts (b) and (e)
$Y(t_0)$ is invertible.
\end{solution}

\item % (b)
Show that if ${\bf k}$ is an arbitrary $n$-vector then the solution
of the initial value problem
$$
{\bf y}'=A(t){\bf y},\quad {\bf y}(t_0)={\bf k}
$$
is
$$
{\bf y}=Y(t)Y^{-1}(t_0){\bf k}.
$$

\begin{solution}
    From  the equivalence of Theorem~\ref{thmtype:10.3.3} parts (a) and (b),
the solution of the initial value problem is ${\bf y}=Y(t){\bf
c}$, where ${\bf c}$ is a constant vector. To satisfy ${\bf
y}(t_0)={\bf k}$, we must have
$Y(t_0){\bf c}={\bf k}$, so ${\bf c}=Y^{-1}(t_0){\bf k}$
and ${\bf y}=Y^{-1}(t_0)Y(t){\bf k}$.
\end{solution}
\end{enumerate}
\end{problem}

\begin{problem}\label{exer:10.3.7}
Let
$$
 A=\begin{bmatrix}2 & 4\\4 & 2\end{bmatrix}, \quad {\bf y}_1=\begin{bmatrix} e^{6t} \\
e^{6t}
\end{bmatrix}, \quad {\bf y}_2=\begin{bmatrix}
e^{-2t} \\
-e^{-2t}\end{bmatrix}, \quad {\bf k}=\begin{bmatrix}-3
\\ 9\end{bmatrix}.
$$
\begin{enumerate}
\item % (a)
 Verify that $\{{\bf y}_1,{\bf y}_2\}$
is a fundamental set of solutions for
${\bf y}'=A{\bf y}$.

\item % (b)
Solve  the initial value problem
$$
{\bf y}'=A{\bf y},\quad   {\bf y}(0)={\bf k}.
\text{ (A)}
$$

\item % (c)
Use the result of Exercise~\ref{exer:10.3.6} (b) to find a formula for
the solution of (A) for an arbitrary initial vector ${\bf
k}$.
\end{enumerate}
\end{problem}

\begin{problem}\label{exer:10.3.8}
 Repeat Exercise~\ref{exer:10.3.7}  with
$$
 A=\begin{bmatrix} -2& -2\\ -5 &1\end{bmatrix}, \quad {\bf y}_1=\begin{bmatrix}
e^{-4t} \\ e^{-4t}\end{bmatrix}, \quad {\bf y}_2=
\begin{bmatrix}-2e^{3t}
\\ 5e^{3t}\end{bmatrix}, \quad {\bf k}=\begin{bmatrix}
10 \\-4\end{bmatrix}.
$$

\begin{solution}
    ${\bf y}=\begin{bmatrix}
e^{-4t} \\ e^{-4t}\end{bmatrix}+c_2
\begin{bmatrix}-2e^{3t}
\\ 5e^{3t}\end{bmatrix}$ where

$$\begin{array}{rcl}
c_1-2c_2&=&10\\c_1+5c_2&=&-4
\end{array},
$$
 so $c_1=6$, $c_2=-2$, and
$${\bf y}=\begin{bmatrix}6e^{-4t}+4e^{3t}\\6e^{-4t}-10e^{3t}
\end{bmatrix}$$.

$Y(t)=\begin{bmatrix}e^{-4t}&-2e^{3t}\\
e^{-4t}&5e^{3t}\end{bmatrix}$;
$Y(0)=\begin{bmatrix}1&-2\\1&5\end{bmatrix}$;
$Y^{-1}(0)=
\frac{1}{7}\begin{bmatrix}5&2\\-1&1\end{bmatrix}$;
${\bf y}=Y(t)Y^{-1}(0){\bf k}=
\frac{1}{7}\begin{bmatrix}5e^{-4t}+2e^{3t}&2e^{-4t}-2e^{3t}
\\5e^{-4t}-5e^{3t}&2e^{-4t}+5e^{3t}\end{bmatrix}{\bf k}$.

\end{solution}
\end{problem}

\begin{problem}\label{exer:10.3.9}
Repeat Exercise~\ref{exer:10.3.7}  with
$$
 A=\begin{bmatrix}-4 & -10\\ 3& 7\end{bmatrix}, \quad
{\bf y}_1=\begin{bmatrix}-5e^{2t} \\ 3e^{2t}
\end{bmatrix}, \quad
{\bf y}_2=\begin{bmatrix} 2e^t \\-e^t
\end{bmatrix}, \quad
{\bf k}=\begin{bmatrix}-19 \\ 11\end{bmatrix}. $$
\end{problem}

\begin{problem}\label{exer:10.3.10}
 Repeat Exercise~\ref{exer:10.3.7} with
$$
 A=\begin{bmatrix} 2 &1\\ 1& 2\end{bmatrix}, \quad {\bf y}_1=\begin{bmatrix} e^{3t} \\
e^{3t}
\end{bmatrix}, \quad {\bf y}_2=\begin{bmatrix}e^t \\
-e^t\end{bmatrix}, \quad {\bf k}=\begin{bmatrix} 2 \\ 8
\end{bmatrix}.$$

\begin{solution}
    ${\bf y}_1=c_1\begin{bmatrix} e^{3t} \\e^{3t}
\end{bmatrix}+c_2\begin{bmatrix}e^t \\
-e^t\end{bmatrix}$,
where

$$\begin{array}{rcl}
c_1+c_2&=&2\\c_1-c_2&=&8
\end{array},
$$

 so $c_1=5$, $c_2=-3$, and
$${\bf y}=\begin{bmatrix}5e^{3t}-3e^t\\5e^{3t}+3e^t
\end{bmatrix}$$

$Y(t)=\begin{bmatrix}e^{3t}&e^t\\
3e^{3t}&-e^t\end{bmatrix}$;
$Y(0)=\begin{bmatrix}1&1\\1&-1\end{bmatrix}$;

$Y^{-1}(0)=
\frac{1}{2}\begin{bmatrix}1&1\\1&-1\end{bmatrix}$;
${\bf y}=Y(t)Y^{-1}(0){\bf k}=
\frac{1}{2}\begin{bmatrix}e^{3t}+e^t&e^{3t}-e^t
\\e^{3t}-e^t&e^{3t}+e^t\end{bmatrix}{\bf k}$.
\end{solution}
\end{problem}

\begin{problem}\label{exer:10.3.11}
Let

$$ A=\begin{bmatrix} 3 &-1 & -1\\ -2 & 3 &
2\\4 & -1 & -2\end{bmatrix}$$

$${\bf y}_1=\begin{bmatrix} e^{2t} \\ 0 \\ e^{2t}\end{bmatrix}, \quad
{\bf y}_2=\begin{bmatrix} e^{3t} \\-e^{3t} \\
e^{3t}\end{bmatrix}, \quad
{\bf y}_3=\begin{bmatrix} e^{-t} \\-3e^{-t} \\
7e^{-t}
\end{bmatrix}, \quad {\bf k}=\begin{bmatrix}
2 \\-7 \\ 20\end{bmatrix}.$$


\begin{enumerate}
\item % (a)
 Verify that $\{{\bf y}_1,{\bf y}_2,{\bf y}_3\}$
is a fundamental set of solutions for
${\bf y}'=A{\bf y}$.

\item % (b)
 Solve the initial value problem
$$
{\bf y}'=A{\bf y}, \quad   {\bf y}(0)={\bf k}.
\text{ (A)}
$$

\item % (c)
Use the result of Exercise~\ref{exer:10.3.6} (b) to find a formula
for
 the solution of  (A)
for an arbitrary initial vector ${\bf k}$.
\end{enumerate}
\end{problem}

\begin{problem}\label{exer:10.3.12}
 Repeat Exercise~\ref{exer:10.3.11} with

$$ A=\begin{bmatrix} 0 & 2 & 2\\ 2 & 0 & 2\\ 2 & 2 & 0\end{bmatrix}$$
$${\bf y}_1=\begin{bmatrix}-e^{-2t} \\ 0 \\ e^{-2t}
\end{bmatrix}, \quad
{\bf y}_2=\begin{bmatrix}-e^{-2t} \\ e^{-2t} \\
0\end{bmatrix}, \quad
{\bf y}_3=\begin{bmatrix} e^{4t} \\ e^{4t} \\ e^{4t}\end{bmatrix}, \quad
{\bf k}=\begin{bmatrix} 0 \\-9 \\ 12\end{bmatrix}$$

\begin{solution}
    ${\bf y}=c_1\begin{bmatrix}-e^{-2t} \\ 0 \\ e^{-2t}
\end{bmatrix}+c_2
\begin{bmatrix}-e^{-2t} \\ e^{-2t} \\
0\end{bmatrix}+
c_3\begin{bmatrix} e^{4t} \\ e^{4t} \\ e^{4t}\end{bmatrix}$,
where

$$\begin{array}{rcr}
-c_1-c_2+c_3&=&0\\c_2+c_3&=&-9\\c_1+c_3&=&12
\end{array},
$$
 so $c_1=11$, $c_2=-10$, $c_3=1$, and
$${\bf y
}=\frac{1}{3}\begin{bmatrix}-e^{-2t}+e^{4t}\\-10e^{-2t}+e^{4t}\\
11e^{-2t}+e^{4t}\end{bmatrix}$$


$Y(t)=
\begin{bmatrix}-e^{-2t}&-e^{-2t}&e^{4t}\\
0&e^{-2t}&e^{4t}\\e^{-2t}&0&e^{4t}\end{bmatrix}$;
$Y(0)=
\begin{bmatrix}-1&-1&1\\0&1&1\\1&0&1\end{bmatrix}$;
$Y^{-1}(0)=
\frac{1}{3}\begin{bmatrix}-1&-1&2\\-1&2&-1
\\1&1&1\end{bmatrix}$;
${\bf y}=Y(t)Y^{-1}(0){\bf k}=
\frac{1}{3}\begin{bmatrix}2e^{-2t}+e^{4t}&-e^{-2t}+e^{4t}
&-e^{-2t}+e^{4t}
\\-e^{-2t}+e^{4t}&2e^{-2t}+e^{4t}&-e^{-2t}+e^{4t}\\
-e^{-2t}+e^{4t}&-e^{-2t}+e^{4t}&2e^{-2t}+e^{4t}
\end{bmatrix}{\bf k}$.

\end{solution}
\end{problem}

\begin{problem}\label{exer:10.3.13}
 Repeat Exercise~\ref{exer:10.3.11}  with

$$ A=\begin{bmatrix} -1 & 2 & 3 \\ 0 & 1 & 6\\
0 & 0 & -2\end{bmatrix}$$
$${\bf y}_1=\begin{bmatrix} e^t \\ e^t \\ 0\end{bmatrix},
\quad
{\bf y}_2=\begin{bmatrix} e^{-t} \\ 0 \\ 0\end{bmatrix},
\quad {\bf y}_3=\begin{bmatrix} e^{-2t} \\-2e^{-2t}
\\ e^{-2t}\end{bmatrix}, \quad
{\bf k}=\begin{bmatrix} 5 \\ 5 \\-1
\end{bmatrix}$$

\end{problem}

\begin{problem}\label{exer:10.3.14}
Suppose $Y$ and $Z$ are fundamental matrices for the $n\times n$
system ${\bf y}'=A(t){\bf y}$. Then some of the four matrices
$YZ^{-1}$, $Y^{-1}Z$, $Z^{-1}Y$, $Z Y^{-1}$ are necessarily
constant. Identify them and prove that they are constant.

\begin{solution}
    If $Y$  and $Z$ are both fundamental matrices for ${\bf y}'=A(t){\bf y
}$, then $Z=CY$, where $C$ is a constant invertible matrix. Therefore,
$ZY^{-1}=C$ and $YZ^{-1}=C^{-1}$.
\end{solution}
\end{problem}

\begin{problem}\label{exer:10.3.15}
Suppose the columns of an $n\times n$ matrix $Y$ are solutions of
the $n\times n$ system ${\bf y}'=A{\bf y}$ and $C$ is an $n \times n$
constant matrix.

\begin{enumerate}
\item % (a)
Show that the matrix $Z=YC$ satisfies the differential equation
$Z'=AZ$.

\item % (b)
Show that $Z$ is a fundamental matrix for ${\bf y}'=A(t){\bf y}$ if
and only if $C$ is invertible and $Y$ is a fundamental matrix for
${\bf y}'=A(t){\bf y}$.
\end{enumerate}
\end{problem}

\begin{problem}\label{exer:10.3.16}
 Suppose the $n\times n$  matrix $A=A(t)$ is
continuous on $(a,b)$ and $t_0$ is in $(a,b)$.
 For $i=1$, $2$, \dots, $n$, let ${\bf y}_i$ be the solution of the initial
value
problem ${\bf y}_i'=A(t){\bf y}_i,\; {\bf y}_i(t_0)={\bf e}_i$, where
$$
{\bf e}_1=\begin{bmatrix} 1\\0\\ \vdots\\0\end{bmatrix},\quad
 {\bf e}_2=\begin{bmatrix} 0\\1\\
\vdots\\0\end{bmatrix},\quad\cdots\quad
 {\bf e}_n=\begin{bmatrix} 0\\0\\ \vdots\\1\end{bmatrix};
$$
that is, the $j$th component of ${\bf e}_i$ is $1$ if $j=i$, or $0$ if
$j\ne i$.

\begin{enumerate}
\item % (a)
 Show that $\{{\bf y}_1,{\bf y}_2,\dots,{\bf y}_n\}$ is a fundamental set of
solutions of ${\bf y}'=A(t){\bf y}$ on  $(a,b)$.

\begin{solution}
    The Wronskian  of $\{{\bf y}_1,{\bf y}_2,\dots,{\bf y}_n\}$ equals
one when $t=t_0$. Apply Theorem~\ref{thmtype:10.3.3} .
\end{solution}

\item % (b)
 Conclude from part (a) and Exercise~\ref{exer:10.3.15}  that ${\bf y}'=
A(t){\bf y}$ has infinitely many fundamental sets of solutions on
$(a,b)$.

\begin{solution}
    Let $Y$ be the matrix with columns $\{{\bf y}_1,{\bf y}_2,\dots,{\bf
y}_n\}$. From part (a), $Y$ is a fundamental matrix for ${\bf
y}'=A(t){\bf y}$ on $(a,b)$. From Exercise~\ref{exer:10.3.15} (b),
so is $Z=YC$ if $C$ is any invertible constant matrix.
\end{solution}
\end{enumerate}
\end{problem}

\begin{problem}\label{exer:10.3.17}
 Show that $Y$ is a
fundamental matrix for the system ${\bf y}'=A(t){\bf y}$ if and only
if $Y^{-1}$ is a fundamental matrix for ${\bf y}'=-
A^T(t){\bf y}$, where $A^T$ denotes the transpose of $A$.
\begin{hint}
    See Exercise \ref{exer:10.2.11}.
\end{hint}
\end{problem}

\begin{problem}\label{exer:10.3.18}
Let $Z$ be the fundamental matrix for the constant coefficient system
 ${\bf y}'=A{\bf y}$ such that $Z(0)=I$.
\begin{enumerate}
\item % (a)
Show that $Z(t)Z(s)=Z(t+s)$ for all $s$ and $t$. 
\begin{hint}
For
fixed
$s$ let $\Gamma_1(t)=Z(t)Z(s)$ and $\Gamma_2(t)=Z(t+s)$. Show that
$\Gamma_1$ and $\Gamma_2$ are both solutions of the matrix initial value
problem $\Gamma'=A\Gamma,\quad\Gamma(0)=Z(s)$. Then conclude from
Theorem~\ref{thmtype:10.2.1} that $\Gamma_1=\Gamma_2$.
\end{hint}

\begin{solution}
    $\Gamma_1'(t)=Z'(t)Z(s)=AZ(t)Z(s)=A\Gamma(t)$ and $\Gamma_1(0)=Z(s)$,
since $Z(0)=I$.
$\Gamma_2'(t)=Z'(t+s)=AZ(t+s)=A\Gamma_2(t)$ (since $A$ is constant)
and $\Gamma_2(0)=Z(s)$. Applying Theorem~\ref{thmtype:10.2.1}  to the columns of
$\Gamma_1$ and $\Gamma_2$  shows that $\Gamma_1=\Gamma_2$.
\end{solution}
\item % (b)
Show that $(Z(t))^{-1}=Z(-t)$.

\begin{solution}
    With $s=-t$, part (a) implies that $Z(t)Z(-t)=Z(0)=I$;
therefore $(Z(t))^{-1}=Z(-t)$.
\end{solution}
\item % (c)
The matrix $Z$ defined above is sometimes denoted by $e^{tA}$. Discuss
 the motivation for this notation.

 \begin{solution}
     $e^{0\cdot A}=I$ is analogous to $e^{o\cdot a}=e^{0}=1$ when
$a$ is a scalar, while $e^{(t+s)A}=e^{tA}e^{sA}$  is analagous
to $e^{(t+s)a}=e^{ta}e^{sa}$ when $a$ is a scalar.
 \end{solution}
\end{enumerate}
\end{problem}


\end{document}