\documentclass{ximera}
%% You can put user macros here
%% However, you cannot make new environments

%\listfiles

% Get the 'old' hints/expandables, for use on ximera.osu.edu
%\def\xmNotHintAsExpandable{true}
%\def\xmNotExpandableAsAccordion{true}



%\graphicspath{{./}{firstExample/}{secondExample/}}
\graphicspath{{./}
{aboutDiffEq/}
{applicationsLeadingToDiffEq/}
{applicationsToCurves/}
{autonomousSecondOrder/}
{basicConcepts/}
{bernoulli/}
{constCoeffHomSysI/}
{constCoeffHomSysII/}
{constCoeffHomSysIII/}
{constantCoeffWithImpulses/}
{constantCoefficientHomogeneousEquations/}
{convolution/}
{coolingActivity/}
{directionFields/}
{drainingTank/}
{epidemicActivity/}
{eulersMethod/}
{exactEquations/}
{existUniqueNonlinear/}
{frobeniusI/}
{frobeniusII/}
{frobeniusIII/}
{global.css/}
{growthDecay/}
{heatingCoolingActivity/}
{higherOrderConstCoeff/}
{homogeneousLinearEquations/}
{homogeneousLinearSys/}
{improvedEuler/}
{integratingFactors/}
{interactExperiment/}
{introToLaplace/}
{introToSystems/}
{inverseLaplace/}
{ivpLaplace/}
{laplaceTable/}
{lawOfCooling/}
{linSysOfDiffEqs/}
{linearFirstOrderDiffEq/}
{linearHigherOrder/}
{mixingProblems/}
{motionUnderCentralForce/}
{nonHomogeneousLinear/}
{nonlinearToSeparable/}
{odesInSage/}
{piecewiseContForcingFn/}
{population/}
{reductionOfOrder/}
{regularSingularPts/}
{reviewOfPowerSeries/}
{rlcCircuit/}
{rungeKutta/}
{secondLawOfMotion/}
{separableEquations/}
{seriesSolNearOrdinaryPtI/}
{seriesSolNearOrdinaryPtII/}
{simplePendulum/}
{springActivity/}
{springProblemsI/}
{springProblemsII/}
{undCoeffHigherOrderEqs/}
{undeterminedCoeff/}
{undeterminedCoeff2/}
{unitStepFunction/}
{varParHigherOrder/}
{varParamNonHomLinSys/}
{variationOfParameters/}
}


\usepackage{tikz}
%\usepackage{tkz-euclide}
\usepackage{tikz-3dplot}
\usepackage{tikz-cd}
\usetikzlibrary{shapes.geometric}
\usetikzlibrary{arrows}
\usetikzlibrary{decorations.pathmorphing,patterns}
\usetikzlibrary{backgrounds} % added by Felipe
% \usetkzobj{all}   % NOT ALLOWED IN RECENT TeX's ...
\pgfplotsset{compat=1.13} % prevents compile error.

\pdfOnly{\renewcommand{\theHsection}{\thepart.section.\thesection}}  %% MAKES LINKS WORK should be added to CLS
\pdfOnly{\renewcommand{\part}[1]{\chapterstyle\title{#1}\begin{abstract}\end{abstract}\maketitle\def\thechaptertitle{#1}}}


\renewcommand{\vec}[1]{\mathbf{#1}}
\newcommand{\RR}{\mathbb{R}}
\providecommand{\dfn}{\textit}
\renewcommand{\dfn}{\textit}
\newcommand{\dotp}{\cdot}
\newcommand{\id}{\text{id}}
\newcommand\norm[1]{\left\lVert#1\right\rVert}
\newcommand{\dst}{\displaystyle}
 
\newtheorem{general}{Generalization}
\newtheorem{initprob}{Exploration Problem}

\tikzstyle geometryDiagrams=[ultra thick,color=blue!50!black]

\usepackage{mathtools}

\title{Spreadsheet Lab} \license{CC BY-NC-SA 4.0}

\begin{document}

\begin{abstract}
\end{abstract}
\maketitle

\begin{onlineOnly}
\section*{Spreadsheet Lab}
\end{onlineOnly}
 
Spreadsheets are a great way to analyze iterative processes such as the numerical methods presented here in Chapter 3.  In this lab you will implement Euler's method using a spreadsheet.
 
 
\begin{exploration}\label{lab3.1:exp1}
Example~\ref{example:3.1.2} asked us to use Euler's method with step sizes $h=0.1$, $h=0.05$, and $h=0.025$ to
find approximate values of the solution of the initial value problem
$$
y'+2y=x^3e^{-2x},\quad y(0)=1
$$
at $x=0, 0.1, 0.2, 0.3, \ldots, 1.0$.
 
The link below takes you to a spreadsheet which can be used to complete this exercise.  To use the spreadsheet, SAVE A COPY of the sheet.  The yellow boxes control the initial condition and the step size. 
     
%\href{https://docs.google.com/spreadsheets/d/1CyQ9jvOL9WMxcphNu4QlM8ZQAkU2oHDQEbK2ORw7k-o/edit?usp=sharing}{LINK TO SPREADSHEET}

\end{exploration}
 
\begin{exploration}\label{lab3.1:exp2}
 It is also not difficult to modify the spreadsheet in the previous exploration for use in other problems.  In this exploration you will try to modify the spreadsheet above to tackle this problem from Example~\ref{3.1.3}:

Use Euler's method with step sizes $h=0.1$, $h=0.05$, and $h=0.025$ to
find approximate values of the solution of the initial value problem
$$
y'=-2y^2+xy+x^2, y(0)=1,
$$
at $x=0, 0.1, 0.2, 0.3, \ldots, 1.0$.

The following video guides you through the process of modifying the spreadsheet for this new problem.

%\youtube{rb45dYb0fp0}
    
\end{exploration}



 
 
 \begin{example}\label{example:3.1.3}
The tables below show analogous results
for the nonlinear initial value problem
\begin{equation} \label{eq:3.1.7}
y'=-2y^2+xy+x^2, y(0)=1,
\end{equation}
except in this case we can't solve $\eqref{eq:3.1.7}$ exactly.
The results in the ``Exact'' column were obtained by using a
more accurate numerical method known as the
\href{https://en.wikipedia.org/wiki/Runge%E2%80%93Kutta_methods}{Runge-Kutta}
method with a small step size. They are exact to eight decimal places.
The following table shows numerical solutions obtained by Euler's method.
$$
\begin{array}{|c|c|c|c|c|}
\hline
x&
h=0.1&
h=0.05&
h=0.025&
\text{``Exact''}\\ \hline
0.0 & 1.000000000 & 1.000000000 & 1.000000000 & 1.000000000 \\
0.1 & 0.800000000 & 0.821375000 & 0.829977007 & 0.837584494 \\
0.2 & 0.681000000 & 0.707795377 & 0.719226253 & 0.729641890 \\
0.3 & 0.605867800 & 0.633776590 & 0.646115227 & 0.657580377 \\
0.4 & 0.559628676 & 0.587454526 & 0.600045701 & 0.611901791 \\
0.5 & 0.535376972 & 0.562906169 & 0.575556391 & 0.587575491 \\
0.6 & 0.529820120 & 0.557143535 & 0.569824171 & 0.581942225 \\
0.7 & 0.541467455 & 0.568716935 & 0.581435423 & 0.593629526 \\
0.8 & 0.569732776 & 0.596951988 & 0.609684903 & 0.621907458 \\
0.9 & 0.614392311 & 0.641457729 & 0.654110862 & 0.666250842 \\
1.0 & 0.675192037 & 0.701764495 & 0.714151626 & 0.726015790\\
\hline
\end{array}
$$
The following table shows the error in approximate solutions obtained by Euler's method.
$$
\begin{array}{|c|c|c|c|}
\hline
x&
h=0.1&
h=0.05&
h=0.025\\ \hline
0.1 & 0.0376 & 0.0162 &0.0076 \\
0.2 & 0.0486 & 0.0218 &0.0104 \\
0.3 & 0.0517 & 0.0238 &0.0115 \\
0.4 & 0.0523 & 0.0244 &0.0119 \\
0.5 & 0.0522 & 0.0247 &0.0121 \\
0.6 & 0.0521 & 0.0248 &0.0121 \\
0.7 & 0.0522 & 0.0249 &0.0122 \\
0.8 & 0.0522 & 0.0250 &0.0122 \\
0.9 & 0.0519 & 0.0248 &0.0121 \\
1.0 & 0.0508 & 0.0243 &0.0119 \\
\hline
\end{array}
$$

\end{example}



 
\end{document}