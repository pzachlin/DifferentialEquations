\documentclass{ximera}
%% You can put user macros here
%% However, you cannot make new environments

%\listfiles

% Get the 'old' hints/expandables, for use on ximera.osu.edu
%\def\xmNotHintAsExpandable{true}
%\def\xmNotExpandableAsAccordion{true}



%\graphicspath{{./}{firstExample/}{secondExample/}}
\graphicspath{{./}
{aboutDiffEq/}
{applicationsLeadingToDiffEq/}
{applicationsToCurves/}
{autonomousSecondOrder/}
{basicConcepts/}
{bernoulli/}
{constCoeffHomSysI/}
{constCoeffHomSysII/}
{constCoeffHomSysIII/}
{constantCoeffWithImpulses/}
{constantCoefficientHomogeneousEquations/}
{convolution/}
{coolingActivity/}
{directionFields/}
{drainingTank/}
{epidemicActivity/}
{eulersMethod/}
{exactEquations/}
{existUniqueNonlinear/}
{frobeniusI/}
{frobeniusII/}
{frobeniusIII/}
{global.css/}
{growthDecay/}
{heatingCoolingActivity/}
{higherOrderConstCoeff/}
{homogeneousLinearEquations/}
{homogeneousLinearSys/}
{improvedEuler/}
{integratingFactors/}
{interactExperiment/}
{introToLaplace/}
{introToSystems/}
{inverseLaplace/}
{ivpLaplace/}
{laplaceTable/}
{lawOfCooling/}
{linSysOfDiffEqs/}
{linearFirstOrderDiffEq/}
{linearHigherOrder/}
{mixingProblems/}
{motionUnderCentralForce/}
{nonHomogeneousLinear/}
{nonlinearToSeparable/}
{odesInSage/}
{piecewiseContForcingFn/}
{population/}
{reductionOfOrder/}
{regularSingularPts/}
{reviewOfPowerSeries/}
{rlcCircuit/}
{rungeKutta/}
{secondLawOfMotion/}
{separableEquations/}
{seriesSolNearOrdinaryPtI/}
{seriesSolNearOrdinaryPtII/}
{simplePendulum/}
{springActivity/}
{springProblemsI/}
{springProblemsII/}
{undCoeffHigherOrderEqs/}
{undeterminedCoeff/}
{undeterminedCoeff2/}
{unitStepFunction/}
{varParHigherOrder/}
{varParamNonHomLinSys/}
{variationOfParameters/}
}


\usepackage{tikz}
%\usepackage{tkz-euclide}
\usepackage{tikz-3dplot}
\usepackage{tikz-cd}
\usetikzlibrary{shapes.geometric}
\usetikzlibrary{arrows}
\usetikzlibrary{decorations.pathmorphing,patterns}
\usetikzlibrary{backgrounds} % added by Felipe
% \usetkzobj{all}   % NOT ALLOWED IN RECENT TeX's ...
\pgfplotsset{compat=1.13} % prevents compile error.

\pdfOnly{\renewcommand{\theHsection}{\thepart.section.\thesection}}  %% MAKES LINKS WORK should be added to CLS
\pdfOnly{\renewcommand{\part}[1]{\chapterstyle\title{#1}\begin{abstract}\end{abstract}\maketitle\def\thechaptertitle{#1}}}


\renewcommand{\vec}[1]{\mathbf{#1}}
\newcommand{\RR}{\mathbb{R}}
\providecommand{\dfn}{\textit}
\renewcommand{\dfn}{\textit}
\newcommand{\dotp}{\cdot}
\newcommand{\id}{\text{id}}
\newcommand\norm[1]{\left\lVert#1\right\rVert}
\newcommand{\dst}{\displaystyle}
 
\newtheorem{general}{Generalization}
\newtheorem{initprob}{Exploration Problem}

\tikzstyle geometryDiagrams=[ultra thick,color=blue!50!black]

\usepackage{mathtools}

\title{Exercises} \license{CC BY-NC-SA 4.0}

\begin{document}

\begin{abstract}
\end{abstract}
\maketitle

\begin{onlineOnly}
\section*{Exercises}
\end{onlineOnly}


\begin{problem}\label{exer:4.2.1}
A thermometer is moved from a room where the temperature is
$70^\circ$F to a freezer where the temperature is $12^\circ
F$.  After 30 seconds the thermometer reads
$40^\circ$F.  What does it read after 2 minutes?
\end{problem}

\begin{problem}\label{exer:4.2.2}
A fluid initially at $100^\circ$C is placed outside on a
day when the temperature is $-10^\circ$C, and the
temperature of the fluid drops $20^\circ$C in one minute.
Find the temperature $T(t)$ of the fluid for $t > 0$.

\begin{solution}
Since $T_0=100$ and $T_M=-10$, $T=-10+110e^{-kt}$. Now
$T(1)=80\Rightarrow 80=-10+110e^{-k}$, so $e^{-k}=\frac{9}{11}$
and $k=\ln\frac{11}{9}$. Therefore, $T=-10+110e^{-t\ln\frac{11}{9}}$.
\end{solution}
\end{problem}

\begin{problem}\label{exer:4.2.3}
At 12:00 {\sc pm} a thermometer reading $10^\circ$F is placed in
a room where the temperature is $70^\circ$F.  It reads
$56^\circ$ when it's placed outside, where the temperature
is $5^\circ$F, at 12:03.  What does it read at 12:05 {\sc pm}?
\end{problem}

\begin{problem}\label{exer:4.2.4}
A thermometer initially reading $212^\circ$F is placed in a
room where the temperature is $70^\circ$F.  After 2
minutes the thermometer reads $125^\circ$F.

\begin{enumerate}
\item %(a)
What does the thermometer read after 4 minutes?

\begin{solution}
Let $T$ be the thermometer reading.
Since $T_0=212$ and $T_M=70$, $T=70+142e^{-kt}$. Now
$T(2)=125\Rightarrow 125=70+142e^{-2k}$, so
$e^{-2k}=\frac{55}{142}$
and $k=\frac{1}{2}\ln\frac{142}{55}$. Therefore,
(A) $T=70+142e^{-\frac{t}{2}\ln\frac{142}{55}}$.

$T(2)=70+142e^{-2\ln\frac{142}{55}}=70+142\left(\frac{55}{142}\right)^2
\approx91.30^\circ$F.
\end{solution}

\item %(b)
When will the thermometer read $72^\circ$F?

\begin{solution}
Let $\tau$ be the time when
 $T(\tau)=72$, so $72=70+142e^{-\frac{\tau}{2}\ln\frac{142}{55}}$, or
$e^{-\frac{\tau}{2}\ln\frac{142}{55}}=\frac{1}{71}$. Therefore,
$\tau=2\frac{\ln71}{\ln\frac{142}{55}}\approx8.99$ min.
\end{solution}

\item %(c)
When will the thermometer read $69^\circ$F?

\begin{solution}
Since (A) implies that $T>70$ for all $t>0$, the
thermometer will never read $69^\circ$F.
\end{solution}
\end{enumerate}
\end{problem}

\begin{problem}\label{exer:4.2.5}
An object with initial temperature $150^\circ$C is placed
outside, where the temperature is $35^\circ$C.  Its
temperatures at 12:15 and 12:20 are $120^\circ$C and
$90^\circ$C, respectively.

\begin{enumerate}
\item %(a)
At what time was the object placed outside?

\item %(b)
When will its temperature be $40^\circ$C?
\end{enumerate}
\end{problem}

\begin{problem}\label{exer:4.2.6}
An object is placed in a room where the temperature is
$20^\circ$C.  The temperature of the object drops by
$5^\circ$C in 4 minutes and by $7^\circ$C in 8 minutes.
What was the temperature of the object when it was initially
placed in the room?

\begin{solution}
Since $T_M=20$, $T=20+(T_0-20)e^{-kt}$. Now
$T_0-5=20+(T_0-20)e^{-4k}$  and $T_0-7=20+(T_0-20)e^{-8k}$. Therefore,
$\frac{T_0-25}{T_0-20}=e^{-4k}$ and $\left(\frac{T_0-27}{T_0-20}\right)=e^{-8k}$, so  $\frac{T_0-27}{T_0-20}=\left(\frac{T_0-25}{T_0-20}\right)^2$,
which implies that $(T_0-20)(T_0-27)=(T_0-25)^2$, or
$T_0^2-47T_0+540=T_0^2-50T_0+625$; hence $3T_0=85$
and  $T_0={(85/3)^\circ C}$.
\end{solution}
\end{problem}

\begin{problem}\label{exer:4.2.7}
A cup of boiling water is placed outside at 1:00 {\sc pm}.  One
minute later the temperature of the water is $152^\circ$F.
After another minute its temperature is $112^\circ$F.  What
is the outside temperature?
\end{problem}

\begin{problem}\label{exer:4.2.8}
A tank initially contains 40 gallons of pure water.  A
solution with 1 gram of salt per gallon of water is added
to the tank at 3 gal/min, and the  resulting
solution dranes out at the same rate.  Find the quantity $Q(t)$
of salt in the tank at time $t > 0$.

\begin{solution}
$Q'=3-\frac{3}{40}Q,\quad Q(0)=0$. Rewrite the differential equation
as (A) $Q'+\frac{3}{40}Q=3$. Since $Q_1=e^{-3t/40}$ is a solution of
the complementary equation, the solutions of (A) are given by
$Q=ue^{-3t/40}$ where $u'e^{-3t/40}=3$. Therefore,$u'=3e^{3t/40}$,
$u=40e^{3t/40}+c$, and $Q=40+ce^{-3t/40}$. Now $Q(0)=0\Rightarrow
c=-40$, so $Q=40(1-e^{-3t/40})$.
\end{solution}
\end{problem}

\begin{problem}\label{exer:4.2.9}
A tank initially contains a solution of 10 pounds of salt in
60 gallons of water.  Water with 1/2 pound of salt per
gallon is added to the tank at 6 gal/min, and the
 resulting solution leaves at the same rate.
Find the quantity $Q(t)$ of salt in the tank at time
$t > 0$.
\end{problem}

\begin{problem}\label{exer:4.2.10}
A  tank initially contains 100 liters of a salt solution
with a concentration of .1 g/liter.  A solution with a
salt concentration of .3 g/liter is added to the tank at
5 liters/min, and the resulting mixture is
drained out at the same rate.  Find the concentration $K(t)$
of salt in the tank as a function of $t$.

\begin{solution}
$Q'=\frac{3}{2}-\frac{Q}{20},\quad Q(0)=10$. Rewrite the
differential equation as (A) $Q'+\frac{Q}{20}=\frac{3}{2}$. Since
$Q_1=e^{-t/20}$ is a solution of the complementary equation, the
solutions of (A) are given by $Q=ue^{-t/20}$ where
$u'e^{-t/20}=\frac{3}{2}$. Therefore,$u'=\frac{3}{2}e^{t/20}$,
$u=30e^{t/20}+c$, and $Q=30+ce^{-t/20}$. Now $Q(0)=10\Rightarrow
c=-20$, so $Q=30-20e^{-t/20}$ and $K=\frac{Q}{100}=.3-.2e^{-t/20}$.
\end{solution}
\end{problem}

\begin{problem}\label{exer:4.2.11}
A 200 gallon tank initially contains 100 gallons of water
with 20 pounds of salt.  A salt solution with 1/4 pound of
salt per gallon is added to the tank at 4 gal/min, and the
resulting  mixture is drained out at 2 gal/min.
Find the quantity of salt in the tank as it's about to
overflow.
\end{problem}

\begin{problem}\label{exer:4.2.12}
Suppose  water is added to a tank at 10 gal/min, but leaks
out at the rate of 1/5 gal/min for each gallon in the tank.
What is the smallest capacity the tank can have if the
process is to continue indefinitely?

\begin{solution}
$Q'=10-\frac{Q}{5}$, or
 (A) $Q'+\frac{Q}{5}=10$. Since $Q_1=e^{-t/5}$ is a solution
of the complementary equation, the solutions of (A) are given by
$Q=ue^{-t/5}$ where $u'e^{-t/5}=10$. Therefore,$u'=10e^{t/5}$,
$u=50e^{t/10}+c$, and $Q=50+ce^{-t/5}$. Since
$\lim_{t\to\infty}Q(t)=50$, the mininum capacity is 50 gallons.
\end{solution}
\end{problem}

\begin{problem}\label{exer:4.2.13}
A chemical reaction in a laboratory with volume $V$ (in ft$^3$)
produces $q_1$ ft$^3$/min of a noxious gas as a byproduct. The gas is
dangerous at concentrations greater than $\overline c$,
 but harmless at
concentrations $\le \overline c$. Intake fans at one end of the
laboratory pull in fresh air at the rate of $q_2$ ft$^3$/min and
exhaust fans at the other end exhaust the mixture of gas and air from
the
laboratory at the same rate. Assuming that the gas is always uniformly
distributed in the room and its initial concentration $c_0$ is at a
safe level, find the smallest value of $q_2$ required to maintain safe
conditions in the laboratory for all time.
\end{problem}

\begin{problem}\label{exer:4.2.14}
A 1200-gallon tank initially contains 40 pounds of salt
dissolved in 600 gallons of water.  Starting at $t_0=0$,
water that contains 1/2 pound of salt per gallon is added to
the tank at the rate of 6 gal/min and the
resulting mixture  is drained from the tank at 4 gal/min.  Find the
quantity $Q(t)$ of salt in the tank at any time $t > 0$
prior to overflow.

\begin{solution}
Since there are $2t+600$ gallons of mixture in the tank at
time $t$ and mixture is being drained at 4
gallons/min, $Q'=3-\frac{2}{t+300}Q,\quad Q(0)=40$.
 Rewrite the differential
equation
as (A) $Q'+\frac{2}{t+300}Q=3$. Since $Q_1=\frac{1}{(t+300)^2}$
is a
solution of the complementary equation, the solutions of (A) are given
by
$Q=\frac{u}{(t+300)^2}$ where $\frac{u'}{(t+300)^2}=3$. Therefore,
$u'=3(t+300)^2$,
$u=(t+300)^3+c$, and $Q=t+300+\frac{c}{(t+300)^2}$.
 Now
$Q(0)=40\Rightarrow c=-234\times10^5$, so
 $Q= t+300-\frac{234 \times 10^5}{(t+300)^2},
\, 0\le t\le300$.
\end{solution}
\end{problem}

\begin{problem}\label{exer:4.2.15}
Tank $T_1$  initially contain 50 gallons of pure water. Starting at
$t_0=0$,  water that contains 1 pound of salt per gallon is poured into
$T_1$ at  the rate of  2 gal/min.
The  mixture is drained from
$T_1$ at the same rate into a second tank $T_2$, which initially
contains  50 gallons of pure water.  Also starting at
$t_0=0$, a mixture from another source that contains 2  pounds of salt
per gallon is poured
into $T_2$  at the rate of 2 gal/min.  The  mixture is drained
from $T_2$ at the rate of 4 gal/min.


\begin{enumerate}
\item % (a)
 Find a differential
equation for the quantity $Q(t)$ of salt in  tank $T_2$ at time $t > 0$.
\item % (b)
Solve the equation derived  in the previous part to determine $Q(t)$.
\item % (c)
 Find $\lim_{t\to\infty}Q(t)$.
\end{enumerate}
\end{problem}

\begin{problem}\label{exer:4.2.16}
Suppose an object  with initial temperature $T_0$
is placed in a sealed container, which is in turn placed in a medium with
temperature $T_m$. Let the initial
temperature of the container be $S_0$. Assume that the temperature of the
object does not affect the temperature of the container, which in turn does
not affect the temperature of the medium. (These assumptions  are
reasonable, for example, if the object is a cup of coffee, the container is
a house, and the medium is the atmosphere.)

\begin{enumerate}
\item % (a)
Assuming that the container and the medium have distinct temperature
decay constants $k$ and $k_m$ respectively, use Newton's law of
cooling to find the temperatures $S(t)$ and $T(t)$ of the container
and object at time $t$.

\begin{solution}
$S'=-k_m(S-T_m),\ S(0)=0$, so (A)
$S=T_m+(S_0-T_m)e^{-k_mt}$.
$T'=-k(T-S)=-k\left(T-T_m-(S_0-T_m)e^{-k_mt}\right)$,
from
(A). Therefore,$T'+kT=kT_m+k(S_0-T_m)e^{-k_mt}$; $T=ue^{-kt}$;
(B) $u'=kT_me^{kt}+k(S_0-T_m)e^{(k-k_m)t}$;
 $u=T_me^{kt}+\frac{k}{k-k_m}(S_0-T_m)e^{(k-k_m)t}+c$;
 $T(0)=T_0\Rightarrow
 c=T_0-T_m-\frac{k}{k-k_m}(S_0-T_m)$;
 $u=T_me^{kt}+\frac{k}{k-k_m}(S_0-T_m)e^{(k-k_m)t}+T_0-T_m-\frac{k}{k-k_m}(S_0-T_m)$;
$T=T_m+(T_0-T_m)e^{-kt}+\frac{k(S_0-T_m)}{(k-k_m)}\left(e^{-k_mt}-e^{-kt}\right)$.
\end{solution}

\item % (b)
Assuming that the container and the medium have the same temperature
decay constant $k$, use Newton's law of cooling to find the
temperatures $S(t)$ and $T(t)$ of the container and object at time
$t$.

\begin{solution}
If $k=k_m$ (B) becomes
(B) $u'=kT_me^{kt}+k(S_0-T_m)$; $u=T_me^{kt}+k(S_0-T_m)t+c$;
$T(0)=T_0\Rightarrow c=T_0-T_m$;
$u=T_me^{kt}+k(S_0-T_m)t+(T_0-T_m)$;
$T=T_m+k(S_0-T_m)te^{-kt}+(T_0-T_m)e^{-kt}$.
\end{solution}

\item % (c)
Find $\lim._{t\to\infty}S(t)$  and $\lim_{t\to\infty}T(t)$ .

\begin{solution}
$\lim_{t\to\infty}T(t)=\lim_{t\to\infty}S(t)=T_m$
in either case.
\end{solution}
\end{enumerate}
\end{problem}

\begin{problem}\label{exer:4.2.17}  %\exercisemolten
In  our previous examples and exercises concerning Newton's law of cooling
we assumed that
the temperature of the medium remains constant.  This model is adequate
if the heat lost or gained by the object is
insignificant compared to the heat required to cause an appreciable change
in the temperature of the medium. If this isn't  so,  we
must use a model that accounts for the heat exchanged between the object
and the medium. Let $T=T(t)$ and $T_m=T_m(t)$  be the temperatures of the
object and the medium, respectively, and let $T_0$  and $T_{m0}$ be
their
initial values.  Again, we assume that $T$  and $T_m$ are related by
Newton's law of cooling,
$$
T'=-k(T-T_m).
\text{(A)}
$$
We also assume that the change in heat of the object as its
temperature changes from $T_0$ to $T$  is $a(T-T_0)$
and that the change in heat of the medium  as its
temperature changes from $T_{m0}$ to $T_m$  is $a_m(T_m-T_{m0})$,
where $a$
and $a_m$ are positive constants  depending  upon the masses and
thermal properties of the
object and medium, respectively.  If we assume that the total heat of
the system consisting of the object and the medium remains constant
(that is, energy is conserved), then
$$
a(T-T_0)+a_m(T_m-T_{m0})=0.
\text{(B)}
$$

\begin{enumerate}
\item % (a)
Equation (A) involves  two unknown functions $T$ and $T_m$.
 Use (A) and (B) to derive a
differential equation involving only $T$.
\item % (b)
Find $T(t)$ and $T_m(t)$ for $t>0$.
\item % (c)
Find $\lim_{t\to\infty}T(t)$ and
 $\lim_{t\to\infty}T_m(t)$.
\end{enumerate}
\end{problem}

\begin{problem}\label{exer:4.2.18}Control mechanisms allow fluid to flow into a tank
at a rate proportional to the volume $V$ of fluid in the tank, and to
flow
out at a rate proportional to $V^2$. Suppose $V(0)=V_0$ and the
constants of proportionality are $a$ and $b$, respectively.
Find $V(t)$  for $t>0$ and find $\lim_{t\to\infty}V(t)$.

\begin{solution}
$V'=aV-bV^2=-bV(V-b/a)$; $\frac{V'}{V(V-a/b)}=-b$;
$\left[\frac{1}{V-a/b}-\frac{1}{V}\right]V'=-a$;
$\ln\left|\frac{V-a/b}{V}\right|=-at+k$; (A) $\frac{V-a/b}{V}=ce^{-at}$; (B) $V=\frac{a}{b}\frac{1}{1-ce^{-at}}$. Since
$V(0)=V_0$,  (A) $\Rightarrow c=\frac{V_0-a/b}{V_0}$. Substituting
this into (B) yields
$V=\frac{a}{b}\frac{V_0}{V_0-\left(V_0-a/b
\right)e^{-at}}$ so $\lim_{t\to\infty}V(t)=a/b$
\end{solution}
\end{problem}

\begin{problem}\label{exer:4.2.19}
Identical tanks $T_1$ and $T_2$ initially contain $W$ gallons each of
pure water. Starting at $t_0=0$, a salt solution with constant
concentration $c$ is pumped into $T_1$ at $r$ gal/min and drained from
$T_1$ into $T_2$ at the same rate. The resulting mixture in $T_2$ is
also drained at the same rate. Find the concentrations $c_1(t)$
and $c_2(t)$ in tanks $T_1$ and $T_2$ for $t>0$.
\end{problem}

\begin{problem}\label{exer:4.2.20}
An infinite sequence of identical tanks $T_1$, $T_2$, \dots, $T_n$, \dots,
initially contain $W$ gallons
each of pure water.  They are hooked together so that fluid drains
from $T_n$ into  $T_{n+1}\,(n=1,2,\cdots)$.
A salt solution is circulated through the tanks so that it enters and
leaves each tank at the constant rate of $r$ gal/min.
The solution has a concentration of
$c$  pounds of salt per
gallon when it enters $T_1$.

\begin{enumerate}
\item % (a)
Find the concentration $c_n(t)$
in tank $T_n$ for $t>0$.

\begin{solution}
If $Q_n(t)$ is the number of pounds of salt in $T_n$ at time $t$, then
$Q_{n+1}'+\frac{r}{W}Q_{n+1}=rc_n(t),\ n=0,1,\dots$, where
$c_0(t)\equiv c$. Therefore,$Q_{n+1}=u_{n+1}e^{-rt/w}$;
(A) $u_{n+1}'=re^{rt/W}c_n(t)$. In particular, with $n=0$,
$u_1=cW(e^{rt/W}-1)$, so $Q_1=cW(1-e^{-rt/W})$  and
$c_1=c(1-e^{-rt/W})$. We will show by induction that
$c_n=c\left(1-e^{-rt/W}\sum_{j=0}^{n-1}\frac{1}{j!}\left(\frac{rt}{W}\right)^j\right)$.
 This is true for $n=1$; if it is
true for a given $n$, then, from (A),
$$
u_{n+1}'=cre^{rt/W}
\left(1-e^{-rt/W}\sum_{j=0}^{n-1}\frac{1}{j!}\left(\frac{rt}{W}\right)^j\right)
=cre^{rt/W}-cr
\sum_{j=0}^{n-1}\frac{1}{j!}\left(\frac{rt}{W}\right)^j,
$$
so (since $Q_{n+1}(0)=0$),
$$
u_{n+1}=cW(e^{rt/W}-1)-c\sum_{j=0}^{n-1}\frac{1}{(j+1)!}
\frac{r^{j+1}}{W^j}t^{j+1}.
$$
Therefore,
$$
c_{n+1}=\frac{1}{W}u_{n+1}e^{-rt/W}=
c\left(1-e^{-rt/W}\sum_{j=0}^n\frac{1}{j!}\left(\frac{rt}{W}\right)^j\right),
$$
which completes the induction.
\end{solution}

\item % (b)
  Find $\lim_{t\to\infty}c_n(t)$ for each $n$.

\begin{solution}
$\lim_{t\to\infty}c_n(t)=c$
\end{solution}
\end{enumerate}
\end{problem}

\begin{problem}\label{exer:4.2.21}
Tanks $T_1$  and $T_2$ have capacities $W_1$ and $W_2$ liters,
respectively.
Initially they are both full of dye solutions with concentrations $c_{1}$
and $c_2$ grams per liter. Starting at $t_0=0$,  the solution from $T_1$
is pumped into $T_2$ at a rate of $r$  liters per minute,
and the solution from $T_2$
is pumped into $T_1$ at the same rate.

\begin{enumerate}
\item % (a)
Find the concentrations $c_1(t)$ and $c_2(t)$ of the dye in $T_1$
and $T_2$ for $t>0$.
\item % (b)
Find $\lim_{t\to\infty}c_1(t)$  and $\lim_{t\to\infty}c_2(t)$.
\end{enumerate}
\end{problem}

\begin{problem}\label{exer:4.2.22}  
Consider the mixing problem of Example~\ref{example:4.2.3}, but without
the assumption that the mixture is stirred instantly so that the salt
is always uniformly distributed throughout the mixture. Assume instead
that the distribution approaches uniformity as $t\to\infty$.
In this case the differential equation for $Q$ is  of the form
$$
Q'+\frac{a(t)}{150}Q=2
$$
where  $\lim_{t\to\infty}a(t)=1$.

\begin{enumerate}
\item % (a)
Assuming that
$Q(0)=Q_0$, can you guess the value of
$\lim_{t\to\infty}Q(t)$?.

\begin{solution}
Since the incoming solution contains 1/2 lb of salt per gallon and there are always 600 gallons in the tank, we conclude
intuitively that $\lim_{t\to\infty}Q(t)=300$.
To verify this rigorously,
note that $Q_1(t)=\exp\left(-\frac{1}{150}\int_0^t
a(\tau)\,d\tau\right)$ is a solution of the complementary equation,
(A) $Q_1(0)=1$, and (B) $\lim_{t\to\infty}Q_1(t)=0$ (since
$\lim_{t\to\infty}a(t)=1$).  Therefore,$Q=Q_1u$; $Q_1u'=2$;
$u'=\frac{2}{Q_1}$;
$u=Q_0+2\int_0^t\frac{d\tau}{Q_1(\tau)}$ (see (A)), and
$Q(t)=Q_0Q_1(t)+2Q_1(t)\int_0^t\frac{d\tau}{Q_1(\tau)}$.
From (B),
$\lim_{t\to\infty}Q(t)=2\lim_{t\to\infty}Q_1(t)
\int_0^t\frac{d\tau}{Q_1(\tau)}$, a $0\cdot\infty$ indeterminate
form. By L'Hospital's rule,
$\lim_{t\to\infty}Q(t)=2\lim_{t\to\infty}
\frac{1}{Q_1(t)}\bigg/\left(\frac{-Q_1'(t)}{Q_1^2(t)}\right)
=-2\lim_{t\to\infty}\frac{Q_1(t)}{Q_1'(t)}=300$.
\end{solution}

\item % (b)
Use numerical methods to  confirm  your guess in the these cases:
\begin{enumerate}
    \item $a(t)=\frac{t}{1+t}$
    \item $a(t)=1-e^{-t^2}$
    \item $a(t)=1-\sin(e^{-t})$
\end{enumerate}

\end{enumerate}
\end{problem}

\begin{problem}\label{exer:4.2.23}  
Consider the mixing problem of Example~\ref{example:4.2.4} in a tank with
infinite capacity, but without the assumption that the mixture is
stirred instantly so that the salt is always uniformly distributed
throughout the mixture. Assume instead that the distribution
approaches uniformity as $t\to\infty$. In this case the differential
equation for $Q$ is of the form
$$
Q'+\frac{a(t)}{t+100}Q=1
$$
where  $\lim_{t\to\infty}a(t)=1$.
\begin{enumerate}
\item % (a)
Let $K(t)$ be the concentration of salt at time $t$. Assuming that
$Q(0)=Q_0$, can you guess the value of $\lim_{t\to\infty}K(t)$?
\item % (b)
Use numerical methods to  confirm  your guess in the these cases:
\begin{enumerate}
    \item $a(t)=\frac{t}{1+t}$
    \item $a(t)=1-e^{-t^2}$
    \item $a(t)=1-\sin(e^{-t})$
\end{enumerate}


\end{enumerate}
\end{problem}

\end{document}