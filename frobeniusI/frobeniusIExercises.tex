\documentclass{ximera}
%% You can put user macros here
%% However, you cannot make new environments

%\listfiles

% Get the 'old' hints/expandables, for use on ximera.osu.edu
%\def\xmNotHintAsExpandable{true}
%\def\xmNotExpandableAsAccordion{true}



%\graphicspath{{./}{firstExample/}{secondExample/}}
\graphicspath{{./}
{aboutDiffEq/}
{applicationsLeadingToDiffEq/}
{applicationsToCurves/}
{autonomousSecondOrder/}
{basicConcepts/}
{bernoulli/}
{constCoeffHomSysI/}
{constCoeffHomSysII/}
{constCoeffHomSysIII/}
{constantCoeffWithImpulses/}
{constantCoefficientHomogeneousEquations/}
{convolution/}
{coolingActivity/}
{directionFields/}
{drainingTank/}
{epidemicActivity/}
{eulersMethod/}
{exactEquations/}
{existUniqueNonlinear/}
{frobeniusI/}
{frobeniusII/}
{frobeniusIII/}
{global.css/}
{growthDecay/}
{heatingCoolingActivity/}
{higherOrderConstCoeff/}
{homogeneousLinearEquations/}
{homogeneousLinearSys/}
{improvedEuler/}
{integratingFactors/}
{interactExperiment/}
{introToLaplace/}
{introToSystems/}
{inverseLaplace/}
{ivpLaplace/}
{laplaceTable/}
{lawOfCooling/}
{linSysOfDiffEqs/}
{linearFirstOrderDiffEq/}
{linearHigherOrder/}
{mixingProblems/}
{motionUnderCentralForce/}
{nonHomogeneousLinear/}
{nonlinearToSeparable/}
{odesInSage/}
{piecewiseContForcingFn/}
{population/}
{reductionOfOrder/}
{regularSingularPts/}
{reviewOfPowerSeries/}
{rlcCircuit/}
{rungeKutta/}
{secondLawOfMotion/}
{separableEquations/}
{seriesSolNearOrdinaryPtI/}
{seriesSolNearOrdinaryPtII/}
{simplePendulum/}
{springActivity/}
{springProblemsI/}
{springProblemsII/}
{undCoeffHigherOrderEqs/}
{undeterminedCoeff/}
{undeterminedCoeff2/}
{unitStepFunction/}
{varParHigherOrder/}
{varParamNonHomLinSys/}
{variationOfParameters/}
}


\usepackage{tikz}
%\usepackage{tkz-euclide}
\usepackage{tikz-3dplot}
\usepackage{tikz-cd}
\usetikzlibrary{shapes.geometric}
\usetikzlibrary{arrows}
\usetikzlibrary{decorations.pathmorphing,patterns}
\usetikzlibrary{backgrounds} % added by Felipe
% \usetkzobj{all}   % NOT ALLOWED IN RECENT TeX's ...
\pgfplotsset{compat=1.13} % prevents compile error.

\pdfOnly{\renewcommand{\theHsection}{\thepart.section.\thesection}}  %% MAKES LINKS WORK should be added to CLS
\pdfOnly{\renewcommand{\part}[1]{\chapterstyle\title{#1}\begin{abstract}\end{abstract}\maketitle\def\thechaptertitle{#1}}}


\renewcommand{\vec}[1]{\mathbf{#1}}
\newcommand{\RR}{\mathbb{R}}
\providecommand{\dfn}{\textit}
\renewcommand{\dfn}{\textit}
\newcommand{\dotp}{\cdot}
\newcommand{\id}{\text{id}}
\newcommand\norm[1]{\left\lVert#1\right\rVert}
\newcommand{\dst}{\displaystyle}
 
\newtheorem{general}{Generalization}
\newtheorem{initprob}{Exploration Problem}

\tikzstyle geometryDiagrams=[ultra thick,color=blue!50!black]

\usepackage{mathtools}

\title{Exercises} \license{CC BY-NC-SA 4.0}

\begin{document}

\begin{abstract}
\end{abstract}
\maketitle

\begin{onlineOnly}
\section*{Exercises}
\end{onlineOnly}




\section*{Verification Procedure}

Let $L$ and
$Y_n(x; r_i)$ be defined by
$$
Ly=
x^2(\alpha_0+\alpha_1x+\alpha_2x^2)y''+x(\beta_0+\beta_1x+\beta_2x^2)y'
+(\gamma_0+\gamma_1x+\gamma_2x^2)y
$$
and
$$
y_N(x;r_i)=x^{r_i}\sum_{n=0}^N a_n(r_i)x^n,
$$
where the coefficients $\{a_n(r_i)\}_{n=0}^N$ are computed as in
(\ref{eq:7.5.12}), Theorem~$\ref{thmtype:7.5.2}$. Compute the error
\begin{equation} \label{eq:7.5.27}
E_N(x;r_i)=x^{-r_i}Ly_N(x;r_i)/\alpha_0
\end{equation}
for various values of $N$ and various values of $x$ in the interval
$(0,\rho)$, with $\rho$ as defined in Theorem~$\ref{thmtype:7.5.2}$.

The multiplier $x^{-r_i}/\alpha_0$ on the right of (\ref{eq:7.5.27})
eliminates the effects of small or large values of $x^{r_i}$ near
$x=0$, and of multiplication by an arbitrary constant. In some
exercises you will be asked to estimate the maximum value of
$E_N(x; r_i)$ on an interval $(0,\delta]$ by computing $E_N(x_m;r_i)$
at the $M$ points $x_m=m\delta/M,\; m=1$, $2$, \dots, $M$, and finding the
maximum of the absolute values:
\begin{equation} \label{eq:7.5.28}
\sigma_N(\delta)=\max\{|E_N(x_m;r_i)|,\;   m=1,2,\dots,M\}.
\end{equation}
(For simplicity, this notation ignores the dependence of
the right side of the equation on $i$ and $M$.)

To implement this procedure, you'll have to write a computer program to
calculate $\{a_n(r_i)\}$ from the applicable recurrence relation, and
to evaluate $E_N(x;r_i)$.

The next exercise set contains five exercises that ask you to implement the verification
procedure. These particular exercises were chosen arbitrarily --   you can
just as well formulate such laboratory problems for any of the
equations in any  of the Exercises~\ref{exer:7.5.1}--\ref{exer:7.5.10},
\ref{exer:7.5.14}--\ref{exer:7.5.25}, and \ref{exer:7.5.28}--\ref{exer:7.5.51}


\begin{problem}\label{exer:7.5.1} 
Find a fundamental
set
of Frobenius solutions. Compute $a_0$, $a_{1}$ \dots, $a_N$ for $N$ at
least
$7$ in
each solution.

$2x^2(1+x+x^2)y''+x(3+3x+5x^2)y'-y=0$

\end{problem}

%%%% HACK:  TO BE REMOVED %%%%
%%%% Something wrong with \begin{solution} in html infra ...
\end{document}

\begin{problem}\label{exer:7.5.2}
Find a fundamental
set
of Frobenius solutions. Compute $a_0$, $a_{1}$ \dots, $a_N$ for $N$ at
least
$7$ in
each solution.

$3x^2y''+2x(1+x-2x^2)y'+(2x-8x^2)y=0$

\begin{solution}
    $p_0(r)=r(3r-1)$;
$p_1(r)=2(r+1)$;
$p_2(r)=-4(r+2)$.

$a_1(r)=-\frac{2}{3r+2}$;
$a_n(r)=-\frac{2a_{n-1}(r)-4a_{n-2}(r)}{3n+3r-1}$,
$n\geq 1$.

$r_1=1/3$;
$a_1(1/3)=-2/3$;
$a_n(1/3)=-\frac{2a_{n-1}(1/3)-4a_{n-2}(1/3)}{3n}$,
$n\geq $1;

$y_1=x^{1/3}\left(1-\frac{2}{3}x+\frac{8}{9}x^2-
\frac{40}{81}x^3+\cdots\right)$.


$r_2=0$;
$a_1(0)=-1$;
$a_n(0)=-\frac{2a_{n-1}(0)-4a_{n-2}(0)}{3n-1}$,
$n\geq 1$;

$y_2=1-x+\frac{6}{5}x^2-\frac{4}{5}x^3+\cdots$.
\end{solution}

\end{problem}

\begin{problem}\label{exer:7.5.3}
Find a fundamental
set
of Frobenius solutions. Compute $a_0$, $a_{1}$ \dots, $a_N$ for $N$ at
least
$7$ in
each solution.

$x^2(3+3x+x^2)y''+x(5+8x+7x^2)y'-(1-2x-9x^2)y=0$

\end{problem}

\begin{problem}\label{exer:7.5.4}
Find a fundamental
set
of Frobenius solutions. Compute $a_0$, $a_{1}$ \dots, $a_N$ for $N$ at
least
$7$ in
each solution.

$4x^2y''+x(7+2x+4x^2)y'-(1-4x-7x^2)y=0$

\begin{solution}
    $p_0(r)=(r+1)(4r-1)$;
$p_1(r)=2(r+2)$;
$p_2(r)=4r+7$.

$a_1(r)=-\frac{2}{ 4r+3}$;
$a_n(r)=-\frac{2}{ 4n+4r-1}a_{n-1}(r)-\frac{1}{ n+r+1}a_{n-2}(r)$,
$n\geq 1$.

$r_1=1/4$;
$a_1(1/4)=-1/2$;
$a_n(1/4)=-\frac{1}{ 2n}a_{n-1}(1/4)-\frac{4}{ 4n+5}a_{n-2}(1/4)$,
$n\geq $1;


$y_1=x^{1/4}\left(1-\frac{1}{2}x-\frac{19}{104}x^2+
\frac{1571}{10608}x^3+\cdots\right)$.

$r_2=-1$;
$a_1(-1)=2$;
$a_n(-1)=-\frac{2}{ 4n-5}a_{n-1}(-1)-\frac{1}{ n}a_{n-2}(-1)$,
$n\geq 1$;

$y_2=x^{-1}\left(1+2x-\frac{11}{6}x^2-\frac{1}{7}x^3+\cdots\right)$.
\end{solution}

\end{problem}

\begin{problem}\label{exer:7.5.5}
Find a fundamental
set
of Frobenius solutions. Compute $a_0$, $a_{1}$ \dots, $a_N$ for $N$ at
least
$7$ in
each solution.

$12x^2(1+x)y''+x(11+35x+3x^2)y'-(1-10x-5x^2)y=0$

\end{problem}

\begin{problem}\label{exer:7.5.6}
Find a fundamental
set
of Frobenius solutions. Compute $a_0$, $a_{1}$ \dots, $a_N$ for $N$ at
least
$7$ in
each solution.

 $x^2(5+x+10x^2)y''+x(4+3x+48x^2)y'+(x+36x^2)y=0$

 \begin{solution}
     $p_0(r)=r(5r-1)$;
$p_1(r)=(r+1)^2$;
$p_2(r)=2(r+2)(5r+9)$.


$a_1(r)=-\frac{r+1}{5r+4}$;
$a_n(r)=-\frac{n+r}{5n+5r-1}a_{n-1}(r)-2a_{n-2}(r)$,
$n\geq 1$.

$r_1=1/5$;
$a_1(1/5)=-6/25$;
$a_n(1/5)=-\frac{5n+1}{25n}a_{n-1}(1/5)-2a_{n-2}(1/5)$,
$n\geq $1;

$y_1=x^{1/5}\left(1-\frac{6}{25}x-\frac{1217}{625}x^2+
\frac{41972}{46875}x^3
+\cdots\right)$.


$r_2=0$;
$a_1(0)=-1/4$;
$a_n(0)=-\frac{n}{5n-1}a_{n-1}(0)-2a_{n-2}(0)$,
$n\geq 1$;

$y_2=x-\frac{1}{4}x^2-\frac{35}{18}x^3+\frac{11}{12}x^4+\cdots$.
 \end{solution}
 \end{problem}

\begin{problem}\label{exer:7.5.7} 
Find a fundamental
set
of Frobenius solutions. Compute $a_0$, $a_{1}$ \dots, $a_N$ for $N$ at
least
$7$ in
each solution.

$8x^2y''-2x(3-4x-x^2)y'+(3+6x+x^2)y=0$

\end{problem}

\begin{problem}\label{exer:7.5.8} 
Find a fundamental
set
of Frobenius solutions. Compute $a_0$, $a_{1}$ \dots, $a_N$ for $N$ at
least
$7$ in
each solution.

$18x^2(1+x)y''+3x(5+11x+x^2)y'-(1-2x-5x^2)y=0$

\begin{solution}
    $p_0(r)=(3r-1)(6r+1)$;
$p_1(r)=(3r+2)(6r+1)$;
$p_2(r)=3r+5$.

$a_1(r)=-\frac{6r+1}{6r+7}$;
$a_n(r)=-\frac{6n+6r-5}{6n+6r+1}a_{n-1}(r)-\frac{1}{6n+6r+1}a_{n-2}(r)$,
$n\geq 1$.

$r_1=1/3$;
$a_1(1/3)=-1/3$;
$a_n(1/3)=-\frac{2n-1}{2n+1}a_{n-1}(1/3)-\frac{1}{6n+3}a_{n-2}(1/3)$,
$n\geq $1;

$y_1=x^{1/3}\left(1-\frac{1}{3}x+\frac{2}{15}x^2-
\frac{5}{63}x^3+\cdots\right)$.


$r_2=-1/6$;
$a_1(-1/6)=0$;
$a_n(-1/6)=-\frac{n-1}{ n}a_{n-1}(-1/6)-\frac{1}{6n}a_{n-2}(-1/6)$,
$n\geq 1$;

$y_2=x^{-1/6}\left(1-\frac{1}{12}x^2+\frac{1}{18}x^3+\cdots\right)$.
\end{solution}
\end{problem}

\begin{problem}\label{exer:7.5.9}
Find a fundamental
set
of Frobenius solutions. Compute $a_0$, $a_{1}$ \dots, $a_N$ for $N$ at
least
$7$ in
each solution.

$x(3+x+x^2)y''+(4+x-x^2)y'+xy=0$
\end{problem}

\begin{problem}\label{exer:7.5.10} 
Find a fundamental
set
of Frobenius solutions. Compute $a_0$, $a_{1}$ \dots, $a_N$ for $N$ at
least
$7$ in
each solution.

$10x^2(1+x+2x^2)y''+x(13+13x+66x^2)y'-(1+4x+10x^2)y=0$

\begin{solution}
    $p_0(r)=(2r+1)(5r-1)$;
$p_1(r)=(2r-1)(5r+4)$;
$p_2(r)=2(2r+5)(5r-1)$.

$a_1(r)=-\frac{2r-1}{2r+3}$;
$a_n(r)=-\frac{2n+2r-3}{2n+2r+1}a_{n-1}(r)-
\frac{10n+10r-22}{5n+5r-1}a_{n-2}(r)$,
$n\geq 1$.

$r_1=1/5$;
$a_1(1/5)=3/17$;
$a_n(1/5)=-\frac{10n-13}{10n+7}a_{n-1}(1/5)-\frac{2n-4}{ n}a_{n-2}(1/5)$,
$n\geq $1;


$y_1=x^{1/5}\left(1+\frac{3}{17}x-\frac{7}{153}x^2-
\frac{547}{5661}x^3+\cdots\right)$.

$r_2=-1/2$;
$a_1(-1/2)=1$;
$a_n(-1/2)=-\frac{n-2}{ n}a_{n-1}(-1/2)-\frac{20n-54}{10n-7}a_{n-2}(-1/2)$,
$n\geq 1$;

$y_2=x^{-1/2}\left(1+x+\frac{14}{13}x^2-\frac{556}{897}x^3+\cdots\right)$.
\end{solution}
\end{problem}

\begin{problem}\label{exer:7.5.11}
The Frobenius solutions of
$$
2x^2(1+x+x^2)y''+x(9+11x+11x^2)y'+(6+10x+7x^2)y=0
$$
obtained in Example~\ref{example:7.5.1} are defined on $(0,\rho)$,
where $\rho$ is defined in Theorem~\ref{thmtype:7.5.2}. Find $\rho$.
Then do the following experiments for each Frobenius solution,
with $M=20$ and $\delta=.5\rho$, $.7\rho$, and $.9\rho$
in the verification procedure described above.
\begin{enumerate}
\item % (a)
Compute  $\sigma_N(\delta)$  (see Eqn.~(\ref{eq:7.5.28}))
for $N=5$, $10$, $15$,\dots, $50$.
\item % (b)
Find $N$ such that $\sigma_N(\delta)<10^{-5}$.
\item % (b)
Find $N$ such that $\sigma_N(\delta)<10^{-10}$.
\end{enumerate}
\end{problem}


\begin{problem}\label{exer:7.5.12}
By Theorem~\ref{thmtype:7.5.2}
the Frobenius solutions of the equation in Exercise~\ref{exer:7.5.4} are
defined on $(0,\infty)$. Do experiments in parts (a), (b), and (c) of
Exercise~\ref{exer:7.5.11} for each Frobenius solution, with $M=20$ and
$\delta=1$, $2$, and $3$ in the verification procedure described at
the end of this section.
\end{problem}


\begin{problem}\label{exer:7.5.13}
The Frobenius solutions of the equation in Exercise~\ref{exer:7.5.6} are
defined on $(0,\rho)$, where $\rho$ is defined in
Theorem~\ref{thmtype:7.5.2}. Find $\rho$ and do experiments (a), (b), and (c) of Exercise~\ref{exer:7.5.11} for each Frobenius solution, with
$M=20$ and $\delta=.3\rho$, $.4\rho$, and $.5\rho$, in the verification
procedure described above.
\end{problem}


\begin{problem}\label{exer:7.5.14} 
Find a
fundamental set of Frobenius solutions. Give explicit formulas for the
coefficients in each solution.

$2x^2y''+x(3+2x)y'-(1-x)y=0$

\begin{solution}
    $p_0(r)=(r+1)(2r-1)$;
$p_1(r)=2r+1$;
$a_n(r)=-\frac{1}{ n+r+1}a_{n-1}(r)$.

$r_1=1/2$; $a_n(1/2)=-\frac{2}{ 2n+3}a_{n-1}(1/2)$;
$y_1=x^{1/2}\sum_{n=0}^\infty\frac{(-2)^n}{\prod_{j=1}^n(2j+3)}x^n$.

$r_2=-1$; $a_n(-1)=-\frac{1}{ n}a_{n-1}(-1)$;
$y_2=x^{-1}\sum_{n=0}^\infty\frac{(-1)^n}{ n!}x^n$.
\end{solution}
\end{problem}

\begin{problem}\label{exer:7.5.15} 
Find a
fundamental set of Frobenius solutions. Give explicit formulas for the
coefficients in each solution.

$x^2(3+x)y''+x(5+4x)y'-(1-2x)y=0$

\end{problem}

\begin{problem}\label{exer:7.5.16} 
Find a
fundamental set of Frobenius solutions. Give explicit formulas for the
coefficients in each solution.

$2x^2y''+x(5+x)y'-(2-3x)y=0$

\begin{solution}
    $p_0(r)=(r+2)(2r-1)$;
$p_1(r)=r+3$;
$a_n(r)=-\frac{1}{ 2n+2r-1}a_{n-1}(r)$.

$r_1=1/2$;
$a_n(1/2)= -\frac{1}{ 2n}a_{n-1}(1/2)$;
$y_1=x^{1/2}\sum_{n=0}^\infty \frac{(-1)^n}{2^nn!}x^n$.


$r_2=-2$;
$a_n(-2)=-\frac{1}{ 2n-5}a_{n-1}(-2)$;
$y_2=\frac{1}{ x^2}\sum_{n=0}^\infty\frac{(-1)^n}{\prod_{j=1}^n(2j-5)}
x^n$.
\end{solution}

\end{problem}

\begin{problem}\label{exer:7.5.17} 
Find a
fundamental set of Frobenius solutions. Give explicit formulas for the
coefficients in each solution.

$3x^2y''+x(1+x)y'-y=0$

\end{problem}

\begin{problem}\label{exer:7.5.18} 
Find a
fundamental set of Frobenius solutions. Give explicit formulas for the
coefficients in each solution.

$2x^2y''-xy'+(1-2x)y=0$

\begin{solution}
    $p_0(r)=(r-1)(2r-1)$;
$p_1(r)= -2$;
$a_n(r)=\frac{2}{ (n+r-1)(2n+2r-1)}a_{n-1}(r)$.

$r_1=1$;
$a_n(1)=\frac{2}{ n(2n+1})a_{n-1}(1)$;
$y_1=x\sum_{n=0}^\infty\frac{2^n}{
n!\prod_{j=1}^n(2j+1)}x^n$.

$r_2=1/2$;
$a_n(1/2)=\frac{2}{ n(2n-1})a_{n-1}(1/2)$;
$y_2=x^{1/2}\sum_{n=0}^\infty\frac{2^n}{
n!\prod_{j=1}^n(2j-1)}x^n$.

\end{solution}

\end{problem}

\begin{problem}\label{exer:7.5.19} 
Find a
fundamental set of Frobenius solutions. Give explicit formulas for the
coefficients in each solution.

$9x^2y''+9xy'-(1+3x)y=0$

\end{problem}

\begin{problem}\label{exer:7.5.20} 
Find a
fundamental set of Frobenius solutions. Give explicit formulas for the
coefficients in each solution.

$3x^2y''+x(1+x)y'-(1+3x)y=0$

\begin{solution}
    $p_0(r)=(r-1)(3r+1)$;
$p_1(r)=r-3$;
$a_n(r)=-\frac{n+r-4}{ (n+r-1)(3n+3r+1)}a_{n-1}(r)$.

$r_1=$;
$a_n(1)=-\frac{n-3}{ n(3n+4)}a_{n-1}(1)$;
$y_1=x\left(1+\frac{2}{7}x+\frac{1}{70}x^2\right)$.

$r_2=-1/3$;
$a_n(-1/3)=-\frac{3n-13}{ 3n(3n-4)}a_{n-1}(-1/3)$;
$y_2=x^{-1/3}\sum_{n=0}^\infty\frac{(-1)^n}{3^nn!}\left(\prod_{j=1}^n
\frac{3j-13}{3j-4}\right)x^n$.
\end{solution}

\end{problem}

\begin{problem}\label{exer:7.5.21} 
Find a
fundamental set of Frobenius solutions. Give explicit formulas for the
coefficients in each solution.

$2x^2(3+x)y''+x(1+5x)y'+(1+x)y=0$

\end{problem}

\begin{problem}\label{exer:7.5.22} 
Find a
fundamental set of Frobenius solutions. Give explicit formulas for the
coefficients in each solution.

$x^2(4+x)y''-x(1-3x)y'+y=0$

\begin{solution}
    $p_0(r)=(r-1)(4r-1)$;
$p_1(r)=r(r+2)$;
$a_n(r)=-\frac{n+r+1}{ 4n+4r-1}a_{n-1}(r)$.

$r_1=1$;
$a_n(1)= -\frac{n+2}{ 4n+3}a_{n-1}(1)$;
$y_1=x\sum_{n=0}^\infty\frac{(-1)^n(n+2)!}{2\prod_{j=1}^n(4j+3)}
x^n$.

$r_2=1/4$;
$a_n(1/4)=-\frac{4n+5}{ 16n}a_{n-1}(1/4)$;
$y_2=x^{1/4}\sum_{n=0}^\infty\frac{(-1)^n}{16^nn!}\prod_{j=1}^n(4j+5)
x^n$
\end{solution}

\end{problem}

\begin{problem}\label{exer:7.5.23} 
Find a
fundamental set of Frobenius solutions. Give explicit formulas for the
coefficients in each solution.

$2x^2y''+5xy'+(1+x)y=0$

\end{problem}

\begin{problem}\label{exer:7.5.24} 
Find a
fundamental set of Frobenius solutions. Give explicit formulas for the
coefficients in each solution.

$x^2(3+4x)y''+x(5+18x)y'-(1-12x)y=0$

\begin{solution}
    $p_0(r)=(r+1)(3r-1)$;
$p_1(r)=2(r+2)(2r+3)$;
$a_n(r)=-2\frac{2n+2r+1}{ 3n+3r-1}a_{n-1}(r)$.

$r_1=1/3$;
$a_n(1/3)=-2\frac{6n+5}{ 9n}a_{n-1}(1/3)$;
$y_1=x^{1/3}\sum_{n=0}^\infty\frac{(-1)^n}{
n!}\left(\frac{2}{9}\right)^n\left(\prod_{j=1}^n(6j+5)\right)
x^n$;

$r_2=-1$;
$a_n(-1)=-2\frac{2n-1}{ 3n-4}a_{n-1}(-1)$;
$y_2=x^{-1}\sum_{n=0}^\infty(-1)^n2^n\left(\prod_{j=1}^n
\frac{2j-1}{3j-4}\right) x^n$
\end{solution}

\end{problem}

\begin{problem}\label{exer:7.5.25} 
Find a
fundamental set of Frobenius solutions. Give explicit formulas for the
coefficients in each solution.

$6x^2y''+x(10-x)y'-(2+x)y=0$

\end{problem}

\begin{problem}\label{exer:7.5.26}
By Theorem~\ref{thmtype:7.5.2} the Frobenius solutions of the equation in
Exercise~\ref{exer:7.5.17} are defined on $(0,\infty)$. Do experiments
in parts (a), (b), and (c) of Exercise~\ref{exer:7.5.11} for each Frobenius
solution,
with $M=20$ and $\delta=3$, $6$, $9$, and $12$ in the verification
procedure described at the end of this section.
\end{problem}

\begin{problem}\label{exer:7.5.27}
The Frobenius solutions of the equation in Exercise~\ref{exer:7.5.22} are
defined on $(0,\rho)$, where $\rho$ is defined in
Theorem~\ref{thmtype:7.5.2}. Find $\rho$ and do experiments in parts (a), (b), and (c) of Exercise~\ref{exer:7.5.11} for each Frobenius solution, with
$M=20$ and
$\delta=.25\rho$, $.5\rho$, and $.75\rho$ in the verification
procedure described at the end of this section.
\end{problem}


\begin{problem}\label{exer:7.5.28} 
Find a
fundamental set of Frobenius solutions. Compute
coefficients $a_0$, \dots, $a_N$ for $N$ at least $7$ in each solution.

$x^2(8+x)y''+x(2+3x)y'+(1+x)y=0$

\begin{solution}
    $p_0(r)=(2r-1)(4r-1)$;
$p_1(r)=(r+1)^2$;
$a_n(r)=-\frac{(n+r)^2}{(2n+2r-1)(4n+4r-1)} a_{n-1}(r)$.

$r_1=1/2$;
$a_n(1/2)=-\frac{4n^2+4n+1}{8n(4n+1)} a_{n-1}(1/2)$;
$y_1=x^{1/2}\left(1-\frac{9}{40}x+\frac{5}{128}x^2-\frac{245}{39936}x^3
+\cdots\right)$.

$r_2=1/4$;
$a_n(1/4)=-\frac{16n^2+8n+1}{32n(4n-1)} a_{n-1}(1/4)$;
$y_2=x^{1/4}\left(1-\frac{25}{96}x+\frac{675}{14336}x^2-
\frac{38025}{5046272}x^3
+\cdots\right)$.
\end{solution}
\end{problem}

\begin{problem}\label{exer:7.5.29}
Find a
fundamental set of Frobenius solutions. Compute
coefficients $a_0$, \dots, $a_N$ for $N$ at least $7$ in each solution.

$x^2(3+4x)y''+x(11+4x)y'-(3+4x)y=0$
\end{problem}

\begin{problem}\label{exer:7.5.30}
Find a
fundamental set of Frobenius solutions. Compute
coefficients $a_0$, \dots, $a_N$ for $N$ at least $7$ in each solution.

$2x^2(2+3x)y''+x(4+11x)y'-(1-x)y=0$

\begin{solution}
    $p_0(r)=(2r-1)(2r+1)$;
$p_1(r)=(2r+1)(3r+1)$;
$a_n(r)=-\frac{(3n+3r-2)}{(2n+2r+1)} a_n(r)$.

$r_1=1/2$;
$a_n(1/2)=-\frac{6n-1}{4(n+1)} a_{n-1}(1/2)$;
$y_1=x^{1/2}\left(1-\frac{5}{8}x+\frac{55}{96}x^2
-\frac{935}{1536}x^3+\cdots\right)$.

$r_2=-1/2$;
$a_n(-1/2)=-\frac{6n-7}{4n} a_{n-1}(-1/2)$;
$y_2=x^{-1/2}\left(1+\frac{1}{4}x-\frac{5}{32}x^2
-\frac{55}{384}x^3+\cdots\right)$.
\end{solution}
\end{problem}

\begin{problem}\label{exer:7.5.31}
Find a
fundamental set of Frobenius solutions. Compute
coefficients $a_0$, \dots, $a_N$ for $N$ at least $7$ in each solution.

$x^2(2+x)y''+5x(1-x)y'-(2-8x)y$
\end{problem}

\begin{problem}\label{exer:7.5.32} 
Find a
fundamental set of Frobenius solutions. Compute
coefficients $a_0$, \dots, $a_N$ for $N$ at least $7$ in each solution.

$x^2(6+x)y''+x(11+4x)y'+(1+2x)y=0$

\begin{solution}
    $p_0(r)=(2r+1)(3r+1)$;
$p_1(r)=(r+1)(r+2)$;
$a_n(r)=\frac{(n+r)(n+r+1)}{(2n+2r+1)(3n+3r+1)} a_n(r)$.

$r_1=-1/3$;
$a_n(-1/3)=-\frac{(3n-1)(3n+2)}{9n(6n+1)}a_{n-1}(-1/3)$;
$y_1=x^{-1/3}\left(1-\frac{10}{63}x+\frac{200}{7371}x^2-
\frac{17600}{3781323}x^3
+\cdots\right)$.

$r_2=-1/2$;
$a_n(-1/2)=-\frac{(2n-1)(2n+1)}{4n(6n-1)} a_{n-1}(-1/2)$;
$y_2=x^{-1/2}\left(1-\frac{3}{20}x+\frac{9}{352}x^2
-\frac{105}{23936}x^3
+\cdots\right)$.
\end{solution}
\end{problem}


\begin{problem}\label{exer:7.5.33} 
Find a
fundamental set of Frobenius solutions. Give explicit formulas for
the coefficients in each solution.

$8x^2y''+x(2+x^2)y'+y=0$
\end{problem}

\begin{problem}\label{exer:7.5.34} 
Find a
fundamental set of Frobenius solutions. Give explicit formulas for
the coefficients in each solution.

$8x^2(1-x^2)y''+2x(1-13x^2)y'+(1-9x^2)y=0$

\begin{solution}
    $p_0(r)=(2r-1)(4r-1)$;
$p_2(r)=-(2r+3)(4r+3)$;
$a_{2m}(r)=\frac{8m+4r-5}{8m+4r-1}a_{2m-2}(r)$.

$r_1=1/2$;
$a_{2m}(1/2)=\frac{8m-3}{8m+1}a_{2m-2}(1/2)$;
$y_1=x^{1/2}\sum_{m=0}^\infty\left(\prod_{j=1}^m\frac{8j-3}{8j+
1}\right)x^{2m}$.

$r_2=1/4$;
$a_{2m}(1/4)=\frac{2m-1}{2m}a_{2m-2}(1/4)$;
$y_2=x^{1/4}\sum_{m=0}^\infty\frac{1}{2^mm!}\left(\prod_{j=1}^m(2j-1)
\right)x^{2m}$
\end{solution}
\end{problem}

\begin{problem}\label{exer:7.5.35} 
Find a
fundamental set of Frobenius solutions. Give explicit formulas for
the coefficients in each solution.

$x^2(1+x^2)y''-2x(2-x^2)y'+4y=0$
\end{problem}

\begin{problem}\label{exer:7.5.36} 
Find a
fundamental set of Frobenius solutions. Give explicit formulas for
the coefficients in each solution.

$x(3+x^2)y''+(2-x^2)y'-8xy=0$

\begin{solution}
    $p_0(r)=r(3r-1)$;
$p_2(r)=(r-4)(r+2)$;
$a_{2m}(r)=-\frac{2m+r-6}{6m+3r-1}a_{2m-2}(r)$.

$r_1=1/3$;
$a_{2m}(1/3)=-\frac{6m-17}{18m}a_{2m-2}(1/3)$;
$y_1=x^{1/3}\sum_{m=0}^\infty\frac{(-1)^m}{18^mm!}\left(\prod_{j=1}^m
(6j-17)\right) x^{2m}$.

$r_2=0$;
$a_{2m}(0)=-\frac{2m-6}{6m-1}a_{2m-2}(0)$;
$y_2=1+\frac{4}{5}x^2+\frac{8}{55}x^4$
\end{solution}
\end{problem}

\begin{problem}\label{exer:7.5.37} 
Find a
fundamental set of Frobenius solutions. Give explicit formulas for
the coefficients in each solution.

$4x^2(1-x^2)y''+x(7-19x^2)y'-(1+14x^2)y=0$
\end{problem}

\begin{problem}\label{exer:7.5.38} 
Find a
fundamental set of Frobenius solutions. Give explicit formulas for
the coefficients in each solution.

$3x^2(2-x^2)y''+x(1-11x^2)y'+(1-5x^2)y=0$

\begin{solution}
    $p_0(r)=(2r-1)(3r-1)$;
$p_2(r)=-(r+1)(3r+5)$;
$a_{2m}(r)=\frac{2m+r-1}{4m+2r-1}a_{2m-2}(r)$.

$r_1=1/2$;
$a_{2m}(1/2)=\frac{4m-1}{8m}a_{2m-2}(1/2)$;
$y_1=x^{1/2}\sum_{m=0}^\infty\frac{1}{8^mm!}\left(\prod_{j=1}^m(4j-1)
 \right)x^{2m}$.

$r_2=1/3$;
$a_{2m}(1/3)=\frac{6m-2}{12m-1}a_{2m-2}(1/3)$;
$y_2=x^{1/3}\sum_{m=0}^\infty2^m\left(\prod_{j=1}^m\frac{3j-1}{12j-1}
\right) x^{2m}$.
\end{solution}
\end{problem}

\begin{problem}\label{exer:7.5.39} 
Find a
fundamental set of Frobenius solutions. Give explicit formulas for
the coefficients in each solution.

$2x^2(2+x^2)y''-x(12-7x^2)y'+(7+3x^2)y=0$
\end{problem}

\begin{problem}\label{exer:7.5.40} 
Find a
fundamental set of Frobenius solutions. Give explicit formulas for
the coefficients in each solution.

$2x^2(2+x^2)y''+x(4+7x^2)y'-(1-3x^2)y=0$

\begin{solution}
    $p_0(r)=(2r-1)(2r+1)$;
$p_1(r)=(r+1)(2r+3)$;
$a_{2m}(r)=-\frac{2m+r-1}{4m+2r+1}a_{2m-2}(r)$.

$r_1=1/2$;
$a_{2m}(1/2)=-\frac{4m-1}{4(2m+1)}a_{2m-2}(1/2)$;
$y_1=x^{1/2}\sum_{m=0}^\infty\frac{(-1)^m}{4^m}\left(\prod_{j=1}^m
\frac{4j-1}{2j+1}\right)x^{2m}$.

$r_2=-1/2$;
$a_{2m}(-1/2)=-\frac{4m-3}{8m}a_{2m-2}(-1/2)$;
$y_2=x^{-1/2}\sum_{m=0}^\infty\frac{(-1)^m}{8^mm!}\left(\prod_{j=1}^m
(4j-3)\right)x^{2m}$
\end{solution}
\end{problem}

\begin{problem}\label{exer:7.5.41} 
Find a
fundamental set of Frobenius solutions. Give explicit formulas for
the coefficients in each solution.

$2x^2(1+2x^2)y''+5x(1+6x^2)y'-(2-40x^2)y=0$
\end{problem}

\begin{problem}\label{exer:7.5.42} 
Find a
fundamental set of Frobenius solutions. Give explicit formulas for
the coefficients in each solution.

$3x^2(1+x^2)y''+5x(1+x^2)y'-(1+5x^2)y=0$

\begin{solution}
    $p_0(r)=(r+1)(3r-1)$;
$p_1(r)=(r-1)(3r+5)$;
$a_{2m}(r)=-\frac{2m+r-3}{2m+r+1}a_{2m-2}(r)$.

$r_1=1/3$;
$a_{2m}(1/3)=-\frac{3m-4}{3m+2}a_{2m-2}(1/3)$;
$y_1=x^{1/3}\sum_{m=0}^\infty(-1)^m\left(\prod_{j=1}^m
\frac{3j-4}{3j+2}\right)x^{2m}$.

$r_2=-1$;
$a_{2m}(-1)=-\frac{m-2}{ m}a_{2m-2}(-1)$;
 $y_2=x^{-1}(1+x^2)$
\end{solution}
\end{problem}

\begin{problem}\label{exer:7.5.43} 
Find a
fundamental set of Frobenius solutions. Give explicit formulas for
the coefficients in each solution.

$x(1+x^2)y''+(4+7x^2)y'+8xy=0$
\end{problem}

\begin{problem}\label{exer:7.5.44} 
Find a
fundamental set of Frobenius solutions. Give explicit formulas for
the coefficients in each solution.

$x^2(2+x^2)y''+x(3+x^2)y'-y=0$

\begin{solution}
    $p_0(r)=(r+1)(2r-1)$;
$p_1(r)=r^2$;
$a_{2m}(r)=-\frac{(2m+r-2)^2}{(2m+r+1)(4m+2r-1)}a_{2m-2}(r)$.

$r_1=1/2$;
$a_{2m}(1/2)=-\frac{(4m-3)^2}{8m(4m+3)}a_{2m-2}(1/2)$;
$y_1=x^{1/2}\sum_{m=0}^\infty\frac{(-1)^m}{8^mm!}\left(\prod_{j=1}^m
\frac{(4j-3)^2}{4j+3}\right)x^{2m}$.

$r_2=-1$;
$a_{2m}(-1)=-\frac{(2m-3)^2}{2m(4m-3)}a_{2m-2}(-1)$;
$y_2=x^{-1}\sum_{m=0}^\infty\frac{(-1)^m}{2^mm!}\left(\prod_{j=1}^m
\frac{(2j-3)^2}{4j-3}\right)x^{2m}$.
\end{solution}
\end{problem}

\begin{problem}\label{exer:7.5.45} 
Find a
fundamental set of Frobenius solutions. Give explicit formulas for
the coefficients in each solution.

$2x^2(1+x^2)y''+x(3+8x^2)y'-(3-4x^2)y=0$
\end{problem}

\begin{problem}\label{exer:7.5.46} 
Find a
fundamental set of Frobenius solutions. Give explicit formulas for
the coefficients in each solution.

$9x^2y''+3x(3+x^2)y'-(1-5x^2)y=0$

\begin{solution}
    $p_0(r)=(3r-1)(3r+1)$;
$p_1(r)=3r+5$;
$a_{2m}(r)=-\frac{1}{6m+3r+1}a_{2m-2}(r)$.

$r_1=1/3$;
$a_{2m}(1/3)=-\frac{1}{2(3m+1)}a_{2m-2}(1/3)$;
$y_1=x^{1/3}\sum_{m=0}^\infty\frac{(-1)^m}{2^m\prod_{j=1}^m(3j+1)}x^{2m}$.

$r_2=-1/3$;
$a_{2m}(-1/3)=-\frac{1}{6m}a_{2m-2}(-1/3)$;
$y_2=x^{-1/3}\sum_{m=0}^\infty\frac{(-1)^m}{6^mm!} x^{2m}$

\end{solution}
\end{problem}


\begin{problem}\label{exer:7.5.47}
Find a
fundamental set of Frobenius solutions. Compute  the
coefficients $a_0$, \dots, $a_{2M}$ for $M$ at least $7$ in each
solution.

$6x^2y''+x(1+6x^2)y'+(1+9x^2)y=0$
\end{problem}

\begin{problem}\label{exer:7.5.48}
Find a
fundamental set of Frobenius solutions. Compute  the
coefficients $a_0$, \dots, $a_{2M}$ for $M$ at least $7$ in each
solution.

$x^2(8+x^2)y''+7x(2+x^2)y'-(2-9x^2)y=0$

\begin{solution}
    $p_0(r)=2(r+1)(4r-1)$;
$p_2(r)=(r+3)^2$;
$a_{2m}(r)=-\frac{2m+r+1}{2(8m+4r-1)} a_{2m-2}(r)$.

$r_1=1/4$;
$a_{2m}(1/4)=-\frac{8m+5}{64m} a_{2m-2}(1/4)$;
$y_1=x^{1/4}\left(1-\frac{13}{64}x^2+\frac{273}{8192}x^4-
\frac{2639}{524288}x^6 +\cdots\right)$.


$r_2=-1$;
$a_{2m}(-1)=-\frac{m}{8m-5} a_{2m-2}(-1)$;
$y_2=x^{-1}\left(1-\frac{1}{3}x^2+\frac{2}{33}x^4-\frac{2}{209}x^6
+\cdots\right)$.
\end{solution}
\end{problem}

\begin{problem}\label{exer:7.5.49}
Find a
fundamental set of Frobenius solutions. Compute  the
coefficients $a_0$, \dots, $a_{2M}$ for $M$ at least $7$ in each
solution.

$9x^2(1+x^2)y''+3x(3+13x^2)y'-(1-25x^2)y=0$
\end{problem}

\begin{problem}\label{exer:7.5.50} 
Find a
fundamental set of Frobenius solutions. Compute  the
coefficients $a_0$, \dots, $a_{2M}$ for $M$ at least $7$ in each
solution.

$4x^2(1+x^2)y''+4x(1+6x^2)y'-(1-25x^2)y=0$

\begin{solution}
    $p_0(r)=(2r-1)(2r+1)$;
$p_2(r)=(2r+5)^2$;
$a_{2m}(r)=-\frac{4m+2r+1}{4m+2r-1} a_{2m-2}(r)$.

$r_1=1/2$;
$a_{2m}(1/2)=-\frac{2m+1}{2m} a_{2m-2}(1/2)$;
$y_1=x^{1/2}\left(1-\frac{3}{2}x^2+\frac{15}{8}x^4-\frac{35}{16}x^6
+\cdots\right)$.


$r_2=-1/2$;
$a_{2m}(-1/2)=-\frac{2m}{2m-1} a_{2m-2}(-1/2)$;
$y_2=x^{-1/2}\left(1-2x^2+\frac{8}{3}x^4-\frac{16}{5}x^6
+\cdots\right)$.
\end{solution}
\end{problem}

\begin{problem}\label{exer:7.5.51}
Find a
fundamental set of Frobenius solutions. Compute  the
coefficients $a_0$, \dots, $a_{2M}$ for $M$ at least $7$ in each
solution.

$8x^2(1+2x^2)y''+2x(5+34x^2)y'-(1-30x^2)y=0$
\end{problem}


\begin{problem}\label{exer:7.5.52} 
Suppose $r_1>r_2$, $a_0=b_0=1$, and the
Frobenius series
$$
y_1=x^{r_1}\sum_{n=0}^\infty a_nx^n\quad\mbox{ and }
\quad y_2=x^{r_2}\sum_{n=0}^\infty b_nx^n
$$
both converge on an interval $(0,\rho)$.

\begin{enumerate}
\item % (a)
Show that $y_1$ and $y_2$
are linearly independent on $(0,\rho)$. 
    \begin{hint}
    Show that if $c_1$ and
$c_2$
are constants such that $c_1y_1+c_2y_2\equiv0$ on $(0,\rho)$, then
$$
c_1x^{r_1-r_2}\sum_{n=0}^\infty a_nx^n+
c_2\sum_{n=0}^\infty b_nx^n=0,\quad 0<x<\rho.
$$
Then let $x\to0+$ to conclude that $c_2=0$.
    \end{hint}

\begin{solution}
    Multiplying (A) $c_1y_1+c_2y_2\equiv 0$ by $x^{-r_2}$ yields
    
$$c_1x^{r_1-r_2}\sum_{n=0}^\infty a_nx^n+
c_2\sum_{n=0}^\infty b_nx^n=0,\quad 0<x<\rho$$

Letting $x\to0+$ shows that $c_2=0$, since $b_0=1$. Now (A)
reduces to $c_1y_1\equiv0$, so $c_1=0$. Therefore, $y_1$
and $y_2$ are linearly independent on $(0,\rho)$.
\end{solution}

\item % (b)
 Use the  result of part (a) to
complete the proof of Theorem~\ref{thmtype:7.5.3}.

\begin{solution}
    Since
$y_1=\sum_{n=0}^\infty a_n(r_1)x^n$ and
$y_2=\sum_{n=0}^\infty a_n(r_2)x^n$ are linearly independent solutions
of $Ly=0$  $(0,\rho)$, $\{y_1,y_2\}$ is a fundamental
set of solutions of $Ly=0$ on $(0,\rho)$, by Theorem~\ref{thmtype:5.1.6}.
\end{solution}
\end{enumerate}
\end{problem}

\begin{problem}\label{exer:7.5.53}
The equation

\begin{equation} \label{eq:7.5.exer:7.5.53A}
 x^2y''+xy'+(x^2-\nu^2)y=0
\end{equation}
is
\href{http://www-history.mcs.st-and.ac.uk/Mathematicians/Bessel.html}
 {\color{blue}\it Bessel's equation of order~$\nu$\/}.
(Here $\nu$ is a parameter, and this use of ``order'' should not be
confused with its usual use as in ``the order of the equation.'') The
solutions of (\ref{eq:7.5.exer:7.5.53A}) are
\href{http://www-history.mcs.st-and.ac.uk/Mathematicians/Bessel.html}
{\color{blue}\it Bessel
functions of order\/} $\nu$.
\begin{enumerate}
\item % (a)
Assuming that $\nu$ isn't
an integer, find a fundamental set of Frobenius solutions of
(\ref{eq:7.5.exer:7.5.53A}).
\item % (b)
 If $\nu=1/2$, the solutions of
(\ref{eq:7.5.exer:7.5.53A}) reduce to familiar elementary functions. Identify these
functions.
\end{enumerate}
\end{problem}

\begin{problem}\label{exer:7.5.54}
\begin{enumerate}
\item % (a)
Verify that
$$
\frac{d}{dx}\left(|x|^rx^n\right)=(n+r)|x|^rx^{n-1}\mbox{\quad and \quad}
\frac{d^2}{dx^2}\left(|x|^rx^n\right)=(n+r)(n+r-1)|x|^rx^{n-2}
$$
if $x\neq 0$.

\begin{solution}
    If $x>0$, then $|x|^rx^n=x^{n+r}$, so the assertions
are obvious. If $x<0$, then $|x|^r=(-x)^r$, so
$\frac{d}{ dx}|x|^r=-r(-x)^{r-1}=\frac{r(-x)^r}{ x}=\frac{r|x|^r}{ x}$.
Therefore,(A) $\frac{d}{ dx}(|x|^rx^n)=\frac{r|x|^r}{
x}x^n+|x|^r(nx^{n-1})=(n+r)|x|^rx^{n-1}$ and
$\frac{d^2}{ dx^2}(|x|^rx^n)=(n+r)\frac{d}{ dx}(|x|^rx^{n-1})=
(n+r)(n+r-1)|x|^rx^{n-2}$, from (A) with $n$ replaced by $n-1$.
\end{solution}

\item % (b)
Let
$$
Ly=
x^2(\alpha_0+\alpha_1x+\alpha_2x^2)y''+x(\beta_0+\beta_1x+\beta_2x^2)y'
+(\gamma_0+\gamma_1x+\gamma_2x^2)y=0.
$$
 Show   that if $x^r\sum_{n=0}^\infty a_nx^n$ is a solution of
$Ly=0$ on
$(0,\rho)$ then  $|x|^r\sum_{n=0}^\infty a_nx^n$ is a solution on
$(-\rho,0)$ and $(0,\rho)$.
\end{enumerate}
\end{problem}

\begin{problem}\label{exer:7.5.55}
\begin{enumerate}
\item % (a)
Deduce from Eqn.~(\ref{eq:7.5.20}) that
$$
a_n(r)=(-1)^n\prod_{j=1}^n\frac{p_1(j+r-1)}{ p_0(j+r)}.
$$
\item % (b)
Conclude that if $p_0(r)=\alpha_0(r-r_1)(r-r_2)$ where $r_1-r_2$
is not an integer, then
$$
y_1=x^{r_1}\sum_{n=0}^\infty a_n(r_1)x^n\quad\mbox{ and }\quad
y_2=x^{r_2}\sum_{n=0}^\infty a_n(r_2)x^n
$$
 form a fundamental set of  Frobenius solutions of
$$
x^2(\alpha_0+\alpha_1x)y''+x(\beta_0+\beta_1x)y'+(\gamma_0+\gamma_1x)y=0.
$$
\item % (c)
Show that if $p_0$ satisfies the hypotheses of part (b) then
$$
y_1=x^{r_1}\sum_{n=0}^\infty \frac{(-1)^n}{ n!\prod_{j=1}^n(j+r_1-r_2)}
\left(\frac{\gamma_1}{\alpha_0}\right)^nx^n
$$
and
$$
y_2=x^{r_2}\sum_{n=0}^\infty \frac{(-1)^n}{ n!\prod_{j=1}^n(j+r_2-r_1)}
\left(\frac{\gamma_1}{\alpha_0}\right)^nx^n
$$
form a fundamental set of  Frobenius solutions of
$$
\alpha_0x^2y''+\beta_0xy'+(\gamma_0+\gamma_1x)y=0.
$$
\end{enumerate}
\end{problem}

\begin{problem}\label{exer:7.5.56}
Let
$$
Ly=x^2(\alpha_0+\alpha_2x^2)y''+x(\beta_0+\beta_2x^2)y'+
(\gamma_0+\gamma_2x^2)y=0
$$
and define
$$
p_0(r)=\alpha_0r(r-1)+\beta_0r+\gamma_0\quad\mbox{ and }\quad
p_2(r)=\alpha_2r(r-1)+\beta_2r+\gamma_2.
$$
\begin{enumerate}
\item % (a)
Use Theorem~\ref{thmtype:7.5.2} to show that if
\begin{equation} \label{eq:7.5.exer:7.5.56A}
\begin{array}{rcl}
a_0(r)&=&1,\\
p_0(2m+r)a_{2m}(r)+p_2(2m+r-2)a_{2m-2}(r)&=&0,\quad m\geq 1,
\end{array}
\end{equation}
then the Frobenius series
 $y(x,r)=x^r\sum_{m=0}^\infty a_{2m}x^{2m}$ satisfies
$Ly(x,r)=p_0(r)x^r$.

\begin{solution}
    Here $p_1\equiv0$, so Eqn.~(\ref{eq:7.5.12}) reduces to
$a_0(r)=1$, $a_1(r)=0$, $a_n(r)=-\frac{p_2(n+r-2)}{
p_0(n+r)}a_{n-2}(r)$, $r\geq 0$, which implies that $a_{2m+1}(r)=0$
for $m=1,2,3,\dots$. Therefore,Eqn.~(\ref{eq:7.5.12}) actually reduces to
$a_0(r)=1$, $a_{2m}(r)=-\frac{p_2(2m+r-2)}{ p_0(2m+r)}$, which
holds because of condition (A).
\end{solution}

\item % (b)
Deduce from (\ref{eq:7.5.exer:7.5.56A}) that  if $p_0(2m+r)$ is nonzero
for every positive integer $m$ then
$$
a_{2m}(r)=(-1)^m\prod_{j=1}^m\frac{p_2(2j+r-2)}{p_0(2j+r)}.
$$

\begin{solution}
    Similar to the proof of Exercise~\ref{exer:7.5.55} (a).
\end{solution}

\item % (c)
Conclude that if $p_0(r)=\alpha_0(r-r_1)(r-r_2)$ where $r_1-r_2$
is not an even integer, then
$$
y_1=x^{r_1}\sum_{m=0}^\infty a_{2m}(r_1)x^{2m}\quad\mbox{ and
}\quad y_2=x^{r_2}\sum_{m=0}^\infty a_{2m}(r_2)x^{2m}
$$
form a fundamental set of  Frobenius solutions of $Ly=0$.

\begin{solution}
    $p_0(2m+r_1)=2m\alpha_0(2m+r_1-r_2)$,
which is nonzero if $m>0$, since $r_1-r_2\geq 0$. Therefore, the
assumptions of Theorem~\ref{thmtype:7.5.2} hold with $r=r_1$, and
$Ly_1=p_0(r_1)x^{r_1}=0$.
If $r_1-r_2$ is not an even integer, then
$p_0(2m+r_2)=2m\alpha_0(2m-r_1+r_2)\neq 0$, $m=1,2,\cdots$.
Hence, the assumptions of Theorem~\ref{thmtype:7.5.2} hold with $r=r_2$ and
 $Ly_2=p_0(r_2)x^{r_2}=0$.
From Exercise~\ref{exer:7.5.52},
$\{y_1,y_2\}$ is a fundamental set of solutions.
\end{solution}

\item % (d)
Show that if $p_0$ satisfies the hypotheses of part (c) then
$$
y_1=x^{r_1}\sum_{m=0}^\infty \frac{(-1)^m}{
2^mm!\prod_{j=1}^m(2j+r_1-r_2)}
\left(\frac{\gamma_2}{\alpha_0}\right)^mx^{2m}
$$
and
$$
y_2=x^{r_2}\sum_{m=0}^\infty \frac{(-1)^m}{
2^mm!\prod_{j=1}^m(2j+r_2-r_1)}
\left(\frac{\gamma_2}{\alpha_0}\right)^mx^{2m}
$$
form a fundamental set of  Frobenius solutions of
$$
\alpha_0x^2y''+\beta_0xy'+(\gamma_0+\gamma_2x^2)y=0.
$$

\begin{solution}
     Similar to the proof of Exercise~\ref{exer:7.5.55} (d).
\end{solution}
\end{enumerate}
\end{problem}

\begin{problem}\label{exer:7.5.57}
 Let
$$
Ly=x^2q_0(x)y''+xq_1(x)y'+q_2(x)y,
$$
where
$$
q_0(x)=\sum_{j=0}^\infty \alpha_jx^j,\quad
q_1(x)=\sum_{j=0}^\infty \beta_jx^j,\quad
q_2(x)=\sum_{j=0}^\infty \gamma_jx^j,
$$
and define
$$
p_j(r)=\alpha_jr(r-1)+\beta_jr+\gamma_j,\quad j=0,1,\dots.
$$
 Let $y=x^r\sum_{n=0}^\infty a_nx^n$. Show that
$$
Ly=x^r\sum_{n=0}^\infty  b_nx^n,
$$
where
$$
b_n=\sum_{j=0}^np_j(n+r-j)a_{n-j}.
$$
\end{problem}

\begin{problem}\label{exer:7.5.58}
\begin{enumerate}
\item % (a)
Let  $L$ be as in Exercise~\ref{exer:7.5.57}. Show that if
$$
y(x,r)=x^r\sum_{n=0}^\infty a_n(r)x^n
$$
where
\begin{eqnarray*}
a_0(r)&=&1,\\
a_n(r)&=&-\frac{1}{p_0(n+r)}\sum_{j=1}^n p_j(n+r-j)a_{n-j}(r),\quad
n\geq 1,
\end{eqnarray*}
then
$$
Ly(x,r)=p_0(r)x^r.
$$

\begin{solution}
    From Exercise~\ref{exer:7.5.57}, $b_n=0$ for $n\geq 1$.
\end{solution}

\item % (b)
Conclude that if
$$
p_0(r)=\alpha_0(r-r_1)(r-r_2)
$$
where $r_1-r_2$ isn't  an integer then $y_1=y(x,r_1)$ and $y_2=y(x,r_2)$
are solutions of $Ly=0$.
\end{enumerate}
\end{problem}

\begin{problem}\label{exer:7.5.59}
Let
$$
Ly=x^2(\alpha_0+\alpha_qx^q)y''+x(\beta_0+\beta_qx^q)y'+
(\gamma_0+\gamma_qx^q)y
$$
where $q$ is a positive integer, and  define
$$
p_0(r)=\alpha_0r(r-1)+\beta_0r+\gamma_0\quad\mbox{ and }\quad
p_q(r)=\alpha_qr(r-1)+\beta_qr+\gamma_q.
$$
\begin{enumerate}
\item % (a)
Show that if
$$
y(x,r)=x^{r}\sum_{m=0}^\infty a_{qm}(r)x^{qm}
$$
where

\begin{equation} \label{eq:7.5.exer:7.5.59A}
\begin{array}{rcl}
a_0(r)&=&1,\\
a_{qm}(r)&=&-\frac{p_q\left(q(m-1)+r\right)}{p_0(qm+r)}a_{q(m-1)}(r),\quad m\geq 1,
\end{array}
\end{equation}
then
$$
Ly(x,r)=p_0(r)x^r
$$
\item % (b)
Deduce from (\ref{eq:7.5.exer:7.5.59A}) that
$$
a_{qm}(r)=(-1)^m\prod_{j=1}^m\frac{p_q\left(q(j-1)+r\right)}{ p_0(qj+r)}.
$$
\item % (c)
Conclude that if $p_0(r)=\alpha_0(r-r_1)(r-r_2)$ where $r_1-r_2$
is not an integer multiple of $q$, then
$$
y_1=x^{r_1}\sum_{m=0}^\infty a_{qm}(r_1)x^{qm}\quad\mbox{ and
}\quad y_2=x^{r_2}\sum_{m=0}^\infty a_{qm}(r_2)x^{qm}
$$
form a fundamental set of Frobenius solutions of $Ly=0$.
\item % (d)
Show that if $p_0$ satisfies the hypotheses of part (c) then
$$
y_1=x^{r_1}\sum_{m=0}^\infty \frac{(-1)^m}{
q^mm!\prod_{j=1}^m(qj+r_1-r_2)}
\left(\frac{\gamma_q}{\alpha_0}\right)^mx^{qm}
$$
and
$$
y_2=x^{r_2}\sum_{m=0}^\infty \frac{(-1)^m}{
q^mm!\prod_{j=1}^m(qj+r_2-r_1)}
\left(\frac{\gamma_q}{\alpha_0}\right)^mx^{qm}
$$
form a fundamental set of Frobenius solutions of
$$
\alpha_0x^2y''+\beta_0xy'+(\gamma_0+\gamma_qx^q)y=0.
$$
\end{enumerate}
\end{problem}

\begin{problem}\label{exer:7.5.60}
\begin{enumerate}
\item % (a)
Suppose $\alpha_0,\alpha_1$, and $\alpha_2$ are real numbers with
$\alpha_0\ne0$, and  $\{a_n\}_{n=0}^\infty$ is
defined by
$$
\alpha_0a_1+\alpha_1a_0=0
$$
and
$$
\alpha_0a_n+\alpha_1a_{n-1}+\alpha_2a_{n-2}=0,\quad n\geq 2.
$$
Show that
$$
(\alpha_0+\alpha_1x+\alpha_2x^2)\sum_{n=0}^\infty a_nx^n=\alpha_0a_0,
$$
and infer that
$$
\sum_{n=0}^\infty
a_nx^n=\frac{\alpha_0a_0}{\alpha_0+\alpha_1x+\alpha_2x^2}.
$$

\begin{solution}
    $(\alpha_0+\alpha_1x+\alpha_2x^2)\sum_{n=0}^\infty a_nx^n=
\alpha_0a_0+ (\alpha_0a_1+\alpha_1a_0)x+
\sum_{n=2}^\infty(\alpha_0a_n+\alpha_1a_{n-1}+\alpha_2a_{n-2})x^n=1$,
so $\sum_{n=0}^\infty
a_nx^n=\frac{\alpha_0a_0}{\alpha_0+\alpha_1x+\alpha_2x^2}$.
\end{solution}

\item % (b)
With $\alpha_0,\alpha_1$, and $\alpha_2$ as in part (a), consider the
equation

\begin{equation} \label{eq:7.5.exer:7.5.60A}
x^2(\alpha_0+\alpha_1x+\alpha_2 x^2)y''+x(\beta_0+\beta_1x+\beta_2x^2)y'+
(\gamma_0+\gamma_1x+\gamma_2x^2)y=0,
\end{equation}
and define
$$
p_j(r)=\alpha_jr(r-1)+\beta_jr+\gamma_j,\quad j=0,1,2.
$$
Suppose
$$
\frac{p_1(r-1)}{p_0(r)}=
\frac{\alpha_1}{\alpha_0},\qquad
\frac{p_2(r-2)}{p_0(r)}=
\frac{\alpha_2}{\alpha_0},
$$
and
$$
p_0(r)=\alpha_0(r-r_1)(r-r_2),
$$
where $r_1>r_2$.  Show that
$$
y_1=\frac{x^{r_1}}{\alpha_0+\alpha_1x+\alpha_2x^2}\quad\mbox{ and }\quad
y_2=\frac{x^{r_2}}{\alpha_0+\alpha_1x+\alpha_2x^2}
$$
form a fundamental set of  Frobenius  solutions of Equation 
(\ref{eq:7.5.exer:7.5.60A}) on any interval $(0,\rho)$ on which
$\alpha_0+\alpha_1x+\alpha_2x^2$ has no zeros.

\begin{solution}
    If
$\frac{p_1(r-1)}{ p_0(r)}=
\frac{\alpha_1}{\alpha_0}$ and
$\frac{p_2(r-2)}{ p_0(r)}=
\frac{\alpha_2}{\alpha_0}$, then Eqn.~(\ref{eq:7.5.12}) is equivalent to
$a_0(r)=1$, $\alpha_0a_1(r)+\alpha_1a_0(r)=0$, $\alpha_0a_n(r)
+\alpha_1a_{n-1}(r)+\alpha_2a_{n-2}(r)=0$,\quad  $n\geq 2$.
Therefore,Theorem~\ref{thmtype:7.5.2} implies the conclusion.
\end{solution}
\end{enumerate}
\end{problem}


\begin{problem}\label{exer:7.5.61}
Use the method
suggested by Exercise~\ref{exer:7.5.60} to find the general solution on
some interval $(0,\rho)$.

$2x^2(1+x)y''-x(1-3x)y'+y=0$
\end{problem}

\begin{problem}\label{exer:7.5.62}
Use the method
suggested by Exercise~\ref{exer:7.5.60} to find the general solution on
some interval $(0,\rho)$.

$6x^2(1+2x^2)y''+x(1+50x^2)y'+(1+30x^2)y=0$

\begin{solution}
    $p_0(r)=(2r-1)(3r-1)$;
$p_1(r)=0$;
$p_2(r)=2(2r+3)(3r+5)$;
$\frac{p_1(r-1)}{ p_0(r)}=0=\frac{\alpha_1}{\alpha_0}$;
$\frac{p_2(r-2)}{ p_0(r)}=2=\frac{\alpha_2}{\alpha_0}$;
$y_1=\frac{x^{1/3}}{1+2x^2}$; $y_2=\frac{x^{1/2}}{1+2x^2}$.
\end{solution}
\end{problem}

\begin{problem}\label{exer:7.5.63}
Use the method
suggested by Exercise~\ref{exer:7.5.60} to find the general solution on
some interval $(0,\rho)$.

$28x^2(1-3x)y''-7x(5+9x)y'+7(2+9x)y=0$
\end{problem}

\begin{problem}\label{exer:7.5.64}
Use the method
suggested by Exercise~\ref{exer:7.5.60} to find the general solution on
some interval $(0,\rho)$.

$9x^2(5+x)y''+9x(5+3x)y'-(5-8x)y=0$

\begin{solution}
    $p_0(r)=5(3r-1)(3r+1)$;
$p_1(r)=(3r+2)(3r+4)$;
$p_2(r)=0$;
$\frac{p_1(r-1)}{ p_0(r)}=\frac{1}{5}=\frac{\alpha_1}{\alpha_0}$;
$\frac{p_2(r-2)}{ p_0(r)}=0=\frac{\alpha_2}{\alpha_0}$;
$y_1=\frac{x^{1/3}}{5+x}$; $y_2=\frac{x^{-1/3}}{5+x}$.

\end{solution}
\end{problem}

\begin{problem}\label{exer:7.5.65}
Use the method
suggested by Exercise~\ref{exer:7.5.60} to find the general solution on
some interval $(0,\rho)$.

$8x^2(2-x^2)y''+2x(10-21x^2)y'-(2+35x^2)y=0$
\end{problem}

\begin{problem}\label{exer:7.5.66}
Use the method
suggested by Exercise~\ref{exer:7.5.60} to find the general solution on
some interval $(0,\rho)$. 

$4x^2(1+3x+x^2)y''-4x(1-3x-3x^2)y'+3(1-x+x^2)y=0$

\begin{solution}
$p_0(r)=(2r-3)(2r-1)$;
$p_1(r)=3(2r-1)(2r+1)$;
$p_2(r)=(2r+1)(2r+3)$;
$\frac{p_1(r-1)}{ p_0(r)}=3=\frac{\alpha_1}{\alpha_0}$;
$\frac{p_2(r-2)}{ p_0(r)}=1=\frac{\alpha_2}{\alpha_0}$;
$y_1=\frac{x^{1/2}}{1+3x+x^2}$;  $y_2=\frac{x^{3/2}}{1+3x+x^2}$.
\end{solution}
\end{problem}

\begin{problem}\label{exer:7.5.67}
Use the method
suggested by Exercise~\ref{exer:7.5.60} to find the general solution on
some interval $(0,\rho)$.

$3x^2(1+x)^2y''-x(1-10x-11x^2)y'+(1+5x^2)y=0$
\end{problem}

\begin{problem}\label{exer:7.5.68}
Use the method
suggested by Exercise~\ref{exer:7.5.60} to find the general solution on
some interval $(0,\rho)$.

$4x^2(3+2x+x^2)y''-x(3-14x-15x^2)y'+(3+7x^2)y=0$

\begin{solution}
    $p_0(r)=3(r-1)(4r-1)$;
$p_1(r)=2r(4r+3)$;
$p_2(r)=(r+1)(4r+7)$;
$\frac{p_1(r-1)}{ p_0(r)}=\frac{2}{3}=\frac{\alpha_1}{\alpha_0}$;
$\frac{p_2(r-2)}{ p_0(r)}=\frac{1}{3}=\frac{\alpha_2}{\alpha_0}$;
$y_1=\frac{x}{3+2x+x^2}$; $y_2=\frac{x^{1/4}}{3+2x+x^2}$.
\end{solution}
\end{problem}

\end{document}