\documentclass{ximera}
 
%% You can put user macros here
%% However, you cannot make new environments

%\listfiles

% Get the 'old' hints/expandables, for use on ximera.osu.edu
%\def\xmNotHintAsExpandable{true}
%\def\xmNotExpandableAsAccordion{true}



%\graphicspath{{./}{firstExample/}{secondExample/}}
\graphicspath{{./}
{aboutDiffEq/}
{applicationsLeadingToDiffEq/}
{applicationsToCurves/}
{autonomousSecondOrder/}
{basicConcepts/}
{bernoulli/}
{constCoeffHomSysI/}
{constCoeffHomSysII/}
{constCoeffHomSysIII/}
{constantCoeffWithImpulses/}
{constantCoefficientHomogeneousEquations/}
{convolution/}
{coolingActivity/}
{directionFields/}
{drainingTank/}
{epidemicActivity/}
{eulersMethod/}
{exactEquations/}
{existUniqueNonlinear/}
{frobeniusI/}
{frobeniusII/}
{frobeniusIII/}
{global.css/}
{growthDecay/}
{heatingCoolingActivity/}
{higherOrderConstCoeff/}
{homogeneousLinearEquations/}
{homogeneousLinearSys/}
{improvedEuler/}
{integratingFactors/}
{interactExperiment/}
{introToLaplace/}
{introToSystems/}
{inverseLaplace/}
{ivpLaplace/}
{laplaceTable/}
{lawOfCooling/}
{linSysOfDiffEqs/}
{linearFirstOrderDiffEq/}
{linearHigherOrder/}
{mixingProblems/}
{motionUnderCentralForce/}
{nonHomogeneousLinear/}
{nonlinearToSeparable/}
{odesInSage/}
{piecewiseContForcingFn/}
{population/}
{reductionOfOrder/}
{regularSingularPts/}
{reviewOfPowerSeries/}
{rlcCircuit/}
{rungeKutta/}
{secondLawOfMotion/}
{separableEquations/}
{seriesSolNearOrdinaryPtI/}
{seriesSolNearOrdinaryPtII/}
{simplePendulum/}
{springActivity/}
{springProblemsI/}
{springProblemsII/}
{undCoeffHigherOrderEqs/}
{undeterminedCoeff/}
{undeterminedCoeff2/}
{unitStepFunction/}
{varParHigherOrder/}
{varParamNonHomLinSys/}
{variationOfParameters/}
}


\usepackage{tikz}
%\usepackage{tkz-euclide}
\usepackage{tikz-3dplot}
\usepackage{tikz-cd}
\usetikzlibrary{shapes.geometric}
\usetikzlibrary{arrows}
\usetikzlibrary{decorations.pathmorphing,patterns}
\usetikzlibrary{backgrounds} % added by Felipe
% \usetkzobj{all}   % NOT ALLOWED IN RECENT TeX's ...
\pgfplotsset{compat=1.13} % prevents compile error.

\pdfOnly{\renewcommand{\theHsection}{\thepart.section.\thesection}}  %% MAKES LINKS WORK should be added to CLS
\pdfOnly{\renewcommand{\part}[1]{\chapterstyle\title{#1}\begin{abstract}\end{abstract}\maketitle\def\thechaptertitle{#1}}}


\renewcommand{\vec}[1]{\mathbf{#1}}
\newcommand{\RR}{\mathbb{R}}
\providecommand{\dfn}{\textit}
\renewcommand{\dfn}{\textit}
\newcommand{\dotp}{\cdot}
\newcommand{\id}{\text{id}}
\newcommand\norm[1]{\left\lVert#1\right\rVert}
\newcommand{\dst}{\displaystyle}
 
\newtheorem{general}{Generalization}
\newtheorem{initprob}{Exploration Problem}

\tikzstyle geometryDiagrams=[ultra thick,color=blue!50!black]

\usepackage{mathtools}
 
 
 
 
\title{Spring Problems I}
 
 
\begin{document}
 
\begin{abstract}
 We study undamped harmonic motion as an application of second order linear differential equations.
\end{abstract}
 
\maketitle
 
\section*{Spring Problems I}
 
 
We consider the motion of an object of mass $m$, suspended from a
spring of negligible mass. We say that the spring--mass system is
in \textit{equilibrium} when the object is at rest and the forces
acting on it sum to zero. The position of the object in this case is
the \textit{equilibrium position}. We define $y$ to be the displacement
of the object from its equilibrium position %(Figure~\ref{figure:6.1.1}),
measured positive upward.

\begin{center}
\begin{tikzpicture}[scale=1]
   % y-axis
  \draw[-stealth, thick] (0,-3) -- (0,4) node[above] {$y$};

\draw[line width=1pt,gray, dashed](2,0.3)--(0,0.3);  
\draw[line width=1pt,gray, dashed](4,-0.7)--(0,-0.7) node[left]{0};
\draw[line width=1pt,gray, dashed](6,-1.7)--(0,-1.7);

% Springs
\draw[decoration={aspect=0.3, segment length=1.5mm, amplitude=3mm,coil},decorate] (2,3)node[above]{(a)} -- (2,0.3);

\draw[decoration={aspect=0.3, segment length=2mm, amplitude=3mm,coil},decorate] (4,3) node[above]{(b)} -- (4,-0.7);
\draw[decoration={aspect=0.3, segment length=2.5mm, amplitude=3mm,coil},decorate] (6,3) node[above]{(c)} -- (6,-1.8);

% Circles
  \draw[black, fill=gray] (2,0) circle(0.3);
  \draw[black, fill=gray] (4,-1) circle(0.3);
  \draw[black, fill=gray] (6,-2) circle(0.3);  
\end{tikzpicture}
\end{center}

%\begin{image}
 % \includegraphics[height=1.5in]{fig060101.jpg}
%\end{image}
 
Our model accounts for the following kinds of forces acting on the
object:
 
\begin{itemize}
\item The force $-mg$, due to gravity.
 
\item A force $F_s$ exerted by the spring resisting change in its
length. The \textit{natural length} of the spring is its length with
no mass attached. We assume that the spring obeys
\href{http://www-history.mcs.st-and.ac.uk/Mathematicians/Hooke.html}
{Hooke's law}:
If the length of the spring is changed by an amount $\Delta L$
from its natural length,  then the spring exerts a force $F_s=k\Delta
L$, where $k$ is a positive number called the \textit{spring constant}.
If the spring is stretched then  $\Delta L>0$ and $F_s>0$, so the
spring force is upward, while if the spring is compressed then $\Delta
L<0$ and $F_s<0$, so the spring force is downward.
 
\item A \textit{damping force} $F_d=-cy'$ that resists the motion with
a force proportional to the velocity of the object. It may be due to
air resistance or friction in the spring. However, a convenient way to
visualize a damping force is to assume that the object is rigidly
attached to a piston with negligible mass immersed in a cylinder
(called a \textit{dashpot}) filled with a viscous liquid
%(Figure~\ref{figure:6.1.2}).

\begin{center}
\begin{tikzpicture}[scale=1]
  
% Spring

\draw[decoration={aspect=0.3, segment length=2.5mm, amplitude=3mm,coil},decorate] (0,3) -- (0,-1.8);

\draw[-, black] (0,-2.3) -- (0,-4);
\fill[gray] (-0.8,-4) rectangle (0.8,-4.2);
% Circles
   \draw[black, fill=gray] (0,-2) circle(0.3);  

   % Tank
  \draw[black] (-1,-2) -- (-1,-6) -- (1,-6) -- (1,-2);
  \draw[gray, dashed] (-1,-2.8) -- (1,-2.8); % Fill line

     \foreach \x in {-1, -0.9,...,1} {
    \foreach \y in {-6,-5.9,...,-2.8} {
      \fill[gray] (\x,\y) circle (0.4pt);
    }
  }
\end{tikzpicture}
\end{center}

%\begin{image}
 % \includegraphics[height=1.5in]{fig060102.jpg}
%\end{image}
 
As the piston moves, the liquid exerts a
damping force. We say that the motion is \textit{undamped} if $c=0$, or
\textit{damped} if $c>0$.
 
 
\item An external force $F$, other than the force due to gravity, that
may vary with $t$, but is independent of displacement and velocity. We
say that the motion is \textit{free} if $F\equiv0$, or \textit{forced}
if $F\not\equiv0$.
 
\end{itemize}
 
 
 
From Newton's second law of motion,
\begin{equation}\label{eq:6.1.1}
my''=-mg+F_d+F_s+F=-mg-cy'+F_s+F.
\end{equation}
We must now relate $F_s$ to $y$. In the absence of external forces the
object stretches the spring by an amount $\Delta l$ to assume its
equilibrium position, as shown in the figure below.
%(Figure~\ref{figure:6.1.3}).

\begin{center}
\begin{tikzpicture}

\node[] (b) at (2,2) {};
\draw[line width=1pt,gray, dashed](-2.1,0)node[left]{0}--(3.5,0) node[right]{Equilibrium Position};
\draw[line width=1pt,gray, dashed](2,2)--(3.5,2) node[right]{Unstretched Position};
\draw[line width=1pt,gray, -stealth](2.5,1)node[right]{$\Delta L$}--(2.5,0);
\draw[line width=1pt,gray, -stealth](2.7,3.5)--(2.7,2);
\draw[line width=1pt,gray, -stealth](2.7,3.5)node[right]{$L$}--(2.7,5);
\draw[line width=1pt,gray, -stealth](2.5,1)--(2.5,2);

\draw[decoration={aspect=0.3, segment length=3mm, amplitude=3mm,coil},decorate] (0,5) -- (0,0);

\draw[decoration={aspect=0.3, segment length=1.5mm, amplitude=3mm,coil},decorate] (2,5) -- (2,2);

\draw[black, fill=gray] (0,-0.3) circle(0.3); 

 \draw[-stealth, thick] (-2,-2) -- (-2,5.5) node[above] {$y$};
\end{tikzpicture}
\end{center}

%\begin{image}
 % \includegraphics[height=1.5in]{fig060103.jpg}
%\end{image}
 
Since the sum of the forces acting on the object
is then zero, Hooke's Law implies that $mg=k\Delta l$. If the object
is displaced $y$ units from its equilibrium position, the total
change in the length of the spring is $\Delta L=\Delta l-y$,
 so Hooke's law implies that
 $$
F_s=k\Delta L=k\Delta l-ky.
$$
Substituting this into \eqref{eq:6.1.1} yields
$$
my''=-mg-cy'+k\Delta L-ky+F.
$$
Since $mg=k\Delta l$ this can be written as
\begin{equation}\label{eq:6.1.2}
my''+cy'+ky=F.
\end{equation}
We call this \textit{the equation of motion}.
 
 
\subsection*{Simple Harmonic Motion}
 
Throughout the rest of this section we'll consider spring--mass
systems without damping; that is, $c=0$. We'll consider systems with
damping in the next section.
 
We first consider the case where the motion is also free; that is,
$F$=0. We begin with an example.
 
\begin{example}\label{example:6.1.1}
An object stretches a spring 6 inches in equilibrium.
\begin{enumerate}
\item \label{item:6.1.1a} % (a)
Set up the equation of motion and find its general solution.
 
\item \label{item:6.1.1b} % (b)
Find the displacement of the object for $t>0$ if it's initially
displaced 18 inches above equilibrium and given a downward velocity of
3 ft/s.
\end{enumerate}
 
 
\begin{explanation}
\ref{item:6.1.1a}
Setting $c=0$ and $F=0$ in \eqref{eq:6.1.2} yields the equation of motion
$$
my''+ky=0,
$$
which we rewrite as
\begin{equation}\label{eq:6.1.3}
y''+\frac{k}{m}y=0.
\end{equation}
Although we would need the weight of the object to obtain $k$ from the
equation $mg=k\Delta l$ we can obtain $k/m$ from $\Delta l$ alone;
thus, $k/m=g/\Delta l$. Consistent with the units used in the problem
statement, we take $g=32$ ft/s$^2$. Although $\Delta l$ is stated in
inches, we must convert it to feet to be consistent with this choice
of $g$; that is, $\Delta l =1/2$ ft. Therefore
$$
\frac{k}{m}=\frac{32}{1/2}=64
$$
and \eqref{eq:6.1.3} becomes
\begin{equation}\label{eq:6.1.4}
y''+64y=0.
\end{equation}
 The characteristic equation of \eqref{eq:6.1.4} is
$$
r^2+64=0,
$$
which has the zeros $r=\pm 8i$. Therefore the general solution of
\eqref{eq:6.1.4} is
\begin{equation}\label{eq:6.1.5}
y=c_1\cos8t+c_2\sin8t.
\end{equation}
 
\ref{item:6.1.1b}
The initial upward displacement of 18 inches is positive and must be
expressed in feet. The initial downward velocity is negative;   thus,
$$
y(0)=\frac{3}{2}\quad\mbox{and}\quad y'(0)=-3.
$$
 Differentiating \eqref{eq:6.1.5} yields
\begin{equation}\label{eq:6.1.6}
y'=-8c_1\sin8t+8c_2\cos8t.
\end{equation}
Setting $t=0$ in \eqref{eq:6.1.5} and \eqref{eq:6.1.6} and imposing the
initial conditions shows that $c_1=3/2$ and $c_2=-3/8$. Therefore
$$
y=\frac{3}{2}\cos8t-\frac{3}{8}\sin8t,
$$
where $y$ is in feet.

\begin{center}
\begin{tikzpicture}[scale=1]
        \begin{axis}[
          domain=0:3.3,
          xmin=-0.4, xmax=3.5,
          ymin=-2, ymax=2,
          width=4in,
          height=3in,
          xtick={0.5,1,1.5,2,2.5,3},
            xticklabels={$0.5$,$1.0$,$1.5$,$2.0$, $2.5$,$3.0$},
        ytick={-2,-1.5,...,1.5},
            axis x line=center,
            axis y line=center,
            xlabel={$t$},
            ylabel={$y$},
            every axis x label/.style={at=(current axis.right of origin),anchor=west}
          ]
\addplot[smooth, thick, blue] {1.5*cos(8*x*180/pi)-(3/8)*sin(8*x*180/pi)};
      \end{axis}
        \end{tikzpicture}
\end{center}

%\begin{image}
 % \includegraphics[height=1.5in]{fig060104.jpg}
%\end{image}
\end{explanation}
\end{example}
 
We'll now consider the equation
$$
my''+ky=0
$$
where $m$ and $k$ are arbitrary positive numbers. Dividing through by
$m$ and defining $\omega_0=\sqrt{k/m}$ yields
$$
y''+\omega_0^2y=0.
$$
The general solution of this equation is
\begin{equation}\label{eq:6.1.7}
y=c_1\cos\omega_0t+c_2\sin\omega_0t.
\end{equation}
We can rewrite this in a more useful form by defining
\begin{equation}\label{eq:6.1.8}
R=\sqrt{c_1^2+c_2^2},
\end{equation}
and
\begin{equation}\label{eq:6.1.9}
c_1=R\cos\phi\quad\mbox{and}\quad c_2=R\sin\phi.
\end{equation}
Substituting from \eqref{eq:6.1.9} into \eqref{eq:6.1.7} and applying the
identity
$$
\cos\omega_0t\cos\phi+\sin\omega_0t\sin\phi=\cos(\omega_0t-\phi)
$$
yields
\begin{equation}\label{eq:6.1.10}
y=R\cos(\omega_0t-\phi).
\end{equation}
 
From \eqref{eq:6.1.8} and \eqref{eq:6.1.9} we see that the $R$ and $\phi$
can be interpreted as polar coordinates of the point with rectangular
coordinates $(c_1,c_2)$.

\begin{center}
\begin{tikzpicture}[scale=2]
 \draw[-stealth, thick] (0,-0.5) -- (0,1.5) node[above] {$y$};
 \draw[-stealth, thick] (-1,0) -- (2,0) node[right] {$x$};
\draw[-,black, dashed](1.5,0)node[below]{$c_1$}--(1.5,1);
\draw[-,black, dashed](0,1)node[left]{$c_2$}--(1.5,1);
\draw[-,blue](0,0)--(1.5,1);
\node at (0.7, 0.6) {$R$};
\node at (0.3, 0.1) {$\theta$};
\draw[dashed, blue] (0.5,0) arc[start angle=0,end angle=35,radius=0.5];

\end{tikzpicture}
\end{center}

%\begin{image}
 % \includegraphics[height=1.5in]{fig060105.jpg}
%\end{image}
 
Given $c_1$ and
$c_2$, we can compute $R$ from \eqref{eq:6.1.8}. From \eqref{eq:6.1.8} and
\eqref{eq:6.1.9}, we see that $\phi$ is related to $c_1$ and $c_2$ by
$$
\cos\phi=\frac{c_1}{\sqrt{c_1^2+c_2^2}}\quad\mbox{and}\quad\sin\phi=
\frac{c_2}{\sqrt{c_1^2+c_2^2}}.
$$
There are infinitely many angles $\phi$, differing
 by integer multiples of $2\pi$, that satisfy these equations. We
will always choose $\phi$ so that $-\pi\leq\phi<\pi$.
 
 
The motion described by \eqref{eq:6.1.7} or \eqref{eq:6.1.10}
is  \textit{simple harmonic motion}. We see from either of these
equations that the motion is periodic, with period
$$
T=2\pi/\omega_0.
$$
This is the time required for the object to complete one full cycle of
oscillation (for example, to move from its highest position to its
lowest position and back to its highest position). Since the highest
and lowest positions of the object are $y=R$ and $y=-R$, we say that
$R$ is the \textit{amplitude} of the oscillation. The angle $\phi$ in
\eqref{eq:6.1.10} is the \textit{phase angle}. It's measured
in
radians. Equation~\eqref{eq:6.1.10} is the \textit{amplitude--phase
form} of the displacement. If $t$ is in seconds then $\omega_0$ is
in radians per second (rad/s);   it's  the \textit{frequency} of
the motion. It is also called the \textit{natural frequency} of
the spring--mass system without damping.
 
\begin{example}\label{example:6.1.2}
We found the displacement of the object
 in Example~\ref{example:6.1.1} to be
$$
y=\frac{3}{2}\cos8t-\frac{3}{8}\sin8t.
$$
Find the frequency, period, amplitude, and phase angle of the motion.
 
\begin{explanation}
The frequency is $\omega_0=8$ rad/s, and the period is
$T=2\pi/\omega_0=\pi/4$ s. Since $c_1=3/2$ and $c_2=-3/8$,
the amplitude is
$$
R=\sqrt{c^2_1+c^2_2}=\sqrt{\left(\frac{3}{2}\right)^2+\left(\frac{3}{8}\right)^2}
=\frac{3}{8}\sqrt{17}.
$$
The phase angle is determined by
\begin{equation}\label{eq:6.1.11}
\cos\phi=\frac{3/2}{(3/8)\sqrt{17}}=\frac{4}{\sqrt{17}}
\end{equation}
and
\begin{equation}\label{eq:6.1.12}
\sin\phi=\frac{-3/8}{(3/8)\sqrt{17}}=-\frac{1}{\sqrt{17}}.
\end{equation}
 Using a calculator, we see from \eqref{eq:6.1.11} that
$$
\phi\approx\pm.245\mbox{ rad}.
$$
Since $\sin\phi<0$ (see \eqref{eq:6.1.12}), the minus sign applies here;
that is,
$$
\phi\approx-.245\mbox{ rad}.
$$
\end{explanation}
\end{example}
 
 
\begin{example}\label{example:6.1.3}
The natural length of a spring is 1 m. An object is attached to it and
the length of the spring increases to 102 cm when the object is in
equilibrium. Then the object is initially displaced downward 1 cm and
given an upward velocity of 14 cm/s. Find the displacement for
$t>0$. Also, find the natural frequency, period, amplitude, and phase
angle of the resulting motion. Express the answers in cgs units.
 
\begin{explanation}
In cgs units $g=980$ cm/s$^2$. Since $\Delta l=2$ cm,
$\omega_0^2=g/\Delta l=490$.
Therefore
$$
y''+490y=0, \quad  y(0)=-1,\quad y'(0)=14.
$$
 The general solution of the differential equation is
$$
y=c_1\cos7\sqrt{10}t+c_2\sin7\sqrt{10}t,
$$
so
$$
y'=7\sqrt{10}\left(-c_1\sin7\sqrt{10}t+c_2\cos7\sqrt{10}t\right).
$$
Substituting the initial conditions into the last two equations yields
$c_1=-1$ and $c_2=2/\sqrt{10}$. Hence,
$$
y=-\cos7\sqrt{10}t+\frac{2}{\sqrt{10}}\sin7\sqrt{10}t.
$$
The frequency is $7\sqrt{10}$ rad/s, and  the period
is $T=2\pi/(7\sqrt{10})$  s.
The amplitude  is
$$
R=\sqrt{c_1^2+c_2^2}=\sqrt{(-1)^2+\left(\frac{2}{\sqrt{10}}\right)^2}=
\sqrt{\frac{7}{5}}\mbox{ cm}.
$$
The  phase angle  is determined by
$$
\cos\phi=\frac{c_1}{R}=-\sqrt{\frac{5}{7}}\quad\mbox{and}\quad\sin\phi=
\frac{c_2}{R}=\sqrt{\frac{2}{7}}.
$$
Therefore $\phi$ is in the second quadrant and
$$
\phi=\cos^{-1}\left(-\sqrt{\frac{5}{7}}\right)\approx2.58\mbox{ rad}.
$$
\end{explanation}
\end{example}
 
 
\subsection*{Undamped Forced Oscillation}
 
In many mechanical problems a device is subjected to periodic external
forces. For example, soldiers marching in cadence on a bridge cause
periodic disturbances in the bridge, and the engines of a propeller
driven aircraft cause periodic disturbances in its wings. In the
absence of sufficient damping forces, such disturbances -- even if
small in magnitude -- can cause structural breakdown if they are at
certain critical frequencies. To illustrate, this we'll consider the
motion of an object in a spring--mass system without damping, subject
to an external force
$$
F(t)=F_0\cos\omega t
$$
where $F_0$ is a constant. In this case the equation of motion
\eqref{eq:6.1.2} is
$$
my''+ky=F_0\cos\omega t,
$$
which we rewrite  as
\begin{equation}\label{eq:6.1.13}
y''+\omega_0^2y=\frac{F_0}{m}\cos\omega t
\end{equation}
with $\omega_0=\sqrt{k/m}$.
We'll see from the next two examples that the solutions of
\eqref{eq:6.1.13}
with $\omega\ne\omega_0$ behave very differently from the solutions with
$\omega=\omega_0$.
 
 
 
 
\begin{example}\label{example:6.1.4}
 Solve the initial value problem
\begin{equation}\label{eq:6.1.14}
y''+\omega_0^2y=\frac{F_0}{m}\cos\omega t, \quad  y(0)=0,\quad y'(0)=0,
\end{equation}
 given that $\omega\neq\omega_0$.
 
 
 
\begin{explanation}
We first obtain a particular solution of \eqref{eq:6.1.13} by the method
of undetermined coefficients. Since $\omega\ne\omega _0$,
 $\cos\omega t$ isn't  a solution of the complementary equation
$$
y''+\omega_0^2y=0.
$$
Therefore \eqref{eq:6.1.13} has a particular solution of the form
$$
y_p=A\cos\omega t+B\sin\omega t.
$$
Since
$$
y''_p=-\omega^2(A\cos\omega t+B\sin\omega t),
$$
$$
y_p''+\omega_0^2y_p=\frac{F_0}{m}\cos\omega t
$$
if and only if
$$
(\omega_0^2-\omega^2)\left(A\cos\omega t+
B\sin\omega t\right)=\frac{F_0}{m}\cos\omega t.
$$
 This holds if and only if
$$
A=\frac{F_0}{m(\omega_0^2-\omega^2)}\quad\mbox{and}\quad B=0,
$$
so
$$
y_p=\frac{F_0}{m(\omega_0^2-\omega^2)}\cos\omega t.
$$
The general solution of \eqref{eq:6.1.13}  is
\begin{equation}\label{eq:6.1.15}
y=\frac{F_0}{m(\omega_0^2-\omega^2)}\cos\omega t
+c_1\cos\omega_0 t+c_2\sin\omega_0t,
\end{equation}
so
$$
y'=\frac{-\omega F_0}{m(\omega_0^2-\omega^2)}\sin\omega t
+\omega_0(-c_1\sin\omega_0 t+c_2\cos\omega_0t).
$$
The initial conditions $y(0)=0$ and $y'(0)=0$ in \eqref{eq:6.1.14} imply
that
$$
c_1=-\frac{F_0}{m(\omega_0^2-\omega^2)}\quad\mbox{and}\quad
c_2=0.
$$
Substituting these into \eqref{eq:6.1.15} yields
\begin{equation}\label{eq:6.1.16}
y=\frac{F_0}{m(\omega_0^2-\omega^2)}(\cos\omega t-\cos\omega
_0t).
\end{equation}
 
It is revealing to write this in a different form.  We start with the
trigonometric identities
\begin{eqnarray*}
\cos(\alpha-\beta)&=&\cos\alpha\cos\beta+\sin\alpha\sin\beta\\
\cos(\alpha+\beta)&=&\cos\alpha\cos\beta-\sin\alpha\sin\beta.
\end{eqnarray*}
Subtracting the second identity from the first yields
\begin{equation}\label{eq:6.1.17}
\cos(\alpha-\beta)-\cos(\alpha+\beta)=2\sin\alpha\sin\beta
\end{equation}
 Now let
\begin{equation}\label{eq:6.1.18}
\alpha-\beta=\omega t\quad\mbox{and}\quad\alpha+\beta=\omega_0t,
\end{equation}
 so that
\begin{equation}\label{eq:6.1.19}
\alpha=\frac{(\omega_0+\omega)t}{2}\quad\mbox{and}\quad\beta=\frac{(\omega_0-\omega
)t}{2}.
\end{equation}
 Substituting \eqref{eq:6.1.18} and \eqref{eq:6.1.19} into \eqref{eq:6.1.17}
yields
 $$
\cos\omega t-\cos\omega_0t=
2\sin\frac{(\omega_0-\omega)t}{2}\sin\frac{(\omega_0+\omega)t}{2},
$$
and substituting this into \eqref{eq:6.1.16} yields
\begin{equation}\label{eq:6.1.20}
y=R(t)\sin\frac{(\omega_0+\omega)t}{2},
\end{equation}
 where
\begin{equation}\label{eq:6.1.21}
R(t)=\frac{2F_0}{m(\omega_0^2-\omega^2)}
\sin\frac{(\omega_0-\omega)t}{2}.
\end{equation}
 
From \eqref{eq:6.1.20} we can regard $y$ as a sinusoidal variation with
frequency $(\omega_0+\omega)/2$ and variable amplitude
$|R(t)|$. In the figure below, the dashed curve above the $t$ axis
is $y=|R(t)|$, the dashed curve below the $t$ axis is $y=-|R(t)|$, and
the displacement $y$ appears as an oscillation bounded by them. The
oscillation of $y$ for $t$ on an interval between successive zeros of
$R(t)$ is called a \textit{beat}.
 
\begin{center}
\geogebra{ybhuwjqu}{800}{600}
\end{center}
 
%\begin{image}
 % \includegraphics[height=1.5in]{fig060106.jpg}
%\end{image}
 
 
 
 
You can see from
\eqref{eq:6.1.20} and \eqref{eq:6.1.21} that
$$
|y(t)|\leq\frac{2|F_0|}{m|\omega_0^2-\omega^2|};
$$
moreover, if $\omega+\omega_0$ is sufficiently large compared with $\omega
-\omega_0$, then $|y|$ assumes values close to (perhaps equal to) this
upper bound during each beat. However, the oscillation remains bounded for
all $t$. (This assumes that the spring can withstand
deflections of this size and continue to obey Hooke's law.)  The next
example shows that this isn't  so if $\omega=\omega_0$.
 
\end{explanation}
\end{example}
 
 
 
\begin{example}\label{example:6.1.5}
Find the general solution of
\begin{equation}\label{eq:6.1.22}
y''+\omega_0^2y=\frac{F_0}{m}\cos\omega_0t.
\end{equation}
 
\begin{explanation}
We first obtain a particular solution $y_p$ of \eqref{eq:6.1.22}. Since
$\cos\omega_0t$ is a solution of the complementary equation, the form
for $y_p$ is
\begin{equation}\label{eq:6.1.23}
y_p=t(A\cos\omega_0t+B\sin\omega_0t).
\end{equation}
 Then
$$
y_p'=A\cos\omega_0t+B\sin\omega_0t
+\omega_0t(-A\sin\omega_0t+B\cos\omega_0t)
$$
and
\begin{equation}\label{eq:6.1.24}
y''_p=2\omega_0(-A\sin\omega_0t
+B\cos\omega_0t)-\omega_0^2t(A\cos\omega_0t+B\sin\omega_0t).
\end{equation}
From \eqref{eq:6.1.23} and \eqref{eq:6.1.24}, we see that $y_p$ satisfies
\eqref{eq:6.1.22} if
$$
-2A\omega_0\sin\omega_0t+2B\omega_0\cos\omega_0t=\frac{F_0}{m}
\cos\omega_0t;
$$
that is, if
$$
A=0\quad\mbox{and}\quad B=\frac{F_0}{2m\omega_0}.
$$
Therefore
$$
y_p=\frac{F_0t}{2m\omega_0}\sin\omega_0t
$$
is a particular solution of \eqref{eq:6.1.22}. The general solution of
 \eqref{eq:6.1.22} is
$$
y=\frac{F_0t}{2m\omega_0}\sin\omega_0t+c_1\cos\omega_0t+c_2\sin\omega_0t.
$$
The graph of $y_p$ is shown below.  It can be
seen that $y_p$ oscillates between the dashed lines
$$
y=\frac{F_0t}{2m\omega_0}\quad\mbox{and}\quad y=-\frac{F_0t}{2m\omega_0}
$$
with increasing amplitude that approaches $\infty$ as $t\rightarrow\infty$. Of
course, this means that the spring must eventually fail to obey Hooke's law
or break.

\begin{center}
\geogebra{pxn4scrz}{800}{600}
\end{center}

%\begin{image}
 % \includegraphics[height=1.5in]{fig060107.jpg}
%\end{image}
\end{explanation}
\end{example}
 
 
 
This phenomenon of unbounded displacements of a spring--mass system in
response to a periodic forcing function at its natural
frequency is called \textit{resonance}. More complicated mechanical
structures can also exhibit resonance--like phenomena. For example,
rhythmic oscillations of a suspension bridge by wind forces or of an
airplane wing by periodic vibrations of reciprocating engines can
cause damage or even failure if the frequencies of the disturbances
are close to critical frequencies determined by the parameters of the
mechanical system in question.
 
\section*{Text Source}
Trench, William F., "Elementary Differential Equations" (2013). Faculty Authored and Edited Books \& CDs. 8. (CC-BY-NC-SA)
 
\href{https://digitalcommons.trinity.edu/mono/8/}{https://digitalcommons.trinity.edu/mono/8/}
 
\end{document}