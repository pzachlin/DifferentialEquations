\documentclass{ximera}
%% You can put user macros here
%% However, you cannot make new environments

%\listfiles

% Get the 'old' hints/expandables, for use on ximera.osu.edu
%\def\xmNotHintAsExpandable{true}
%\def\xmNotExpandableAsAccordion{true}



%\graphicspath{{./}{firstExample/}{secondExample/}}
\graphicspath{{./}
{aboutDiffEq/}
{applicationsLeadingToDiffEq/}
{applicationsToCurves/}
{autonomousSecondOrder/}
{basicConcepts/}
{bernoulli/}
{constCoeffHomSysI/}
{constCoeffHomSysII/}
{constCoeffHomSysIII/}
{constantCoeffWithImpulses/}
{constantCoefficientHomogeneousEquations/}
{convolution/}
{coolingActivity/}
{directionFields/}
{drainingTank/}
{epidemicActivity/}
{eulersMethod/}
{exactEquations/}
{existUniqueNonlinear/}
{frobeniusI/}
{frobeniusII/}
{frobeniusIII/}
{global.css/}
{growthDecay/}
{heatingCoolingActivity/}
{higherOrderConstCoeff/}
{homogeneousLinearEquations/}
{homogeneousLinearSys/}
{improvedEuler/}
{integratingFactors/}
{interactExperiment/}
{introToLaplace/}
{introToSystems/}
{inverseLaplace/}
{ivpLaplace/}
{laplaceTable/}
{lawOfCooling/}
{linSysOfDiffEqs/}
{linearFirstOrderDiffEq/}
{linearHigherOrder/}
{mixingProblems/}
{motionUnderCentralForce/}
{nonHomogeneousLinear/}
{nonlinearToSeparable/}
{odesInSage/}
{piecewiseContForcingFn/}
{population/}
{reductionOfOrder/}
{regularSingularPts/}
{reviewOfPowerSeries/}
{rlcCircuit/}
{rungeKutta/}
{secondLawOfMotion/}
{separableEquations/}
{seriesSolNearOrdinaryPtI/}
{seriesSolNearOrdinaryPtII/}
{simplePendulum/}
{springActivity/}
{springProblemsI/}
{springProblemsII/}
{undCoeffHigherOrderEqs/}
{undeterminedCoeff/}
{undeterminedCoeff2/}
{unitStepFunction/}
{varParHigherOrder/}
{varParamNonHomLinSys/}
{variationOfParameters/}
}


\usepackage{tikz}
%\usepackage{tkz-euclide}
\usepackage{tikz-3dplot}
\usepackage{tikz-cd}
\usetikzlibrary{shapes.geometric}
\usetikzlibrary{arrows}
\usetikzlibrary{decorations.pathmorphing,patterns}
\usetikzlibrary{backgrounds} % added by Felipe
% \usetkzobj{all}   % NOT ALLOWED IN RECENT TeX's ...
\pgfplotsset{compat=1.13} % prevents compile error.

\pdfOnly{\renewcommand{\theHsection}{\thepart.section.\thesection}}  %% MAKES LINKS WORK should be added to CLS
\pdfOnly{\renewcommand{\part}[1]{\chapterstyle\title{#1}\begin{abstract}\end{abstract}\maketitle\def\thechaptertitle{#1}}}


\renewcommand{\vec}[1]{\mathbf{#1}}
\newcommand{\RR}{\mathbb{R}}
\providecommand{\dfn}{\textit}
\renewcommand{\dfn}{\textit}
\newcommand{\dotp}{\cdot}
\newcommand{\id}{\text{id}}
\newcommand\norm[1]{\left\lVert#1\right\rVert}
\newcommand{\dst}{\displaystyle}
 
\newtheorem{general}{Generalization}
\newtheorem{initprob}{Exploration Problem}

\tikzstyle geometryDiagrams=[ultra thick,color=blue!50!black]

\usepackage{mathtools}

\title{Exercises} \license{CC BY-NC-SA 4.0}

\begin{document}

\begin{abstract}
\end{abstract}
\maketitle

\begin{onlineOnly}
\section*{Exercises}
\end{onlineOnly}

In the following exercises assume that there's no damping.

\begin{problem}\label{exer:6.1.1}
An object stretches a spring 4 inches in equilibrium.
Find and graph its displacement for $t>0$ if it's initially displaced
36 inches above equilibrium and given a downward velocity of 2 ft/s.
\end{problem}

\begin{problem}\label{exer:6.1.2}
An object  stretches a string 1.2 inches in equilibrium.
Find its displacement for $t>0$ if it's initially displaced 3 inches
below equilibrium and given a downward velocity of 2 ft/s.

\begin{solution}
    Since $\frac{k}{m}=\frac{g}{\Delta l}=\frac{32}{0.1}=320$ the
equation of motion is (A) $y''+320y=0$. The general solution of (A) is
$y=c_1\cos8\sqrt5t+c_2\sin8\sqrt5t$, so $y'=8\sqrt5(-c_1\sin8\sqrt5
t+c_2\cos8\sqrt5t)$. Now $y(0)=-\frac{1}{4}\Rightarrow
c_1=-\frac{1}{4}$ and $y'(0)=-2\Rightarrow
c_2=-\frac{1}{4\sqrt5}$, so $y=-\frac{1}{4}\cos8\sqrt{5}t-\frac{1}{4\sqrt{5}}\sin8\sqrt{5}t$ ft.
\end{solution}
\end{problem}

\begin{problem}\label{exer:6.1.3}
A spring with natural length .5 m has length 50.5 cm with a mass of 2
gm suspended from it. The mass is initially displaced 1.5 cm below
equilibrium and released with zero velocity. Find its displacement for
$t>0$.
\end{problem}

\begin{problem}\label{exer:6.1.4}
An object stretches a spring 6 inches in equilibrium. Find its
displacement for $t>0$ if it's initially displaced 3 inches above
equilibrium and given a downward velocity of 6 inches/s. Find the
frequency, period, amplitude and phase angle of the motion.

\begin{solution}
Since $\frac{k}{m}=\frac{g}{\Delta l}=\frac{32}{.5}=64$ the
equation of motion is (A) $y''+64y=0$. The general solution of (A) is
$y=c_1\cos8t+c_2\sin8t$, so $y'=8(-c_1\sin8t+c_2\cos8t)$. Now
$y(0)=\frac{1}{4}\Rightarrow c_1=\frac{1}{4}$ and
$y'(0)=-\frac{1}{2}\Rightarrow c_2=-\frac{1}{16}$, so
$y=\frac{1}{4}\cos8t-\frac{1}{16}\sin8t$ ft;\;
$R=\frac{\sqrt{17}}{16}$ ft;\ $\omega_0=8$ rad/s;\ $T=\pi/4$ s;
$\phi\approx-.245\text{ rad}\approx -14.04^\circ$.

\end{solution}
\end{problem}

\begin{problem}\label{exer:6.1.5}
An object stretches a spring 5 cm in equilibrium. It is initially
displaced 10 cm above equilibrium and given an upward velocity of .25
m/s. Find and graph its displacement for $t>0$. Find the frequency,
period, amplitude, and phase angle of the motion.
\end{problem}

\begin{problem}\label{exer:6.1.6}
A 10 kg mass stretches a spring 70 cm in equilibrium. Suppose a 2
kg mass is
attached to the spring, initially displaced 25 cm below equilibrium,
and given an upward velocity of 2 m/s. Find its displacement for
$t>0$. Find the frequency, period, amplitude, and phase angle of the
motion.

\begin{solution}
    Since $\frac{k}{m}=\frac{g}{\Delta l}=\frac{32}{0.5}=64$ the
equation of motion is (A) $y''+64y=0$. The general solution of (A) is
$y=c_1\cos8t+c_2\sin8t$, so $y'=8(-c_1\sin8t+c_2\cos8t)$. Now
$y(0)=\frac{1}{4}\Rightarrow c_1=\frac{1}{4}$ and
$y'(0)=-\frac{1}{2}\Rightarrow c_2=-\frac{1}{16}$, so
$y=\frac{1}{4}\cos8t-\frac{1}{16}\sin8t$ ft;\;
$R=\frac{\sqrt{17}}{16}$ ft;\ $\omega_0=8$ rad/s;\ $T=\pi/4$ s;
$\phi\approx-.245\mbox{ rad}\approx -14.04^\circ$.
\end{solution}
\end{problem}

\begin{problem}\label{exer:6.1.7}
A  weight stretches a spring 1.5 inches in equilibrium. The weight
is initially displaced 8 inches above equilibrium and given a downward
velocity of 4 ft/s. Find its displacement for $t > 0$.
\end{problem}

\begin{problem}\label{exer:6.1.8}
A weight stretches a spring 6 inches in equilibrium. The weight is
initially displaced 6 inches above equilibrium and given a downward
velocity of 3 ft/s. Find its displacement for $t>0$.

\begin{solution}
Since $\frac{k}{m}=\frac{g}{\Delta l}=\frac{32}{1/2}=64$ the
equation of motion is (A) $y''+64y=0$. The general solution of (A) is
$y=c_1\cos8t+c_2\sin8t$, so $y'=8(-c_1\sin8t+c_2\cos8t)$. Now
$y(0)=\frac{1}{2}\Rightarrow c_1=\frac{1}{2}$ and
$y'(0)=-3\Rightarrow c_2=-\frac{3}{8}$, so
$y=\frac{1}{2}\cos8t-\frac{3}{8}\sin 8t$\text{ ft}.
\end{solution}
\end{problem}

\begin{problem}\label{exer:6.1.9}
A spring--mass system has natural frequency $7\sqrt{10}$ rad/s. The
natural length of the spring is .7 m. What is the length of the spring
when the mass is in equilibrium?
\end{problem}

\begin{problem}\label{exer:6.1.10}
A 64 lb weight is attached to a spring with constant $k=8$ lb/ft and
subjected to an external force $F(t)=2\sin t$. The weight is initially
displaced 6 inches above equilibrium and given an upward velocity of 2
ft/s. Find its displacement for $t>0$.

\begin{solution}
$m=\frac{64}{32}=2$, so the equation of motion is $2y''+8y=2\sin t$,
or (A) $y''+4y=\sin t$. Let $y_p=A\cos t+B\sin t$; then $y_p''=-A\cos
t-B\sin t$, so $y_p''+4y_p=3A\cos t+3B\sin t= \sin t$ if $3A=0$,
$3B=1$. Therefore,$A=0$, $B=\frac{1}{3}$, and $y_p=\frac{1}{3}\sin
t$. The general solution of (A) is (B) $y=\frac{1}{3}\sin
t+c_1\cos2t+c_2\sin2t$, so $y(0)=\frac{1}{2}\Rightarrow
c_1=\frac{1}{2}$. Differentiating (B) yields $y'=\frac{1}{3}\cos
t-2c_1\sin2t+2c_2\cos2t$, so $y'(0)=2\Rightarrow
2=\frac{1}{3}+2c_2$, so $c_2=\frac{5}{6}$. Therefore,
$y=\frac{1}{3}\sin t+\frac{1}{2}\cos2t+\frac{5}{6}\sin2t$ ft.
\end{solution}
\end{problem}

\begin{problem}\label{exer:6.1.11}
A unit mass hangs in equilibrium from a spring with constant $k=1/16$.
Starting at $t=0$, a force $F(t)=3\sin t$ is applied to the mass. Find
its displacement for $t>0$.
\end{problem}

\begin{problem}\label{exer:6.1.12} 
A 4 lb weight stretches a spring 1 ft in equilibrium. An external
force $F(t)=.25\sin8 t$ lb is applied to the weight, which is
initially displaced 4 inches above equilibrium and given a downward
velocity of 1 ft/s. Find and graph its displacement for $t>0$.

\begin{solution}
$m=\frac{4}{32}=\frac{1}{8}$ and $k=\frac{mg}{\Delta l}=4$, so
the equation of motion is $\frac{1}{8}y''+4y =\frac{1}{4}\sin8t$,
or (A) $y''+32y=2\sin8t$. Let $y_p=A\cos8t+B\sin8t$; then
$y_p''=-64A\cos t-64B\sin8t$, so $y_p''+32y_p=-32A\cos8t-32B\sin8t=
2\sin8t$ if $-32A=0$, $-32B=2$. Therefore,$A=0$, $B=-\frac{1}{16}$,
and $y_p=-\frac{1}{16}\sin8t$. The general solution of (A) is (B)
$y=-\frac{1}{16}\sin8t+c_1\cos4\sqrt2t+c_2\sin4\sqrt2t$, so
$y(0)=\frac{1}{3}\Rightarrow c_1=\frac{1}{3}$. Differentiating (B)
yields $y'=-\frac{1}{2}\cos
8t+4\sqrt2(-c_1\sin4\sqrt2t+c_2\cos4\sqrt2t)$, so $y'(0)=-1\Rightarrow
-1=-\frac{1}{2}+4\sqrt2c_2$, so $c_2=-\frac{1}{8\sqrt2}$.
Therefore,$y=-\frac{1}{16}\sin8t+\frac{1}{3}\cos4\sqrt2t
-\frac{1}{8\sqrt2 }\sin 4\sqrt2t$~ft.
\end{solution}
\end{problem}

\begin{problem}\label{exer:6.1.13}
A 2 lb weight stretches a spring 6 inches in equilibrium.  An external
force $F(t)=\sin8t$ lb is applied to the weight, which is released
from rest 2 inches below equilibrium. Find its displacement
for $t>0$.
\end{problem}

\begin{problem}\label{exer:6.1.14}
A 10 gm mass suspended on a spring moves in simple harmonic
motion with period 4 s.  Find the period of the simple
harmonic motion of a 20 gm mass suspended from the same spring.

\begin{solution}
    Since $T=\frac{2\pi}{\omega_0}=2\pi\sqrt\frac{m}{k}$
the period is proportional to the square root of the mass.
Therefore, doubling the mass mutiplies the period by $\sqrt2$;
hence the period of the system with the 20 gm mass is
 $T=4\sqrt{2}$~s.
\end{solution}
\end{problem}

\begin{problem}\label{exer:6.1.15}
A 6 lb weight stretches a spring 6 inches in equilibrium. Suppose
an external force $F(t)=\frac{3}{16}\sin\omega
t+\frac{3}{8}\cos\omega t $ lb is applied to the weight. For what
value
of $\omega$ will the displacement be unbounded? Find the displacement
if $\omega$ has this value. Assume that the motion starts from
equilibrium with zero initial velocity.
\end{problem}

\begin{problem}\label{exer:6.1.16} 
A 6 lb weight stretches a spring 4 inches in equilibrium. Suppose
an external force $ F(t)=4\sin\omega t-6\cos\omega t $ lb is applied
to the weight. For what value of $\omega$ will the displacement be
unbounded? Find and graph the displacement if $\omega$ has this value.
Assume that the motion starts from equilibrium with zero initial
velocity.

\begin{solution}
$m=\frac{6}{32}=\frac{3}{16}$ and $k=\frac{mg}{\Delta
l}=\frac{6}{1/3}=18$ so the equation of motion is
$\frac{3}{16}y''+18y=4\sin\omega t-6\cos\omega t$, or (A)
$y''+96y=\frac{64}{3}\sin\omega t-32\cos\omega t$. The displacement
will be unbounded if $\omega=\sqrt{96}=4\sqrt6$, in which case (A)
becomes (B) $y''+96y=\frac{64}{3}\sin4\sqrt6t-32\cos4\sqrt6t$. Let
\begin{eqnarray*}
y_p&=&At\cos4\sqrt6t+Bt\sin4\sqrt6t; \text{ then}\\
y_p'&=&(A+4\sqrt6Bt)\cos4\sqrt6t+(B-4\sqrt6At)\sin4\sqrt6t\\
y_p''&=&(8\sqrt6B-96At)\cos4\sqrt6t-(8\sqrt6A+96Bt)\sin4\sqrt6t,
\text{ so}
\end{eqnarray*}
$$
 y_p''+96y_p=8\sqrt6B\cos4\sqrt6t-8\sqrt6A\sin4\sqrt6t=
\frac{64}{3}\sin4\sqrt6t-32\cos4\sqrt6t
$$
if $8\sqrt6B=-32$, $-8\sqrt6A=\frac{64}{3}$. Therefore,
$A=-\frac{8}{3\sqrt6}$, $B=-\frac{4}{\sqrt6}$, and
$y_p=-\frac{t}{\sqrt6}\left(\frac{8}{3}\cos4\sqrt6t+
4\sin4\sqrt6t\right)$.
The general solution of (B) is
$$
y=-\frac{t}{\sqrt6}\left(\frac{8}{3}\cos4\sqrt6t+4\sin4\sqrt6t\right)
+c_1\cos4\sqrt6t+c_2\sin4\sqrt6t,
\text{(C)}
$$
so $y(0)=0\Rightarrow c_1=0$. Differentiating
(C) yields
\begin{eqnarray*}
y'&=&-\left(\frac{8}{3\sqrt6}\cos4\sqrt6t+
\frac{4}{\sqrt6} \sin4\sqrt6t\right)
-4t\left(-\frac{8}{3} \sin4\sqrt6t+
4\cos4\sqrt6t\right)\\
&&+4\sqrt6(-c_1\sin4\sqrt6t+c_2\cos\sqrt6t),
\end{eqnarray*}
so $y'(0)=0\Rightarrow 0=-\frac{8}{3\sqrt6}+4\sqrt6c_2$, and
$c_2=\frac{1}{9}$. Therefore,
$$
y=-\frac{t}{\sqrt6} \left(\frac{8}{3}\cos4\sqrt6t+4\sin
4\sqrt6t\right)+\frac{1}{9}\sin 4\sqrt{6t}\text{ ft}.
$$
\end{solution}
\end{problem}

\begin{problem}\label{exer:6.1.17}
A mass of one kg is attached to a spring with constant $k=4$ N/m. An
external force $F(t)=-\cos\omega t-2\sin\omega t$ n is applied to the
mass. Find the displacement $y$ for $t>0$ if $\omega$ equals the
natural frequency of the spring--mass system. Assume that the
mass is initially displaced 3 m above equilibrium and given an upward
velocity of 450 cm/s.
\end{problem}

\begin{problem}\label{exer:6.1.18}
An object is in simple harmonic motion
with frequency $\omega_0$, with $y(0)=y_0$ and $y'(0)=v_0$. Find
its displacement for $t>0$. Also, find the amplitude of the
oscillation and give formulas for the sine and cosine of the initial
phase angle.

\begin{solution}
    The equation of motion is (A) $y''+\omega_0^2y=0$. The general
solution of (A) is $y=c_1\cos\omega_0t+c_2\sin\omega_0t$.
Now $y(0)=y_0\Rightarrow c_1=y_0$. Since
 $y'=\omega_0(-c_1\sin\omega_0t+c_2\cos\omega_0t)$,
$y'(0)=v_0\Rightarrow c_2=\frac{v_0}{\omega_0}$. Therefore,
$y=y_0\cos\omega_0 t+\frac{v_0}{\omega_0}\sin\omega_0t$;
$$
R=\frac{1}{\omega_0} \sqrt{(\omega_0y_0)^2+(v_0)^2};\;
\cos\phi=\frac{y_0\omega_0}{
\sqrt{(\omega_0y_0)^2+(v_0)^2}}\,;\ \sin\phi=\frac{v_0}{
\sqrt{(\omega_0y_0)^2+(v_0)^2}}\,.
$$
\end{solution}
\end{problem}

\begin{remark}
    In Exercises~\ref{exer:6.1.19}-\ref{exer:6.1.21} we use
the fact that in a spring--mass system with mass $m$ and spring
constant $k$ the period of the motion is $T=2\pi\sqrt\frac{m}{
k}$. Therefore, if we have two systems with masses $m_1$ and $m_2$
and spring constants $k_1$ and $k_2$, then the periods are related by
$\frac{T_2}{ T_1}=\sqrt\frac{m_2k_1}{ m_1k_2}$. We will use this
formula in the solutions of these exercises.
\end{remark}

\begin{problem}\label{exer:6.1.19}
Two objects suspended from identical springs are set into
motion.  The period of one object is twice the period of the other.
How are the weights of the two objects related?
\end{problem}

\begin{problem}\label{exer:6.1.20}
Two objects suspended from identical springs are set into motion. The
weight of one object is twice the weight of the other. How are the
periods of the resulting motions related?

\begin{solution}
we use
the fact that in a spring--mass system with mass $m$ and spring
constant $k$ the period of the motion is $T=2\pi\sqrt{\frac{m}{k}}$. Therefore, if we have two systems with masses $m_1$ and $m_2$
and spring constants $k_1$ and $k_2$, then the periods are related by
$\frac{T_2}{T_1}=\sqrt{\frac{m_2k_1}{m_1k_2}}$. We will use this
formula in this solution.

Let $m_2=2m_1$. Since $k_1=k_2$, $\frac{T_2}{T_1} =\sqrt{\frac{2m_1}{m_1}} =\sqrt2$, so $T_2=\sqrt2T_1$.

\end{solution}
\end{problem}

\begin{problem}\label{exer:6.1.21}
Two identical objects suspended from different springs are set into
motion. The period of one motion is 3 times the period of the
other. How are the two spring constants related?
\end{problem}

\end{document}

