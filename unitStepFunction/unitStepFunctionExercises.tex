\documentclass{ximera}
%% You can put user macros here
%% However, you cannot make new environments

%\listfiles

% Get the 'old' hints/expandables, for use on ximera.osu.edu
%\def\xmNotHintAsExpandable{true}
%\def\xmNotExpandableAsAccordion{true}



%\graphicspath{{./}{firstExample/}{secondExample/}}
\graphicspath{{./}
{aboutDiffEq/}
{applicationsLeadingToDiffEq/}
{applicationsToCurves/}
{autonomousSecondOrder/}
{basicConcepts/}
{bernoulli/}
{constCoeffHomSysI/}
{constCoeffHomSysII/}
{constCoeffHomSysIII/}
{constantCoeffWithImpulses/}
{constantCoefficientHomogeneousEquations/}
{convolution/}
{coolingActivity/}
{directionFields/}
{drainingTank/}
{epidemicActivity/}
{eulersMethod/}
{exactEquations/}
{existUniqueNonlinear/}
{frobeniusI/}
{frobeniusII/}
{frobeniusIII/}
{global.css/}
{growthDecay/}
{heatingCoolingActivity/}
{higherOrderConstCoeff/}
{homogeneousLinearEquations/}
{homogeneousLinearSys/}
{improvedEuler/}
{integratingFactors/}
{interactExperiment/}
{introToLaplace/}
{introToSystems/}
{inverseLaplace/}
{ivpLaplace/}
{laplaceTable/}
{lawOfCooling/}
{linSysOfDiffEqs/}
{linearFirstOrderDiffEq/}
{linearHigherOrder/}
{mixingProblems/}
{motionUnderCentralForce/}
{nonHomogeneousLinear/}
{nonlinearToSeparable/}
{odesInSage/}
{piecewiseContForcingFn/}
{population/}
{reductionOfOrder/}
{regularSingularPts/}
{reviewOfPowerSeries/}
{rlcCircuit/}
{rungeKutta/}
{secondLawOfMotion/}
{separableEquations/}
{seriesSolNearOrdinaryPtI/}
{seriesSolNearOrdinaryPtII/}
{simplePendulum/}
{springActivity/}
{springProblemsI/}
{springProblemsII/}
{undCoeffHigherOrderEqs/}
{undeterminedCoeff/}
{undeterminedCoeff2/}
{unitStepFunction/}
{varParHigherOrder/}
{varParamNonHomLinSys/}
{variationOfParameters/}
}


\usepackage{tikz}
%\usepackage{tkz-euclide}
\usepackage{tikz-3dplot}
\usepackage{tikz-cd}
\usetikzlibrary{shapes.geometric}
\usetikzlibrary{arrows}
\usetikzlibrary{decorations.pathmorphing,patterns}
\usetikzlibrary{backgrounds} % added by Felipe
% \usetkzobj{all}   % NOT ALLOWED IN RECENT TeX's ...
\pgfplotsset{compat=1.13} % prevents compile error.

\pdfOnly{\renewcommand{\theHsection}{\thepart.section.\thesection}}  %% MAKES LINKS WORK should be added to CLS
\pdfOnly{\renewcommand{\part}[1]{\chapterstyle\title{#1}\begin{abstract}\end{abstract}\maketitle\def\thechaptertitle{#1}}}


\renewcommand{\vec}[1]{\mathbf{#1}}
\newcommand{\RR}{\mathbb{R}}
\providecommand{\dfn}{\textit}
\renewcommand{\dfn}{\textit}
\newcommand{\dotp}{\cdot}
\newcommand{\id}{\text{id}}
\newcommand\norm[1]{\left\lVert#1\right\rVert}
\newcommand{\dst}{\displaystyle}
 
\newtheorem{general}{Generalization}
\newtheorem{initprob}{Exploration Problem}

\tikzstyle geometryDiagrams=[ultra thick,color=blue!50!black]

\usepackage{mathtools}

\title{Exercises} \license{CC BY-NC-SA 4.0}

\begin{document}

\begin{abstract}
\end{abstract}
\maketitle

\begin{onlineOnly}
\section*{Exercises}
\end{onlineOnly}

\begin{problem}\label{exer:8.4.1} 
Find the Laplace transform by
the method of Example~\ref{example:8.4.1}. Then express the given function
$f$ in terms of unit step functions as in Eqn.~\eqref{eq:8.4.6}, and use
Theorem~\ref{thmtype:8.4.1} to find ${\cal L}(f)$.
$f(t)=\left\{\begin{array}{cl} 1,&0
\le t<4,\\ t,&t\ge4.\end{array}\right.$
\end{problem}

\begin{problem}\label{exer:8.4.2} 
Find the Laplace transform by
the method of Example~\ref{example:8.4.1}. Then express the given function
$f$ in terms of unit step functions as in Eqn.~\eqref{eq:8.4.6}, and use
Theorem~\ref{thmtype:8.4.1} to find ${\cal L}(f)$.
$f(t)=\left\{\begin{array}{cl} t,&0
\le t<1,\\ 1,&t\ge1.\end{array}\right.$

\begin{solution}
$$
{\cal L}(f)  =\int_0^\infty e^{-st}f(t)\, dt
 =\int_0^1 e^{-st}t\,dt+\int_1^\infty e^{-st}\,dt.
\text{ (A)}
$$
To relate the first term to a Laplace transform we add and subtract
$\int_1^\infty e^{-st}t\, dt$
  in (A) to obtain
$$
{\cal L}(f)=\int_0^\infty e^{-st}t\,dt+
\int_1^\infty e^{-st}(1-t)\, dt
={\cal L}(t)-\int_1^\infty e^{-st}(t-1)\,dt.
\text{ (B)}
$$
Letting $t=x+1$  in the last integral yields
$$
\int_1^\infty e^{-st}(t-1)\,dt=-\int_0^\infty e^{-s(x+1)}x\,dx
 =e^{-s}{\cal L}(t).
$$
This and (B) imply that
${\cal L}(f)=(1-e^{-s}){\cal L}(t)
= \frac{1-e^{-s}}{s^2}$.

Alternatively,
$f(t)=t-u(t-1)(t-1)\leftrightarrow(1-e^{-s}){\cal
L}(t)= \frac{1-e^{-s}}{s^2}$.
\end{solution}
\end{problem}

\begin{problem}\label{exer:8.4.3}
Find the Laplace transform by
the method of Example~\ref{example:8.4.1}. Then express the given function
$f$ in terms of unit step functions as in Eqn.~\eqref{eq:8.4.6}, and use
Theorem~\ref{thmtype:8.4.1} to find ${\cal L}(f)$.  Graph $f$.
$f(t)=\left\{\begin{array}{cl} 2t-1,&
0\le t<2,\\  t,&t\ge2.\end{array}\right.$
\end{problem}

\begin{problem}\label{exer:8.4.4}
Find the Laplace transform by
the method of Example~\ref{example:8.4.1}. Then express the given function
$f$ in terms of unit step functions as in Eqn.~\eqref{eq:8.4.6}, and use
Theorem~\ref{thmtype:8.4.1} to find ${\cal L}(f)$.  Graph $f$.
$f(t)=\left\{\begin{array}{cl}1,
&0\le t<1,\\ t+2,&t\ge1.\end{array}\right.$

\begin{solution}
$$
{\cal L}(f)  =\int_0^\infty e^{-st}f(t)\, dt
 =\int_0^1 e^{-st}\,dt+\int_1^\infty e^{-st}(t+2)\,dt.
\text{(A)}
$$
To relate the first term to a Laplace transform we  add and subtract
$\int_1^\infty e^{-st}\, dt$
  in (A) to obtain
$$
{\cal L}(f)=\int_0^\infty e^{-st}\,dt+
\int_1^\infty e^{-st}(t+1)\, dt
={\cal L}(t)+\int_1^\infty e^{-st}(t+1)\,dt.
\text{(B)}
$$
Letting $t=x+1$  in the last integral yields
$$
\int_1^\infty e^{-st}(t+1)\,dt=\int_0^\infty e^{-s(x+1)}(x+2)\,dx
 =e^{-s}{\cal L}(t+2).
$$
This and (B) imply that
$ {\cal L}(f)={\cal L}(1)+e^{-s}{\cal L}(t+2)
=\frac{1}{s}+e^{-s}\left(\frac{1}{s^2}+\frac{2}{s}\right)$.

Alternatively,

 $f(t)= 1+u(t-1)(t+1)\leftrightarrow {\cal L}(1)+e^{-s}{\cal
L}(t+2)=
\frac{1}{s}+e^{-s}\left(\frac{1}{s^2}+\frac{2}{s}\right)$.
\end{solution}
\end{problem}

\begin{problem}\label{exer:8.4.5}
Find the Laplace transform by
the method of Example~\ref{example:8.4.1}. Then express the given function
$f$ in terms of unit step functions as in Eqn.~\eqref{eq:8.4.6}, and use
Theorem~\ref{thmtype:8.4.1} to find ${\cal L}(f)$.
$f(t)=\left\{\begin{array}{cl} t-1,& 0\le
t<2,\\ 4,&t\ge2.\end{array}\right.$
\end{problem}

 \begin{problem}\label{exer:8.4.6}
Find the Laplace transform by
the method of Example~\ref{example:8.4.1}. Then express the given function
$f$ in terms of unit step functions as in Eqn.~\eqref{eq:8.4.6}, and use
Theorem~\ref{thmtype:8.4.1} to find ${\cal L}(f)$.
$f(t)=\left\{\begin{array}{cl} t^2,& 0\le
t<1,\\ 0,&t\ge1.\end{array}\right.$

\begin{solution}
$$
{\cal L}(f)  =\int_0^\infty e^{-st}f(t)\, dt
 =\int_0^1 e^{-st}t^2={\cal L}(t^2)-\int_1^\infty t^2\,dt.
\text{ (A)}
$$
Letting $t=x+1$  in the last integral yields
$$
\int_1^\infty e^{-st}t^2\,dt=\int_0^\infty e^{-s(x+1)}(t^2+2t+1)\,dx
 =e^{-s}{\cal L}(t^2+2t+1).
$$
This and (A) imply that
$$
{\cal L}(f)={\cal L}(t^2)+e^{-s}{\cal L} (t^2+2t+1)
=\frac{2}{s^3}-e^{-s}\left(\frac{2}{s^3}+\frac{2}{s^2}+\frac{1}{s}\right).
$$

Alternatively,
$$
f(t)=t^2\left(1-u(t-1)\right)\leftrightarrow
{\cal L}(t^2)+e^{-s}{\cal L} (t^2+2t+1)=
\frac{2}{s^3}-e^{-s}\left(\frac{2}{s^3}+\frac{2}{s^2}+\frac{1}{s}\right)
$$
\end{solution}
\end{problem}

\begin{problem}\label{exer:8.4.7}
Express the
given function $f$ in terms of unit step functions
 and use Theorem~\ref{thmtype:8.4.1} to find ${\cal L}(f)$.
$f(t)=\left\{\begin{array}{cl} 0, &0\le
t<2,\\ t^2+3t,&t\ge2.\end{array}\right.$
\end{problem}

 \begin{problem}\label{exer:8.4.8}
Express the
given function $f$ in terms of unit step functions
 and use Theorem~\ref{thmtype:8.4.1} to find ${\cal L}(f)$.
 $f(t)=\left\{\begin{array}{cl} t^2+2, &0\le
t<1,\\ t,&t\ge1.\end{array}\right.$

\begin{solution}
$f(t)=t^2+2+u(t-1)(t-t^2-2)$.
Since $t^2+2\leftrightarrow\frac{2}{s^3}+\frac{2}{s}$  and
\begin{eqnarray*}
{\cal L}\left(u(t-1)(t-t^2-2)\right)&=&e^{-s}{\cal
L}\left((t+1)-(t+1)^2-2\right)\\&=&-e^{-s}{\cal
L}(t^2+t+2)=-e^{-s}\left(\frac{2}{s^3}+\frac{1}{s^2}+\frac{2}{s}\right),
\end{eqnarray*}
it follows that $F(s)=\frac{2}{s^3} +\frac{2}{s}-e^{-s}\left(\frac{2}{s^3}+\frac{1}{s^2}+\frac{2}{s}\right)$.
\end{solution}
\end{problem}
 \begin{problem}\label{exer:8.4.9}
Express the
given function $f$ in terms of unit step functions
 and use Theorem~\ref{thmtype:8.4.1} to find ${\cal L}(f)$.
 $f(t)=\left\{\begin{array}{cl} te^t,& 0\le t
<1,\\ e^t,&t\ge1.\end{array}\right.$
\end{problem}

\begin{problem}\label{exer:8.4.10} 
Express the
given function $f$ in terms of unit step functions
 and use Theorem~\ref{thmtype:8.4.1} to find ${\cal L}(f)$.
$f(t)=\left\{\begin{array}{cl}
e^{2-t}, &0\le t<1,\\ 
e^{-2t},&t\ge1.\end{array}\right.$

\begin{solution}
$f(t)=e^{-t}+u(t-1)(e^{-2t}-e^{-t})
\leftrightarrow {\cal L}(e^{-t})+e^{-s}{\cal
L}(e^{-2(t+1)})-e^{-s}{\cal L}(e^{-t-1})=
 {\cal L}(e^{-t})+e^{-(s+2)}{\cal
L}(e^{-2t})-e^{-(s+1)}{\cal L}(e^{-t})=
\frac{1-e^{-(s+1)}}{s+1}+\frac{e^{-(s+2)}}{s+2}$.
\end{solution}
\end{problem}

\begin{problem}\label{exer:8.4.11}
Express the
given function $f$ in terms of unit step functions
 and use Theorem~\ref{thmtype:8.4.1} to find ${\cal L}(f)$. 
$f(t)=\left\{\begin{array}{cl} -t,&0 \le
t<2,\\ t-4,&2\le t<3,\\ 1,&t\ge3.
\end{array}\right.$
\end{problem}

\begin{problem}\label{exer:8.4.12} $f(t)=\left\{\begin{array}{cl} 0,&0 \le
t<1,\\ t,&1\le t<2,\\ 0,&t\ge2.\end{array}\right.$

\begin{solution}
$f(t)=\left[u(t-1)-u(t-2)\right]t\leftrightarrow
e^{-s}{\cal L}(t+1)-e^{-2s}{\cal L}(t+2)\\=
e^{-s}\left(\frac{1}{s^2}+\frac{1}{s}\right)-e^{-2s}\left(\frac{1}{s^2}
+\frac{2}{s}\right)$.
\end{solution}
\end{problem}

\begin{problem}\label{exer:8.4.13} 
 Express the
given function $f$ in terms of unit step functions
 and use Theorem~\ref{thmtype:8.4.1} to find ${\cal L}(f)$.
 $f(t)=\left\{\begin{array}{cl} t,&0 \le
t<1,\\ t^2,&1\le t<2,\\ 0,&t\ge2. \end{array}\right.$
\end{problem}

 \begin{problem}\label{exer:8.4.14} 
 Express the
given function $f$ in terms of unit step functions
 and use Theorem~\ref{thmtype:8.4.1} to find ${\cal L}(f)$.
 $f(t)=\left\{\begin{array}{cl} t,&0\le
t<1,\\ 2-t,&1\le t<2,\\ 6,&t > 2. \end{array}\right.$

\begin{solution}
\begin{eqnarray*}
f(t)&=&t-2u(t-1)(t-1)+u(t-2)(t+4)\leftrightarrow
   \frac{1}{s^2}-2e^{-s}{\cal L}(t)+e^{-2s}{\cal L}(t+6)\\
  &=&\frac{1}{s^2}-\frac{2e^{-s}}{s^2}
+e^{-2s}\left(\frac{1}{s^2}+\frac{6}{s}\right).
\end{eqnarray*}
\end{solution}
\end{problem}

\begin{problem}\label{exer:8.4.15} 
Express the given function $f$ in terms of unit step functions and use Theorem~\ref{thmtype:8.4.1} to find ${\cal L}(f)$.  Graph $f$.
 $f(t)=\left\{\begin{array}{cl}
2
\sin t,&0\le t<\frac{\pi}{2},\\ 2\sin t,&
\frac{\pi}{2}\le t<\pi,\\ 2\cos t,
&t\ge\pi.\end{array}\right.$
\end{problem}

\begin{problem}\label{exer:8.4.16} Express the given function $f$ in terms of unit step functions and use Theorem~\ref{thmtype:8.4.1} to find ${\cal L}(f)$.  Graph $f$.
$f(t)=\left\{\begin{array}{cl}- 2,&0\le
t<1,\\ -2t+2,&1\le t<3,\\ -3t,&t\ge
3.\end{array}\right.$

\begin{solution}
$f(t)=2-2u(t-1)t+u(t-3)(5t-2)
\leftrightarrow {\cal L}(2)-2e^{-s}{\cal L}(t+1)+e^{-3s}{\cal
L}(5t+13)=
\frac{2}{s}-e^{-s}\left(\frac{2}{s^2}+\frac{2}{s}\right)+e^{-3s}\left(\frac{5}{s^2}+\frac{13}{s}\right)$.
\end{solution}
\end{problem}

\begin{problem}\label{exer:8.4.17}  Express the given function $f$ in terms of unit step functions and use Theorem~\ref{thmtype:8.4.1} to find ${\cal L}(f)$.  Graph $f$.
$f(t)=\left\{\begin{array}{cl}3,&0\le t<2,\\ 3t+2,&2\le
t<4,\\ 4t,&t\ge
4.\end{array}\right.$
\end{problem}

\begin{problem}\label{exer:8.4.18}  Express the given function $f$ in terms of unit step functions and use Theorem~\ref{thmtype:8.4.1} to find ${\cal L}(f)$.  Graph $f$.
$f(t)=\left\{\begin{array}{ll}(t+1)^2,&0\le t<1,
\\(t+2)^2,&t\ge1.\end{array}\right.$

\begin{solution}
$f(t)=(t+1)^2+u(t-1)\left((t+2)^2-(t+1)^2\right)
=t^2+2t+1+u(t-1)(2t+3)
\leftrightarrow
{\cal L}(t^2+2t+1)+e^{-s}{\cal L}(2t+5)=
\frac{2}{s^3 }+\frac{2}{s^2}+\frac{1}{s}+e^{-s}\left(\frac{2}{s^2}+\frac{5}{s}\right)$.
\end{solution}
\end{problem}

\begin{problem}\label{exer:8.4.19} Use Theorem~\ref{thmtype:8.4.2} to express the inverse transforms in terms of step functions, and then find distinct formulas for the inverse transforms on the appropriate intervals, as in
Example~\ref{example:8.4.7}.
$H(s)=\frac{e^{-2s}}{s-2}$
\end{problem}

\begin{problem}\label{exer:8.4.20}
Use Theorem~\ref{thmtype:8.4.2} to express the inverse transforms in terms of step functions, and then find distinct formulas for the inverse transforms on the appropriate intervals, as in
Example~\ref{example:8.4.7}.
$H(s)=\frac{e^{-s}}{s(s+1)}$

\begin{solution}
$\frac{1}{s(s+1)}=\frac{1}{s}-\frac{1}{s+1}\leftrightarrow
 1-e^{-t}
\Rightarrow
e^{-s}\frac{1}{s(s+1)}\leftrightarrow
u(t-1)\left(1-e^{-(t-1)}\right)=
\left\{\begin{array}{cl} 0,&0\le t<1,\\
1-e^{-(t-1)},&t\ge1.\end{array}\right.$
\end{solution}
\end{problem}

\begin{problem}\label{exer:8.4.21}
Use Theorem~\ref{thmtype:8.4.2} to express the inverse transforms in terms of step functions, and then find distinct formulas for the inverse transforms on the appropriate intervals, as in
Example~\ref{example:8.4.7}.  Graph the inverse transform. 
$H(s)=\frac{e^{-s}}{s^3}+
\frac{e^{-2s}}{s^2}$
\end{problem}

\begin{problem}\label{exer:8.4.22} Use Theorem~\ref{thmtype:8.4.2} to express the inverse transforms in terms of step functions, and then find distinct formulas for the inverse transforms on the appropriate intervals, as in
Example~\ref{example:8.4.7}.  Graph the inverse transform.  $H(s)=\left(\frac{2}{s}+\frac{1}{s^2}\right)
+e^{-s}\left(\frac{3}{s}-\frac{1}{s^2}\right)+e^{-3s}\left(\frac{1}{s}+\frac{1}{s^2}\right)$

\begin{solution}
$$
\frac{3}{s}-\frac{1}{s^2}\leftrightarrow 3-t\Rightarrow
e^{-s}\left(\frac{3}{s}-\frac{1}{s^2}\right)\leftrightarrow
u(t-1)\left(3-(t-1)\right)=u(t-1)(4-t);
$$
$$
\frac{1}{s}+\frac{1}{s^2}\leftrightarrow 1+t\Rightarrow
e^{-3s}\left(\frac{1}{s}+\frac{1}{s^2}\right)\leftrightarrow
u(t-3)\left(1+(t-3)\right)=u(t-3)(t-2);
$$
therefore
$$
h(t)=2+t+u(t-1)(4-t)+u(t-3)(t-2)=\left\{\begin{array}{cl} 2+t,&
0\le t<1,\\  6,&1\le t<3,\\  t+4,&t\ge
3.\end{array}\right.
$$
\end{solution}
\end{problem}

\begin{problem}\label{exer:8.4.23} Use Theorem~\ref{thmtype:8.4.2} to express the inverse transforms in terms of step functions, and then find distinct formulas for the inverse transforms on the appropriate intervals, as in
Example~\ref{example:8.4.7}.
$H(s)=\left(\frac{5}{s}-\frac{1}{s^2}\right)
+e^{-3s}\left(\frac{6}{s}+\frac{7}{s^2}\right)+\frac{3e^{-6s}}{s^3}$
\end{problem}

\begin{problem}\label{exer:8.4.24}
Use Theorem~\ref{thmtype:8.4.2} to express the inverse transforms in terms of step functions, and then find distinct formulas for the inverse transforms on the appropriate intervals, as in
Example~\ref{example:8.4.7}.
$H(s)=\frac{e^{-\pi s} (1-2s)}{s^2+4s+5}$

\begin{solution}
$$
\frac{1-2s}{s^2+4s+5}=\frac{5-2(s+2)}{(s+2)^2+1}\leftrightarrow
e^{-2t}(5\sin t-2\cos t);
$$
therefore,
\begin{eqnarray*}
h(t)&=&u(t-\pi)e^{-2(t-\pi)} \left(5\sin(t-\pi)-2\cos(t-\pi)\right)\\
&=&u(t-\pi)e^{-2(t-\pi)} (2\cos t-5\sin t)\\
&=&\left\{\begin{array}{cl} 0, &0\le t<\pi,\\
e^{-2(t-\pi)}(2\cos t-5\sin t),&t\ge\pi. \end{array}\right..
\end{eqnarray*}
\end{solution}
\end{problem}

\begin{problem}\label{exer:8.4.25} Use Theorem~\ref{thmtype:8.4.2} to express the inverse transforms in terms of step functions, and then find distinct formulas for the inverse transforms on the appropriate intervals, as in
Example~\ref{example:8.4.7}.  Graph the inverse transform.
$H(s)=\left(\frac{1}{s}-\frac{s}{s^2+1}\right)+e^{-\frac{\pi}{2}s}\left(\frac{3s-1}{s^2+1}\right)$
\end{problem}

\begin{problem}\label{exer:8.4.26} Use Theorem~\ref{thmtype:8.4.2} to express the inverse transforms in terms of step functions, and then find distinct formulas for the inverse transforms on the appropriate intervals, as in
Example~\ref{example:8.4.7}.  $H(s)= e^{-2s}\left[\frac{3(s-3)}{(s+1)(s-2)}-\frac{s+1}{(s-1)(s-2)}\right]$

\begin{solution}
Denote $F(s)=\frac{3(s-3)}{(s+1)(s-2)}-\frac{s+1}{(s-1)(s-2)}$.
Since $\frac{3(s-3)}{(s+1)(s-2)}=\frac{4}{s+1}-\frac{1}{s-2}$ and
$\frac{s+1}{(s-1)(s-2)}=\frac{3}{s-2}-\frac{2}{s-1}$,
  $F(s)=\frac{4}{s+1}-\frac{4}{s-2}+\frac{2}{s-1}
\leftrightarrow4e^{-t}-4e^{2t}+2e^t$.
Therefore,$e^{-2s}F(s)\leftrightarrow $
$u(t-2)\left
(4e^{-(t-2)}-4e^{2(t-2)}+2e^{(t-2)}\right)=
\left\{\begin{array}{cl} 0,&0\le
t<2,\\4e^{-(t-2)}-4e^{2(t-2)}+2e^{(t-2)},&t\ge
2.\end{array}\right.$
\end{solution}
\end{problem}

\begin{problem}\label{exer:8.4.27} Use Theorem~\ref{thmtype:8.4.2} to express the inverse transforms in terms of step functions, and then find distinct formulas for the inverse transforms on the appropriate intervals, as in
Example~\ref{example:8.4.7}.
$H(s)=\frac{1}{s}+\frac{1}{s^2}+e^{-s}\left(\frac{3}{s}+\frac{2}{s^2}\right) +e^{-3s}\left(\frac{4}{s}+\frac{3}{s^2}\right)$
\end{problem}

\begin{problem}\label{exer:8.4.28}
Use Theorem~\ref{thmtype:8.4.2} to express the inverse transforms in terms of step functions, and then find distinct formulas for the inverse transforms on the appropriate intervals, as in
Example~\ref{example:8.4.7}.
$H(s)=\frac{1}{s}-\frac{2}{s^3}+e^{-2s}\left(\frac{3}{s}-\frac{1}{s^3}\right) +\frac{e^{-4s}}{s^2}$

\begin{solution}
$$
\frac{3}{s}-\frac{1}{s^3}\leftrightarrow 3-{t^2\over2}\Rightarrow
e^{-2s}\left(\frac{3}{s}-\frac{1}{s^3}\right)\leftrightarrow
u(t-2)\left(3-\frac{(t-2)^2}{2}\right)=u(t-2)
\left(-\frac{t^2}{2}+2t+1\right);
$$
$$
\frac{1}{s^2}\leftrightarrow t\Rightarrow
\frac{e^{-4s}}{s^2}\leftrightarrow u(t-4)(t-4);
$$
therefore
\begin{eqnarray*}
h(t)&=&1-t^2+u(t-2)\left(-\frac{t^2}{2}+2t+1\right) +u(t-4)(t-4)
\\&=&
\left\{\begin{array}{cl} 1-t^2,&0\le
t<2\\ -\frac{3t^2}{2} +2t+2,&2\le
t<4,\\ -\frac{3t^2}{2}+3t-2,&t\ge 4.\end{array}\right.
\end{eqnarray*}
\end{solution}
\end{problem}

\begin{problem}\label{exer:8.4.29} Find ${\cal L}\left(u(t-\tau)\right)$.
\end{problem}

\begin{problem}\label{exer:8.4.30}
Let $\{t_m\}_{m=0}^\infty$ be a sequence of points such that $t_0=0$,
$t_{m+1}>t_m$, and $\lim_{m\to\infty}t_m=\infty$. For each nonnegative
integer $m$, let $f_m$ be continuous on $[t_m,\infty)$, and let $f$ be
defined on $[0,\infty)$ by
$$
f(t)=f_m(t),\,t_m\le t<t_{m+1}\quad (m=0,1,\dots).
$$
Show that $f$ is piecewise continuous on $[0,\infty)$  and
that it has the step function representation
$$
f(t)=f_0(t)+\sum_{m=1}^\infty u(t-t_m)\left(f_m(t)-f_{m-1}(t)\right),\,
0\le t<\infty.
$$
How do we know that the series on the right converges for all $t$
in $[0,\infty)$?

\begin{solution}
Let $T$ be an arbitrary positive number.
Since $\lim_{m\to\infty}t_m=\infty$, only finitely many
members of  $\{t_m\}$ are in $[0,T]$.
Since $f_m$ is continuous on $[t_m,\infty)$ for each $m$,
$f$ is piecewise continuous on $[0,T]$.
If $t_M\le t<t_{M+1}$, then   $u(t-t_m)=1$ if $m\le M$, while
$u(t-t_m)=0$ if $m>M$. Therefore,
$f(t)= f_0(t)+\sum_{m=1}^M(f_m(t)-f_{m-1}(t))=f_M(t)$
\end{solution}
\end{problem}

\begin{problem}\label{exer:8.4.31}
In addition to the assumptions of Exercise~\ref{exer:8.4.30},
assume that
$$
|f_m(t)|\le Me^{s_0t},\,t\ge t_m,\,m=0,1,\dots,
\text{(A)}
$$
and that the series
$$
\sum_{m=0}^\infty e^{-\rho t_m}
\text{(B)}
$$
converges for some $\rho>0$. Using the  steps listed below, show
that
${\cal L}(f)$ is defined for $s>s_0$  and
$$
{\cal L}(f)={\cal L}(f_0)+\sum_{m=1}^\infty e^{-st_m}{\cal L}(g_m)
\text{(C)}
$$
for $s>s_0+\rho$,
where
$$
g_m(t)=f_m(t+t_m)-f_{m-1}(t+t_m).
$$

\begin{enumerate} % (a)
\item
Use (A) and Theorem~8.1.6 to show that
$$
{\cal L}(f)=\sum_{m=0}^\infty\int_{t_m}^{t_{m+1}}e^{-st}f_m(t)\,dt
\text{(D)}
$$
is defined for $s>s_0$.
\item % (b)
Show that  (D) can be rewritten as
$$
{\cal L}(f)=\sum_{m=0}^\infty\left(\int_{t_m}^\infty e^{-st}f_m(t)\,dt
-\int_{t_{m+1}}^\infty e^{-st}f_m(t)\,dt\right).
\text{(E)}
$$
\item % (c)
Use  (A), the assumed convergence of  (B), and the
comparison test  to show that the series
$$
\sum_{m=0}^\infty\int_{t_m}^\infty e^{-st}f_m(t)\,dt\mbox{\quad and \quad}
\sum_{m=0}^\infty\int_{t_{m+1}}^\infty e^{-st}f_m(t)\,dt
$$
both converge (absolutely) if $s>s_0+\rho$.
\item % (d)
Show that  (E) can be rewritten as
 $$
{\cal L}(f)={\cal L}(f_0)+\sum_{m=1}^\infty\int_{t_m}^\infty e^{-st}
\left(f_m(t)-f_{m-1}(t)\right)\,dt
$$
if $s>s_0+\rho$.
\item % (e)
Complete the proof of  (C).
\end{enumerate}
\end{problem}

\begin{problem}\label{exer:8.4.32} Suppose $\{t_m\}_{m=0}^\infty$ and
$\{f_m\}_{m=0}^\infty$ satisfy the assumptions of
Exercises~\ref{exer:8.4.30} and \ref{exer:8.4.31}, and
there is a positive constant $K$ such that $t_m\ge Km$ for $m$
sufficiently large. Show that the series (B) of
Exercise~\ref{exer:8.4.31} converges for any $\rho>0$, and conclude from
this that (C) of Exercise~\ref{exer:8.4.31} holds for $s>s_0$.

\begin{solution}
Since $\sum_{m=0}^\infty e^{-\rho Km}$
converges if $\rho>0$, $\sum_{m=0}^\infty e^{-\rho t_m}$ converges
if $\rho>0$, by the comparison test. Therefore,(C) of
Exercise~8.3.31 holds if $s>s_0+\rho$ if $\rho$ is any positive
number. This implies that it holds if $s>s_0$.
\end{solution}
\end{problem}



\begin{problem}\label{exer:8.4.33}
 Find the step
function representation of $f$ and use the result of
Exercise~\ref{exer:8.4.32} to find ${\cal L}(f)$. (You will need
formulas related to the formula for the sum of a geometric series.)
$f(t)=m+1,\,m\le t<m+1\; (m=0,1,2,\dots)$
\end{problem}

\begin{problem}\label{exer:8.4.34}
 Find the step
function representation of $f$ and use the result of
Exercise~\ref{exer:8.4.32} to find ${\cal L}(f)$. (You will need
formulas related to the formula for the sum of a geometric series.)
$f(t)=(-1)^m,\,m\le t<m+1\; (m=0,1,2,\dots)$

\begin{solution}
Let $t_m=m$ and $f_m(t)=(-1)^m$, $m=0,1,2,\dots$.
Then $f_m(t)-f_{m-1}(t)=(-1)^m2$, so
$f(t)=1+2\sum_{m=1}^\infty (-1)^mu(t-m)$ and
$F(s)=\frac{1}{s}\left(1+2\sum_{m=1}^\infty(-1)^me^{-ms}
\right)$
Substituting $x=e^{-s}$ in the identity $\dst\sum_{m=1}^\infty
(-1)^mx^m=-\frac{x}{1+x}$ ($|x|<1$) yields
$F(s)=\frac{1}{s}\left(1-\frac{2e^{-s}}{1+e^{-s}}\right)=
\frac{1}{s}\ \frac{1-e^{-s}}{1+e^{-s}}$.
\end{solution}
\end{problem}

\begin{problem}\label{exer:8.4.35}
Find the step function representation of $f$ and use the result of Exercise~\ref{exer:8.4.32} to find ${\cal L}(f)$. (You will need
formulas related to the formula for the sum of a geometric series.)
$f(t)=(m+1)^2,\,m\le t<m+1\; (m=0,1,2,\dots)$
\end{problem}

\begin{problem}\label{exer:8.4.36}
 Find the step
function representation of $f$ and use the result of
Exercise~\ref{exer:8.4.32} to find ${\cal L}(f)$. (You will need
formulas related to the formula for the sum of a geometric series.)
$f(t)=(-1)^mm,\,m\le t<m+1\; (m=0,1,2,\dots)$

\begin{solution}
Let $t_m=m$ and $f_m(t)=(-1)^mm$, $m=0,1,2,\dots$.
Then $f_m(t)-f_{m-1}(t)=(-1)^m(2m-1)$, so
$f(t)=\sum_{m=1}^\infty(-1)^m(2m-1)u(t-m)$and
$F(s)=\frac{1}{s}\sum_{m=1}^\infty(-1)^m(2m-1)e^{-ms}$.
Substituting $x=e^{-s}$ in the identities $\sum_{m=1}^\infty
(-1)^mx^m=-\frac{x}{1+x}$ and $\sum_{m=1}^\infty
(-1)^mmx^m=-{x\over(1+x)^2}$
($|x|<1$) yields
$F(s)=\frac{1}{s}\left[\frac{e^{-s}}{1+e^{-s}}-\frac{2e^{-s}}{(1+e^{-s})^2}\right]=
\frac{1}{s}\frac{(1-e^s)}{(1+e^s)^2}$.
\end{solution}
\end{problem}


\end{document}