\documentclass{ximera}
%% You can put user macros here
%% However, you cannot make new environments

%\listfiles

% Get the 'old' hints/expandables, for use on ximera.osu.edu
%\def\xmNotHintAsExpandable{true}
%\def\xmNotExpandableAsAccordion{true}



%\graphicspath{{./}{firstExample/}{secondExample/}}
\graphicspath{{./}
{aboutDiffEq/}
{applicationsLeadingToDiffEq/}
{applicationsToCurves/}
{autonomousSecondOrder/}
{basicConcepts/}
{bernoulli/}
{constCoeffHomSysI/}
{constCoeffHomSysII/}
{constCoeffHomSysIII/}
{constantCoeffWithImpulses/}
{constantCoefficientHomogeneousEquations/}
{convolution/}
{coolingActivity/}
{directionFields/}
{drainingTank/}
{epidemicActivity/}
{eulersMethod/}
{exactEquations/}
{existUniqueNonlinear/}
{frobeniusI/}
{frobeniusII/}
{frobeniusIII/}
{global.css/}
{growthDecay/}
{heatingCoolingActivity/}
{higherOrderConstCoeff/}
{homogeneousLinearEquations/}
{homogeneousLinearSys/}
{improvedEuler/}
{integratingFactors/}
{interactExperiment/}
{introToLaplace/}
{introToSystems/}
{inverseLaplace/}
{ivpLaplace/}
{laplaceTable/}
{lawOfCooling/}
{linSysOfDiffEqs/}
{linearFirstOrderDiffEq/}
{linearHigherOrder/}
{mixingProblems/}
{motionUnderCentralForce/}
{nonHomogeneousLinear/}
{nonlinearToSeparable/}
{odesInSage/}
{piecewiseContForcingFn/}
{population/}
{reductionOfOrder/}
{regularSingularPts/}
{reviewOfPowerSeries/}
{rlcCircuit/}
{rungeKutta/}
{secondLawOfMotion/}
{separableEquations/}
{seriesSolNearOrdinaryPtI/}
{seriesSolNearOrdinaryPtII/}
{simplePendulum/}
{springActivity/}
{springProblemsI/}
{springProblemsII/}
{undCoeffHigherOrderEqs/}
{undeterminedCoeff/}
{undeterminedCoeff2/}
{unitStepFunction/}
{varParHigherOrder/}
{varParamNonHomLinSys/}
{variationOfParameters/}
}


\usepackage{tikz}
%\usepackage{tkz-euclide}
\usepackage{tikz-3dplot}
\usepackage{tikz-cd}
\usetikzlibrary{shapes.geometric}
\usetikzlibrary{arrows}
\usetikzlibrary{decorations.pathmorphing,patterns}
\usetikzlibrary{backgrounds} % added by Felipe
% \usetkzobj{all}   % NOT ALLOWED IN RECENT TeX's ...
\pgfplotsset{compat=1.13} % prevents compile error.

\pdfOnly{\renewcommand{\theHsection}{\thepart.section.\thesection}}  %% MAKES LINKS WORK should be added to CLS
\pdfOnly{\renewcommand{\part}[1]{\chapterstyle\title{#1}\begin{abstract}\end{abstract}\maketitle\def\thechaptertitle{#1}}}


\renewcommand{\vec}[1]{\mathbf{#1}}
\newcommand{\RR}{\mathbb{R}}
\providecommand{\dfn}{\textit}
\renewcommand{\dfn}{\textit}
\newcommand{\dotp}{\cdot}
\newcommand{\id}{\text{id}}
\newcommand\norm[1]{\left\lVert#1\right\rVert}
\newcommand{\dst}{\displaystyle}
 
\newtheorem{general}{Generalization}
\newtheorem{initprob}{Exploration Problem}

\tikzstyle geometryDiagrams=[ultra thick,color=blue!50!black]

\usepackage{mathtools}

\title{Exercises} \license{CC BY-NC-SA 4.0}

\begin{document}

\begin{abstract}
\end{abstract}
\maketitle

\begin{onlineOnly}
\section*{Exercises}
\end{onlineOnly}

\begin{problem}\label{exer:8.4.1} 
Find the Laplace transform by
the method of Example~\ref{example:8.4.1}. Then express the given function
$f$ in terms of unit step functions as in Eqn.~\eqref{eq:8.4.6}, and use
Theorem~\ref{thmtype:8.4.1} to find ${\cal L}(f)$.
$f(t)=\left\{\begin{array}{cl} 1,&0
\le t<4,\\ t,&t\ge4.\end{array}\right.$
\end{problem}

\begin{problem}\label{exer:8.4.2} 
Find the Laplace transform by
the method of Example~\ref{example:8.4.1}. Then express the given function
$f$ in terms of unit step functions as in Eqn.~\eqref{eq:8.4.6}, and use
Theorem~\ref{thmtype:8.4.1} to find ${\cal L}(f)$.
$f(t)=\left\{\begin{array}{cl} t,&0
\le t<1,\\ 1,&t\ge1.\end{array}\right.$
\end{problem}

\begin{problem}\label{exer:8.4.3} 
Find the Laplace transform by
the method of Example~\ref{example:8.4.1}. Then express the given function
$f$ in terms of unit step functions as in Eqn.~\eqref{eq:8.4.6}, and use
Theorem~\ref{thmtype:8.4.1} to find ${\cal L}(f)$.  Graph $f$.
$f(t)=\left\{\begin{array}{cl} 2t-1,&
0\le t<2,\\  t,&t\ge2.\end{array}\right.$
\end{problem}

\begin{problem}\label{exer:8.4.4}
Find the Laplace transform by
the method of Example~\ref{example:8.4.1}. Then express the given function
$f$ in terms of unit step functions as in Eqn.~\eqref{eq:8.4.6}, and use
Theorem~\ref{thmtype:8.4.1} to find ${\cal L}(f)$.  Graph $f$.
$f(t)=\left\{\begin{array}{cl}1,
&0\le t<1,\\ t+2,&t\ge1.\end{array}\right.$
\end{problem}

\begin{problem}\label{exer:8.4.5}
Find the Laplace transform by
the method of Example~\ref{example:8.4.1}. Then express the given function
$f$ in terms of unit step functions as in Eqn.~\eqref{eq:8.4.6}, and use
Theorem~\ref{thmtype:8.4.1} to find ${\cal L}(f)$.
$f(t)=\left\{\begin{array}{cl} t-1,& 0\le
t<2,\\ 4,&t\ge2.\end{array}\right.$
\end{problem}

 \begin{problem}\label{exer:8.4.6}
Find the Laplace transform by
the method of Example~\ref{example:8.4.1}. Then express the given function
$f$ in terms of unit step functions as in Eqn.~\eqref{eq:8.4.6}, and use
Theorem~\ref{thmtype:8.4.1} to find ${\cal L}(f)$.
$f(t)=\left\{\begin{array}{cl} t^2,& 0\le
t<1,\\ 0,&t\ge1.\end{array}\right.$
\end{problem}

\begin{problem}\label{exer:8.4.7}
Express the
given function $f$ in terms of unit step functions
 and use Theorem~\ref{thmtype:8.4.1} to find ${\cal L}(f)$.
$f(t)=\left\{\begin{array}{cl} 0, &0\le
t<2,\\ t^2+3t,&t\ge2.\end{array}\right.$
\end{problem}

 \begin{problem}\label{exer:8.4.8}
Express the
given function $f$ in terms of unit step functions
 and use Theorem~\ref{thmtype:8.4.1} to find ${\cal L}(f)$.
 $f(t)=\left\{\begin{array}{cl} t^2+2, &0\le
t<1,\\ t,&t\ge1.\end{array}\right.$
\end{problem}


 \begin{problem}\label{exer:8.4.9}
Express the
given function $f$ in terms of unit step functions
 and use Theorem~\ref{thmtype:8.4.1} to find ${\cal L}(f)$.
 $f(t)=\left\{\begin{array}{cl} te^t,& 0\le t
<1,\\ e^t,&t\ge1.\end{array}\right.$
\end{problem}

\begin{problem}\label{exer:8.4.10} 
Express the
given function $f$ in terms of unit step functions
 and use Theorem~\ref{thmtype:8.4.1} to find ${\cal L}(f)$.
$f(t)=\left\{\begin{array}{cl}
e^{2-t}, &0\le t<1,\\ 
e^{-2t},&t\ge1.\end{array}\right.$
\end{problem}

\begin{problem}\label{exer:8.4.11}
Express the
given function $f$ in terms of unit step functions
 and use Theorem~\ref{thmtype:8.4.1} to find ${\cal L}(f)$. 
$f(t)=\left\{\begin{array}{cl} -t,&0 \le
t<2,\\ t-4,&2\le t<3,\\ 1,&t\ge3.
\end{array}\right.$
\end{problem}

\begin{problem}\label{exer:8.4.12} $f(t)=\left\{\begin{array}{cl} 0,&0 \le
t<1,\\ t,&1\le t<2,\\ 0,&t\ge2.\end{array}\right.$
\end{problem}

\begin{problem}\label{exer:8.4.13} 
 Express the
given function $f$ in terms of unit step functions
 and use Theorem~\ref{thmtype:8.4.1} to find ${\cal L}(f)$.
 $f(t)=\left\{\begin{array}{cl} t,&0 \le
t<1,\\ t^2,&1\le t<2,\\ 0,&t\ge2. \end{array}\right.$
\end{problem}

 \begin{problem}\label{exer:8.4.14} 
 Express the
given function $f$ in terms of unit step functions
 and use Theorem~\ref{thmtype:8.4.1} to find ${\cal L}(f)$.
 $f(t)=\left\{\begin{array}{cl} t,&0\le
t<1,\\ 2-t,&1\le t<2,\\ 6,&t > 2. \end{array}\right.$
\end{problem}

\begin{problem}\label{exer:8.4.15} 
Express the given function $f$ in terms of unit step functions and use Theorem~\ref{thmtype:8.4.1} to find ${\cal L}(f)$.  Graph $f$.
 $f(t)=\left\{\begin{array}{cl}
2
\sin t,&0\le t<\frac{\pi}{2},\\ 2\sin t,&
\frac{\pi}{2}\le t<\pi,\\ 2\cos t,
&t\ge\pi.\end{array}\right.$
\end{problem}

\begin{problem}\label{exer:8.4.16} Express the given function $f$ in terms of unit step functions and use Theorem~\ref{thmtype:8.4.1} to find ${\cal L}(f)$.  Graph $f$.
$f(t)=\left\{\begin{array}{cl}- 2,&0\le
t<1,\\ -2t+2,&1\le t<3,\\ -3t,&t\ge
3.\end{array}\right.$
\end{problem}

\begin{problem}\label{exer:8.4.17}  Express the given function $f$ in terms of unit step functions and use Theorem~\ref{thmtype:8.4.1} to find ${\cal L}(f)$.  Graph $f$.
$f(t)=\left\{\begin{array}{cl}3,&0\le t<2,\\ 3t+2,&2\le
t<4,\\ 4t,&t\ge
4.\end{array}\right.$
\end{problem}

\begin{problem}\label{exer:8.4.18}  Express the given function $f$ in terms of unit step functions and use Theorem~\ref{thmtype:8.4.1} to find ${\cal L}(f)$.  Graph $f$.
$f(t)=\left\{\begin{array}{ll}(t+1)^2,&0\le t<1,
\\(t+2)^2,&t\ge1.\end{array}\right.$
\end{problem}

\begin{problem}\label{exer:8.4.19} Use Theorem~\ref{thmtype:8.4.2} to express the inverse transforms in terms of step functions, and then find distinct formulas for the inverse transforms on the appropriate intervals, as in
Example~\ref{example:8.4.7}.
$H(s)=\frac{e^{-2s}}{s-2}$
\end{problem}

\begin{problem}\label{exer:8.4.20}
Use Theorem~\ref{thmtype:8.4.2} to express the inverse transforms in terms of step functions, and then find distinct formulas for the inverse transforms on the appropriate intervals, as in
Example~\ref{example:8.4.7}.
$H(s)=\frac{e^{-s}}{s(s+1)}$
\end{problem}

\begin{problem}\label{exer:8.4.21}
Use Theorem~\ref{thmtype:8.4.2} to express the inverse transforms in terms of step functions, and then find distinct formulas for the inverse transforms on the appropriate intervals, as in
Example~\ref{example:8.4.7}.  Graph the inverse transform. 
$H(s)=\frac{e^{-s}}{s^3}+
\frac{e^{-2s}}{s^2}$
\end{problem}

\begin{problem}\label{exer:8.4.22} Use Theorem~\ref{thmtype:8.4.2} to express the inverse transforms in terms of step functions, and then find distinct formulas for the inverse transforms on the appropriate intervals, as in
Example~\ref{example:8.4.7}.  Graph the inverse transform.  $H(s)=\left(\frac{2}{s}+\frac{1}{s^2}\right)
+e^{-s}\left(\frac{3}{s}-\frac{1}{s^2}\right)+e^{-3s}\left(\frac{1}{s}+\frac{1}{s^2}\right)$
\end{problem}

\begin{problem}\label{exer:8.4.23} Use Theorem~\ref{thmtype:8.4.2} to express the inverse transforms in terms of step functions, and then find distinct formulas for the inverse transforms on the appropriate intervals, as in
Example~\ref{example:8.4.7}.
$H(s)=\left(\frac{5}{s}-\frac{1}{s^2}\right)
+e^{-3s}\left(\frac{6}{s}+\frac{7}{s^2}\right)+\frac{3e^{-6s}}{s^3}$
\end{problem}

\begin{problem}\label{exer:8.4.24}
Use Theorem~\ref{thmtype:8.4.2} to express the inverse transforms in terms of step functions, and then find distinct formulas for the inverse transforms on the appropriate intervals, as in
Example~\ref{example:8.4.7}.
$H(s)=\frac{e^{-\pi s} (1-2s)}{s^2+4s+5}$
\end{problem}

\begin{problem}\label{exer:8.4.25} Use Theorem~\ref{thmtype:8.4.2} to express the inverse transforms in terms of step functions, and then find distinct formulas for the inverse transforms on the appropriate intervals, as in
Example~\ref{example:8.4.7}.  Graph the inverse transform.
$H(s)=\left(\frac{1}{s}-\frac{s}{s^2+1}\right)+e^{-\frac{\pi}{2}s}\left(\frac{3s-1}{s^2+1}\right)$
\end{problem}

\begin{problem}\label{exer:8.4.26} Use Theorem~\ref{thmtype:8.4.2} to express the inverse transforms in terms of step functions, and then find distinct formulas for the inverse transforms on the appropriate intervals, as in
Example~\ref{example:8.4.7}.  $H(s)= e^{-2s}\left[\frac{3(s-3)}{(s+1)(s-2)}-\frac{s+1}{(s-1)(s-2)}\right]$
\end{problem}

\begin{problem}\label{exer:8.4.27} Use Theorem~\ref{thmtype:8.4.2} to express the inverse transforms in terms of step functions, and then find distinct formulas for the inverse transforms on the appropriate intervals, as in
Example~\ref{example:8.4.7}.
$H(s)=\frac{1}{s}+\frac{1}{s^2}+e^{-s}\left(\frac{3}{s}+\frac{2}{s^2}\right) +e^{-3s}\left(\frac{4}{s}+\frac{3}{s^2}\right)$
\end{problem}

\begin{problem}\label{exer:8.4.28}
Use Theorem~\ref{thmtype:8.4.2} to express the inverse transforms in terms of step functions, and then find distinct formulas for the inverse transforms on the appropriate intervals, as in
Example~\ref{example:8.4.7}.
$H(s)=\frac{1}{s}-\frac{2}{s^3}+e^{-2s}\left(\frac{3}{s}-\frac{1}{s^3}\right) +\frac{e^{-4s}}{s^2}$
\end{problem}

\begin{problem}\label{exer:8.4.29} Find ${\cal L}\left(u(t-\tau)\right)$.
\end{problem}

\begin{problem}\label{exer:8.4.30}
Let $\{t_m\}_{m=0}^\infty$ be a sequence of points such that $t_0=0$,
$t_{m+1}>t_m$, and $\lim_{m\to\infty}t_m=\infty$. For each nonnegative
integer $m$, let $f_m$ be continuous on $[t_m,\infty)$, and let $f$ be
defined on $[0,\infty)$ by
$$
f(t)=f_m(t),\,t_m\le t<t_{m+1}\quad (m=0,1,\dots).
$$
Show that $f$ is piecewise continuous on $[0,\infty)$  and
that it has the step function representation
$$
f(t)=f_0(t)+\sum_{m=1}^\infty u(t-t_m)\left(f_m(t)-f_{m-1}(t)\right),\,
0\le t<\infty.
$$
How do we know that the series on the right converges for all $t$
in $[0,\infty)$?
\end{problem}

\begin{problem}\label{exer:8.4.31}
In addition to the assumptions of Exercise~\ref{exer:8.4.30},
assume that
$$
|f_m(t)|\le Me^{s_0t},\,t\ge t_m,\,m=0,1,\dots,
\text{(A)}
$$
and that the series
$$
\sum_{m=0}^\infty e^{-\rho t_m}
\text{(B)}
$$
converges for some $\rho>0$. Using the  steps listed below, show
that
${\cal L}(f)$ is defined for $s>s_0$  and
$$
{\cal L}(f)={\cal L}(f_0)+\sum_{m=1}^\infty e^{-st_m}{\cal L}(g_m)
\text{(C)}
$$
for $s>s_0+\rho$,
where
$$
g_m(t)=f_m(t+t_m)-f_{m-1}(t+t_m).
$$

\begin{enumerate} % (a)
\item
Use (A) and Theorem~8.1.6 to show that
$$
{\cal L}(f)=\sum_{m=0}^\infty\int_{t_m}^{t_{m+1}}e^{-st}f_m(t)\,dt
\text{(D)}
$$
is defined for $s>s_0$.
\item % (b)
Show that  (D) can be rewritten as
$$
{\cal L}(f)=\sum_{m=0}^\infty\left(\int_{t_m}^\infty e^{-st}f_m(t)\,dt
-\int_{t_{m+1}}^\infty e^{-st}f_m(t)\,dt\right).
\text{(E)}
$$
\item % (c)
Use  (A), the assumed convergence of  (B), and the
comparison test  to show that the series
$$
\sum_{m=0}^\infty\int_{t_m}^\infty e^{-st}f_m(t)\,dt\mbox{\quad and \quad}
\sum_{m=0}^\infty\int_{t_{m+1}}^\infty e^{-st}f_m(t)\,dt
$$
both converge (absolutely) if $s>s_0+\rho$.
\item % (d)
Show that  (E) can be rewritten as
 $$
{\cal L}(f)={\cal L}(f_0)+\sum_{m=1}^\infty\int_{t_m}^\infty e^{-st}
\left(f_m(t)-f_{m-1}(t)\right)\,dt
$$
if $s>s_0+\rho$.
\item % (e)
Complete the proof of  (C).
\end{enumerate}
\end{problem}

\begin{problem}\label{exer:8.4.32} Suppose $\{t_m\}_{m=0}^\infty$ and
$\{f_m\}_{m=0}^\infty$ satisfy the assumptions of
Exercises~\ref{exer:8.4.30} and \ref{exer:8.4.31}, and
there is a positive constant $K$ such that $t_m\ge Km$ for $m$
sufficiently large. Show that the series (B) of
Exercise~\ref{exer:8.4.31} converges for any $\rho>0$, and conclude from
this that (C) of Exercise~\ref{exer:8.4.31} holds for $s>s_0$.
\end{problem}



\begin{problem}\label{exer:8.4.33}
 Find the step
function representation of $f$ and use the result of
Exercise~\ref{exer:8.4.32} to find ${\cal L}(f)$. (You will need
formulas related to the formula for the sum of a geometric series.)
$f(t)=m+1,\,m\le t<m+1\; (m=0,1,2,\dots)$
\end{problem}

\begin{problem}\label{exer:8.4.34}
 Find the step
function representation of $f$ and use the result of
Exercise~\ref{exer:8.4.32} to find ${\cal L}(f)$. (You will need
formulas related to the formula for the sum of a geometric series.)
$f(t)=(-1)^m,\,m\le t<m+1\; (m=0,1,2,\dots)$
\end{problem}

\begin{problem}\label{exer:8.4.35}
Find the step function representation of $f$ and use the result of Exercise~\ref{exer:8.4.32} to find ${\cal L}(f)$. (You will need
formulas related to the formula for the sum of a geometric series.)
$f(t)=(m+1)^2,\,m\le t<m+1\; (m=0,1,2,\dots)$
\end{problem}

\begin{problem}\label{exer:8.4.36}
 Find the step
function representation of $f$ and use the result of
Exercise~\ref{exer:8.4.32} to find ${\cal L}(f)$. (You will need
formulas related to the formula for the sum of a geometric series.)
$f(t)=(-1)^mm,\,m\le t<m+1\; (m=0,1,2,\dots)$
\end{problem}


\end{document}