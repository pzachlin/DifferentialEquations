\documentclass{ximera}
%% You can put user macros here
%% However, you cannot make new environments

%\listfiles

% Get the 'old' hints/expandables, for use on ximera.osu.edu
%\def\xmNotHintAsExpandable{true}
%\def\xmNotExpandableAsAccordion{true}



%\graphicspath{{./}{firstExample/}{secondExample/}}
\graphicspath{{./}
{aboutDiffEq/}
{applicationsLeadingToDiffEq/}
{applicationsToCurves/}
{autonomousSecondOrder/}
{basicConcepts/}
{bernoulli/}
{constCoeffHomSysI/}
{constCoeffHomSysII/}
{constCoeffHomSysIII/}
{constantCoeffWithImpulses/}
{constantCoefficientHomogeneousEquations/}
{convolution/}
{coolingActivity/}
{directionFields/}
{drainingTank/}
{epidemicActivity/}
{eulersMethod/}
{exactEquations/}
{existUniqueNonlinear/}
{frobeniusI/}
{frobeniusII/}
{frobeniusIII/}
{global.css/}
{growthDecay/}
{heatingCoolingActivity/}
{higherOrderConstCoeff/}
{homogeneousLinearEquations/}
{homogeneousLinearSys/}
{improvedEuler/}
{integratingFactors/}
{interactExperiment/}
{introToLaplace/}
{introToSystems/}
{inverseLaplace/}
{ivpLaplace/}
{laplaceTable/}
{lawOfCooling/}
{linSysOfDiffEqs/}
{linearFirstOrderDiffEq/}
{linearHigherOrder/}
{mixingProblems/}
{motionUnderCentralForce/}
{nonHomogeneousLinear/}
{nonlinearToSeparable/}
{odesInSage/}
{piecewiseContForcingFn/}
{population/}
{reductionOfOrder/}
{regularSingularPts/}
{reviewOfPowerSeries/}
{rlcCircuit/}
{rungeKutta/}
{secondLawOfMotion/}
{separableEquations/}
{seriesSolNearOrdinaryPtI/}
{seriesSolNearOrdinaryPtII/}
{simplePendulum/}
{springActivity/}
{springProblemsI/}
{springProblemsII/}
{undCoeffHigherOrderEqs/}
{undeterminedCoeff/}
{undeterminedCoeff2/}
{unitStepFunction/}
{varParHigherOrder/}
{varParamNonHomLinSys/}
{variationOfParameters/}
}


\usepackage{tikz}
%\usepackage{tkz-euclide}
\usepackage{tikz-3dplot}
\usepackage{tikz-cd}
\usetikzlibrary{shapes.geometric}
\usetikzlibrary{arrows}
\usetikzlibrary{decorations.pathmorphing,patterns}
\usetikzlibrary{backgrounds} % added by Felipe
% \usetkzobj{all}   % NOT ALLOWED IN RECENT TeX's ...
\pgfplotsset{compat=1.13} % prevents compile error.

\pdfOnly{\renewcommand{\theHsection}{\thepart.section.\thesection}}  %% MAKES LINKS WORK should be added to CLS
\pdfOnly{\renewcommand{\part}[1]{\chapterstyle\title{#1}\begin{abstract}\end{abstract}\maketitle\def\thechaptertitle{#1}}}


\renewcommand{\vec}[1]{\mathbf{#1}}
\newcommand{\RR}{\mathbb{R}}
\providecommand{\dfn}{\textit}
\renewcommand{\dfn}{\textit}
\newcommand{\dotp}{\cdot}
\newcommand{\id}{\text{id}}
\newcommand\norm[1]{\left\lVert#1\right\rVert}
\newcommand{\dst}{\displaystyle}
 
\newtheorem{general}{Generalization}
\newtheorem{initprob}{Exploration Problem}

\tikzstyle geometryDiagrams=[ultra thick,color=blue!50!black]

\usepackage{mathtools}

\title{Exercises} \license{CC BY-NC-SA 4.0}

\begin{document}

\begin{abstract}
\end{abstract}
\maketitle

\begin{onlineOnly}
\section*{Exercises}
\end{onlineOnly}


 \begin{problem}\label{exer:10.6.1}  
 Find the general solution.
 
 $ {\bf y}'= \begin{bmatrix}-1&2\\-5&5\end{bmatrix}{\bf y}$
\end{problem}

 \begin{problem}\label{exer:10.6.2}  
 Find the general solution.
 
 $ {\bf y}'= \begin{bmatrix}-11&4\\-26&9\end{bmatrix}{\bf y}$

 \begin{solution}
     $\begin{vmatrix}-11-\lambda&4\\-26&9-\lambda
\end{vmatrix}=(\lambda+1)^2+4$.
The augmented matrix of
$\left(A-\left(-1+2i\right)I\right){\bf x}={\bf 0}$ is
$ \begin{bmatrix}-10-2i&4&\vdots&0\\
-26&10-2i&\vdots&0  \end{bmatrix}$,
which is row equivalent to
$ \begin{bmatrix} 1&(-5+i)/13&\vdots&0\\
0&0&\vdots&0
 \end{bmatrix}$.
Therefore,    $x_1=(5-i)x_2/13$. Taking $x_2=13$ yields the eigenvector
${\bf x}= \begin{bmatrix}5-i\\13 \end{bmatrix}$.
Taking  real and imaginary parts of
$e^{-t}(\cos2t+i\sin2t)\begin{bmatrix}5-i\\13 \end{bmatrix}$ yields
$$
 {\bf y}= c_1e^{-t} \begin{bmatrix}5\cos2t+\sin2t\\13\cos2t \end{bmatrix}
+c_2e^{-t} \begin{bmatrix}5\sin2t-\cos2t\\13\sin2t \end{bmatrix}.
$$
 \end{solution}
\end{problem}


 \begin{problem}\label{exer:10.6.3}  
 Find the general solution.
 
 $ {\bf y}'= \begin{bmatrix}1&2\\ -4&5\end{bmatrix}{\bf y}$
\end{problem}

 \begin{problem}\label{exer:10.6.4}  
 Find the general solution.
 
 $ {\bf y}'= \begin{bmatrix}5&-6\\3&-1\end{bmatrix}{\bf y}$

 \begin{solution}
     $\begin{vmatrix} 5-\lambda&-6\\ 3&-1-\lambda
\end{vmatrix}=(\lambda-2)^2+9$.
Hence, $\lambda=2+3i$ is an eigenvalue of $A$. The associated
eigenvectors satisfy $\left(A-\left(2+3i\right)I\right){\bf x}={\bf
0}$. The augmented matrix of this system is
$ \begin{bmatrix} 3-3i&-6&\vdots&0\\
3&-3-3i&\vdots&0   \end{bmatrix}$,
which is row equivalent to
$ \begin{bmatrix} 1&-1-i&\vdots&0\\
0&0&\vdots&0
 \end{bmatrix}$.
Therefore,    $x_1=(1+i)x_2$. Taking $x_2=1$ yields $x_1=1+i$, so
${\bf x}= \begin{bmatrix}1+i\\1 \end{bmatrix}$
is an eigenvector.  Taking  real and imaginary parts of
$e^{2t}(\cos3t+i\sin3t)
 \begin{bmatrix}1+i\\1 \end{bmatrix}$ yields
 ${\bf y}=c_1e^{2t} \begin{bmatrix}\cos3t-\sin3t\\\cos3t \end{bmatrix}
+c_2e^{2t} \begin{bmatrix}\sin3t+\cos3t\\\sin3t \end{bmatrix}$.

 \end{solution}
\end{problem}

 \begin{problem}\label{exer:10.6.5}  
 Find the general solution.
 
 $ {\bf
y}'= \begin{bmatrix}3&-3&1\\0&2&2\\5&1&1\end{bmatrix}{\bf y}$
\end{problem}

 \begin{problem}\label{exer:10.6.6}
 Find the general solution.
 
 $ {\bf
y}'= \begin{bmatrix}-3&3&1\\1&-5&-3\\-3&7&3\end{bmatrix}{\bf y}$

\begin{solution}
    $\begin{vmatrix}-3-\lambda&3&1\\1&-5-\lambda&-3\\-3
&7&3-\lambda\end{vmatrix}=-(\lambda+1)\left((\lambda+2)^2+4\right)$.
 The augmented matrix of $(A+I){\bf x=0}$ is
$ \begin{bmatrix}-2&3&1&\vdots&0\\1&-4&-3&\vdots&0\\
-3&7&4&\vdots&0 \end{bmatrix}$,
which is row equivalent to
$ \begin{bmatrix} 1&0&1&\vdots&0\\ 0&1&1&
\vdots&0\\ 0&0&0&\vdots&0 \end{bmatrix}$.
Therefore,    $x_1=x_2=-x_3$.  Taking $x_3=1$ yields
 ${\bf y}_1= \begin{bmatrix}-1\\-1\\1 \end{bmatrix}e^{-t}$.
The augmented matrix of
$\left(A-(-2+2i)I\right){\bf x=0}$ is
$ \begin{bmatrix}-1-2i&3&1&\vdots&0\\1&
-3-2i&-3&\vdots&0\\-3&7&5-2i&\vdots&0
 \end{bmatrix}$,
which is row equivalent to
$ \begin{bmatrix} 1&0&-(1+i)/2&\vdots&0\\
0&1&(1-i)/2&
\vdots&0\\ 0&0&0&\vdots&0 \end{bmatrix}$.
Therefore,    $x_1=\frac{(1+i) }{2}x_3$ and $x_2=-\frac{(1-i) }{2}x_3$.
Taking $x_3=2$ yields the eigenvector
${\bf x}_2= \begin{bmatrix}1+i\\-1+i\\2 \end{bmatrix}$.
The real and imaginary parts of
$e^{-2t}(\cos2t+i\sin2t) \begin{bmatrix}1+i\\-1+i\\2 \end{bmatrix}$
 are
${\bf y}_2=e^{-2t} \begin{bmatrix}\cos2t-\sin2t\\-\cos2t-\sin2t\\2\cos2t \end{bmatrix}$
 and
${\bf y}_3=e^{-2t} \begin{bmatrix}\sin2t+\cos2t\\-\sin2t+\cos2t\\2\sin2t \end{bmatrix}$.
 Therefore,   
$$
{\bf y}=c_1 \begin{bmatrix}-1\\-1\\1 \end{bmatrix}e^{-t}+c_2e^{-2t} \begin{bmatrix}\cos2t-\sin
2t\\-\cos2t-\sin  2t\\2\cos2t \end{bmatrix}+c_3e^{-2t} \begin{bmatrix}
\sin2t+\cos2t\\-\sin2t+\cos2t\\2\sin2t \end{bmatrix}.
$$
\end{solution}
\end{problem}


 \begin{problem}\label{exer:10.6.7}
 Find the general solution.
 
 $ {\bf
y}'= \begin{bmatrix}2&1&-1\\0&1&1\\1&0&1\end{bmatrix}{\bf y}$
\end{problem}

 \begin{problem}\label{exer:10.6.8}
 Find the general solution.
 
 $ {\bf
y}'= \begin{bmatrix}-3&1&-3\\4&-1&2\\4&-2&3\end{bmatrix}{\bf y}$

\begin{solution}
    $\begin{vmatrix}-3-\lambda&1&-3\\4&-1-\lambda&2\\4
&-2&3-\lambda\end{vmatrix}=-(\lambda-1)\left((\lambda+1)^2+4\right)$.
 The augmented matrix of $(A-I){\bf x=0}$ is
$ \begin{bmatrix}-4&1&-3&\vdots&0\\4&-2&2&\vdots&0\\
4&-2&2&\vdots&0 \end{bmatrix}$,
which is row equivalent to
$ \begin{bmatrix} 1&0&1&\vdots&0\\ 0&1&1&
\vdots&0\\ 0&0&0&\vdots&0 \end{bmatrix}$.
Therefore,   $x_1=x_2=-x_3$.  Taking $x_3=1$ yields
${\bf y}_1= \begin{bmatrix}-1\\-1\\1 \end{bmatrix}e^t$. The augmented matrix of
$\left(A-(-1+2i)I\right){\bf x=0}$ is
$ \begin{bmatrix}-2-2i&1&-3&\vdots&0\\4&-2i&2&\vdots&0\\
4&-2&4-2i&\vdots&0 \end{bmatrix}$,
which is row equivalent to
$ \begin{bmatrix} 1&0&(1-i)/2&\vdots&0\\
0&1&-1&\vdots&0\\ 0&0&0&\vdots&0 \end{bmatrix}$.
Therefore,   $x_1=-\frac{1-i}{2}x_3$ and $x_2=x_3$.
Taking $x_3=2$ yields the eigenvector
${\bf x}_2= \begin{bmatrix}-1+i\\2\\2 \end{bmatrix}$.
The real and imaginary parts of
$e^{-t}(\cos2t+i\sin2t) \begin{bmatrix}-1+i\\2\\2 \end{bmatrix}$
are ${\bf y}_2=e^{-t} \begin{bmatrix}-\sin2t-\cos2t\\2\cos2t\\2\cos2t \end{bmatrix}$
and ${\bf y}_3=e^{-t} \begin{bmatrix} \cos2t-\sin2t\\2\sin2t\\2\sin2t \end{bmatrix}$.
Therefore,   
$$
 {\bf
y}=c_1 \begin{bmatrix}-1\\1\\1 \end{bmatrix}e^{-t}+c_2e^{-t} \begin{bmatrix}-\sin
2t-\cos2t\\2\cos2t\\2\cos2t \end{bmatrix}+c_3e^{-t} \begin{bmatrix}
\cos2t-\sin2t\\2\sin2t\\2\sin2t \end{bmatrix}.
$$
\end{solution}
\end{problem}


 \begin{problem}\label{exer:10.6.9} 
 Find the general solution.
 
 $ {\bf y}'= \begin{bmatrix}5&-4\\10&1\end{bmatrix}{\bf
y}$
\end{problem}

  \begin{problem}\label{exer:10.6.10} 
  Find the general solution.
  
  $ {\bf
y}'=\frac{1}{3} \begin{bmatrix}7&-5\\2&5\end{bmatrix}{\bf y}$

\begin{solution}
    $\frac{1 }{3}\begin{vmatrix}7-3\lambda&-5\\2&5-3\lambda
\end{vmatrix}= \left(\lambda-2\right)^2+1$.
The augmented matrix of
$ \left(A-(2+i)I\right){\bf x}={\bf 0}$ is
$\frac{1 }{3} \begin{bmatrix}1-3i&-5&\vdots&0\\2&-1-3i&\vdots&0
 \end{bmatrix}$, which is row equivalent to
$ \begin{bmatrix} 1&-(1+3i)/2&\vdots&0\\ 0&0&\vdots&0
 \end{bmatrix}$. Therefore,  $x_1=\frac{1+3i }{2}x_2$. Taking
$x_2=2$ yields the eigenvector
${\bf x}= \begin{bmatrix}1+3i\\2 \end{bmatrix}$.
 Taking real and imaginary parts of
$e^{2t}(\cos t+i\sin t)
 \begin{bmatrix}1+3i\\2 \end{bmatrix}$
yields
$$
 {\bf y}=c_1e^{2t} \begin{bmatrix}\cos t-3\sin t
\\2\cos t \end{bmatrix}+
c_2e^{2t} \begin{bmatrix}
\sin t+3\cos t\\ 2\sin t \end{bmatrix}.
$$

\end{solution}
\end{problem}
 

 \begin{problem}\label{exer:10.6.11} 
 Find the general solution.
 
 $ {\bf y}'= \begin{bmatrix}3&2\\-5&1\end{bmatrix}{\bf y}$
 \end{problem}

 \begin{problem}\label{exer:10.6.12}  
 Find the general solution.
 
 $ {\bf
y}'= \begin{bmatrix}34&52\\-20&-30\end{bmatrix}{\bf y}$

\begin{solution}
    $\begin{vmatrix}34-\lambda&52\\-20&-30-\lambda
\end{vmatrix}=(\lambda-2)^2+16$. The augmented matrix of
$\left(A-\left(2+4i\right)I\right){\bf x}={\bf 0}$ is
$ \begin{bmatrix}32-4i&52&\vdots&0\\-20&-32-4i&\vdots&0
 \end{bmatrix}$, which is row equivalent to
$ \begin{bmatrix} 1&(8+i)/5&\vdots&0\\ 0&0&\vdots&0
 \end{bmatrix}$. Therefore,   $x_1=-\frac{(8+i)}{5}x_2$. Taking
$x_2=5$ yields the eigenvector ${\bf
x}= \begin{bmatrix}-8-i\\5 \end{bmatrix}$.
Taking real and imaginary parts of $e^{2t}(\cos4t+i\sin
4t) \begin{bmatrix}-8-i\\5 \end{bmatrix}$ yields ${\bf
y}=c_1e^{2t} \begin{bmatrix}\sin4t-8\cos4t\\
5\cos4t \end{bmatrix}+c_2e^{2t} \begin{bmatrix}
-\cos4t-8\sin4t\\ 5\sin4t \end{bmatrix}$.
\end{solution}
\end{problem}
 


 \begin{problem}\label{exer:10.6.13} 
 Find the general solution.
 
 $ {\bf y}'
= \begin{bmatrix}1&1&2\\1&0&-1\\-1&-2&-1\end{bmatrix}{\bf y}$
\end{problem}

 \begin{problem}\label{exer:10.6.14} 
 Find the general solution.
 
 $ {\bf y}'
= \begin{bmatrix}3&-4&-2\\-5&7&-8\\-10&13&-8\end{bmatrix}{\bf y}$

\begin{solution}
    $\begin{vmatrix}3-\lambda&-4&-2\\-5&7-\lambda&-8\\-10
&13&-8-\lambda\end{vmatrix}=-(\lambda+2)\left((\lambda-2)^2+9\right)$.
 The augmented matrix of $(A+2I){\bf x=0}$ is
$ \begin{bmatrix}5&-4&-2&\vdots&0\\-5&9&-8&\vdots&0\\
-10&13&-6&\vdots&0 \end{bmatrix}$,
which is row equivalent to
$ \begin{bmatrix} 1&0&-2&\vdots&0\\ 0&1&-2&
\vdots&0\\ 0&0&0&\vdots&0 \end{bmatrix}$.
Therefore,  $x_1=x_2=2x_3$.  Taking $x_3=1$ yields
${\bf y}_1= \begin{bmatrix}2\\2\\1 \end{bmatrix}e^{-2t}$.
The augmented matrix of
$\left(A-(2+3i)I\right){\bf x=0}$ is
$ \begin{bmatrix}1-3i&-4&-2&\vdots&0\\-5&5-3i&-8&\vdots&0\\
-10&13&-10-3i&\vdots&0 \end{bmatrix}$,
which is row equivalent to
$ \begin{bmatrix} 1&0&1-i&\vdots&0\\
0&1&-i&\vdots&0\\ 0&0&0&\vdots&0 \end{bmatrix}$.
Therefore,  $x_1=-(1-i)x_3$ and $x_2=ix_3$.
Taking $x_3=1$ yields the eigenvector
${\bf x}_2= \begin{bmatrix}-1+i\\i\\1 \end{bmatrix}$.
The real and imaginary parts of $e^{2t}(\cos3t+i\sin3t)
 \begin{bmatrix}-1+i\\i\\1 \end{bmatrix}$ are
${\bf y}_2=e^{2t} \begin{bmatrix}-\cos3t-\sin3t\\-\sin3t\\
\cos3t \end{bmatrix}$ and ${\bf y}_3=
c_3e^{2t} \begin{bmatrix}-\sin3t+\cos3t\\\cos3t\\
\sin3t \end{bmatrix}$. Therefore,  
$$
 {\bf
y}=c_1 \begin{bmatrix} 2\\ 2\\1  \end{bmatrix}e^{-2t}+
c_2e^{2t} \begin{bmatrix}-\cos3t-\sin3t\\-\sin3t\\
\cos3t \end{bmatrix}+c_3e^{2t} \begin{bmatrix}{c}
-\sin3t+\cos3t\\\cos3t\\\sin3t \end{bmatrix}.
$$

\end{solution}
\end{problem}
 


 \begin{problem}\label{exer:10.6.15}  
 Find the general solution.
 
 $ {\bf
y}'= \begin{bmatrix}6&0&-3\\-3&3&3\\1&-2&6\end{bmatrix}{\bf y}'$
\end{problem}

 \begin{problem}\label{exer:10.6.16} 
 Find the general solution.
 
 $ {\bf
y}'= \begin{bmatrix}1&2&-2\\0&2&-1\\1&0&0\end{bmatrix}{\bf y}'$

\begin{solution}
    $\begin{vmatrix}1-\lambda&2&-2\\0&2-\lambda&-1\\1
&0&-\lambda\end{vmatrix}=-(\lambda-2)\left((\lambda-1)^2+1\right)$.
The augmented matrix of $(A-I){\bf x=0}$ is
$ \begin{bmatrix}0&2&-2&\vdots&0\\0&
1&-1&\vdots&0\\ 1&0&-1&\vdots&0
 \end{bmatrix}$,
which is row equivalent to
$ \begin{bmatrix}1&0&-1&\vdots&0\\ 0&1&-1&
\vdots&0\\ 0&0&0&\vdots&0 \end{bmatrix}$.
Therefore,  $x_1=x_2=1$.  Taking $x_3=1$ yields
 ${\bf y}_1= \begin{bmatrix}1\\1\\1 \end{bmatrix}e^t$.
The augmented matrix of
$\left(A-(1+i)I\right){\bf x=0}$ is
$ \begin{bmatrix}-i&2&-2&\vdots&0\\0&
1-i&-1&\vdots&0\\ 1&0&-1-i&\vdots&0
 \end{bmatrix}$,
which is row equivalent to
$ \begin{bmatrix} 1&0&-1-i&\vdots&0\\ 0&1&-(1+i)/2&
\vdots&0\\ 0&0&0&\vdots&0 \end{bmatrix}$.
Therefore,  $x_1=(1+i)x_3$ and $x_2=\frac{(1+i)}{2}x_3$.
Taking $x_3=2$ yields the eigenvector
${\bf x}_2= \begin{bmatrix}2+2i\\1+i\\2 \end{bmatrix}$.
The real and imaginary parts of
$e^{4t}(\cos t+i\sin t) \begin{bmatrix}2+2i
\\1+i\\2 \end{bmatrix}$  are
${\bf y}_2=e^t \begin{bmatrix} 2\cos t-2\sin t\\\cos t-\sin
t\\ 2\cos t \end{bmatrix}$ and
${\bf y}_3=
c_3e^t \begin{bmatrix} 2\sin t+2\cos t\\\cos t+\sin t\\
2\sin t \end{bmatrix}$. Therefore,  
$$
 {\bf y}=c_1 \begin{bmatrix} 1\\1\\1
 \end{bmatrix}e^t+
c_2e^t \begin{bmatrix} 2\cos t-2\sin t
\\\cos t-\sin t\\ 2\cos t \end{bmatrix}+
c_3e^t \begin{bmatrix} 2\sin t+2\cos t\\\cos t+\sin t\\
2\sin t \end{bmatrix}.
$$
\end{solution}
\end{problem}
 

 \begin{problem}\label{exer:10.6.17}
 Solve the initial value problem.
 
$ {\bf y}'= \begin{bmatrix}4&-6\\3&-2\end{bmatrix}{\bf y},\quad {\bf
y}(0)= \begin{bmatrix}5\\2\end{bmatrix}$
\end{problem}

 \begin{problem}\label{exer:10.6.18}
 Solve the initial value problem.
 
$ {\bf y}'= \begin{bmatrix}7&15\\-3&1\end{bmatrix}{\bf y},\quad {\bf
y}(0)= \begin{bmatrix}5\\1\end{bmatrix}$

\begin{solution}
    $\begin{vmatrix}7-\lambda&15\\-3&1-\lambda
\end{vmatrix}=(\lambda-4)^2+36$. The augmented matrix of
$\left(A-\left(4+6i\right)I\right){\bf x}={\bf 0}$ is
$ \begin{bmatrix}3-6i&15&\vdots&0\\-3&-3-6i&\vdots&0
 \end{bmatrix}$, which is row equivalent to
$ \begin{bmatrix} 1&1+2i&\vdots&0\\ 0&0&\vdots&0
 \end{bmatrix}$. Therefore,  $x_1=-(1+2i)x_2$. Taking $x_2=1$ yields
the eigenvector ${\bf
x}= \begin{bmatrix}-1-2i\\1 \end{bmatrix}$.
Taking real and imaginary parts of
$e^{4t}(\cos6t+i\sin6t) \begin{bmatrix}-1-2i\\1 \end{bmatrix}$
yields ${\bf y}= c_1e^{4t}\begin{bmatrix}2\sin6t-\cos6t\\\cos6t \end{bmatrix}
+c_2e^{4t}\begin{bmatrix}-2\cos6t-\sin6t\\\sin6t \end{bmatrix}$. Now ${\bf
y}(0)=\begin{bmatrix}5\\1 \end{bmatrix}\Rightarrow
\begin{bmatrix}-1&-2\\1&0 \end{bmatrix}\begin{bmatrix}c_1\\c_2 \end{bmatrix}=\begin{bmatrix}5\\1 \end{bmatrix}$, so $c_1=1$, $c_2=-3$,
and $ {\bf
y}=e^{4t} \begin{bmatrix}5\cos6t+5\sin6t\\\cos6t-3\sin6t
 \end{bmatrix}$.
\end{solution}
\end{problem}

 \begin{problem}\label{exer:10.6.19}
 Solve the initial value problem.
 
$ {\bf y}'= \begin{bmatrix}7&-15\\3&-5\end{bmatrix}{\bf y},\quad {\bf
y}(0)= \begin{bmatrix}17\\7\end{bmatrix}$
\end{problem}

 \begin{problem}\label{exer:10.6.20}
 Solve the initial value problem.
 
$ {\bf y}'=\frac{1}{6} \begin{bmatrix}4&-2\\5&2\end{bmatrix}{\bf y},\quad {\bf
y}(0)= \begin{bmatrix}1\\-1\end{bmatrix}$

\begin{solution}
    $\frac{1}{6}\begin{vmatrix}4-6\lambda&-2\\5&2-6\lambda
\end{vmatrix}= \left(\lambda-
\frac{1}{2}\right)^2+\frac{1}{4}$.
The augmented matrix of
$ \left(A-\frac{1+i}{2}I\right){\bf x}={\bf 0}$ is
$\frac{1}{6} \begin{bmatrix}1-3i&-2&\vdots&0\\5&-1-3i&\vdots&0
 \end{bmatrix}$, which is row equivalent to
$ \begin{bmatrix} 1&-(1+3i)/5&\vdots&0\\ 0&0&\vdots&0
 \end{bmatrix}$. Therefore,  $x_1=\frac{1+3i}{5}x_2$. Taking
$x_2=5$ yields the eigenvector
${\bf x}= \begin{bmatrix}1+3i\\5 \end{bmatrix}$.
 Taking real and imaginary parts of
$e^{t/2}(\cos t/2+i\sin
t/2) \begin{bmatrix}1+3i\\5 \end{bmatrix}$
yields ${\bf y}= c_1e^{t/2}\begin{bmatrix}\cos t/2-3\sin t/2\\5\cos t/2 \end{bmatrix}
+c_2e^{t/2}\begin{bmatrix}\sin t/2+3\cos t/2\\5\sin t/2 \end{bmatrix}$. Now ${\bf
y}(0)=\begin{bmatrix}1\\-1 \end{bmatrix}\Rightarrow
\begin{bmatrix}1&3\\5&0 \end{bmatrix}\begin{bmatrix}c_1\\c_2 \end{bmatrix}=\begin{bmatrix}1\\-1 \end{bmatrix}$, so
$c_1=-\frac{1}{5}$,
$c_2=\frac{2}{5}$, and
${\bf y}= e^{t/2} \begin{bmatrix}\cos(t/2)+
\sin(t/2)\\-\cos(t/2)+2\sin(t/2)
 \end{bmatrix}$.
\end{solution}
\end{problem}

 \begin{problem}\label{exer:10.6.21}
 Solve the initial value problem.
 
$ {\bf y}'= \begin{bmatrix}5&2&-1\\-3&2&2\\1&3&2\end{bmatrix}{\bf y},\quad {\bf
y}(0)= \begin{bmatrix}4\\0\\6\end{bmatrix}$
\end{problem}

 \begin{problem}\label{exer:10.6.22}
 Solve the initial value problem.
 
$ {\bf y}'= \begin{bmatrix}4&4&0\\8&10&-20\\2&3&-2\end{bmatrix}{\bf y},\quad {\bf
y}(0)= \begin{bmatrix}8\\6\\5\end{bmatrix}$

\begin{solution}
    $\begin{vmatrix}4-\lambda&4&0\\8&10-\lambda&-20\\2
&3&-2-\lambda\end{vmatrix}=-(\lambda-8)\left((\lambda-2)^2+4\right)$.
 The augmented matrix of $(A-8I){\bf x=0}$ is
$ \begin{bmatrix}0&4&0&\vdots&0\\8&6&-20&\vdots&0\\
2&3&-6&\vdots&0 \end{bmatrix}$,
which is row equivalent to
$ \begin{bmatrix}1&0&-2&\vdots&0\\ 0&1&-2&
\vdots&0\\ 0&0&0&\vdots&0 \end{bmatrix}$.
Therefore,  $x_1=x_2=2x_3$. Taking $x_3=2$ yields
${\bf y}_1= \begin{bmatrix}2\\2\\1 \end{bmatrix}e^{8t}$.
The augmented matrix of
$\left(A-(2+2i)I\right){\bf x=0}$ is
$ \begin{bmatrix}2-2i&4&0&\vdots&0\\8&
8-2i&-20&\vdots&0\\2&3&-4-2i&\vdots&0 \end{bmatrix}$,
which is row equivalent to
$ \begin{bmatrix} 1&0&-2+2i&\vdots&0\\
0&1&-2i&\vdots&0\\ 0&0&0&\vdots&0 \end{bmatrix}$.
Therefore,  $x_1=(2-2i)x_3$ and $x_2=2ix_3$.
Taking $x_3=1$ yields the eigenvector
${\bf x}_2= \begin{bmatrix}2-2i\\2i\\1 \end{bmatrix}$.
The real and imaginary parts of
$e^{2t}(\cos2t+i\sin
2t) \begin{bmatrix}2-2i\\2i\\1 \end{bmatrix}$
are ${\bf y}_2=e^{2t} \begin{bmatrix}2\cos2t+2\sin2t
\\-2\sin2t\\2\cos2t \end{bmatrix}$ and
${\bf y}_3=e^{2t} \begin{bmatrix}2\sin2t+2\cos2t
\\2\cos2t\\\sin2t \end{bmatrix}$, so the general solution
is ${\bf y}=c_1 \begin{bmatrix}2\\2\\1 \end{bmatrix}e^{8t}+c_2
e^{2t} \begin{bmatrix}2\cos2t+2\sin2t
\\-2\sin2t\\2\cos2t \end{bmatrix}+
c_3e^{2t} \begin{bmatrix}2\sin2t+2\cos2t
\\2\cos2t\\\sin2t \end{bmatrix}$.
Now ${\bf y}(0)= \begin{bmatrix}8\\6\\5 \end{bmatrix}\Rightarrow \begin{bmatrix}2&2&-2\\2&0&2\\1&1&0 \end{bmatrix}
 \begin{bmatrix}c_1\\c_2\\c_3 \end{bmatrix}= \begin{bmatrix}8\\6\\5 \end{bmatrix}$, so $c_1=2$, $c_2=3$,
$c_3=1$, and $ {\bf y}= \begin{bmatrix}4\\4\\2 \end{bmatrix}e^{8t}
+e^{2t} \begin{bmatrix}4\cos2t+8\sin2t\\
-6\sin2t+2\cos2t\\3\cos2t+\sin2t
 \end{bmatrix}$.

\end{solution}
\end{problem}

 \begin{problem}\label{exer:10.6.23}
 Solve the initial value problem.
 
$ {\bf
y}'= \begin{bmatrix}1&15&-15\\-6&18&-22\\-3&11&-15\end{bmatrix}{\bf
y},\quad {\bf y}(0)= \begin{bmatrix}15\\17\\10\end{bmatrix}$
\end{problem}

 \begin{problem}\label{exer:10.6.24}
 Solve the initial value problem.
 
$ {\bf y}'= \begin{bmatrix}4&-4&4\\-10&3&15\\2&-3&1\end{bmatrix}{\bf y},\quad
{\bf y}(0)= \begin{bmatrix}16\\14\\6\end{bmatrix}$

\begin{solution}
    $\begin{vmatrix}4-\lambda&-4&4\\-10&3-\lambda&15\\2
&-3&1-\lambda\end{vmatrix}=-(\lambda-8)(\lambda^2+16)$.
 The augmented matrix of $(A-8I){\bf x=0}$ is
$ \begin{bmatrix}-4&-4&4&\vdots&0\\-10&-5&15&\vdots&0\\
2&-3&-7&\vdots&0 \end{bmatrix}$,
which is row equivalent to
$ \begin{bmatrix} 1&0&-2&\vdots&0\\ 0&1&1&
\vdots&0\\ 0&0&0&\vdots&0 \end{bmatrix}$.
Therefore,  $x_1=2x_3$ and $x_2=-x_3$. Taking $x_3=1$ yields
${\bf y}_1= \begin{bmatrix}2\\-1\\1 \end{bmatrix}e^{8t}$.
The augmented matrix of
$\left(A-4iI\right){\bf x=0}$ is
$ \begin{bmatrix}4-4i&-4&4&\vdots&0\\-10&3-4i&15&\vdots&0\\
2&-3&1-4i&\vdots&0 \end{bmatrix}$,
which is row equivalent to
$ \begin{bmatrix}1&0&-1+i&\vdots&0\\
0&1&-1+2i&\vdots&0\\ 0&0&0&\vdots&0 \end{bmatrix}$.
Therefore,  $x_1=(1-i)x_3$ and $x_2=(1-2i)x_3$.
Taking $x_3=1$ yields the eigenvector
${\bf x}_2= \begin{bmatrix}1-i\\1-2i\\1 \end{bmatrix}$.
The real and imaginary parts of
$(\cos4t+i\sin
4t) \begin{bmatrix}1-i\\1-2i\\1 \end{bmatrix}$
are ${\bf y}_2= \begin{bmatrix}\cos4t+\sin4t
\\\cos4t+2\sin4t\\\cos4t \end{bmatrix}$ and
${\bf y}_3= \begin{bmatrix}\sin4t-\cos4t
\\\sin4t-2\cos4t\\\sin4t \end{bmatrix}$, so the general solution
is ${\bf y}=c_1  \begin{bmatrix}2\\-1\\1 \end{bmatrix}e^{8t}+
c_2 \begin{bmatrix}\cos4t+\sin4t
\\\cos4t+2\sin4t\\\cos4t \end{bmatrix}
+c_3 \begin{bmatrix}\sin4t-\cos4t
\\\sin4t-2\cos4t\\\sin4t \end{bmatrix}$.
Now ${\bf
y}(0)= \begin{bmatrix}16\\14\\6 \end{bmatrix}\Rightarrow \begin{bmatrix}2&1&-1\\-1&1&-\\1&1&0 \end{bmatrix}
 \begin{bmatrix}c_1\\c_2\\c_3 \end{bmatrix}= \begin{bmatrix}16\\14\\6 \end{bmatrix}$, so $c_1=3$, $c_2=3$,
$c_3=-7$, and $ {\bf y}= \begin{bmatrix}6\\-3\\3 \end{bmatrix}e^{8t}
+ \begin{bmatrix}10\cos4t-4\sin4t\\17\cos4t-\sin4t\\3\cos4t-7\sin4t
 \end{bmatrix}$.

\end{solution}
\end{problem}

 \begin{problem}\label{exer:10.6.25}
Suppose an $n\times n$ matrix $A$ with
real entries has a complex eigenvalue $\lambda=\alpha+i\beta$
($\beta\ne0$) with associated eigenvector ${\bf x}={\bf u}+i{\bf v}$,
where ${\bf u}$ and ${\bf v}$ have real components. Show that ${\bf
u}$ and ${\bf v}$ are both nonzero.
\end{problem}

 \begin{problem}\label{exer:10.6.26}
Verify that
$$
{\bf y}_1=e^{\alpha t}({\bf u}\cos\beta t-{\bf v}\sin\beta t)
\mbox{\quad and\quad}
{\bf y}_2=e^{\alpha t}({\bf u}\sin\beta t+{\bf v}\cos\beta t),
$$
are the real and imaginary parts of
$$
e^{\alpha t}(\cos\beta t+i\sin\beta t)({\bf u}+i{\bf v}).
$$
\end{problem}

 \begin{problem}\label{exer:10.6.27}
 Show that if the vectors ${\bf u}$ and ${\bf v}$ are not both ${\bf
0}$ and $\beta\ne0$ then the vector functions
$$
{\bf y}_1=e^{\alpha t}({\bf u}\cos\beta t-{\bf v}\sin\beta t)\quad
\mbox{ and }\quad
{\bf y}_2=e^{\alpha t}({\bf u}\sin\beta t+{\bf v}\cos\beta t)
$$
are linearly independent on every interval. 
\begin{hint}There are
two
cases to consider: {\rm(i)} $\{{\bf u},{\bf v}\}$ linearly
independent, and {\rm(ii)}
 $\{{\bf u},{\bf v}\}$ linearly dependent. In either case, exploit the
the linear independence of $\{\cos\beta t,\sin\beta t\}$
on every interval.
\end{hint}
\end{problem}

 \begin{problem}\label{exer:10.6.28}
Suppose ${\bf u}=  \begin{bmatrix}u_1\\u_2\end{bmatrix}$ and ${\bf
v}=  \begin{bmatrix}v_1\\v_2\end{bmatrix}$ are not orthogonal; that is, $({\bf
u},{\bf v})\ne0$.
\begin{enumerate}
\item % (a)
Show that  the quadratic equation
$$
({\bf u},{\bf v})k^2+(||{\bf v}||^2-||{\bf u}||^2)k-({\bf u},{\bf
v})=0
$$
has a positive root $k_1$ and a negative root $k_2=-1/k_1$.

\begin{solution}
    From the quadratic formula the roots are
$$
k_1=\frac{||{\bf u}||^2-||{\bf v}^2||+\sqrt{(||{\bf u}||^2-||{\bf
v}^2||)^2+4({\bf u},{\bf v})^2}}{2({\bf u},{\bf v})}
$$
$$
k_2=\frac{||{\bf u}||^2-||{\bf v}^2||-\sqrt{(||{\bf u}||^2-||{\bf
v}^2||)^2+4({\bf u},{\bf v})^2}}{2({\bf u},{\bf v})}
$$
Clearly $k_1>0$ and $k_2<0$. Moreover,
$$
k_1k_2=\frac{(||{\bf u}||^2-||{\bf v}^2||)^2-\left[(||{\bf u}||^2-||{\bf
v}^2||)^2+4({\bf u},{\bf v})^2\right]}{4({\bf u},{\bf v})^2}=-1.
$$
\end{solution}

\item % (b)
Let ${\bf u}_1^{(1)}={\bf u}-k_1{\bf v}$, ${\bf v}_1^{(1)}={\bf
v}+k_1{\bf u}$, ${\bf u}_1^{(2)}={\bf u}-k_2{\bf v}$, and ${\bf
v}_1^{(2)}={\bf v}+k_2{\bf u}$, so that $({\bf u}_1^{(1)},{\bf
v}_1^{(1)}) =({\bf u}_1^{(2)},{\bf v}_1^{(2)})=0$, from the discussion
given above. Show that
$$
{\bf u}_1^{(2)}=\frac{{\bf v}_1^{(1)}}{ k_1}
\quad\text{ and }\quad
{\bf v}_1^{(2)}=-\frac{{\bf u}_1^{(1)}}{ k_1}.
$$

\begin{solution}
    Since $k_2=-1/k_1$,
$$
{\bf u}_1^{(2)}={\bf u}-k_2{\bf v}={\bf u}+\frac{1}{k_1}{\bf
v}=\frac{1}{k_1}({\bf v}+k_1{\bf u})=\frac{1}{k_1}{\bf v}_1^{(1)}
$$
$$
{\bf v}_1^{(2)}={\bf v}+k_2{\bf u}={\bf v}-\frac{1}{k_1}{\bf
u}=-\frac{1}{k_1}({\bf u}-k_1{\bf v})=-\frac{1}{k_1}{\bf u}_1^{(1)}.
$$
\end{solution}

\item % (c)
Let ${\bf U}_1$, ${\bf V}_1$, ${\bf U}_2$, and ${\bf V}_2$ be unit
vectors in the directions of ${\bf u}_1^{(1)}$, ${\bf v}_1^{(1)}$,
${\bf u}_1^{(2)}$, and ${\bf v}_1^{(2)}$, respectively. Conclude from
part (a) that ${\bf U}_2={\bf V}_1$ and ${\bf V}_2=-{\bf U}_1$, and
that therefore the counterclockwise angles from ${\bf U}_1$ to ${\bf
V}_1$ and from ${\bf U}_2$ to ${\bf V}_2$ are both $\pi/2$ or both
$-\pi/2$.
\end{enumerate}
\end{problem}

 \begin{problem}\label{exer:10.6.29}
 Find vectors
${\bf U}$ and ${\bf V}$ parallel to the axes of symmetry of the
trajectories, and plot some typical trajectories.

$ {\bf y}'= \begin{bmatrix}3&-5\\5&-3\end{bmatrix}{\bf
y}$
\end{problem}

 \begin{problem}\label{exer:10.6.30}   
 Find vectors
${\bf U}$ and ${\bf V}$ parallel to the axes of symmetry of the
trajectories, and plot some typical trajectories.

$ {\bf y}'= \begin{bmatrix}-15&10\\-25&15\end{bmatrix}{\bf y}$

\begin{solution}
    $\begin{vmatrix}-15-\lambda&10\\-25&15-\lambda
\end{vmatrix}=\lambda^2+25$.
The augmented matrix of $(A-5iI){\bf x}={\bf 0}$ is
$\begin{bmatrix}-15-5i&10&\vdots&0\\
-25&15-5i&\vdots&0 \end{bmatrix}$,
which is row equivalent to
$\begin{bmatrix} 1&(-3+i)/5&\vdots&0\\ 0&0&\vdots&0
\end{bmatrix}$.
Therefore, $x_1=\frac{(3-i)}{5}x_2$. Taking $x_2=5$
yields the eigenvector
 ${\bf x}=\begin{bmatrix}3-i\\5\end{bmatrix}$,
so ${\bf u}=  \begin{bmatrix}3\\5\end{bmatrix}$ and ${\bf v}=  \begin{bmatrix}-1\\0\end{bmatrix}$.
The quadratic equation is $-3k^2-33k+3=0$, with positive root
$k\approx.0902$.
 Routine calculations yield
 ${\bf U}\approx  \begin{bmatrix}.5257\\.8507\end{bmatrix}$,
${\bf V}\approx  \begin{bmatrix}-.8507\\.5257\end{bmatrix}$.
\end{solution}
 \end{problem}
 


 \begin{problem}\label{exer:10.6.31}   
 Find vectors
${\bf U}$ and ${\bf V}$ parallel to the axes of symmetry of the
trajectories, and plot some typical trajectories.

$ {\bf y}'= \begin{bmatrix}-4&8\\-4&4\end{bmatrix}{\bf y}$
 \end{problem}

 \begin{problem}\label{exer:10.6.32}  
 Find vectors
${\bf U}$ and ${\bf V}$ parallel to the axes of symmetry of the
trajectories, and plot some typical trajectories.

$ {\bf y}'= \begin{bmatrix}-3&-15\\3&3\end{bmatrix}{\bf y}$

\begin{solution}
    $\begin{vmatrix}-3-\lambda&-15\\3&3-\lambda
\end{vmatrix}=\lambda^2+36$.
The augmented matrix of $(A-6iI){\bf x}={\bf 0}$ is
$ \begin{bmatrix}-3-6i&-15&\vdots&0\\
3&3-6i&\vdots&0  \end{bmatrix}$,
which is row equivalent to
$ \begin{bmatrix} 1&1-2i&\vdots&0\\ 0&0&\vdots&0
 \end{bmatrix}$.
Therefore, $x_1=-(1-2i)x_2$. Taking $x_2=1$
yields the eigenvector
 ${\bf x}= \begin{bmatrix}-1+2i\\1 \end{bmatrix}$,
so ${\bf u}=  \begin{bmatrix}-1\\1\end{bmatrix}$ and ${\bf v}=  \begin{bmatrix}2\\0\end{bmatrix}$.
The quadratic equation is $-2k^2+2k+2=0$, with positive root
$k\approx1.6180$.
 Routine calculations yield
 ${\bf U}\approx  \begin{bmatrix}-.9732\\.2298\end{bmatrix}$,
${\bf V}\approx  \begin{bmatrix}.2298\\.9732\end{bmatrix}$.
\end{solution}
 \end{problem}
 

 \begin{problem}\label{exer:10.6.33}   
 Find vectors
${\bf U}$ and ${\bf V}$ parallel to the axes of symmetry of the shadow
trajectories, and plot a typical trajectory.

$ {\bf y}'= \begin{bmatrix}-5&6\\-12&7\end{bmatrix}{\bf y}$
 \end{problem}

 \begin{problem}\label{exer:10.6.34}  
 Find vectors
${\bf U}$ and ${\bf V}$ parallel to the axes of symmetry of the shadow
trajectories, and plot a typical trajectory.

$ {\bf y}'= \begin{bmatrix}5&-12\\6&-7\end{bmatrix}{\bf y}$

\begin{solution}
    $\begin{vmatrix}5-\lambda&-12\\6&-7-\lambda
\end{vmatrix}=(\lambda+1)^2+36$.
The augmented matrix of $(A-(-1+6i)I){\bf x}={\bf 0}$ is
$ \begin{bmatrix}6-6i&-12&\vdots&0\\
6&-6-6i&\vdots&0  \end{bmatrix}$,
which is row equivalent to
$ \begin{bmatrix} 1&-(1+i)&\vdots&0\\ 0&0&\vdots&0
 \end{bmatrix}$.
Therefore, $x_1=(1+i)x_2$. Taking $x_2=1$
yields the eigenvector
 ${\bf x}= \begin{bmatrix}1+i\\1 \end{bmatrix}$,
so ${\bf u}=  \begin{bmatrix}1\\1\end{bmatrix}$ and ${\bf v}=  \begin{bmatrix}1\\0\end{bmatrix}$.
The quadratic equation is $k^2-k-1=0$, with positive root
$k\approx1.6180$.
 Routine calculations yield
 ${\bf U}\approx  \begin{bmatrix}-.5257\\.8507\end{bmatrix}$,
${\bf V}\approx  \begin{bmatrix}.8507\\.5257\end{bmatrix}$.
\end{solution}
 \end{problem}
 


 \begin{problem}\label{exer:10.6.35}  
 Find vectors
${\bf U}$ and ${\bf V}$ parallel to the axes of symmetry of the shadow
trajectories, and plot a typical trajectory.

$ {\bf y}'= \begin{bmatrix}4&-5\\9&-2\end{bmatrix}{\bf y}$
 \end{problem}

 \begin{problem}\label{exer:10.6.36}  
 Find vectors
${\bf U}$ and ${\bf V}$ parallel to the axes of symmetry of the shadow
trajectories, and plot a typical trajectory.

$ {\bf y}'= \begin{bmatrix}-4&9\\-5&2\end{bmatrix}{\bf y}$

\begin{solution}
    $\begin{vmatrix}-4-\lambda&9\\-5&2-\lambda
\end{vmatrix}=(\lambda+1)^2+36$.
The augmented matrix of $(A-(-1+6i)I){\bf x}={\bf 0}$ is
$ \begin{bmatrix}-3-6i&9&\vdots&0\\
-5&3-6i&\vdots&0  \end{bmatrix}$,
which is row equivalent to
$ \begin{bmatrix} 1&-(3-6i)/5&\vdots&0\\ 0&0&\vdots&0
 \end{bmatrix}$.
Therefore, $x_1=\frac{3-6i}{5}x_2$. Taking $x_2=5$
yields the eigenvector
 ${\bf x}= \begin{bmatrix}3-6i\\5 \end{bmatrix}$,
so ${\bf u}=  \begin{bmatrix}3\\5 \end{bmatrix}$ and ${\bf v}=  \begin{bmatrix}-6\\0 \end{bmatrix}$.
The quadratic equation is $-18k^2+2k+18=0$, with positive root
$k\approx1.0571$.
 Routine calculations yield
 ${\bf U}\approx  \begin{bmatrix}.8817\\.4719 \end{bmatrix}$,
${\bf V}\approx  \begin{bmatrix}-.4719\\.8817 \end{bmatrix}$.
\end{solution}
 \end{problem}
 


 \begin{problem}\label{exer:10.6.37}   
 Find vectors
${\bf U}$ and ${\bf V}$ parallel to the axes of symmetry of the shadow
trajectories, and plot a typical trajectory.

$ {\bf y}'= \begin{bmatrix}-1&10\\-10&-1\end{bmatrix}{\bf y}$
 \end{problem}

 \begin{problem}\label{exer:10.6.38}    
 Find vectors
${\bf U}$ and ${\bf V}$ parallel to the axes of symmetry of the shadow
trajectories, and plot a typical trajectory.

$ {\bf y}'= \begin{bmatrix}-1&-5\\20&-1\end{bmatrix}{\bf y}$

\begin{solution}
    $\begin{vmatrix}-1-\lambda&-5\\20&-1-\lambda
\end{vmatrix}=(\lambda+1)^2+100$.
The augmented matrix of $(A-(-1+10i)I){\bf x}={\bf 0}$ is
$ \begin{bmatrix}-10i&-5&\vdots&0\\
20&-10i&\vdots&0  \end{bmatrix}$,
which is row equivalent to
$ \begin{bmatrix} 1&-i/2&\vdots&0\\ 0&0&\vdots&0
 \end{bmatrix}$.
Therefore, $x_1=\frac{i}{2}x_2$. Taking $x_2=2$
yields the eigenvector
 ${\bf x}= \begin{bmatrix}i\\2 \end{bmatrix}$,
so ${\bf u}=  \begin{bmatrix}0\\2\end{bmatrix}$ and ${\bf v}=  \begin{bmatrix}1\\0\end{bmatrix}$.
Since $({\bf u},{\bf v})=0$ we just normalize ${\bf u}$ and ${\bf v}$
to obtain
${\bf U}=  \begin{bmatrix}0\\1\end{bmatrix}$, ${\bf V}=  \begin{bmatrix}1\\0\end{bmatrix}$.
\end{solution}
 \end{problem}
 


 \begin{problem}\label{exer:10.6.39}  
 Find vectors
${\bf U}$ and ${\bf V}$ parallel to the axes of symmetry of the shadow
trajectories, and plot a typical trajectory.

$ {\bf y}'= \begin{bmatrix}-7&10\\-10&9\end{bmatrix}{\bf y}$
 \end{problem}

 \begin{problem}\label{exer:10.6.40}  
 Find vectors
${\bf U}$ and ${\bf V}$ parallel to the axes of symmetry of the shadow
trajectories, and plot a typical trajectory.

$ {\bf y}'= \begin{bmatrix}-7&6\\-12&5\end{bmatrix}{\bf y}$

\begin{solution}
    $\begin{vmatrix}-7-\lambda&6\\-12&5-\lambda
\end{vmatrix}=(\lambda+1)^2+36$.
The augmented matrix of $(A-(-1+6i)I){\bf x}={\bf 0}$ is
$ \begin{bmatrix}-6-6i&6&\vdots&0\\
-12&6-6i&\vdots&0  \end{bmatrix}$,
which is row equivalent to
$ \begin{bmatrix} 1&-(1-i)/2&\vdots&0\\ 0&0&\vdots&0
 \end{bmatrix}$.
Therefore, $x_1=\frac{1-i }{2}x_2$. Taking $x_2=2$
yields the eigenvector
 ${\bf x}= \begin{bmatrix}1-i\\2 \end{bmatrix}$,
so ${\bf u}=  \begin{bmatrix}1\\2\end{bmatrix}$ and ${\bf v}=  \begin{bmatrix}-1\\0\end{bmatrix}$.
The quadratic equation is $-k^2-4k+1=0$, with positive root
$k\approx.2361$.
 Routine calculations yield
 ${\bf U}\approx  \begin{bmatrix}.5257\\.8507\end{bmatrix}$,
${\bf V}\approx  \begin{bmatrix}-.8507\\.5257\end{bmatrix}$.
\end{solution}
 \end{problem}
 
\end{document}